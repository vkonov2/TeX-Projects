\chapter{Параметрическое оценивание}\label{lec:4}

\section{Оптимальные и несмещенные оценки}\label{lec:4/sec:1}

Пусть $X = (X_1, \dots, X_n)$ - случайное наблюдение, т.е. случайным вектор со значениями в $(\mathbb{R}^n, \mathcal{B}(\mathbb{R}^n))$. Пусть $P_X$ - распределение $X$, т.е.:
$$P_X (A) = P(X \in A), \; A \in \mathcal{B}(\mathbb{R}^n)$$
Будем предполагать, что $\displaystyle P_X \in \Set{P_{\theta}}{\theta \in \Theta \subseteq \mathbb{R}^1}$.\\
Тройка $(\mathbb{R}^n, \mathcal{B}(\mathbb{R}^n), \Set{P_{\theta}}{\theta \in \Theta})$ - статистическая модель.\\
Распределение $P_{\theta}$ известно с точностью до параметра $\theta$. Его надо оценить.

\begin{definition}\label{lec:4/def:1}
  	\red{Оценка параметра} $\theta$ - это любая борелевскя функция $\hat{\theta}_n = \varphi(x) \in \mathbb{R}^1$, зависящая только от наблюдений, но не от $\theta$.
\end{definition}  

\begin{definition}\label{lec:4/def:2}
	\blue{Качество оценки} будем характеризовать средне квадратическим риском:
	$$R_n (\hat{\theta}_n, \theta) := E_{\theta} (\hat{\theta}_n - \theta)^2$$
\end{definition}

\begin{remem}\label{lec:4/remem:1}
	$$E_{\theta} \varphi(x) = \underset{\mathbb{R}^n}{\overset{}{\int}}\varphi(x) P_{\theta}(dx)$$
\end{remem}

\begin{remark}\label{lec:4/remark:1}
	Пусть наблюдение $X = (X_1, \dots, X_n)$ имеет распределение $P_X$, и определено на вероятностном пространстве $(\Omega, \mathcal{F}, P)$. Обычно явный вид этого пространства в рассмотрении не участвует, но иногда его удобно конкретизировать. Например, пусть $X$ имеет плотность по мере $\mu$, т.е.:
	$$P_X (A) = \underset{A}{\overset{}{\int}}p(x) \mu(dx), \; a\in \mathcal{B}(\mathbb{R}^n)$$
	Пусть $N_P = \Set{x}{p(x) > 0}$ - носитель плотности. Тогда полагают:
	$$(\Omega, \mathcal{F}, P) = (N_P, \mathcal{B}(N_P), P_X), \;\;X(w) = X(x) = x$$
	Здесь $\mathcal{B}(N_p)$ - сигма-алгебра борелевских подмножеств $N_P$. При таком выборе распределение случайного вектора $X(x) = x$ есть $P_X$.

	При $P_X \in \Set{P_{\theta}}{\theta \in \Theta \subseteq \mathbb{R}^1}$ получаем:
	$$(\Omega, \mathcal{F}, P) = (N_P, \mathcal{B}(N_P), P_{\theta}) \text{ при некотором неизвестном } \theta \in \Theta$$
\end{remark}

\begin{definition}\label{lec:4/def:3}
	Оценка $\hat{\theta}_n$ называется \red{оптимальной} (наилучшей) в средне квадратическом смысле, если:
	$$R_n (\hat{\theta}_n, \theta) \le R_n (\tilde{\theta}_n, \theta) \; \forall \theta \in \Theta \text{ и } \forall \tilde{\theta}_n \text{ с конечной дисперсией}\eqno(7)$$
\end{definition}

\textbf{НО:} оптимальные оценки в смысле $(7)$ существуют лишь в вырожденных случаях.

Действительно, положим $\tilde{\theta}_n = \theta_0 \in \Theta$. Тогда $R_n (\tilde{\theta}_n, \theta_0) = 0$, и если $(7)$ верно, то $R_n (\hat{\theta}_n, \theta_0) = E_{\theta_0} (\hat{\theta}_n - \theta_0)^2 = 0$. Т.к. $\theta_0$ может быть любой точкой $\Theta$, получаем:
$$E_{\theta} (\hat{\theta}_n - \theta)^2 = 0 \; \forall \theta \in \Theta$$
Значит, $\hat{\theta}_n = \hat{\theta}_n (X) = \theta \text{ п.н. по } P_{\theta} \text{ мере}$. Это и означает, что ситуация вырожденная, и наблюдение $X$ полностью определяет $\theta$.

\begin{example}[]\label{lec:4/example:1}
	$X = (X_1)$, $X_1 \sim R(\theta, \theta+1), \; \theta \in \Theta = \mathbb{N}$. Тогда, если $\hat{\theta}_n = [X_1]$, то $\hat{\theta}_n = \theta \text{ п.н.}$. 
\end{example}


Сузим класс оценко, и будем искать оптимальные внутри более узкого класса.
Ради общности будем далее оценивать $\tau (\theta) \in \mathbb{R}^1$. Оценка $\hat{\tau_n} = \hat{\tau_n} (X) \in \mathbb{R}^1$. Тогда:
$$R_n (\hat{\tau_n}, \tau(\theta)) := E_{\theta} (\hat{\tau_n} - \tau(\theta))^2 =$$
$$ = E_{\theta} (\hat{\tau_n} - E_{\theta} \hat{\tau_n} + E_{\theta}\hat{\tau_n} - \tau(\theta))^2 = D_{\theta} \hat{\tau_n} + (E_{\theta} \hat{\tau_n} - \tau(\theta))^2\eqno(1)$$

\begin{definition}\label{lec:4/def:4}
	Величина $b(\theta) := E \hat{\theta}_n - \tau(\theta)$ называется \blue{смещением оценки} $\hat{\tau_n}$ в точке $\theta$. Если $b(\theta) = 0 \; \forall \theta \in \Theta$, то $\hat{\tau_n}$ называется \red{несмещенной} оценкой.
\end{definition}

Для несмещенной оценки в силу $(1)$: $\displaystyle R_n (\hat{\tau_n}, \tau(\theta)) = D_{\theta} \hat{\tau_n}$.

\begin{example}[]\label{lec:4/example:2}
	$X = (X_1, \dots, X_n), \; \{X_i\}$ н.о.р., $X_1 \sim N(\theta, \sigma^2), \; \theta \in \Theta = \mathbb{R}^1, \;\; \overline{X} = n^{-1} \underset{i=1}{\overset{n}{\sum}}X_i$. Тогда $\overline{X}$ - несмещенная оценка $\tau(\theta) = \theta$.
\end{example}

\begin{example}[]\label{lec:4/example:3}
	$X = (X_1, \dots, X_n), \; \{X_i\}$ - н.о.р., $X_1 \sim Pois (\theta), \theta > 0$, т.е. $\Theta = \mathbb{R}^{+}$. Пусть $\tau(\theta) = \frac{1}{\theta}$. Условие несмещнности для $\hat{\tau_1 (X_1)}$:
	$$E_{\theta} \hat{\tau_1} (X_1) = \underset{k \ge 0}{\overset{}{\sum}} \hat{\tau_1} (k) \frac{\theta^k}{k!}e^{-\theta} = \frac{1}{\theta} \; \forall \theta > 0$$
	Значит:
	$$\underset{k \ge 0}{\overset{}{\sum}} \hat{\tau_1} (k) \frac{\theta^{k+1}}{k!} = e^{\theta} = \underset{r \ge 0}{\overset{}{\sum}}\frac{\theta^r}{r!} \; \forall \theta > 0$$
	Но это невозможно, т.е. все коэффициенты рядов должны совпадать, а слева коэффициенты при $\theta^0$ есть ноль, а справа - единица.
	
	Т.о. \textbf{нет несмещенный оценок для $\tau(\theta) = \frac{1}{\theta}$}.
\end{example}

\begin{definition}\label{lec:4/def:5}
	Несмещенная оценка с конечной дисперсией $\hat{\tau_n} = \hat{\tau_n}(X)$ функции $\tau(\theta)$ называется \red{с.к. оптимальной}, если:
	$$D_{\theta} \hat{\tau_n} \le D_{\theta} \tilde{\tau_n} \; \forall \theta \in \Theta \text{ и } \forall \text{ несмещенной } \tilde{\tau_n} \text{ с конечной диспресией}$$
\end{definition}

\begin{remark}\label{lec:4/remark:2}
	Иногда рассмтаривают класс $\mathbb{C}$ несмещенный оценок с конечной дисперсией и некоторым дополнительным условием, например: 
	$$E_{\theta} \hat{\tau_n}^2 = \alpha < \infty \; \forall \theta \in \Theta$$ 
	Тогда в определение надо добавить $\hat{\tau_n}, \tilde{\tau_n} \in \mathbb{C}$. Это с.к. оптимальность в $\mathbb{C}$.
\end{remark}

\section{Неравенство Рао-Крамера и информация Фишера}\label{lec:4/sec:2}

Пусть распределение $P_{\theta}$ имеет плотность $p(x, \theta)$ в абсолютно непрерывном случае по мере $\mu$. Тогда:
$$E_{\theta} \varphi(x) = \underset{N_P}{\overset{}{\int}} \varphi(x) p(x, \theta) \mu (dx) = \begin{cases}
	\underset{N_p}{\overset{}{\int}}\varphi(x) p(x, \theta) dx \text{ в абс. непр. случае}\\
	\underset{i}{\overset{}{\sum}}\varphi(x_i) P(X = x_i) \text{ в дискретном случае}
\end{cases}$$

\begin{condition}[\red{R}]\label{lec:4/sec:1}
	Перечислим ряд условий:
	\begin{itemize}
		\item[$(i)$] Носитель $N_P = \Set{x}{p(x, \theta) > 0}$ не зависит от $\theta$.
		\item[$(ii)$] $\Theta$ - интервал, и $\forall x \in N_P$ существует производная $\frac{\partial}{\partial \theta} \ln p(x, \theta)$ при любом $\theta \in \Theta$.
		\item[$(iii)$] 
			\begin{itemize}
				\item[$(a)$] Верно равенство:
				$$\frac{\partial}{\partial \theta} \underset{N_P}{\overset{}{\int}}p(x, \theta) \mu (dx) = \underset{N_P}{\overset{}{\int}}\frac{\partial}{\partial \theta} p(x, \theta) \mu(dx) = 0 \; \forall \theta \in \Theta$$
				\item[$(b)$] Верно соотношение:
				$$\tau' (\theta) = \frac{\partial}{\partial \theta} \underset{N_P}{\overset{}{\int}} \hat{\tau_n}(x) p(x, \theta) \mu (dx) = \underset{N_P}{\overset{}{\int}} \hat{\tau_n}(x)\frac{\partial}{\partial \theta} p(x, \theta) \mu(dx) = 0 \; \forall \theta \in \Theta$$
			\end{itemize}
		\item[$(iv)$] Существует величина:
		$$I(\theta) := E_{\theta} \left( \frac{\partial}{\partial \theta} \ln p(X, \theta)\right)^2, \;\; 0 < I(\theta) < \infty$$
		$I(\theta)$ называется \blue{информацией Фишера} о $\theta$, содержащейся в $X$.
	\end{itemize}
\end{condition}

\begin{theorem}[\red{неравенство Рао-Крамера}]\label{lec:4/the:1}
	Пусть $\hat{\tau_n}(x)$ - несмещенная оценка для $\tau(\theta)$ с конечной при всех $\theta \in \Theta$ диспресией. Пусть выполнено условие $(R)$. Тогда:
	$$D_{\theta} \hat{\tau_n} (x) \ge \frac{(\tau'(\theta))^2}{I(\theta)} \; \forall \theta \in \Theta$$
\end{theorem}
\begin{Proof}
	В силу условия $(iii)(a)$:
	$$E_{\theta} \frac{\partial}{\partial \theta} \ln p(X, \theta) = \underset{N_P}{\overset{}{\int}}\left( \ln p(x, \theta) \right)p(x, \theta) \mu(dx) = \underset{N_P}{\overset{}{\int}} \frac{\partial}{\partial \theta} p(x, \theta) \mu(dx) = 0 \; \forall \theta \in \Theta\eqno(2)$$
	В силу условия $(iii)(b)$ и $(2)$:
	$$\tau'(\theta) = \underset{N_P}{\overset{}{\int}}\hat{\tau_n}(x) \frac{\partial p(x, \theta)}{\partial \theta} \mu(dx) = E_{\theta} \left( \hat{\tau_n}(x) \frac{\partial}{\partial \theta} \ln p(X, \theta) \right) =$$
	$$=E_{\theta} \left\{ \left( \hat{\tau_n}(x) - \tau(\theta) \right) \frac{\partial}{\partial \theta} \ln p(X, \theta) \right\}\eqno(3)$$
	В силу неравенства Коши-Буняковского: $\displaystyle \left\{ E(\xi \eta) \right\}^2 \le E\xi^2 \cdot E \eta^2$.\\
	Равенство достигается тогда и только тогда, когда $\eta \PNdef a \xi$. Тогда из $(3)$ следует:
	$$\begin{gathered}
		\left(\tau'(\theta)\right)^2 = \left\{ E_{\theta} \left\{ \left( \hat{\tau_n}(x) - \tau(\theta) \right) \frac{\partial}{\partial \theta} \ln p(x, \theta) \right\} \right\}^2 \le \\
		\le  E_{\theta} \left( \hat{\tau_n}(x) - \tau(\theta) \right)^2 \cdot E_{\theta} \left( \frac{\partial}{\partial \theta} \ln p(x, \theta) \right)^2
	\end{gathered}$$
	Т.е. получаем $\displaystyle D_{\theta} \hat{\tau_n}(x) I(\theta) \ge \left( \tau'(\theta) \right)^2 \; \forall \theta \in \Theta$. Значит $\displaystyle D_{\theta} \hat{\tau_n}(x) \ge \frac{\left( \tau'(\theta) \right)^2}{I(\theta)} \; \forall \theta \in \Theta$.
\end{Proof}

\begin{remark}\label{lec:4/remark:3}
	Пусть $X = (X_1, \dots, X_n)$ и $\{X_i\}$ - н.о.р., $X_1 \sim f(x, \theta)$ по мере $\nu$. Тогда $\displaystyle \;\; p(x_1, \dots, x_n, \theta) \PNdef	\underset{i=1}{\overset{n}{\Pi}} f(x_i, \theta) \text{ по мере } \nu$.
\end{remark}

Предположим, что $\forall \theta \in \Theta$ имеем:
$$E_{\theta} \frac{\partial}{\partial \theta} \ln f(x, \theta) = 0 \text{ и } 0 < E_{\theta} \left( \frac{\partial}{\partial \theta} \ln f(X_1, \theta) \right)^2 < \infty$$

\begin{definition}\label{lec:4/def:5}
	Величина $i(\theta) = E_{\theta} \left( \frac{\partial}{\partial \theta} \ln f(X_1, \theta) \right)^2$ называется \red{информацией Фишера} о параметре $\theta$, содержащейся в одном наблюдении $X_1$.
\end{definition}

Очевидно, что $\displaystyle i(\theta) = D_{\theta} \left( \frac{\partial}{\partial \theta} \ln f(X_1, \theta) \right)$. Имеем:
$$\begin{gathered}
	I(\theta) = E_{\theta} \left( \frac{\partial}{\partial \theta} \ln f(X_1, \theta) \right)^2 = E_{\theta} \left( \frac{\partial}{\partial \theta} \ln \underset{i=1}{\overset{n}{\Pi}}f(X_i, \theta) \right)^2 = E_{\theta} \left( \underset{i=1}{\overset{n}{\sum}}\frac{\partial}{\partial \theta} \ln f(X_i, \theta) \right)^2 = \\
	= D_{\theta} \left( \underset{i=1}{\overset{n}{\sum}}\frac{\partial}{\partial \theta} \ln f(X_i, \theta) \right)^2 = n D_{\theta} \left( \frac{\partial}{\partial \theta} \ln f(X_1, \theta) \right)^2 = n i(\theta)
\end{gathered}$$
Итак: $I(\theta) = n i(\theta)$ и неравенство Рао-Крамера имеет вид:
$$D_{\theta} \hat{\tau_n}(x) \ge \frac{\left( \tau'(\theta) \right)^2}{n i(\theta)} \; \forall \theta \in \Theta$$

\section{Эффикетивные оценки, необходимое и достаточное условия равенства в НРК}\label{lec:4/sec:3}

Обозначим $\mathbb{C}_R$ класс несмещенных оценок для $\tau(\theta)$ с конечной дисперсией и удовлетворяющих условию $(R)$.

\begin{definition}\label{lec:4/def:6}
	Если для оценки $\hat{\tau_n} \in \mathbb{C}_R$ в неравенстве Рао-Крамера достигается равенство, т.е.
	$$D_{\theta} \hat{\tau_n} = \frac{\left( \tau'(\theta) \right)^2}{I(\theta)} \; \forall \theta \in \Theta$$
	то $\hat{\tau_n}$ называется \red{эффективной} в $\mathbb{C}_R$.
\end{definition}

Тогда:
$$\forall \tilde{\tau_n} \in \mathbb{C}_R \; D_{\theta} \tilde{\tau_n} = \frac{\left( \tau'(\theta) \right)^2}{I(\theta)} = D_{\theta} \hat{\tau_n} \; \forall \theta \in \Theta$$
Значит, эффективная в $\mathbb{C}_R$ оценка является оптимальной в $\mathbb{C}_R$.\\

Каковы условия равенства в неравенстве Рао-Крамера?

\begin{definition}\label{lec:4/def:7}
	Пусть вектор $X$ имеет плотность $p(x, \theta), \; \theta \in \Theta \subseteq \mathbb{R}^k$, относительно меры $\mu$. Если эта плотность представима в виде:
	$$p(x, \theta) = exp \left\{ \underset{j=1}{\overset{k}{\sum}} a_j (\theta) u_j (x) + b(\theta) \right\} \overline{h}(x), \; x \in N_P$$
	то распределение $P_{\theta}$ вектора $X$ принадлежит \blue{экспоненциальному семейству}.
\end{definition}

Обычно требуют, чтобы функции $a_0 (\theta) = 1, a_1 (\theta), \dots, a_k(\theta)$ были именно независимы на $\Theta$.

\begin{problem}
	Пусть $X = (X_1, \dots, X_n)$, $\{X_i\}$ - н.о.р. Показать: если $X_1 \sim N(\theta, \sigma^2), \; N(c_1, \theta), \; N(\theta_1, \theta_2), \; Exp(\theta), \; Pois (\theta), \; Bin(1, \theta)$, то распределение $X$ принадлежит экспоненциальному семейству.
\end{problem}

\begin{theorem}[\blue{необходимое условие равенства в неравенстве Рао-Крамера}]\label{lec:4/the:2}
	Пусть $\hat{\tau_n}$ - несмещенная оценка $\tau(\theta), \; 0 < D_{\theta} \hat{\tau_n} < \infty \; \forall \theta \in \Theta$. Пусть выполнено условие $(R)$. Пусть функции $\frac{\partial}{\partial \theta} \ln p(x, \theta)$ для $x \in N_P$, $I(\theta)$ и $\tau'(\theta)$ непрерывны по $\theta$. Тогда, если в неравенстве Рао-Крамера достигается равенство, то:
	$$p(x, \theta) = exp \left\{ \underset{j=1}{\overset{k}{\sum}} a_j (\theta) u_j (x) + b(\theta) \right\} \overline{h}(x), \; x \in N_P, \; \theta \in \Theta\eqno(4)$$ 
\end{theorem}
\begin{remark}\label{lec:4/remark:4}
	Теорема \ref{lec:4/the:1} означает, что если эффективная в $\mathbb{C}_R$ оценка для $\tau(\theta)$ существует, то $p(x, \theta)$ есть плотность из экспоненциального семейства специального вида $(4)$.
\end{remark}
\begin{Proof}
	Из доказательства неравенства Рао-Крамера следует, что равенство в этом неравенстве достигается тогда и только тогда, когда при фиксированном $\theta \in \Theta$:
	$$\hat{\tau_n}(x) - \tau(\theta) = a(\theta) \frac{\partial}{\partial \theta} \ln p(X, \theta) \;\; (P_{\theta} \text{ - п.н.})\eqno(5)$$
	Таким образом, из последнего равенства надо получить $(4)$. У нас $(\Omega, \mathcal{F}, P) = (N_P, \mathcal{B}(N_P), P_{\theta}), \; X(x) = x$. Поэтому упомянутое равенство $(5)$ эквивалентно:
	$$\hat{\tau_n}(x) - \tau(\theta) = a(\theta) \frac{\partial}{\partial \theta} \ln p(x, \theta) \text{ для } P_{\theta}\text{-п.в.}, \; x \in N_P\eqno(6)$$
	При фиксированном $\theta$ соотношение $(6)$ не выполнено при $x \in A_{\theta}, \; P_{\theta} (A_{\theta}) = 0$. При $x \in \overline{A_{\theta}}$ $(6)$ выполнено, $P_{\theta}(\overline{A_{\theta}}) = 1$ ($A_{\theta} \in N_P, \; \overline{A_{\theta}} = N_P \setminus A_{\theta}$).\\
	
	\noindentРассмотрим $(5)$. Домножим $(5)$ на $\frac{\partial}{\partial \theta} \ln p(X, \theta)$ и возьмем среднее:
	$$\begin{gathered}
		E_{\theta} \left\{ \hat{\tau_n}(x) \frac{\partial}{\partial \theta} \ln p(X, \theta) \right\} = a(\theta) I(\theta) = \tau'(\theta) \\
		\left(\text{воспользовались условием } (R)\right)
	\end{gathered}$$
	Значит, $a(\theta) = \frac{\tau' (\theta)}{I(\theta)}$ - непрерывная функция, и $a(\theta) \not = 0$, т.к. $\tau'(\theta) \not = 0$ из-за условия $D_{\theta} \hat{\tau_n} > 0$.\\
	
	\noindentРассмотрим $(6)$. $\displaystyle P_{\theta}(A_{\theta}) = \underset{A_{\theta}}{\overset{}{\int}}p(x, \theta) \mu(dx) = 0 \; \Rightarrow \; \mu(A_{\theta}) = 0$. 
	Пусть $A = \underset{\theta \in \mathbb{Q}}{\overset{}{\bigcup}} A_{\theta}$, тогда $\mu (A) = 0$, но $\overline{A} = \underset{\theta \in \mathbb{Q}}{\overset{}{\bigcap}} \overline{A_{\theta}}$, и при $x \in \overline{A}$ соотношение $(6)$ выполнено при всех рациональных $\theta$. Но левая и правая части $(6)$ непрерывны по $\theta$. Значит, при $x \in \overline{A}$ $(6)$ верно при всех $\theta$. Тогда при любом $x \in \overline{A}$ из $(6)$ следует:
	$$\begin{gathered}
		\frac{\partial}{\partial \theta} \ln p(x, \theta) = \frac{\hat{\tau_n}(x)}{a(\theta)} - \frac{\tau(\theta)}{a(\theta)} \; \Rightarrow \\
		\Rightarrow \; \ln p(x, \theta) = \hat{\tau_n} (x) \underset{\theta_1}{\overset{\theta}{\int}}\frac{d\theta}{a(\theta)} - \underset{\theta_1}{\overset{\theta}{\int}} \frac{\tau'(\theta)}{a(\theta)}d\theta + \ln p(x, \theta_1)
	\end{gathered}$$
	$$\underset{\theta_1}{\overset{\theta}{\int}}\frac{d\theta}{a(\theta)} = A(\theta), \;\; - \underset{\theta_1}{\overset{\theta}{\int}} \frac{\tau'(\theta)}{a(\theta)}d\theta = B(\theta), \;\; \ln p(x, \theta_1) = \overline{h}(x)$$
	Отсюда: $\displaystyle p(x, \theta) = exp\{\hat{\tau_n}(x) A(\theta) + B(\theta)\} \overline{h}(x), \; x \in \overline{A}$.
	На множестве $A, \; \mu(A) = 0$, значения плотности вещественны. Т.о. $(4)$ верно при всех $x \in N_P, \; \theta \in \Theta$.
\end{Proof}

\begin{theorem}[\blue{достаточное условие равенства в неравенстве Рао-Крамера}]\label{lec:4/the:3}
	Пусть $\hat{\tau_n}$ - несмещенная оценка $\tau(\theta), \; 0 < D_{\theta} \hat{\tau_n} < \infty \; \forall \theta \in \Theta$. Пусть выполнено условие $(R)$. Тогда, если:
	$$p(x, \theta) = exp\{\hat{\tau_n}(x) A(\theta) + B(\theta)\}\overline{h}(x), \; x \in N_P\eqno(7)$$
	то в неравенстве Рао-Крамера достигается равенство.
\end{theorem}
\begin{Proof}
	В силу $(7)$ при $x \in N_p, \; \theta \in \Theta$:
	$$\ln p(x, \theta) = \hat{\tau_n}(x) A(\theta) + B(\theta) + \ln \overline{h}(x)$$
	Значит:
	$$\begin{gathered}
		\frac{\partial}{\partial \theta} \ln p(x, \theta) = \hat{\tau_n}(x) A'(\theta) + B'(\theta) = A'(\theta)\left(\hat{\tau_n}(x) + \frac{B'(\theta)}{A'(\theta)}\right) =\\
		= A'(\theta) (\hat{\tau_n}(x) - \tau (\theta)), \; x \in N_P, \; \theta \in \Theta
	\end{gathered}$$
	Последнее соотношение влечет $(6)$, а значит и $(5)$.
\end{Proof}

Итак, в силу теорем 2 и 3 равенство в неравенстве Рао-Крамера достигается лишь для плотностей
$$\begin{gathered}
	p(x, \theta) = exp\left\{ \hat{\tau_n}(x) A(\theta) + B(\theta) \right\} \overline{h}(x), \; x \in N_P, \; \theta \in \Theta, \\
	\text{причем } -\frac{B'(\theta)}{A'(\theta)} = \tau(\theta)
\end{gathered}$$
Это очень специальный вид плотности из экспоненциального семейства. Т.о., эффикетивных оценок мало.

\begin{example}[]\label{lec:4/example:4}
	$X = (X_1, \dots, X_n), \; \{X_i\}$ - н.о.р., $X_1 \sim N(\theta, \sigma^2), \; \theta \in \in \Theta = \mathbb{R}^1$. $\tau(\theta) = \theta$. Найти эффективную оценку.
\end{example}
\begin{solution}
	Здесь $\tau(\theta) = \theta$.
	$$\begin{gathered}
		p(x, \theta) = \left(\frac{1}{\sqrt{2 \pi} \sigma}\right)^n exp\left\{ -\frac{1}{2 \sigma^2} \underset{i=1}{\overset{n}{\sum}} (x_i - \theta)^2 \right\} =\\
		= \left(\frac{1}{\sqrt{2 \pi} \sigma}\right)^n exp\left\{ \overline{X} \cdot \frac{n \theta}{\sigma^2} - \frac{n \theta^2}{2 \sigma^2} \right\} \cdot exp \left\{ -\frac{1}{2 \sigma^2} \underset{i=1}{\overset{n}{\sum}}x_i^2 \right\}, \; \overline{X} = n^{-1} \underset{i=1}{\overset{n}{\sum}}x_i 
	\end{gathered}$$
	Здесь $\hat{\tau_n}(x) = \overline{X}, \; A(\theta) = \frac{n \theta}{\sigma^2}, \; B(\theta) = \frac{n \theta^2}{2 \theta^2}, \; -\frac{B'(\theta)}{A'(\theta)} = \theta = \tau (\theta)$.\\
	Прочие условия теоремы 3 выполнены. В силу теоремы 3 $\hat{\tau_n}(x) = \overline{X}$ - эффективная оценка $\tau(\theta) = \theta$.
\end{solution}

\vspace{0.3cm}
Можно показать, что если некоторая функция $\tau(\theta)$ допускает эффективное оценивание $\hat{\tau_n}(x)$, то эффективно можно оценить еще функцию $a \tau(\theta) + b$ ($a,b$ - константы) и никакие другие. Оценка - $a \hat{\tau_n}(x) + b$.

Значит, в последнем примере все функции, допускающие эффективное оценивание, имеют вид $\tau(\theta) = a \theta + b$, а их оценки $\hat{\tau_n}(x) = a \overline{X} + b$.













