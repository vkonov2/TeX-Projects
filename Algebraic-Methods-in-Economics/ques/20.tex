\chapter{Критерий оптимальности допустимого плана транспортной задачи в терминах потенциалов}
\label{cha:20}

\epigraph{
	\textit{Новые замыслы, новые планы, новые разветвления!}}
{-- Салтыков-Щедрин М.Е.}

Пусть в заданных m городах
$$A_1, \dots, A_m\eqno(57)$$
производится некоторый однородный продукт в количествах $a_1, \dots, a_m > 0$.

Этот продукт перевозится в заданные n городов
$$B_1, \dots, B_n\eqno(58)$$
где он полностью потребляется в количествах $b_1, \dots, b_n > 0$. Предполагаются заданными стоимости $c_{ij} \ge 0$ перевозок единицы продукта из $A_i$ в $B_j$.

Назовем \textit{планом перевозок} неотрицательную матрицу $X = (x_{ij})$ размера $m \times n$, в которой $x_{ij} \ge 0$ указывает количество продукта, перевозимого из $A_i$ в $B_j$, $1 \le i \le m$, $1 \le j \le n$. Стоимость перевозок является линейной функцией от X:
$$z(X) = \underset{i,j}{\overset{}{\sum}}c_{ij}x_{ij}, \; x_{ij} \ge 0\eqno(59)$$
В задаче требуется найти такой план перевозок X, чтобы его стоимость была бы минимальной, весь продукт был вывезен из (57) и потребности городов (58) были полностью удовлетворены.

Транспортная задача является задачей линейного программирования. Действительно, весь продукт, производимый в (57), вывозится в (58), где он полностью потребляется, причем все потребности удовлетворены. Таким образом, возникают следующие условия:
$$\begin{gathered}
	\underset{j=1}{\overset{n}{\sum}}x_{ij} = a_i, \; i = \ton m \\
	\underset{i=1}{\overset{m}{\sum}}x_{ij} = b_j, \; j = \ton n \\
	x_{ij} \ge 0
\end{gathered}\eqno(60)$$
Требуется в условиях (60) найти минимум функции (59). Ввиду специфичности условий (60) можно предложить более специальный метод потенциалов решения этой задачи.

Назовем план перевозок X \textit{допустимым}, если выполнены условия (60).

\begin{propose}\label{cha:20/propose:1}
	Полиэдр P, задаваемый условиями (60), непуст тогда и только тогда, когда
	$$\underset{i}{\overset{}{\sum}}a_i = \underset{j}{\overset{}{\sum}}b_j\eqno(61)$$
\end{propose}
\begin{Proof}
	Если $X = (x_{ij})$ – допустимый план, то по (60):
	$$\underset{i}{\overset{}{\sum}}a_i = \underset{i}{\overset{}{\sum}}\underset{j}{\overset{}{\sum}}x_{ij} = \underset{j}{\overset{}{\sum}}\underset{i}{\overset{}{\sum}}x_{ij} = \underset{j}{\overset{}{\sum}}b_j$$
	Обратно, пусть выполнено условие (61). Построим матрицу первоначального плана $X^0 = (x_{ij}^0)$ методом минимального элемента. Выберем клетку $(i,j)$ с минимальным значением $c_{ij}$. В эту клетку ставим число $x_{ij}^0 = \min (a_i, b_j)$. Если $x_{ij}^0 = b_j$, то в остальные клетки столбца j ставим 0, а число $a_i$ заменяем на $a_i - b_j$. Дуальным образом поступаем, если $x_{ij}^0 = a_i$. Если $a_i \le b_j$, то из $A_i$ весь продукт вывезен в $B_j$ и потребность $B_j$ становится равной $b_j - a_i$. Если же $b_j < a_i$, то в $B_j$ весь необходимый продукт завезен, и в $A_i$ осталось $a_i - b_j$ продукта. Таким образом, либо число пунктов $A_t$, либо пунктов $B_s$ уменьшилось.

	Повторяя эту процедуру, получим первоначальный допустимый план $X^0 = (x_{ij}^0)$.
\end{Proof}

\begin{propose}\label{cha:20/propose:2}
	Полиэдр P, задаваемый условиями (60), ограничен.
\end{propose}
\begin{Proof}
	Если $X = (x_{ij})$ – допустимый план, то для любых индексов $i,j$ имеем $0 \le x_{ij} \le \min (a_i, b_j)$.
\end{Proof}

\begin{conseq}[]\label{cha:20/conseq:1}
	Транспортная задача (60) имеет решение тогда и только тогда, когда выполнено равенство (61).
\end{conseq}
\begin{Proof}
	По предложению \ref{cha:20/propose:1} полиэдр P допустимых планов непуст. По предложению \ref{cha:20/propose:2} он ограничен. Следовательно, непрерывная функция $z(X)$ достигает на P минимума.
\end{Proof}

\begin{theorem}[]\label{cha:20/the:1}
	Для того, чтобы допустимый план перевозок X был оптимальным необходимо и достаточно, чтобы существовали числа (потенциалы) \\ $u_1, \dots, u_m, v_1, \dots, v_n$, для которых:
	\begin{itemize}
		\item[1)] $u_i + v_j \le c_{ij}$ при всех $i, j$
		\item[2)] $u_i + v_j = c_{ij}$, если $x_{ij} > 0$
	\end{itemize}
\end{theorem}
\begin{Proof}
	Проверим достаточность. Пусть $X = (x_{ij})$ – допустимый план, и $u_1, \dots, u_m, v_1, \dots, v_n$ из условий теоремы. Предположим, что $Y = (y_{ij})$ – произвольный допустимый план. Из (60) и условий 1), 2) следует, что:
	$$\begin{gathered}
		z(Y) = \underset{i, j}{\overset{}{\sum}}c_{ij} y_{ij} \ge \underset{i, j}{\overset{}{\sum}}(u_i + v_j) y_{ij} = \underset{i}{\overset{}{\sum}}u_i \underset{j}{\overset{}{\sum}}y_{ij} + \underset{j}{\overset{}{\sum}}v_j \underset{i}{\overset{}{\sum}}y_{ij} = \\
		= \underset{i}{\overset{}{\sum}}u_i a_i + \underset{j}{\overset{}{\sum}}v_j b_j = \underset{i}{\overset{}{\sum}}u_i \underset{j}{\overset{}{\sum}}x_{ij} + \underset{j}{\overset{}{\sum}}v_j \underset{i}{\overset{}{\sum}}x_{ij} = \\
		= \underset{i, j}{\overset{}{\sum}}(u_i + v_j) x_{ij} = \underset{i, j}{\overset{}{\sum}}c_{ij} x_{ij} = z(X)
	\end{gathered}$$
	Проверим теперь необходимость. Рассмотрим задачу, двойственную к транспортной задаче. Для этого перепишем ограничения из (60) и целевую функцию в виде:
	$$\begin{gathered}
		-z(X) = \underset{i, j}{\overset{}{\sum}}(-c_{ij}x_{ij}) \to \max \\
		\underset{j=1}{\overset{n}{\sum}}x_{ij} \le a_i, \; \underset{j=1}{\overset{n}{\sum}}(-x_{ij}) \le -a_i, \; i = \ton m \\
		\underset{i=1}{\overset{m}{\sum}}x_{ij} \le b_j, \; \underset{i=1}{\overset{m}{\sum}}(-x_{ij}) \le b_j, \; j = \ton n \\
		x_{ij} \ge 0
	\end{gathered}$$
	Тогда по определению \ref{cha:16/def:1} двойственная задача с переменными
	$$w_1, \dots, w_m, w'_1, \dots, w'_m, s_1, \dots, s_n, s'_1, \dots, s'_n \ge 0$$
	к транспортной задаче имеет вид:
	$$\begin{gathered}
		T' = \underset{i=1}{\overset{m}{\sum}}(w_i - w'_i)a_i + \underset{j=1}{\overset{n}{\sum}}(s_j - s'_j)b_j \to \min \\
		w_i - w'_i + s_j - s'_j \ge -c_{ij}, \; 1 \le i \le m, \; 1 \le j \le n \\
		w_1, \dots, w_m, w'_1, \dots, w'_m, s_1, \dots, s_n, s'_1, \dots, s'_n \ge 0
	\end{gathered}\eqno(62)$$
	Положим $u_i = w'_i - w_i$, $v_j = s'_j - s_j$. Тогда (62) записывается в виде:
	$$\begin{gathered}
		u_i + v_j \le c_{ij} \\
		T = - T' = \underset{i=1}{\overset{m}{\sum}}u_i a_i + \underset{j=1}{\overset{n}{\sum}}v_j b_j \to \max
	\end{gathered}\eqno(63)$$
	Как отмечено в предложении \ref{cha:20/propose:2}, условия (60) задают ограниченный полиэдр P. Таким образом, по следствиию \ref{cha:20/conseq:1} функция $z(X)$ на P достигает минимум в некоторой точке $X^0$. Заметим, что полиэдр Q, задаваемый неравенствами (63) содержит начало координат, поскольку $c_{ij} \ge 0$. Следовательно, Q непусто и по теореме \ref{cha:17/the:1}:
	$$\min \left( z(X) \right) = - \max \left( -z(X) \right) = - \min \left( T' \right) = \max T$$
	Пусть в точке $(u_1, \dots, u_m, v_1, \dots, v_n) \in Q$ достигается максимум. По теореме \ref{cha:18/the:1} о равновесии получаем $u_i + v_j = c_{ij}$, если $x_{ij}^0 > 0$.
\end{Proof}































