\chapter{Совпадение ответов прямой и двойственной задач линейного программирования}
\label{cha:17}

\epigraph{
	\textit{О, я согласен, что совокупность фактов, совпадение фактов действительно довольно красноречивы.}}
{-- Достоевский Ф.М.}

\begin{propose}\label{cha:17/propose:1}
	Пусть $x \in P$, $y \in Q$. Тогда $^t px \le ^t by$. В частности $\underset{x \in P}{\max} (^t px) \le \underset{y \in Q}{\min} (^t by)$.
\end{propose}
\begin{Proof}
	Пусть $x \in P, y \in Q$. Тогда $^t px \le ^t y A x \le ^t y b = ^t b y$. Отсюда $\underset{x \in P}{\max} (^t px) \le \underset{y \in Q}{\min} (^t by)$.
\end{Proof}

\begin{theorem}[]\label{cha:17/the:1}
	Пусть полиэдры P, Q непусты. Тогда существуют $\underset{x \in P}{\max} (^t px)$, $\underset{y \in Q}{\min} (^t by)$ и они равны.
\end{theorem}
\begin{Proof}
	Рассмотрим в $\mathbb{A}^{n+m} = \Set{(x,y)}{x \in\mathbb{A}^n, y \in \mathbb{A}^m}$ полиэдр M, задаваемый неравенствами:
	$$\begin{cases}
		A x \le b, \; ^t A y \ge p \\
		^t p x \ge ^t b y, \; x \ge 0, \; y \ge 0
	\end{cases} \eqno(43)$$ 
	Неравенства (43) можно записать в следующем виде:
	$$\begin{pmatrix}
		-A & 0 \\
		0 & ^t A \\
		^t p & -^t b \\
		E_n & 0 \\
		0 & E_m
	\end{pmatrix}\begin{pmatrix}
		x \\ y
	\end{pmatrix} + \begin{pmatrix}
		b \\ -p \\ 0 \\ 0 \\ 0
	\end{pmatrix} \ge 0\eqno(44)$$
	Предположим сначала, что система неравенств (44) несовместна. По второму следствию из теоремы Фаркаша существуют такие неотрицательные векторы $c_1, c_4 \in \mathbb{R}^m$, $c_2, c_5 \in \mathbb{R}^n$ и неотрицательное число $c_3$, что:
	$$\begin{gathered}
		\begin{pmatrix}
			-^t A & 0 & p & E_n & 0 \\
			0 & A & -b & 0 & E_m 
		\end{pmatrix}\begin{pmatrix}
			c_1 \\ c_2 \\ c_3 \\ c_4 \\ c_5
		\end{pmatrix} = 0 \\
		\begin{pmatrix}
			^t b & - ^t p
		\end{pmatrix}\begin{pmatrix}
			c_1 \\ c_2
		\end{pmatrix} < 0
	\end{gathered}$$
	Это равносильно следующей системе:
	$$\begin{cases}
		^t A c_1 = p c_3 + c_4 \ge p c_3 = 0, \; A c_2 = b c_3 - c_5 \le b c_2 \\
		^t b c_1 < ^t p c_2, \; c_i \ge 0
	\end{cases}\eqno(45)$$
	Если $c_3 = 0$ и $x_0 \in P, \; y_0 \in Q$, то:
	$$\begin{gathered}
		^t b c_1 = ^t c_1 b \ge ^t c_1 A x_0 = ^t x_0 ^t A c_1 \ge ^t x_0 p c_3 = 0 \\
		^t p c_2 = ^t c_2 p \le ^t c_2 ^t A y_0 = ^t y_0 A c_2 \le ^t y_0 b c_3 = 0
	\end{gathered}$$
	Отсюда $^t p c_2 \le 0 \le ^t b c_1$, что противоречит (45).

	Итак, $c_3 > 0$. По (45):
	$$^t A \left( \frac{c_1}{c_3} \right) \ge p, \; A \left( \frac{c_2}{c_3} \right), \; ^t b c_1 < ^t p c_2, \; c_i \ge 0$$
	В этом случае:
	$$\frac{c_2}{c_3} \in P, \; \frac{c_1}{c_3} \in Q, \; ^tp \frac{c_2}{c_3} > ^t b \frac{c_1}{c_3}$$
	что противоречит предложению \ref{cha:17/propose:1}.

	Таким образом, полиэдр M непуст. Пусть $(x', y') \in M$. В этом случае $x' \in P, y' \in Q$, причем $^t p x'\ge ^t b y'$. Отсюда $^t p x' = ^t b y'$ по предложению \ref{cha:17/propose:1}, и для любого $x \in P$ получаем $^t p x \le ^t b y' = ^t p x'$, т.е. $\underset{x \in P}{\max} ^t p x = ^t p x'$. Аналогично, $\underset{y \in Q}{\min} ^t b y = ^t p x' = \underset{x \in P}{\max} ^t p x$.
\end{Proof}


