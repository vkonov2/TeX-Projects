\chapter{Нормированные векторные пространства и алгебры, примеры. Индуцированные нормы на алгебре матриц}
\label{cha:26}

\epigraph{
	\textit{Вот в этих нормах ваших и спрятаны все основы социального консерватизма.}}
{-- Горький Максим}

Всюду в этой главе под $\mathbb{F}$ понимается либо поле вещественных чисел $\mathbb{R}$, либо поле комплексных чисел $\mathbb{C}$.

\begin{definition}\label{cha:26/def:1}
	\textit{Нормированным векторным пространством} называется векторное пространство V над полем $\mathbb{F}$ с нормой $||x||$, если на V задана функция $x \to ||x||$, принимающая неотрицательные вещественные значения, причем:
	\begin{itemize}
		\item[1)] $||x|| = 0$ тогда и только тогда, когда $x = 0$
		\item[2)] $||\lambda x|| = |\lambda| ||x||$ для всех $x \in V$, $\lambda \in \mathbb{F}$
		\item[3)] $||x + y|| \le ||x|| + ||y||$
	\end{itemize}
\end{definition}

\begin{propose}\label{cha:26/propose:1}
	Для любых элементов x, y из нормированного векторного пространства справедливо неравенство $||x − y|| \ge \Big| ||x|| - ||y|| \Big|$.
\end{propose}
\begin{Proof}
	По свойству 3) имеем $||x|| = ||(x−y)+y|| \le ||x−y||+||y||$. Отсюда $||x||−||y|| \le ||x−y||$. Аналогично $||y||−||x|| \le ||y − x|| = | − 1|||x − y|| = ||x − y||$. Из полученных двух неравенств вытекает утверждение.
\end{Proof}

\textbf{Пример}

\begin{itemize}
	\item[1)] 
		Пусть V – евклидово или эрмитово пространство. Тогда V является нормированным пространством, если положить $||x|| = \sqrt{(x, x)}$.
	\item[2)]
		Пусть $\mathbb{F}^n$ – пространство строк длины n с коэффициентами из $\mathbb{F}$. Для любого вещественного числа $p \ge 1$ положим: 
		$$\displaystyle ||x||_p = \sqrt[p]{\underset{j}{\overset{}{\sum}}|x_j|^p}$$ 
		Тогда $\mathbb{F}^n$ является нормированным векторным пространством с нормой $||x||_p$. Эта норма называется \textit{$l_p$-нормой}.
	\item[3)] 
		В пространстве $\mathbb{F}^n$ положим $||x||_{\infty} = \underset{j}{\max}|x_j|$. Тогда $\mathbb{F}^n$ является нормированным векторным пространством с нормой $||x||_{\infty}$. Эта норма называется \textit{$l_{\infty}$-нормой}.
	\item[4)] 
		Пространство всех непрерывных функций на отрезке $[0, 1]$ является нормированным пространством с нормой $||f|| = \underset{x}{\max}|f(x)|$, а также с нормами:
		$$\sqrt[p]{\underset{0}{\overset{1}{\int}}|f(x)|^p dx}$$
		где $p \ge 1$ – заданное вещественное число.
\end{itemize}

\begin{theorem}[]\label{cha:26/the:1}
	Пусть V - конечномерное нормированное векторное пространство над полем $\mathbb{F}$. Зафиксируем базис $e = (e_1, \dots, e_n)$ в пространстве $V = \mathbb{F}^n$ и рассмотрим функцию $f : \mathbb{F}^n \to \mathbb{R}$, где
	$$f (x_1, \dots, x_n) = ||x_1 e_1 + \dots + x_n e_n||$$ 
	Тогда функция f непрерывна.
\end{theorem}
\begin{Proof}
	По предложению \ref{cha:26/propose:1} для любых $x_1, \dots, x_n, y_1, \dots, y_n \in \mathbb{F}$. Имеем:
	$$\begin{gathered}
		|f(x_1, \dots, x_n) − f(y_1, \dots, yn)| = \Big| ||x|| − ||y|| \Big| \le ||x − y|| = \\
		= \Big| \Big| \underset{j}{\overset{}{\sum}}(x_j - y_j)e_j \Big| \Big| \le \underset{j}{\overset{}{\sum}}|x_j - y_j| ||e_j||
	\end{gathered}\eqno(87)$$
	Положим $M = \underset{j}{\max} ||e_j||$. Тогда в (87) получаем:
	$$|f(x_1, \dots, x_n) - f(y_1, \dots, y_n)| \le M n \underset{j}{\max} |x_j - y_j|$$
	Отсюда следует утверждение.
\end{Proof}

\begin{theorem}[]\label{cha:26/the:2}
	Пусть V - конечномерное нормированное пространство с двумя нормами $||x||$ и $||x||'$. Тогда существуют такие положительные вещественные числа $C_1,C_2$, что для всех $x \in V$ справедливы неравенства:
	$$C_1 ||x|| \le ||x||' \le C_2 ||x||$$
\end{theorem}
\begin{Proof}
	Без ограничения общности можно предполагать, что одна из норм, например, $||x||$ – евклидова (эрмирова) норма. Выберем в V ортонормированный базис $e = (e_1, \dots, e_n)$ и обозначим через S множество всех таких $x \in V$ , что $\underset{j}{\overset{}{\sum}} |x_j^2 | = 1$. Тогда S является n-мерной сферой и, следовательно, компактом. Отсюда в силу теоремы \ref{cha:26/the:1} вытекает, что функция $||x||'$ на S, принимающая положительные значения, ограничена сверху и снизу, т.е. существуют такие положительные вещественные числа $C_1,C_2$, что для всех $x \in S$ справедливы неравенства $C_1 \le ||x||' \le C_2$. Если $x \in V \setminus 0$, то $y = \frac{x}{||x||} \in S$. Таким образом:
	$$C_1 \le ||y||' = \Big| \Big| \frac{x}{||x||} \Big| \Big|' = \frac{||x||'}{||x||} \le C_2$$
	Отсюда вытекает утверждение.
\end{Proof}

Пусть $x_n$ – последовательность элементов нормированного пространства V. Скажем, что эта последовательность сходится к элементу $x \in V$, если для любого $\varepsilon > 0$ существует такое натуральное число N, что для всех $n > N$ справедливо неравенство $||x_n - x|| < \varepsilon$. Аналогичным образом определяются последовательности Коши и полные нормированные векторные пространства.

Из теоремы \ref{cha:26/the:2} вытекает:
\begin{conseq}[]\label{cha:26/conseq:1}
	Если последовательность сходится в конечномерном векторном пространстве относительно одной нормы, то она сходится и относительно любой другой нормы.
\end{conseq}

\textit{Алгеброй A} над полем $\mathbb{F}$ называется векторное пространство над этим полем, являющееся ассоциативным кольцом, причем для любых $x, y \in A$ и $\alpha \in \mathbb{F}$ справедливы равенства $\alpha(xy) = (\alpha x)y = x(\alpha y)$. Примерами алгебр являются алгебра многочленов $\mathbb{F}[X]$, алгебра матриц $Mat(n, \mathbb{F})$ размера n с коэффициентами из $\mathbb{F}$.

Алгебра A над полем $\mathbb{F}$ называется \textit{нормированной}, если A является нормированным векторным пространством с нормой $||x||$, причем для всех $x, y \in A$ выполняется неравенство $||xy|| \le ||x|| ||y||$.

\textbf{Пример}

Алгебра матриц $Mat(n, \mathbb{F})$ является нормированной алгеброй относительно следующих норм. Пусть $A = (a_{ij}) \in Mat(n,\mathbb{F})$. Тогда:
\begin{itemize}
	\item[1)] $||A||_{l_1} = \underset{i, j}{\overset{}{\sum}}|a_{i,j}|$
	\item[2)] $||A||_p = \sqrt[p]{\underset{i, j}{\overset{}{\sum}}|a_{i,j}|^p}$, где $p \ge 1, \; p \not = 2$
	\item[3)] $||A||_E = \sqrt{\underset{i,j}{\overset{}{\sum}}|a_{i,j}|^2}$ 
	\item[4)] $||A|| = n ||A||_{l_{\infty}}$, где $||A||_{l_{\infty}} = \underset{i,j}{\max} |a_{i,j}|$
	\item[5)] $||A||_1 = \underset{j}{\max} \left( \underset{i}{\overset{}{\sum}}|a_{i,j}| \right)$
	\item[6)] $||A||_{\infty} = \underset{i}{\max} \left( \underset{j}{\overset{}{\sum}} |a_{i,j}| \right)$
	\item[7)] $||A||_2$ - максимум из квадратных корней собственных значений матрицы $^t \overline{A}A$
\end{itemize}

Укажем теперь естественный способ построения нормированных алгебр. Пусть V – нормированное векторное пространство и $\mathcal{L}(V)$ - алгебра линейных операторов в V. Для $\mathcal{A} \in \mathcal{L}(V)$ положим:
$$||\mathcal{A}|| = \underset{||x||=1}{\sup}||\mathcal{A}x||\eqno(88)$$

\begin{theorem}[]\label{cha:26/the:3}
	Пусть V – конечномерное векторное пространство. Тогда (88) превращает $\mathcal{L}(V)$ в нормированную алгебру.
\end{theorem}
\begin{Proof}
	Заметим, что $||A||$ определена корректно и принимает конечные значения, поскольку по теореме \ref{cha:26/the:1} функция $f(A,x) = ||Ax||$ непрерывна относительно координат вектора x и относительно элементов матрицы A. 

	\begin{uprazh}[]\label{cha:26/uprazh:1}
		Доказать, что в конечномерном нормированном пространстве V для любых вещественных чисел $a < b$ множество всех таких $x \in V$, что a$ \le ||x|| \le b$ компактно.
	\end{uprazh}

	В силу упражнения \ref{cha:26/uprazh:1} множество всех таких x, что $||x|| = 1$, компактно. Таким образом, на этом множестве функция $f(A,x)$ ограничена.

	Если $||A|| = 0$, то $||Ax|| = 0$ для всех векторов x с условием $||x|| = 1$. Отсюда следует, что $||Ay|| = 0$ для всех векторов y, т.е. $A = 0$.

	\begin{uprazh}[]\label{cha:26/uprazh:2}
		Показать, что $||A||$ в (88) – это инфинум всех таких $C \in \mathbb{R}$, что $||Ax|| \le C||x||$ для всех $x \in V$.
	\end{uprazh}
	
	Заметим далее, что:
	$$\begin{gathered}
		||A+B|| = \underset{||x||=1}{\sup}||Ax+Bx|| \le \underset{||x||=1}{\sup}\left( ||Ax||+||Bx|| \right) \le \\
		\le \underset{||x||=1}{\sup}||Ax|| + \underset{||x||=1}{\sup}||Bx|| = ||A|| + ||B||
	\end{gathered}$$
	$$||\lambda A|| = \underset{||x||=1}{\sup}||\lambda A x|| = |\lambda| \underset{||x||=1}{\sup}||A x|| = |\lambda| ||A||$$
	$$||AB|| = \underset{||x||=1}{\sup}||ABx|| \le \underset{||x||=1}{\sup}||A||||Bx|| = ||A||||B||$$
	Отметим, что при доказательстве последнего неравенства использовано упражнение \ref{cha:26/uprazh:2}.
\end{Proof}

























