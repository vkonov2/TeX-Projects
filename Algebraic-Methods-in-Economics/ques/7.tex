\chapter{Решение игры в чистых стратегиях}
\label{cha:7}

\epigraph{
  \textit{И каждый из этих проектов, основанных на стратегии и тактике, противоречит один другому.}}
{-- Толстой Л.Н.}

Пусть имеются два игрока: первый — $\alpha$ и второй — $\beta$. Развитие игры во времени состоит из нескольких этапов или \textit{ходов}, осуществляемых участниками игры. В нашей терминологии игра заканчивается всегда \textit{выигрышем} первого игрока, выраженным числом. Это число называется \textit{проигрышем} второго игрока. Разумеется, игра может закончится по существу выигрышем второго игрока. В нашей терминологии это будет выражаться тем, что выигрыш первого игрока отрицателен.

\textit{Стратегией} игрока называется система правил, однозначно определяющая его выбор при производстве каждого отдельного хода в зависимости от ситуации, сложивщейся в игре. Мы рассмотрим лишь игры, в которых каждый из игроков имеет только конечное число стратегий. Предположим, что первый игрок имеет n стратегии $\alpha_1, \dots, \alpha_n$, а второй игрок — m стратегий $\beta_1, \dots, \beta_m$. Выигрыш при выборе пары стратегий $\alpha_i, \beta_j$ обозначим $a_{ij}$. Эти числа можно свести в матрицу размера $n \times m$, называемую \textit{матрицей игры} или \textit{платежной матрицей}:

\begin{center}
  \begin{tabular}{ | l || c | c | c | c | c | c | r |}
    \hline
     & $\beta_1$ & $\dots$ & $\beta_m$ \\ \hline \hline
    $\alpha_1$ & $a_{11}$ & $\dots$ & $a_{1m}$ \\ \hline
    $\alpha_2$ & $a_{21}$ & $\dots$ & $a_{2m}$ \\ \hline
    $\vdots$ & $\vdots$ & $\vdots$ & $\vdots$ \\ \hline
    $\alpha_n$ & $a_{n1}$ & $\dots$ & $a_{nm}$ \\ \hline
  \end{tabular}
\end{center}

Стая себе целью иметь \textit{гарантированный} выигрыш (проигрыш) каждый из игроков может выбрать \textit{оптимальную стратегию}. Если игрок $\alpha$ выбирает стратегию $\alpha_i$, то он должен рассчитывать на то, что игрок $\beta$ ответит на нее той стратегией $\beta_j$, для которой выигрыш $a_{ij}$ минимален. Таким образом, стратегия $\alpha_i$ гарантирует выигрыш $\psi_i = \underset{j}{\min} a_{ij}$. Следовательно, максимальный выигрыш по всем стратегиям совпадает с $\displaystyle a = \underset{i}{\max} \psi_i = \underset{i}{\max} \; \underset{j}{\min} a_{ij}$. Положим $a = a_{i_1j_1}$. Тогда стратегия $\alpha_{i_1}$ оптимальна для игрока $\alpha$. Какую бы стратегию игрок $\beta$ не применял, выигрыш будет не меньше a. Аналогично, выбирая стратегию $\beta_j$ игрок $\beta$ должен исходить из того, что игрок $\alpha$ ответит на нее стратегией $\alpha_i$, при которой проигрыш $a_{ij}$ максимален, т. е. стратегия $\beta_j$ гарантирует проигрыш не выше $\varphi_j = \underset{i}{\max} a_{ij}$. При удачном выборе стратегии проигрыш не превзойдет $\displaystyle b = \underset{j}{\min} \varphi_j = \underset{j}{\min} \; \underset{i}{\max} a_{ij}$. Если $b = a_{i_0j_0}$, то стратегия $\beta_{j_0}$ оптимальна для игрока $\beta$. Какую бы стратегию игрок $\alpha$ не применял, проигрыш игрока $\beta$ будет не больше b.

Заметим, что мы находимся в системе представления, сопровождающих теорему фон Неймана. Действительно, матрица игры определяет функцию двух переменных:
$$F: \mathbb{R}^n \times \mathbb{R}^m \to \mathbb{R}, \; F(x,y) = \underset{i,j}{\overset{}{\sum}}x_i y_j a_{ij}$$
При этом $a_{ij} = F(e_i,e'_j)$, где $e_1, \dots, e_n$ и $e'_1, \dots, e'_m$ – стандартные базисы в $\mathbb{R}^n$ и $\mathbb{R}^m$ соответственно. Как отмечено в теореме фон Неймана, имеет место неравенство:
$$b = \underset{j}{\min} \varphi_j = \underset{j}{\min} \; \underset{i}{\max} a_{ij} \ge a = \underset{i}{\max} \psi_i = \underset{i}{\max} \; \underset{j}{\min} a_{ij}\eqno(13)$$
Применение обоими игроками своих оптимальных стратегий $\alpha_{i_1}$ и $\beta_{j_0}$ приводит к выигрышу $a_{i_1 j_0}$ , при котором выполняются неравенства:
$$b = a_{i_0 j_0} \ge a_{i_1 j_0} \ge a_{i_1 j_1} = a\eqno(14)$$
Если неравенство (13) строгое, то оба неравенства (14) также могут быть строгими. Тогда положение, при котором оба игрока применяют свои оптимальные стратегии, может оказать неустойчивым. Например, получив сведение о том, что игрок $\alpha$ применяет оптимальную стратегию, игрок $\beta$ может ответить стратегией $\beta_{j_1}$, что приведет к выигрышу $a_{i_1j_1} < a_{i_1j_0}$. В аналогичной ситуации игрок $\alpha$ может ответить стратегией $\alpha_{i_0}$ и получить выигрыш $a_{i_0j_0} > a_{i_1j_0}$.

Иное дело, если в (13) имеет место равенство. Элемент $a_{i_0j_0}$ в этом случае называется \textit{седловой точкой}. Применение оптимальных стратегий обоими игроками теперь устойчиво: если одна из сторон применяет оптимальную стратегию, то для другой невыгодно уклоняться от своей.


