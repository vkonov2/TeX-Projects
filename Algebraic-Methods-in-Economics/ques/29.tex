\chapter{Теорема о с.в. положительной матрицы, для которых
собственное значение равно по модулю $\rho(A)$}
\label{cha:29}

\epigraph{
	\textit{В своей книге он борется с самим собой, сводит счеты с собственной стихийной натурой.}}
{-- Бердяев Н.А.}

% \begin{definition}\label{cha:29/def:1}
% 	Пусть A, B – прямоугольные вещественные матрицы. Скажем, что $A \ge B$ ($A > B$), если $a_{ij} \ge b_{ij}$ ($a_{ij} > b_{ij}$) для любых элементов $a_{ij}$, $b_{ij}$ матриц A и B. Матрица A \textit{положительна} (\textit{неотрицательна}), если $A \ge 0$ ($A > 0$). В частности, мы будем говорить о положительных (неотрицательных) векторах и квадратных матрицах.
% \end{definition}

% \begin{definition}\label{cha:29/def:2}
% 	Пусть $A = (a_{ij}) \in Mat(n,\mathbb{R})$. Тогда $|A| = (|a_{ij}|)$.
% \end{definition}

% \begin{propose}\label{cha:29/propose:1}
% 	Cправедливы следующие соотношения:
% 	\begin{itemize}
% 		\item[1)] $AB \le |AB| \le |A||B|$
% 		\item[2)] $|A+B|\le|A|+|B|$
% 		\item[3)] $|\alpha A| = |\alpha||A|, \; \alpha \in \mathbb{R}$
% 	\end{itemize}
% \end{propose}
% \begin{Proof}
% 	Пусть $A, B \in Mat(n, \mathbb{R})$. Тогда:
% 	$$\underset{j}{\overset{}{\sum}}a_{ij}b_{jk} \le \Big| \underset{j}{\overset{}{\sum}}a_{ij}b_{jk} \Big| \le \underset{j}{\overset{}{\sum}}|a_{ij}||b_{jk}|$$
% 	откуда вытекает первое свойство.
	
% 	Остальные свойства проверяются аналогично.
% \end{Proof}

% \begin{propose}\label{cha:29/propose:2}
% 	Пусть $A, B \in Mat(n, \mathbb{R})$. Если $|A| \le B$, то $\rho(A) \le \rho(|A|) \le \rho(B)$.
% \end{propose}
% \begin{Proof}
% 	По предложению \ref{cha:29/propose:1} для любого натурального числа k имеем $A^k \le |A^k| \le |A|^k \le B^k$. Рассмотрим норму $||\cdot||_E$ ($||A||_E = \sqrt{\underset{i,j}{\overset{}{\sum}}|a_{i,j}|^2}$) на алгебре матриц. Тогда $\displaystyle ||A^k||_E = \Big|\Big| |A^k| \Big|\Big|_E \le \Big|\Big| |A|^k \Big|\Big|_E \le ||B^k||_E$. Отсюда $\displaystyle ||A^k||_E^{\frac{1}{k}} \le \Big|\Big| |A|^k \Big|\Big|_E^{\frac{1}{k}} \le ||B^k||_E^{\frac{1}{k}}$.

% 	Остается воспользоваться теоремой \ref{cha:28/the:1}.
% \end{Proof}

% \begin{conseq}[]\label{cha:29/conseq:3}
% 	Пусть неотрицательная матрица A имеет положительный собственный вектор с собственным значением $\lambda$. Тогда $\lambda = \rho(A)$.
% \end{conseq}
% \begin{Proof}
% 	Пусть $Ax = \lambda x$, где x – положительный вектор. Применим следствие \ref{cha:29/conseq:2} с $\alpha = \beta = \lambda$.
% \end{Proof}

% \begin{conseq}[]\label{cha:29/conseq:4}
% 	Пусть $A \ge 0$, причем $\underset{j}{\overset{}{\sum}} a_{ij} > 0$ для всех $i = \ton n$. Тогда $\rho(A) > 0$.
% \end{conseq}

\begin{propose}\label{cha:30/propose:1}
	Пусть A – положительная матрица и x – неотрицательный ненулевой вектор. Тогда вектор $Ax$ положителен.
\end{propose}
\begin{Proof}
	Пусть $x_k > 0$. Для любого индекса $i = \ton n$ имеем $\underset{j=1}{\overset{n}{\sum}} a_{ij}x_j \ge a_{ik}x_k > 0$.
\end{Proof}

\begin{theorem}[]\label{cha:30/the:2}
	Пусть A – положительная матрица и $Ax = \lambda x$ для некоторого ненулевого вектора x, причем $|\lambda| = \rho(A)$. Тогда:
	\begin{itemize}
		\item[1)] $A|x| = \rho(A)|x|$
		\item[2)] $|x| > 0$
		\item[3)] $x = e^{i\theta}|x|$ для некоторого $\theta \in \mathbb{R}$
		\item[4)] $\lambda = \rho(A)$
	\end{itemize}
\end{theorem}
\begin{Proof}
	Имеем:
	$$\rho(A)|x| = |\lambda||x| = |\lambda x| = |Ax| \le |A||x| = A|x|\eqno(93)$$
	Положим $y = A|x| − \rho(A)|x|$. По (93) вектор y неотрицателен.

	Предположим сначала, что $y \not = 0$. По предложению \ref{cha:30/propose:1} вектор Ay положителен. Положим $z = A|x|$. В силу предложения \ref{cha:30/propose:1} этот вектор также положителен. Отсюда $0 < Ay = Az- \rho(A)z$ и поэтому $Az > \rho(A)z$. Это противоречит следствию \ref{cha:29/conseq:2}.

	Итак, $y = 0$, т.е. $A|x| = \rho(A)|x|$. Кроме того, $|x| = \rho(A)^{−1}A|x| > 0$ по предложению \ref{cha:30/propose:1}. Поэтому для любой координаты $x_k$ вектора x имеем:
	$$\begin{gathered}
		\rho(A)|x_k| = |\lambda||x_k| = |\lambda x_k| = \Big| \underset{j}{\overset{}{\sum}} a_{kj}x_j \Big| \le \\
		\le \underset{j}{\overset{}{\sum}}|a_{kj}||x_j| = \underset{j}{\overset{}{\sum}}a_{kj}|x_j| = \rho(A)|x_k|
	\end{gathered}$$
	Таким образом, $| \underset{j}{\overset{}{\sum}} a_{kj} x_j | = \underset{j}{\overset{}{\sum}} a_{kj} |x_j |$, а, значит, все $x_j$ расположены на одном луче в комплексной области. В частности, существует такой угол $\theta$, что $e^{−i\theta}x_j > 0$ для всех j. Отсюда $e^{−i\theta}x = |x|$, т.е. x – собственный вектор A c собственным значением $\rho(A)$.
\end{Proof}

\begin{conseq}[]\label{cha:30/conseq:1}
	Пусть A – положительная матрица. Тогда $\rho(A)$ – положительное собственное значение A. Существует положительный собственный вектор с собственным значением $\rho(A)$.
\end{conseq}
\begin{Proof}
	Пусть $|\lambda| = rho(A)$ для некоторого собственного значения $\lambda$ матрицы A и $Ax = \lambda x$, где $x \not = 0$. По теореме $A|x| = \rho(A)|x|$, причем $|x| > 0$.
\end{Proof}

\begin{conseq}[]\label{cha:30/conseq:2}
	Пусть $A > 0$. Если $\lambda$ – собственное значение матрицы A, причем $\lambda \not = \rho(A)$, то $|\lambda| < \rho(A)$.
\end{conseq}




















