\chapter{Доказать, что неотрицательная матрица $A$ размера $n$ неразложима $\Longleftrightarrow$ матрица $(E + A)^{n−1}$ положительна}
\label{cha:34}

\epigraph{
	\textit{Взял, разобрал ножичком, разложил; опять сладил, отдал. Идут часы.}}
{-- Толстой Л.Н.}

Квадратная матрица называется \textit{перестановочной}, если она получается из единичной перестановкой строк (столбцов). Квадратная матрица A размера n называется \textit{разложимой}, если выполнено одно из условий:
\begin{itemize}
	\item[1)] 
		$n = 1$ и $A = 0$
	\item[2)] 
		$n \ge 2$ и существует такая перестановочная матрица P, что
		$$P^{-1}AP = \begin{pmatrix}[c | c]
		B &  C \\ \hline
		0 &  D
	\end{pmatrix}$$
	где B, D – собственные квадратные подматрицы.
\end{itemize}
Матрицы, не являющиеся разложимыми, называются \textit{неразложимыми}.

\begin{theorem}[]\label{cha:34/the:1}
	Пусть задана неотрицательная неразложимая матрица A размера n. Тогда матрица $(E + A)^{n−1}$ положительна.
\end{theorem}
\begin{Proof}
	Пусть $\Gamma$ – ориентированный граф с множеством вершин \\ $\{1, 2, \dots, n\}$. При этом существует дуга $i \to j$, если либо $a_{ij} \not = 0$, либо $i = j$.

	\begin{lemma}\label{cha:34/lemma:1}
		Пусть в $\Gamma$ существует путь из $i \to j$. Тогда в $\Gamma$ существует путь из i в j длины (числа дуг) $\le n − 1$.
	\end{lemma}

	Для любой вершины p из $\Gamma$ обозначим через $C_p$ связную компоненту p, т. е. множество всех концов всевозможных путей из p в графе $\Gamma$. По лемме \ref{cha:34/lemma:1} можно считать, что любой такой путь имеет длину не больше $n − 1$.

	\begin{lemma}\label{cha:34/lemma:2}
		Пусть $i \in C_p$, $j \not \in C_p$. Тогда $a_{ij} = 0$.
	\end{lemma}
	\begin{Proof}
		По условию существует путь $p \to i$. Если $a_{ij} \not = 0$, то существует также путь $i \to j$. Тогда в $\Gamma$ существует путь $p \to j$, что невозможно.
	\end{Proof}

	Предположим противное, что в матрице:
	$$(E + A)^{n-1} = E + \underset{k=1}{\overset{n-1}{\sum}}\begin{pmatrix}
		n-1 \\ k
	\end{pmatrix} A^k$$
	на некотором месте $(p,q)$ стоит нулевой элемент. В этом случае $p \not = q$ и указанный элемент имеет вид:
	$$\underset{k=1}{\overset{n-1}{\sum}}\underset{}{\overset{}{\sum}}\begin{pmatrix}
		n-1 \\ k
	\end{pmatrix}a_{j_1, j_2} \dots a_{j_k, j_{k+1}} = 0\eqno(95)$$
	где внутренний знак суммы означает суммирование по множеству всех таких наборов $(j_1, \dots, j_{k+1})$, что $p = j_1, q = j_{k+1}$, и $k \le n − 1$. Так как все слагаемые в (95) неотрицательны, то каждое произведение
	$$a_{j_1, j_2} \dots a_{j_k, j_{k+1}}\eqno(96)$$
	равно нулю для всех $k = 1, \dots, n − 1$ и для всех наборов ($p = j_1, \dots, j_{k+1} = q$). Это означает, что $q \not \in C_p$. Перенумеруем номера строк матрицы A таким образом, чтобы $C_p = \{k + 1, \dots, n\}$, $k < n$. Эта перенумерация соответствует переходу $A \to P^{−1}AP$ для некоторой перестановочной матрицы P. Тогда $k + 1 \le p \le n$ и для всех $1 \le j \le k$ по лемме \ref{cha:34/lemma:2} получаем $a_{ij} = 0$ при $i > k$, $j \le k$. Таким образом, A содержит угол из нулей.
\end{Proof}

















