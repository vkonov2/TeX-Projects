\chapter{Отсутствие циклов в улучшенном плане}
\label{cha:24}

\epigraph{
	\textit{Печерин верил, что Россия вместе с Соединенными Штатами начнет новый цикл истории.}}
{-- Бердяев Н.А.}

\begin{propose}\label{cha:24/propose:1}
	Если допустимый план X не содержал циклов, то и новый план $X'$ не содержит циклов.
\end{propose}
\begin{Proof}
	Пусть план $X'$ содержит цикл $x'_{p_1q_1}, x'_{p_1q_2}, \dots, x'_{p_sq_s}, x'_{p_sq_1}$. Как и в следствии \ref{cha:21/conseq:1} можно считать, что все индексы $p_1, p_2, \dots, p_s$, (соответственно, $q_1, q_2, \dots, q_s$) различны. Eсли $x'_{ij} \not = 0$ и $(i, j) \not = (i_0, j_0)$, то $x_{ij} \not = 0$. Так как X не содержит циклов, то среди элементов $x_{p_1q_1}, x_{p_1q_2}, \dots, x_{p_sq_s}, x_{p_sq_1}$ встречается элемент $x_{i_0j_0} = 0$. Пусть, например, $(i_0,j_0) = (p_r,q_r)$. Тогда имеем последовательность ненулевых элементов $x_{p_rq_{r+1}}, x_{p_{r+1}q_{r+1}}, , x_{p_sq_s}, x_{p_sq_1}, \dots, x_{p_{r−1}q_r}$, не содержащую $x_{i_0j_0} \not= 0$. По предложению \ref{cha:21/propose:2} получаем, что
	$$\begin{gathered}
		(p_r, \dots, p_s, p_1, \dots, p_{r−1}) = (i_0, \dots, i_k) \\
		(q_{r+1}, \dots, q_s, q_1, \dots, q_r) = (j_0, \dots, j_k)
	\end{gathered}$$
	В силу выбора $\theta$ один из элементов $x'_{p_{i−1}q_i} = 0$. Следовательно, этот случай невозможен.

	Пусть $(i_0, j_0) = (p_{r−1}, q_r)$. Тогда имеем последовательность ненулевых элементов $x_{p_{r−1}q_{r−1}}, x_{p_{r−2}q_{r−1}}, \dots, x_{p_1q_1}, x_{p_sq_1}, \dots, x_{p_rq_r}$, что снова приводит к противоречию.

\end{Proof}
