\chapter{Теорема Вейля. Задание многогранников системой аффинных неравенств}
\label{cha:13}

\epigraph{
	\textit{Пол был землею, потолок — небом, а их соединяли, точно могучие древесные стволы, круглые и многогранные колонны.}}
{-- Куприн А.И.}

\begin{theorem}[\red{Вейля}]\label{cha:13/the:1}
	Конечнопорожденный конус является полиэдром.
\end{theorem}
\begin{Proof}
	Пусть конус $K \subseteq \mathbb{A}^n$ с вершиной O порождается точками $A_1, \dots, A_m$. Можно считать, что $dim K = n$. Пусть O – начало координат и $A_j = (a_{j1}, \dots, a_{jn})$, $1 \le j \le m$. Заметим, что:
	$$n = dim K = dim <\overline{OA_1}, \dots, \overline{OA_m}> = rank 
	\begin{pmatrix}
		a_{11} & \dots & a_{1n} \\
		\vdots & \ddots & \vdots \\
		a_{m1} & \dots & a_{mn}
	\end{pmatrix}\eqno(27)$$
	Предположим, что $B = (b_1, \dots, b_n) \not \in K$. Рассмотрим аффинные функции:
	$$f_j (x_1, \dots, x_n) = \underset{t}{\overset{}{\sum}}a_{jt}x_t, \; j = \ton m \text{ и } g(x_1, \dots, x_n) = \underset{t}{\overset{}{\sum}}b_t x_t$$
	По теореме Фаркаша неравенство $g \ge 0$ не является следствием совместной системы неравенств (5). Следовательно, найдется такая точка $z = (z_1, \dots, z_n) \in \mathbb{A}^n$, что:
	$$f_1 (z) \ge 0, \dots, f_m (z) \ge 0, \; g(z) = a < 0$$
	Рассмотрим полиэдр $P^0$, задаваемый системой неравенств:
	$$f_1 \ge 0, \dots, f_m \ge 0, \; -g + a \ge 0\eqno(28)$$
	Полиэдр $P^0$ непуст, так как он содержит точку z. По теореме Фань Цзы и (27) полиэдр $P^0$ имеет вершину $C = (c_1, \dots, c_n)$. Среди неравенств (28) n линейно независимых функций должны обращаться в нуль. Если n линейно независимых функций среди $f_1, \dots, f_m$ обращается в точке C в нуль, то в силу определения функций $f_1, \dots, f_m$, получаем, что точка C – начало координат. В этом случае $−g(C) + a = a < 0$, что невозможно. Итак, только $n − 1$ независимая функция среди $f_1, \dots, f_m$, обращается в нуль. Кроме того, $−g(C) + a = 0$.

	Рассмотрим уравнение $h(x) = c_1x_1 + \dots + c_nx_n = 0$. По построению $n − 1$ точка среди $A_1, \dots, A_m$ удовлетворяет этому уравнению. Для остальных точек $A_j$ получаем $h(A_j) > 0$. Кроме того, $h(B) = g(C) = a < 0$. Итак, плоскость $\Pi$, задаваемая уравнением $h(x) = 0$ разделяет K и B, причем $\Pi \cap K$ является гранью размерности $n − 1$. Тем самым эти грани определяют полупространства, пересечением которых совпадает с K. Эти полупространства связаны с выбором $n−1$ независимой точки среди $A_1, \dots, A_m$ и проходят через эти точки и начало координат.
\end{Proof}

\begin{theorem}[]\label{cha:13/the:2}
	Конечнопорожденный многогранник является полиэдром.
\end{theorem}
\begin{Proof}
	Пусть многогранник M порождается точками:
	$$A_1, \dots, A_m\eqno(29)$$
	и $\Pi$ – наименьшая плоскость, содержащая M. Можно считать, что $\Pi = \mathbb{A}^n$. Вложим $\mathbb{A}^n$ в $\mathbb{A}^{n+1}$ и возьмем точку $O \in \mathbb{A}^{n+1} \ \mathbb{A}^n$. Пусть K – конус с вершиной O, порожденный точками (29). Заметим, что множество $K \cap \Pi$ выпукло и содержит точки (29). Следовательно, $M \subseteq K \cap \Pi$. Покажем, что $K \cap \Pi \subseteq M$. Выберем в $\mathbb{A}^{n+1}$ систему координат с началом в O и с базисом $e_0, \dots, e_n$, причем $e_0 = \overline{OA_1}$ и $<e_1, \dots, e_n> = <\overline{A_1 A_2}, \dots, \overline{A_1 A_m}>$. Тогда $\Pi = A_1 + <e_1, \dots, e_n>$. Рассмотрим точку:
	$$A = O + \lambda_1 \overline{OA_1} + \dots + \lambda_m \overline{OA_m} \in K \cap \Pi, \; \lambda_1, \dots, \lambda_m \ge 0$$
	Тогда:
	$$\begin{gathered}
		A = O + \lambda_1 \overline{OA_1} + \dots + \lambda_m \overline{OA_m} = O + \left( \underset{i=1}{\overset{m}{\sum}}\lambda_i \right) \overline{OA_1} + \underset{i \ge 2}{\overset{}{\sum}}\lambda_i \overline{A_1 A_i} \in \Pi = \\
		= A_1 + <e_1, \dots, e_n> = O + \overline{OA_1} + <e_1, \dots, e_n>
	\end{gathered}$$
	Отсюда $\underset{i=1}{\overset{m}{\sum}}\lambda_i = 1$ и $A \in M$ по предложению \ref{cha:11/propose:1}. Итак, $M = K \cap \Pi$. Поэтому M задается неравенствами, определяющими K и уравнениями, определяющими $\Pi$.
	
\end{Proof}
















