\chapter{Изменение системы главных неизвестных. Достаточные условия сходимости симплекс-метода}
\label{cha:15}

\epigraph{
	\textit{Я совершенно озадачен. Вчера, в этот самый момент, когда я думал, что все уже распуталось, найдены все иксы — в моем уравнении появились новые неизвестные.}}
{-- Замятин Е.И.}

% \subsection*{Симплекc-метод. Второй вариант}\label{cha:15/sec:1/subsec:1}

% Пусть система ограничений, задающих полиэдр M, имеет вид (33). Необходимо найти минимум функции $z = p_1x_1 + \dots + p_nx_n + q$. Изложим алгоритм решения этой задачи. Он применим лишь в случае отсутствия вырождения.

% \textit{ПЕРВЫЙ ЭТАП – НАХОЖДЕНИЕ ВЕРШИНЫ ПОЛИЭДРА M}

% ШАГ 1.
% Составим матрицу из коэффициентов:
% $$\text{\begin{tabular}{ | l || c | c | c || r |}
% 	    \hline
% 	     & $x_1$ & $\dots$ & $x_n$ & \\ \hline \hline
% 	    $y_1$ & $a_{11}$ & $\dots$ & $a_{1n}$ & $b_1$ \\ \hline
% 	    $\vdots$ & $\vdots$ & $\vdots$ & $\vdots$ & $\vdots$ \\ \hline
% 	    $y_m$ & $a_{m1}$ & $\dots$ & $a_{mn}$ & $b_m$ \\ \hline \hline
% 	    $z$ & $p_1$ & $\dots$ & $p_m$ & $q$ \\ \hline
% 	\end{tabular}}\eqno(37)$$
% ШАГ 2.
% Если все коэффициенты $b_1, \dots, b_m \ge 0$, то точка $O = (0, \dots, 0)$
% является вершиной. В этом случае переходим ко второму этапу (см. ниже ШАГ 6).

% ШАГ 3. Пусть $b_r < 0$ для некоторого r. Если все элементы $a_{r1}, \dots, a_{rn}$ из r-ой строки матрицы (37) неположительны, то система неравенств (33) несовместна и потому задача не имеет решения.

% ШАГ 4.
% Пусть $b_r < 0$ и $a_{rs} > 0$ для некоторого s. Выберем такой индекс j, чтобы число $-\frac{b_j}{a_{js}}$ было положительным и минимальным. В силу отсутствия зацикливания такой индекс j найдется однозначно.
% ШАГ 5.
% Из j-го уравнения выразим $x_s$ через
% $$x_1, \dots, x_{s−1}, y_j, x_{s+1}, \dots, x_n \eqno(38)$$
% и выразим все остальные функции $y_1, \dots, y_{j−1}, y_{j+1}, \dots, y_m, z$ через неизвестные (38).

% \begin{propose}\label{cha:15/propose:1}
% 	При преобразовании, совершенном в шаге 5, для $i \not = j$ все положительные коэффициенты $b_i$ остаются положительными, а отрицательные – отрицательными. Кроме того, значение свободного члена $b_r$ увеличивается, а свободный член в j-ом уравнении становится положительным.
% \end{propose}
% \begin{Proof}
% 	Пусть $y_j = a_{j1}x_1 + \dots + a_{jn}x_n + b_j$. Тогда:
% 	$$x_s = \frac{y_j}{a_{js}} - \frac{a_{j1}x_1}{a_{js}} - \dots - \frac{a_{jn}x_n}{a_{js}} - \frac{b_j}{a_{js}}$$
% 	Если же взять k-ую строку матрицы (37), $k \not= j$, то она имеет вид $y_k = a_{k1}x_1 + \dots + a_{kn}x_n + b_k$. Переписав $y_k$ через неизвестные (38), получаем свободный член:
% 	$$b_k + a_{ks} \left( -\frac{b_j}{a_{js}} \right)\eqno(39)$$
% 	Если $b_k > 0$ и $a_{ks} \ge 0$, то $\displaystyle b_k + a_{ks}\left( -\frac{b_j}{a_{js}} \right) \ge b_k > 0$. Если $b_k > 0$ и $a_{ks} < 0$, то $\displaystyle b_k + a_{ks}\left( -\frac{b_j}{a_{js}} \right) > 0$ в силу выбора j. Если же $b_k < 0$, то при $a_{ks} < 0$ $\displaystyle b_k + a_{ks}\left( -\frac{b_j}{a_{js}} \right) < b_k < 0$. Если $b_k < 0$ и $a_{ks} \ge 0$, то $\displaystyle b_k + a_{ks}\left( -\frac{b_j}{a_{js}} \right) < 0$ в силу выбора j. Кроме того, при $k = r \not = j$ по (39) получаем $\displaystyle 0 \ge b_k + a_{ks}\left( -\frac{b_j}{a_{js}} \right) > b_r$, поскольку $b_r < 0$ и $a_{rs} > 0$. Кроме того, коэффициент $b_j$ заменяется на $-\frac{b_j}{a_{js}} > 0$. Отсюда вытекает утвреждение.
% \end{Proof}

% По предложению \ref{cha:15/propose:1} получаем, что после ШАГА 5 коэффициент $b_r$ увеличивается, причем если $j = r$, то коэффициент $b_r$ становится положительным. При этом от набора $x_1, \dots, x_s, \dots, x_n$ в верхней строке (37) мы перешли к набору $x_1, \dots, y_j, \dots, x_n$. Поскольку в верхней строке (37) могут быть лишь n элементов среди $x_1, \dots, x_n, y_1, \dots, y_m$, то, учитывая увеличение $b_r$ при каждом шаге, мы через конечное число шагов мы добьемся того, что либо коэффициент $b_r$ становится положительным, либо убедимся, что полиэдр ограничений пуст. При этом по предложению \ref{cha:15/propose:1} общее число положительных коэффициентов в последнем столбце на каждом шаге не уменьшается.

% Итак, совершая конечное число шагов типа 2-5 можно добиться, чтобы либо все свободные коэффициенты стали положительными, либо установили, что задача не имеет решения.

% Предположим теперь, что в матрице (37) все коэффициенты $b_i > 0$.

% \textit{ВТОРОЙ ЭТАП – НАХОЖДЕНИЕ min z}

% ШАГ 6.
% Если все коэффициенты $p_i$ неотрицательны, то $\min z = q$, так как $x_i \ge 0$.

% ШАГ 7.
% Пусть $p_s < 0$ для некоторого s. Если все коэффициенты s-го
% столбца $a_{1s}, \dots, a_{ms}$ матрицы (37) неотрицательны, то луч $\displaystyle (0, \dots, 0, \overset{s}{t}, 0, \dots, 0) \in M, \; t \ge 0$. При этом $\displaystyle z(0, \dots, 0, \overset{s}{t}, 0, \dots, 0) = t p_s + q$ принимает сколь угодно большие отрицательные значения. Следовательно, функция z не имеет минимума на M.

% ШАГ 8.
% Пусть $p_s < 0$ для некоторого s и существует $a_{rs} < 0$. Выберем индекс j так, чтобы число $-\frac{b_j}{a_{js}}$ было положительным и минимальным. Переходим к шагу 5. Повторяя доказательство предложения \ref{cha:15/propose:1}, получаем, что все $b_k$ остаются положительными. Кроме того, коэффициент при $y_j$ в z становится равным $\frac{p_s}{a_{js}} > 0$. Свободный член в z становится равным $\displaystyle q + p_s \left( -\frac{b_j}{a_{js}} \right) < q$, поскольку $p_s <0, \; -\frac{b_j}{a_{js}} > 0$.

% Итак, совершив конечное число шагов, мы перейдем к вершине M, в которой функция z достигает минимума, либо выясним, что задача не имеет решения.

% \section*{Явление зацикливания}\label{cha:15/sec:2}

% Условие отсутствия вырождения существенно для применения симплекс-метода. Если через начало координат проходит граничная гиперплоскость, отличная от координатных, то смещение вдоль координатной полуоси вообще может оказаться невозможным, так как уводит начало координат в отрицательное полупространство этой гиперплоскости. Количество граничных гиперплоскостей, отделяющих начало координат от полиэдра ограничений, при этом возрастает и процесс теряет свою направленность. Наличие вырождения в численной реализации алгоритма проявляется в неоднозначности выбора разрешающего элемента. Запоминание уже испытанных ребер алгоритмом не обеспечивается и не исключена возможность, что, продолжая серию итерационных шагов, мы вернемся в уже испытанную вершину. Такое явление называется зацикливанием. Попробуем разобраться, велика ли вероятность столкнуться на практике с явлением зацикливания.

% Будем рассматривать задачи линейного программирования в $\mathbb{A}^n$ с фиксированным числом линейных неравенств. Тогда количества N коэффициентов левых частей этих неравенств также фиксировано. Каждая отдельная задача может интерпретироваться точкой N-мерного пространства, изображаемой строкой коэффициентов левых частей неравенств, занумерованных произвольным образом.

% Наличие вырождения означает, что n+1 граничных гиперплоскостей имеет непустое пересечение, т.е. соответствующая система линейных уравнений совместна. По теореме Кронекера-Капелли из линейной алгебры ранг матрицы системы равен рангу ее расширенной матрицы. Значит, ранг расширенной матрицы строго меньше $n + 1$ и ее детерминант равен нулю. Равенство нулю детерминанта определяет алгебраическое многообразие в N-мерном пространстве и точка, отвечающая задаче с вырождением, лежит на этом многообразии.

% Алгебраическое многообразие имеет нулевую меру. В условиях, когда все задачи равновероятны, вероятность встретить задачу, изображаемую точкой N-мерного пространства, лежащую на фиксированном алгебраическом многообразии, равна нулю. В свете этого представления задачи с вырождением встречаются крайне редко.

\textit{ВТОРОЙ ЭТАП – НАХОЖДЕНИЕ max z}

ШАГ 4.
Пусть $a_{m+1,r} < 0$. Рассмотрим коэффициенты при r-ой переменной.

\begin{propose}\label{cha:14/propose:3}
	Если коэффициенты $a_{1r}, \dots, a_{mr}$ матрицы (35) отрицательны, то максимума у функции z нет.
\end{propose}
\begin{Proof}
	Придадим свободной переменной $x_r$ произвольное значение $k > 0$, а всем остальным свободным переменным придадим нулевое значение. Тогда значение главного неизвестного из i-го уравнения равно $b_i − a_{ir}k > 0$. Таким образом, получаем точку $X(k) \in M$. При этом $z(X(k)) = −a_{m+1,r}k$ принимает сколь угодно большие значения. Следовательно, функция z не имеет максимума на M.
\end{Proof}

Пусть $a_{m+1,r} < 0$ и $a_{ir} > 0$ для некоторого $i=\ton m$. Переходим к ШАГУ 3. Предположим, что у нас были свободные переменные $x_{i_1}, \dots, x_{i_m}$ и в ШАГЕ 3 мы выбрали элемент $a_{sr}$, причем в s-ое уравнение с коэффициентом 1 входила главная переменная $x_{i_j}$. Совершим ШАГ 3 мы заменим $x_{i_j}$ на другую главную переменную $x_s$. Тем самым получим новую систему главных переменных $x_{i_1}, \dots, x_{i_{j−1}}, x_s, x_{i_{j+1}}, \dots, x_{i_m}$.

\begin{propose}\label{cha:14/propose:4}
	При пременении ШАГА 3 значение свободного члена $b_{m+1}$ увеличивается.
\end{propose}
\begin{Proof}
	Как и в предложении \ref{cha:14/propose:1} свободный член в z заменяется на $b_{m+1} − a_{m+1,r} b_s$ для некоторого s. Но $a_{m+1,r} < 0$, а $b_s > 0$. Поэтому $b_{m+1} − a_{m+1,r}b_s > b_{m+1}$.
\end{Proof}

Итак, мы совершаем различные выборы свободных и главных переменных, т.е. получаем вершины из M. При этом мы никогда не вернемся к выбранной ранее вершине, поскольку на каждом шаге значение целевой функции z увеличивается. Итак, совершив конечное число шагов, мы перейдем к вершине M, в которой функция z достигает максимума, либо выясним, что задача не имеет решения.















