\chapter{Теорема фон Неймана}
\label{cha:6}

\epigraph{
	\textit{И опять в неясную и мутную молитву отчетливо, выпукло, звонко врывалась кощунственная фраза…}}
{-- Короленко В.Г.}

\begin{propose}\label{cha:6/propose:1}
	Пусть множество точек $T \subseteq \mathbb{A}^n$ – компактно и выпукло, $N'$ – компактное выпуклое множество аффинных функций на $\mathbb{A}^n$. Предположим, что для любой точки $a \in T$ найдется такая функция $f \in N'$, что $f(a) \ge 0$. Тогда существует такая аффинная функция $f_0 \in N'$, что $f_0 |_T \ge 0$.
\end{propose}
\begin{Proof}
	Обозначим через K – множество всех аффинных функций на $\mathbb{A}^n$, принимающих на T неотрицательные значения. Тогда K является замкнутым выпуклым конусом в линейном пространстве всех аффинных функций.

	\begin{lemma}\label{cha:6/lemma:1}
		Пусть $b \in \mathbb{A}^n$ и $f(b) \ge 0$ для всех $f \in K$. Тогда $b \in T$.
	\end{lemma}
	\begin{Proof}
		Если бы $b \not \in T$, то по теореме \ref{cha:5/the:1} существовала бы такая аффинная функция f, что $f |_T \ge 0$ и $f(b) < 0$. Эта функция f принадлежит K. Получается противоречие с условием леммы.
	\end{Proof}

	Продолжим доказательство предложения. Предположим, что $K \cap N' = \emptyset$. Выберем в $\mathbb{A}^n$ систему координат $O, e_1, \dots, e_n$. Каждая аффинная функция f представляется в виде $f(x) = a_0 + \underset{i}{\overset{}{\sum}} a_i x_i$. По предложению \ref{cha:6/propose:1} существует такая аффинная функция $h(z) = \underset{i=0}{\overset{n}{\sum}} b_i z_i$ на пространстве аффинных функций на $\mathbb{A}^n$, что $h(f) = \underset{i=0}{\overset{n}{\sum}} b_i a_i \ge 0$ для всех $f \in K$ и $h(g) < 0$ для всех $g \in N'$. Так как функция $f = 1$ лежит в K, то $h(1) = b_0 \ge 0$.

	Предположим, что $b_0 > 0$. Тогда точка $b = \left( \frac{b_1}{b_0}, \dots, \frac{b_n}{b_0} \right)$ обладает тем свойством, что $f(b) = \frac{1}{b_0} h(f) \ge 0$ для всех $f \in K$ и поэтому в силу леммы \ref{cha:6/lemma:1} получаем $b \in T$. С другой стороны, $g(b) = b_0^{−1} h(g) < 0$ для всех $g \in N'$, что противоречит условиям предложения.

	Предположим теперь, что $b_0 = 0$. В этом случае:
	$$\begin{gathered}
		h(f) = \underset{i=1}{\overset{n}{\sum}}b_i a_i \ge 0, \; h(g) = \underset{i=1}{\overset{n}{\sum}}b_i c_i < 0 \\
		\text{для всех } f = a_0 + \underset{i=1}{\overset{n}{\sum}}a_i x_i \in K, \; g = c_0 + \underset{i=1}{\overset{n}{\sum}}c_i x_i \in N'
	\end{gathered}$$
	В частности, $b \not = 0$. Возьмем точку $z = (z_1, \dots, z_n) \in T$. Для любого $\mu \ge 0$ и любого $f \in K$ получаем:
	$$f(z + \mu b) = a_0 + \underset{i=1}{\overset{n}{\sum}}a_i z_i + \mu \underset{i=1}{\overset{n}{\sum}}a_i b_i \ge 0$$
	По лемме \ref{cha:6/lemma:1} точка $z + \mu b \in T$ для всех $\mu \ge 0$. Но это противоречит компактности T, поскольку $b \not = 0$. Следовательно, $K \cap N'$ непусто.
\end{Proof}

\begin{propose}\label{cha:6/propose:2}
	Пусть заданы компактные замкнутые выпуклые подмножества $N \subseteq \mathbb{A}^n$, $M \subseteq \mathbb{A}^m$ и $F (x, y)$ – непрерывная функция, где $x \in N$, $y \in M$. Тогда:
	$$\underset{x \in N}{\max} \; \underset{y \in M}{\min} F(x,y) \le \underset{y \in M}{\min} \; \underset{x \in N}{\max} F(x, y)$$
\end{propose}
\begin{Proof}
	$$\begin{gathered}
		F(x,y) \le \underset{x \in N}{\max} F(x,y) \; \Rightarrow \; \underset{y \in M}{\min} F(x,y) \le \underset{y \in M}{\min} \; \underset{x \in N}{\max} F(x,y) \; \Rightarrow \\
		\Rightarrow \; \underset{x \in N}{\max} \; \underset{y \in M}{\min} F(x,y) \le \underset{y \in M}{\min} \; \underset{x \in N}{\max} F(x, y)
	\end{gathered}$$
\end{Proof}

\begin{theorem}[\red{фон Неймана}]\label{cha:6/the:1}
	Пусть
	$$F(x,y) = \underset{i,j}{\overset{}{\sum}} a_{ij}x_i y_j + \underset{i}{\overset{}{\sum}}l_i x_i + \underset{j}{\overset{}{\sum}}r_j y_j + c$$
	где $x = (x_1, \dots, x_n) \in \mathbb{A}^n$, $y = (y_1, \dots, y_m) \in \mathbb{A}^m$. Предположим, что заданы компактные замкнутые выпуклые подмножества $N \subseteq \mathbb{A}^n$, $M \subseteq \mathbb{A}^m$. Тогда:
	\begin{itemize}
		\item[(1)] 
			существуют такие точки $x^* \in N$, $y^* \in M$, что для всех $x \in N$, $y \in M$ выполнены неравенства
			$$F(x, y^*) \le F(x^*, y^*) \le F(x^*, y)$$
		\item[(2)] $\displaystyle \underset{x \in N}{\max} \; \underset{y \in M}{\min} F(x,y) = \underset{y \in M}{\min} \; \underset{x \in N}{\max} F(x,y) = F(x^*, y^*)$
	\end{itemize}
\end{theorem}
\begin{Proof}
	Без ограничения общности, меняя константу c можно считать, что $\underset{x \in N}{\max} \; \underset{y \in M}{\min} F(x,y) = 0$. По предложению \ref{cha:6/propose:2} имеем:
	$$\underset{x \in N}{\max} \; \underset{y \in M}{\min} F(x,y) = 0 \le \underset{y \in M}{\min} \; \underset{x \in N}{\max} F(x,y)$$
	Обозначим через $N'$ множество всех аффинных функций вида:
	$$h_x (y) = F(x.y), \; y \in M$$
	Это множество компактно, замкнуто и выпукло. Из условия $\displaystyle \underset{y \in M}{\min} \; \underset{x \in N}{\max} F(x,y) \ge 0$ следует, что для любого $y \in M$ найдется такой $x \in N$, что $h_x(y) \ge 0$. По предложению \ref{cha:6/propose:1} найдется такая точка $x^* \in N$, что $h_{x^*}(y) = F(x^*,y) \ge 0$ для всех $y \in M$. Таким образом:
	$$0 = \underset{x \in N}{\max} \; \underset{y \in M}{\min} F(x,y) \ge \underset{y \in M}{\min} F(x^*, y) \ge 0$$
	откуда $\displaystyle \underset{y \in M}{\min} F(x^*, y) = 0$. Значит:
	$$0 = \underset{y \in M}{\min} F(x^*, y) \le F(x^*, y^*) \le \underset{x \in N}{\max} F(x, y^*) = 0$$
	Откуда получаем:
	$$F(x^*, y^*) = \underset{y \in M}{\min} F(x^*, y) = \underset{x \in N}{\max} F(x, y^*) = 0$$
	При этом:
	$$\begin{gathered}
		F(x, y^*) \le \underset{x \in N}{\max} F(x, y^*) = 0 = F(x^*, y^*) = \underset{y \in M}{\min} F(x^*, y) \le 0 = F(x^*, y^*) \\
		\underset{x \in N}{\max} \; \underset{y \in M}{\min} F(x,y) = 0 \le \underset{y \in M}{\min} \; \underset{x \in N}{\max} F(x,y) \le \underset{x \in N}{\max} F(x, y^*) \le F(x^*, y^*) = 0
	\end{gathered}$$
\end{Proof}

\begin{definition}\label{cha:6/def:1}
	Точка $(x^*, y^*)$ из теоремы фон Неймана называется \textit{седловой}.
\end{definition}























