\chapter{Внутренние точки полиэдра}
\label{cha:9}

\epigraph{
	\textit{Но с более глубокой, внутренней точки зрения сами эти пространства можно рассматривать как внутренний, духовный факт в русской судьбе.}}
{-- Бердяев Н.А.}

\begin{definition}\label{cha:9/def:1}
	\blue{Полиэдром} P называется множество всех точек $x \in \mathbb{A}^n$, удовлетворяющей заданной системе аффинных неравенств (5).
\end{definition}

Иначе говоря, полиэдр – это пересечение конечного числа полупространств. 

\textit{Размерностью} полиэдра P называется размерность наименьшей плоскости, содержащей P . Другими словами, размерность P совпадает с рангом системы векторов $\{ \overline{AB} | A, B \in P \}$. 

Пусть $f_1, \dots, f_r$ – все аффинные функции из (5), обращающиеся в нуль в точке A. Обозначим через П плоскость, задаваемую уравнениями $f_1 = \dots = f_r = 0$. Гранью $\Gamma_A$ точки A в P называется пересечение $\Pi \cap P$.

Отметим, что каждая грань является полиэдром, поскольку она задается неравенствами (5) и неравенствами $−f_1 \ge 0, \dots, −f_r \ge 0$.

\begin{definition}\label{cha:9/def:2}
	\textit{Вершиной} полиэдра называется грань нулевой размерности. Грань размерности 1 называется \textit{ребром}.
\end{definition}

\begin{theorem}[]\label{cha:9/the:1}
	Предположим, что полиэдр P задается системой неравенств:
	$$\begin{cases}
		a_{11}x_1 + \dots + a_{1n} x_n \le b_1 \\
		\ldots\ldots\ldots\ldots \\
		a_{m1}x_1 + \dots + a_{mn} x_n \le b_m \\
		x_1, \dots, x_n \ge 0
	\end{cases}\eqno(24)$$
	где $b_1, \dots, b_m \ge 0$. Тогда начало координат $O = (0, \dots, 0)$ является вершиной полиэдра P. Если $b_1, \dots, b_m > 0$, то из O выходит ровно n ребер.
\end{theorem}
\begin{Proof}
	Заметим сначала, что точка O удовлетворяет ограничениям (24), поскольку $b_1, \dots, b_m \ge 0$. Кроме того, в грани $\Gamma_O$ выполнены уравнения $x_1 = \dots = x_n = 0$. Отсюда в силу определения \ref{cha:9/def:2} получаем первое утверждение.

	Пусть $b_1, \dots, b_n > 0$. Зафиксируем индекс $1 \le i \le n$. Существует такое $\varepsilon > 0$, что $a_{ji}\varepsilon < b_j$ для всех $j = \ton m$. Тогда точка $B = (0, ... , 0,\varepsilon,0 ... , 0)$, где $\varepsilon$ стоит на месте i, удовлетворяет всем ограничениям (24), т. е. $B \in P$. Но координаты точки B удовлетворяют уравнениям $x_1 = \dots = x_{i−1} = x_{i+1} = \dots = x_n = 0$. Поэтому грань $\Gamma_B$ содержит все точки B с достаточно малым $x_i = \varepsilon$ и потому $dim \Gamma_B = 1$. Таким образом, изменяя $i = \ton n$ получаем n ребер. Так как каждое $b_j > 0$, то вершина O не обращает в равенство ни одно из первых m неравенств в (24).
\end{Proof}

\begin{conseq}[]\label{cha:9/conseq:1}
	Пусть полиэдр M задается системой уравнений и неравенств:
	$$\begin{cases}
		a_{11}x_1 + \dots + a_{1n} x_n + x_{n+1} = b_1 \\
		\ldots\ldots\ldots\ldots \\
		a_{m1}x_1 + \dots + a_{mn} x_n + x_{n+m} = b_m \\
		x_1, \dots, x_{n+m} \ge 0
	\end{cases}\eqno(25)$$
	где $b_1, \dots, b_m \ge 0$. Тогда точка $C = (0, \dots, 0, b_1, \dots, b_m)$ является вершиной полиэдра M. Если $b_1, \dots, b_m > 0$, то из C выходит n ребер.
\end{conseq}
\begin{Proof}
	Так как $b_1, \dots, b_m \ge 0$, то точка C удовлетворяет всем условиям (25) и потому лежит в M.

	Заметим, что при естественной проекции $\mathbb{A}^{n+m}$ на $\mathbb{A}^n$, $\displaystyle (x_1, \dots, x_{n+m}) \to$ \\ $(x_1, \dots, x_n)$, полиэдр M биективно проектируется на полиэдр P из теоремы \ref{cha:9/the:1}. Остается воспользоваться утверждением теоремы \ref{cha:9/the:1}.
\end{Proof}

\begin{definition}\label{cha:9/def:3}
	Пусть x – точка полиэдра P и $\Pi$ – наименьшая по включению плоскость, содержащая P. Точка x называется \textit{(относительно) внутренней} для P, если некоторая ее окрестность в $\Pi$ содержится в P.
\end{definition}

\begin{theorem}[]\label{cha:9/the:2}
	В непустом полиэдре есть внутренние точки.
\end{theorem}
\begin{Proof}
	Пусть полиэдр P имеет размерность s и векторы $\overline{A_0 A_1}, \dots, \overline{A_0 A_s}$ линейно независимы где $A_0, \dots, A_s \in P$. Тогда любая точка:
	$$A_0 + \left( 1 - \underset{i=1}{\overset{s}{\sum}}\lambda_i \right) \overline{A_0 A_0} + \underset{i=1}{\overset{s}{\sum}} \lambda_i \overline{A_0 A_i} = A_0 + \underset{i=1}{\overset{s}{\sum}}\lambda_i \overline{A_0 A_i}, \; \lambda_0 = 1 - \underset{i=1}{\overset{s}{\sum}}\lambda_i \ge 0$$
	лежит в P по теореме \ref{cha:9/the:1}, если $\lambda_1, \dots, \lambda_s \ge 0$, $\underset{i=1}{\overset{s}{\sum}} \lambda_i \le 1$. Более того, эта точка является внутренней, если все $\lambda_i > 0$ при $0 \le i \le s$.
\end{Proof}
