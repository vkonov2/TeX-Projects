\chapter{Теорема отделимости для замкнутого выпуклого конуса и замкнутого выпуклого компакта вне конуса}
\label{cha:4}

\epigraph{
	\textit{Памятник возвышался в цветах; его пьедестал образовал конус цветов, небывалый ворох, сползающий осыпями жасмина, роз и магнолий.}}
{-- Грин Александр}

\begin{theorem}[]\label{cha:4/the:1}
	Пусть K – замкнутый выпуклый конус с вершиной O и N – компактное выпуклое множество, не пересекающееся с K. Тогда существует такая линейная функция g, что $g(x) \ge 0$ для всех $x \in K$, $g(O) = 0$ и $g(y) < 0$ для всех $y \in N$.
\end{theorem}
\begin{Proof}
	По теореме \ref{cha:2/the:1} существует ближайшая к A точка $B\in K$. Пусть $e_1, \dots, e_n$ – базис, и $g = x_1 − \frac{r}{2}$ – аффинная функция, построенная в теореме \ref{cha:2/the:1} и следствии \ref{cha:2/conseq:1}. Покажем, что $g(O) = 0$. Пусть это не так, т.е. $g(O) > 0$. Поскольку $g(A) = −r < 0$, то:
	$$\cos \angle OBA = \frac{\left( BO, BA \right)}{||OB||\cdot ||BA||} = \frac{g(O) \cdot g(A)}{||OB|| \cdot ||BA||} < 0$$
	Таким образом, $\angle OBA > \frac{\pi}{2}$. Следовательно, перпендикуляр, опущенный из A на прямую OB пересекает ее в точке P, лежащей на луче OB, причем точки P и O лежат на этом луче по разные стороны от B.

	Отсюда $\overline{OP} = \lambda \overline{OB}, \lambda > 1$, и поэтому $P \in K$. Но $||AP||<||AB||$, что противоречит выбору B. Полученное противоречие доказывает теорему.
\end{Proof}
