\chapter{Теорема Фаркаша и ее следствия}
\label{cha:5}

\epigraph{
	\textit{Идея этого неравенства нашего нравилась…}}
{-- Достоевский Ф.М.}

\begin{definition}\label{cha:5/def:1}
	Система аффинных неравенств
	$$f_1 (x) \ge 0, \dots, f_m (x) \ge 0\eqno(5)$$
	называется \textit{совместной}, если она имеет решение. Говорят, что аффинное неравенство $f \ge 0$ является следствием (5), если для любого $x \in \mathbb{A}^n$, удовлетворяющего (5), выполнено неравенство $f(x) \ge 0$.
\end{definition}

\begin{theorem}[\red{Фаркаша}]\label{cha:5/the:1}
	Аффинное неравенство $f \ge 0$ является следствием совместной системы аффинных неравенств (5) тогда и только тогда, когда существуют такие неотрицательные числа $c_0, \dots, c_m$, что $f = c_0 + c_1f_1 + \dots + c_mf_m$.
\end{theorem}
\begin{Proof}
	Достаточно показать, что следствие f имеет указанное представление. Зафиксируем систему координат $O, e_1, \dots, e_n$ в $\mathbb{A}^n$. Каждую аффинную функцию $а(x) = a_0 + \underset{i}{\overset{}{\sum}}a_i x_i$ можно отождествить с точкой $(a_0, \dots, a_n) \in \mathbb{R}^{n+1}$ из пространства $(\mathbb{R}^{n+1}, \mathbb{R}^{n+1}, +)$. Пусть при этом отождествлении функциям
	$$\begin{gathered}
		f = u_0 + u_1 x_1 + \dots + u_n x_n \\
		f_i = a_{i0} + a_{i1} x_1 + \dots + a_{in} x_n, \; i = \ton m
	\end{gathered}$$
	соответствуют точки
	$$\begin{gathered}
		f \to (u_0, u_1, \dots, u_n) \\
		f_1 \to (a_{10}, a_{11}, \dots, a_{1n}) \\
		\ldots \\
		f_m \to (a_{m0}, a_{m1}, \dots, a_{mn})
	\end{gathered}$$
	Все функции g, представимые в виде $c_0 + c_1f_1 + \dots+ c_mf_m$, $c_i \ge 0$, образуют конус K в пространстве $(\mathbb{R}^{n+1}, \mathbb{R}^{n+1}, +)$ с вершиной в нулевом векторе, порождаемый точками
	$(a_{10}, a_{11}, \dots, a_{1n}), \ldots, (a_{m0}, a_{m1}, \dots, a_{mn}), (1, 0, \dots, 0)$. 

	Пусть $f \not \in K$. По теореме \ref{cha:4/the:1} существует такая аффинная функция $v(y_0, \dots, y_n) = \underset{i=0}{\overset{n}{\sum}}v_i y_i$ на $(\mathbb{R}^{n+1}, \mathbb{R}^{n+1}, +)$, что $v(z) \ge 0$ для $z \in K$ и
	$$v (u_0, \dots, u_n) < 0 \eqno(6)$$
	В частности, $v(1, 0, \dots, 0) = v_0 \ge 0$. Предположим сначала, что $v_0 > 0$. Для любого $1 \le i \le m$ имеем:
	$$f_i (v_0^{-1} v_1, \dots, v_0^{-1}v_n) = v_0^{-1} v(a_{i0}, \dots, a_{in}) \ge 0$$
	Так как f является следствием (5), то
	$$f (v_0^{-1} v_1, \dots, v_0^{-1}v_n) = v_0^{-1} v (u_0, \dots, u_n) \ge 0$$
	Получаем противоречие с (6). Итак, $v_0 = 0$. При этом:
	$$\underset{j=1}{\overset{n}{\sum}}a_{ij}v_j \ge 0, \; 1 \le i \le m \eqno(7)$$
	$$\underset{j=1}{\overset{n}{\sum}}u_j v_j < 0 \eqno(8)$$
	Так как система неравенств (5) совместна, то существует такая точка $(x_1, \dots, x_n) \in \mathbb{A}^n$, что:
	$$f_i (x_1,\dots,x_n) = a_{i0} +\underset{i}{\overset{}{\sum}}a_{ij}x_j \ge 0, \; 1 \le i \le m \eqno(9)$$
	Выберем такое вещественное число $\mu > 0$, что:
	$$f (x1+\mu v_1, \dots,x_n+ \mu v_n) = u_0+ \underset{i}{\overset{}{\sum}}u_i x_i + \mu  \left( \underset{i}{\overset{}{\sum}} u_i v_i \right) < 0 \eqno(10)$$
	Это возможно в силу (8). Для любого $i = \ton m$ в силу условий (7) и (9) имеем:
	$$f_i (x1+\mu v_1, \dots,x_n+ \mu v_n) = f_i (x_1, \dots, x_n) + \mu  \left( \underset{j}{\overset{}{\sum}} a_{ij} v_j \right) \ge 0$$
	что противоречит (10). Но тогда $f \ge 0$ не является следствием (5). Это противоречие показывает, что $f \in K$.
\end{Proof}

\begin{conseq}[]\label{cha:5/conseq:1}
	Система аффинных неравенств (5) несовместна тогда и только тогда, когда существуют такие неотрицательные числа $c_0, \dots, c_m$, что $c_0 > 0$ и $c_0 + \underset{i=1}{\overset{m}{\sum}}c_i f_i = 0$.
\end{conseq}
\begin{Proof}
	Пусть $f_i (x) = a_{i0} + \underset{j}{\overset{}{\sum}}a_{ij} x_j$. Рассмотрим в $\mathbb{A}^{n+1}$ систему аффинных неравенств $\displaystyle a_{i0} x_0 + \underset{j}{\overset{}{\sum}}a_{ij}x_j \ge 0$. Она совместна, поскольку нулевой вектор является ее решением. Для любого решения $(x_0, \dots, x_n)$ этой системы имеем $x_0 \le 0$. Действительно, если бы $x_0 > 0$, то набор $(x_0^{−1} x_1 , \dots, x_0^{−1} x_n)$ являлся бы решением исходной системы неравенств, что невозможно. Итак, неравенство $−x_0 \ge 0$ является следствием исходной системы неравенств. Поэтому в силу теоремы Фаркаша:
	$$-x_0 = c'_0 + \underset{i}{\overset{}{\sum}}c_i \left( a_{i0} x_0 + \underset{j}{\overset{}{\sum}} a_{ij} x_j \right), \; c'_0, c_i \ge 0 \eqno(11)$$
	Полагая в (11) $x_0 = x_1 = \dots = x_n = 0$, получаем $c'_0 = 0$. Остается в равенстве (11) положить $x_0 = 1$.
\end{Proof}

\begin{conseq}[]\label{cha:5/conseq:2}
	Пусть A – матрица размера $m\times n$, x – столбец неизвестных высоты n и b – столбец свободных членов высоты m. Тогда либо система неравенств $A x+ b \ge 0$ совместна, либо существует такой столбец \\ $\displaystyle c = ^t (c_1, \dots, c_m) \ge 0$ высоты m, что $^t A c = 0$ и $(b,c) < 0$.
\end{conseq}
\begin{Proof}
	Пусть:
	$$\begin{gathered}
		A = (a_{ij}), ^t b = (b_1, \dots, b_m), ^t x = (x_1, \dots, x_n) \\
		f_i (x) = \underset{j}{\overset{}{\sum}} a_{ij} x_j + b_j, \; i = \ton m
	\end{gathered}$$
	Система неравенств $A x + b \ge 0$ имеет вид $f_1 \ge 0 , \dots, f_m \ge 0$. Если эта система несовместна, то существует такой столбец $c = ^t (c_1, \dots, c_m) \ge 0$ и положительное число $c_0$, что:
	$$0 = c_0 + \underset{i}{\overset{}{\sum}}c_i f_i = c_0 + \underset{i}{\overset{}{\sum}} c_i (b_i + \underset{j}{\overset{}{\sum}} a_{ij} x_j) = $$
	$$ = c_0 + \underset{i}{\overset{}{\sum}}b_i c_i + \underset{i j}{\overset{}{\sum}} c_i a_{ij} x_j  = c_0 + (b, c) + ^t c A x\eqno(12)$$
	Так как вектор x произволен, то равенство (12) эквивалентно условиям $(b,c) < (b,c) + c_0 = 0$, $^t A c = 0$.
\end{Proof}




















