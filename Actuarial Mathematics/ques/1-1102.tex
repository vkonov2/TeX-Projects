\chapter{Лекция №1 - 11.02.2021. Страхование жизни}\label{lec:1}


\begin{remark}
	Предполагается спокойное, мирное время. Страхование жизни основано на закономерностях уровня смертности различных групп населения.
\end{remark}

\section{Вероятность выживания} % (fold)
\label{lec:1sec:alive_prob}

Рассмотрим однородную группу людей. Время измеряем в годах.

\begin{definition}
	
	$x$ - возраст гражданина на момент заключения договора;

	$T_x$ - остаточное время жихни - случайная величина;

	$F(t) = P(T_x \leq t) \MYdef {}_{t}q_x$ - функция распределения $T_x$, обладающая плотностью распределения;

	$f(t)$ - плотность распределения $T_x$;

	${}_{t}p_x \MYdef P(T_x > t)$ - вероятность, что с момента $x$ человек проживет более $t$ лет;

	$\omega - x, $ - максимальное время жизни заключившего договор; $\omega$ обычно берут равным 100;
\end{definition}

\begin{definition}
	При заключении договора:
	\begin{enumerate}
		\item фиксируется продолжительность заболеваний страхователя;
		\item возраст страхователя;
	\end{enumerate}

	Тогда $\lambda$ - время, прошедшее между медицинским обследованием (началом заболевания) и началом страхования.
\end{definition}
	


\begin{clair}

	$${}_{\omega - x}p_x = 0.$$

	$${}_{0}p_x = 1.$$

\end{clair}

\begin{clair}

	Пусть 
	$${}_{\frac{t}{t'}}q_x \MYdef P(t < T_x \leq t + t').$$

	Тогда 
	$${}_{\frac{t}{t'}}q_x = {}_{t}p_x - {}_{t+t'}p_x > 0.$$

\end{clair}
\begin{Proof}

	$$P(T_x > t) = P(t \leq T_x \leq t+t') + P(T_x > t+t') \; \Rightarrow \; {}_{t}p_x = {}_{\frac{t}{t'}}q_x + {}_{t+t'}p_x. $$

\end{Proof}

\begin{conseq}
	Из доказательства очевидно вытекает
	\[ {}_{\frac{t}{t'}}q_x \;\rightarrow \; 0, \;\; t'\;\rightarrow\;0\]
\end{conseq}

\begin{definition}
	Рассмотрим $$P(t \leq T_x \leq t + \Delta t| T_x > t) = \frac{P(t < T_x \leq t + \Delta t)}{P(T_x > t)} $$

	Тогда $$\lim\limits_{\Delta t \rightarrow 0}\frac{P(t \leq T_x \leq t + \Delta t| T_x > t)}{\Delta t} = \lim\limits_{\Delta t \rightarrow 0}\frac{F(t + \Delta t) - F(t)}{\Delta t \pre{t}p_x} = \frac{f(t)}{1 - F(t)} \MYdef \mu_x(t)$$ называется \red{мгновенной смертностью в момент $x + t$} или \red{интенсивностью смерти}.
\end{definition}


\begin{properties}
	\begin{enumerate}
		\item $\mu_x(t) \geq 0 $ при $t \in [0, \omega - x]$
		\item $\int\limits_0^b\mu_x(t)dt = \infty$, где $b > \omega - x$
	\end{enumerate}
\end{properties}

\begin{remark}
	\begin{gather*}
		\mu_x(t) = \frac{F'(t)}{1-F(t)} = \frac{-\frac{\partial}{\partial t}(1-F(t))}{1-F(t)}
		= \frac{-\frac{\partial}{\partial t}(\pre{t}p_x)}{\pre{t}p_x} = -\frac{d}{dt}\ln\pre{t}p_x
	\end{gather*}
		
	

	\[\Rightarrow \pre{t} p_x = e^{-\int\limits_0^t\mu_x(u)du}\]
\end{remark}


\section{Модели смертности} % (fold)
\label{lec:1sec:death_mod}

\begin{description}
	\item[\blue{Муавр}] \begin{gather*}
		T_x[0, \omega - x] - \text{равномерно распределена}\\
		f(t) = \begin{cases}
			\frac{1}{\omega - x}, 0 \leq t \leq \omega-x\\
			0, else
		\end{cases}
		\end{gather*}
	\item[\blue{Гомпретц}]
		\[\mu_x(t) = AC^{x+t}, (A>0, C>1)
		\]
		Отсюда $$\pre{t}p_x \rightarrow 0 (t \rightarrow \infty)$$

	\item[\blue{Мэйкхэм}]
		\[\mu_x(t) =D+ AC^{x+t}, (A>0, C>1, D>0)
		\]
		Тут берётся во внимние тот факт, что смерть может наступать не только от старости.

		\[
		{}_tp_x = e^{-\int\limits_0^tD+AC^{x+t}} = e^{-Dt - \frac{A(C^{x+t} - C^x)}{\ln C}}
		\]

	
\end{description}

\section{Оценка числа людей, доживших до момента t}
\label{lec:1sec:eval}

Пусть
\[
L_x- \text{\red{число людей возраста х из однородной группы}.}\]

Выбираем людей тщательно, ибо время считается в годах.


Определим индикатор
\[
X_i(t) = \begin{cases}
	1, i-\text{ый человек проживет больше, чем}t\\
	0, \; \;else
	
\end{cases}
\]

Тогда

\begin{gather*}
	EX_i(t) = {}_tp_x \\
	DX_i(t) = {}_tp_x - ({}_tp_x)^2 = {}_tp_x{}_tq_x.
\end{gather*}

Положим

\[
L_{x+t} \MYdef X_1(t) +...+ X_{L_x}(t) \]

Тогда

\begin{gather*}
	EL_{x+t} = L_x{}_tp_x \MYdef l_{x+t}\;\;(\Rightarrow EL_x = L_x)\\
	DL_{x+t} = L_x{}_tp_x{}_tq_x\;\text{(в силу независимости смертей)}.
\end{gather*}

\begin{remark}
	Видно, что \[{}_tp_x = \frac{l_{x+t}}{l_x} \]
\end{remark}