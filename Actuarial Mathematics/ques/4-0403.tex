\chapter{Лекция № 4 - 4.03.2021. Ренты} % (fold)
\section{Детерминированные ренты} % (fold)

\begin{definition}
	\red{Рента} - периодические взносы или выплаты, производимые в конце или в начале обусловленного периода времени.

	\red{ Рента, выплачиваемая вперед (пренумерандо)} - рента, взносы по которой производятся в начале каждого периода.

	\red{ Рента за истекшее время (постнумерандо)} - рента, взносы по которой производятся в конце каждого периода.
\end{definition}

\begin{enumerate}
	\item \red{Рента за истекшее время}

	\begin{gather*}
		v + ... + v^n = a_{x:n\urcorner} - \text{ настоящее значение ренты }\\
		a_{x:\infty\urcorner} - \text{ ----- бесконечной ренты}\\
		S_{x:n\urcorner} - \text{ накопленное значение ренты }
	\end{gather*}

	\begin{example}
		Пусть А - ежегодная сумма и $ i = 0.04$. Тогда 
		\[  a_{x:\infty\urcorner} = A \frac{1}{0.04}=25A,\]
		то есть настоящее значение бесконечной ренты равно 25-кратной сумме годовой выплаты.
	\end{example}

	\begin{example}
		Найти накопленное значение ренты сразц после последнего платежа при регулярных перечислениях 25\$ каждые 2 месяца в течение 10 лет при годовой процентной ставке $ i = 6\% = 0.06$

		Имеем
		\[ (1+\frac{i^{(6)}}{6}) = 1+i \;\Rightarrow\; j = \frac{i^{(6)}}{6} \approx 0.01\]
		\[ S_{x:n\urcorner} = 2041, 74\]
	\end{example}

	\item \red{ Рента, выплачиваемая вперед}

	\begin{gather*}
		1+v + ... + v^{n-1} = \ddot{a}_{x:n\urcorner} - \text{ настоящее значение ренты }\\
		\Rightarrow\;\; v a_{x:n\urcorner} = \ddot{a}_{x:n\urcorner}\\
		\ddot{a}_{x:\infty\urcorner} - \text{ ----- бесконечной ренты}\\
		\Rightarrow\;\; {a}_{x:\infty\urcorner} = \ddot{a}_{x:\infty\urcorner} + 1\\
		\ddot{S}_{x:n\urcorner} - \text{ накопленное значение ренты }
	\end{gather*}

	\item \red{ Отсроченная рента }

	\[{}_{m|}a_{x:n\urcorner}- \]
	настоящее значение ренты за истекающий срок (отсроченной на m лет);
	\[ {}_{m|}\ddot{a}_{x:n\urcorner}- \]
	настоящее значение ренты, выплачиваемой вперед (отсроченной на m лет)

	\begin{remark}
		Мы можем рассматривать современную стоимость ренты, которая обеспечит заданный размер годичной ренты (или пенсии).

		Естественно, чтобы подобные задачи были вполне определенными , надо знать
		\begin{itemize}
			\item какой тип ренты предполагается
			\item в течение скольких лет она должна выплачиваться 
			\item какова принятая норма роста денег
		\end{itemize}
	\end{remark}
\end{enumerate}

\section{Страховые ренты} % (fold)

\begin{enumerate}
	\item \red{Страхование на всю жизнь}
	Предположим, что клент платит по 1 рублю в моменты 0,1, ...., k до тех пор, пока он жив. Это модель ренты, выплачиваемой вперед; k случайно.

	\red{ Единовременная нетто-ставка } - математическое ожидание настоящего значения этих платежей (платежей клиента)

	\[  \ddot{a}_x = EY = \sum\limits_{k=0}^{\infty}\ddot{a}_{x:k+1\urcorner}{}_kp_xq_{x+k} \]
	Также
	\[1+v+..+v^k = \ddot{a}_{x:k+1\urcorner}=\frac{1-v^{k+1}}{d}=\frac{1-Z}{d},\]
	где Z - настоящее значение страховой суммы равной 1, выплачиваемой в конце года смерти клиента.

	Устанавливаем связь единовременной нетто-ставки единичных платежей с известными нам нетто-премиями:

	\begin{gather*}
		\ddot{a}_x = E \ddot{a}_{x:k+1\urcorner} = \frac{1-EZ}{d}=\frac{1-A_x}{d}
	\end{gather*}
	\begin{mind}
		Преобразуем формулу к некоторому более удобному виду. Имеем:
		\begin{gather*}
			\ddot{a}_x = \sum\limits_{k=0}^{\infty}\ddot{a}_{x:k+1\urcorner}{}_kp_xq_{x+k} \\
			\ddot{a}_{x:k+1\urcorner} = 1+v+..+v^k = \frac{1-v^{k+1}}{d}\\
			q_{x+k} = P(T_{x+k}<1)\;\;;\;\;\;\;{}_kp_x = P(T_x>k)\\
			{}_{k+1}p_x={}_kp_x(1 - q_{x+k})\\
			\Rightarrow \;\;q_{x+k}=1- \frac{{}_{k+1}p_x}{{}_kp_x}\\
		\end{gather*}
		Тогда 
		\begin{gather*}
			\ddot{a}_{x} = \sum\limits_{k=0}^{\infty}\frac{1-v^{k+1}}{d}{}_kp_x(1- \frac{{}_{k+1}p_x}{{}_kp_x})=\\
			=\sum\limits_{k=0}^{\infty}\frac{1-v^{k+1}}{d}{}_kp_x - \sum\limits_{k=0}^{\infty}\frac{1-v^{k+1}}{d}{}_{k+1}p_x=\\
			=[\text{во втором слагаемом сначала заменим k+1 на k,}\\
			\text{а затем просуммируем не от 1, а от 0 (это не изменит суммы)}]=\\
			=\sum\limits_{k=0}^{\infty}\frac{1-v^{k+1}-1+v^{k}}{d}{}_kp_x =[d =1-v]= \sum\limits_{k=0}^{\infty}v^k{}_kp_x
		\end{gather*}
	\end{mind}
	\begin{remark}
		Формула
		\[  \ddot{a}_x =\frac{1-A_x}{d}\]
		дает связь между страхованием (риском смерти) и аннуитетом (риск не дожить до очередной выплаты. В этом случае рента прекращается, дальнейших платежей нет)
	\end{remark}

	\item \red{ Страхование на конечный промежуток времени}
	Приведенные значения сумм, выплачиваемых страхователем
	\[
		Y=\begin{cases}
			\ddot{a}_{k+1\urcorner}, \;\; k=0,1,2,...,n-1\\
			\ddot{a}_{n\urcorner}, \;\; k=n,n+1,...
		\end{cases}
	\]
	Отсюда
	\[ EY = \ddot{a}_{x:n\urcorner}=\sum\limits_{k=0}^{n-1}\ddot{a}_{x:k+1\urcorner}{}_kp_xq_{x+k}+\ddot{a}_{n\urcorner}{}_np_x \]

	Мы знаем, что 
	\[
		Y = \ddot{a}_{k+1\urcorner}=\frac{1-v^{k+1}}{d}=\frac{1-Z}{d},
	\]
	где
	\[
		Z=\begin{cases}
			v^{k+1}, \;\;k=\overline{0,n-1}\\
			v^n, \;\;k=n,n+1,...
		\end{cases}
	\]
	Мы имеем случай endowments и значит 
	\[ EY = \frac{1-EZ}{d}=\ddot{a}_{x:n\urcorner}=\frac{1- A_{x:n\urcorner}}{d}, \]
	где $ \ddot{a}_{x:n\urcorner}$- математическое ожидание настоящего значения платежей клиента (страхователя),

	$ A_{x:n\urcorner}$- математическое ожидание настоящего значения платежа страховой компании (страховащика).

	\item В случае ренты за истекшее время (постнумерандо), когда клиент платит по 1 рублю в моменты 1,2, ..., k, пока он жив.

	\[ Y = v+v^2+...+v^k = a_{k\urcorner}.\]

	\textbf{Связь}: единовременная нетто-ставка платежей клиента (постнумерандо, договор на всю жизнь):
	\[ a_{n\urcorner} = \ddot{a}_{n\urcorner} - 1 \]
\end{enumerate}

\subsection{Периодические нетто-ставки (нетто-премии)} % (fold)

Страховой полис устанавливает с одной стороны выплаты страховщика (страховой компании)(the benefits), которые могут представлять собой единичную выплату или серию выплат, и с другой стороны премию (сумму), выплачиваемую страхователем.

Существует три формы премий выплачиваемых (вносимых) страхователем:
\begin{enumerate}
	\item \red{одноразовая нетто-премия}
	\item \red{периодические нетто-премии одной величины}
	\item \red{периодические нетто-премии переменной величины}
\end{enumerate}
Для периодических премий продолжительность и частота уплаты премий должны быть точно определены в дополнение к величине премии.

Относительно страхового полиса определяется \red{общий убыток страховщика}.
\[EL=0\]
- \red{принцип эквивалентности}. С помощью данного принципа определяются одноразоввые нетто-премии и периодические нетто-премии одной величины. Для периодических нетто-премий переменной величины этот принцип не подходит.

\begin{example}
	Будем рассматривать случай term insurance для возраста 40 лет, на период 10 лет (выплата страховой суммы происходит только в том случае, если клиент умрет в течение этих 10 лет) и страховая сумма $ C$ выплачивается в конце года смерти, премия $ \pi$ платится ежегодно вперед, пока страхователь жив, но не более 10 лет, $ i = 0.04.$
\end{example}
\begin{solution}
	Убыток $ L$ страховщика в этом случае будет 
	\[
		L=\begin{cases}
			Cv^{k+1} - \pi \ddot{a}_{k+1\urcorner}, \;\;k=0,1,...,9\\
			-\pi \ddot{a}_{10\urcorner}, \;\; k =10,11, ...	
		\end{cases}
	\]
	Определим годовую нетто-премию, используя принцип эквивалентности
	\[ EL = 0 \;\; \Rightarrow \;\; C A_{40:10\urcorner}^{1} - \pi \ddot{a}_{40:10\urcorner} = 0.\]

	\begin{enumerate}
		\item При подсчете $ A_{40:10\urcorner}^{1}$ вероятность умереть в течение одного года (любого из оставшихся) равна $ \frac{1}{60}$ (равномерная модель Муавра):
		\[  \omega - 40 = 100 -40 = 60.\]

		Вероятность смерти в течение одного определенного года равна $ \frac{1}{60}$, ибо это условная вероятность умереть в течение определенного года (чтобы умереть в течение именно этого года, надо дожить до начала этого года), т.е.
		\[ {}_kp_{40}q_{40+k} = \frac{1}{60} \]

		\item \[\pi = \frac{C A_{40:10\urcorner}^{1}}{ \ddot{a}_{40:10\urcorner}}\]


	\end{enumerate}
	Используя закон Муавра, находим $ {}_{10}p_x = \frac{5}{6},$
	\begin{gather*}
		A_{40:10\urcorner}^{1} = 0.1352\\
		A_{40:10\urcorner}^{\;\;\;\;1} = 0.5630\\
		A_{40:10\urcorner}= 0.6982\\
		\ddot{a}_{40:10\urcorner} = \frac{1- A_{40:10\urcorner}}{d} = \frac{1-0.6982}{\frac{i}{1+i}} = 7.8476\\
		\pi = 0.01723C
	\end{gather*}
\end{solution}


% subsubsection subsubsection_name (end)

% section страховые_ренты (end)

% section детерминированные_ренты (end)

% chapter лекция_4_4_03_2021_ренты (end)