\chapter{Предварительные сведения}\label{cha:1}

\section{Мера, распределение}\label{cha:1/sec:1}

Пусть $\Omega = \{ w\}$ - произвольное множество, а $\mathcal{F}$ - сигма-алгебра его подмножеств. Т.е. $\mathcal{F}$ такая система множеств, что:
\begin{itemize}
	\item[1)] $\Omega \in \mathcal{F}$
	\item[2)] если $A \in \mathcal{F}$, то $\overline{A} := \Omega - \mathcal{F} \in \mathcal{F}$
	\item[3)] если $A_1, A_2, \dots \in \mathcal{F}$, то $\underset{i}{\bigcup}A_i \in \mathcal{F}, \; \overset{i}{\bigcap}A_i \in \mathcal{F}$
\end{itemize}

\begin{definition}\label{cha:1/def:1}
	Пусть $\Omega = R$, а $\mathcal{F}$ - наименьшая сигма-алгебра, содержащая все интервалы $(\alpha, \beta)$. Такая $\mathcal{F}$ обозначается $\mathcal{B}(\mathbb{R})$ и называется \blue{борелевской} сигма-алгеброй.
\end{definition}

\begin{definition}\label{cha:1/def:2}
	Мера $\mu$, определенная на $\mathcal{F}$, называется \blue{сигма-аддитивной}, если:
	\begin{itemize}
		\item[1)] 
			это неотрицательная функция $\displaystyle \mu (A) \ge 0 \; \forall A \in \mathcal{F}$
		\item[2)] 
			она удовлетворяет условию сигма-аддитивности: $$\displaystyle \mu (\underset{i}{\overset{}{\sum}}A_i) = \underset{i}{\overset{}{\sum}}\mu(A_i), \; A_i \in \mathcal{F}, \; A_i A_j = \emptyset \text{ при } i \not = j$$
	\end{itemize}
\end{definition}

\begin{definition}\label{cha:1/def:3}
	Мера $\mu$ называется \blue{сигма-конечной}, если существуют множества $A_i \in \mathcal{F}$ такие, что $\underset{i}{\bigcup}A_i = \Omega$ и $\mu(A_i) < \infty$.
\end{definition}

\colorbox{DarkSeaGreen}{Считающая мера}: пусть $\Omega$ - счетное, $\mathcal{F}$ - множество всех подмножеств $\Omega$. Положим для $A \in \mathcal{F}$
$\displaystyle \mu(A) := \{ \text{число точек } \Omega, \text{ попавших в } A \}$. Такая мера $\mu$ называется считающей, она сигма-конечна.\\

\colorbox{DarkSeaGreen}{Лебегова мера}: пусть $\Omega = \mathbb{R}, \; \mathcal{F} = \mathcal{B}(\mathbb{R})$. Существует единственная мера $\mu$ на $\mathcal{B}(\mathbb{R})$ такая, что $\displaystyle \mu\left( (\alpha, \beta] \right) = \beta - \alpha$. Эта мера Лебега, она сигма-конечна.\\

\vspace{0.5cm}
$(\Omega, \mathcal{F})$ - измеримое пространство, $(\Omega, \mathcal{F}, \mu)$ - пространство с мерой.

\begin{definition}\label{cha:1/def:4}
	Если $\mu(\Omega) = 1$, то $\mu$ - \blue{вероятностная мера}, она обозначается через $P$. Тройка $(\Omega, \mathcal{F}, \mu)$ - \blue{вероятностное пространство}.
\end{definition}

\begin{definition}\label{cha:1/def:5}
	Измеримая функция $\xi: (\Omega, \mathcal{F}) \to (\mathbb{R}, \mathcal{B}(\mathbb{R}))$ называется \blue{случайной величиной}. Измеримость означает, что:
	$$\forall B \in \mathcal{B}(\mathbb{R}) \;\; \xi^{-1}(B) := (w \; : \; \xi(w) \in B) \in \mathcal{F}$$ Измеримая функция $\varphi: (\mathbb{R}, \mathcal{B}(\mathbb{R})) \to (\mathbb{R}, \mathcal{B}(\mathbb{R}))$ называется \blue{борелевской}.
\end{definition}

\begin{definition}\label{cha:1/def:6}
	Рассмотрим случайную величину $\xi \in \mathbb{R}^1$. Для $x \in \mathbb{R}^1$ функция $F(x) = P(w: \xi(w) \le x) = P(\xi \le x)$ называется \blue{функцией распределения}.
\end{definition}

\begin{definition}\label{cha:1/def:7}
	Мера $P_{\xi} (A) = P(w: \xi(w) \in A), \; A \in \mathcal{B}(\mathbb{R})$, называется \blue{распределением} случайной величины $\xi$.
\end{definition}

Тогда $F(x) = P_{\xi} \left( (-\infty, x] \right)$, т.е. $P_{\xi}$ определяет $F(x)$. 

Обратно: $P(\alpha < \xi \le \beta) = F(\beta) - F(\alpha)$, и существует единственная вероятностная мера $P_{\xi}$ такая, что $P_{\xi} \left( (\alpha, \beta] \right) = F(\beta) - F(\alpha)$, т.е. $F(x)$ определяет $P_{\xi}$.

\begin{definition}\label{cha:1/def:8}
	Пусть на $(\mathbb{R}, \mathcal{B}(\mathbb{R}))$ задана сигма-конечная мера $\mu$. Если существует борелевская функция $f(x), \; f(x) \ge 0$, такая, что
	$$P_{\xi} (A) = \underset{A}{\overset{}{\int}}f(x) \mu(dx) \;\; \forall A \in \mathcal{B}(\mathbb{R})$$
	то $f(x)$ называется \blue{плотностью} вероятности по мере $\mu$.
\end{definition}

Если $\mu$ - мера Лебега, то $f(x)$ - обычная плотность случайной величины $\xi$, введенная на 2-ом курсе. Если же $\xi$ дискретна со значениями $x_1, x_2, \dots$, а $\mu$ - считающая мера, сосредоточенная в этих точках, то, очевидно:
$$P_{\xi}(A) = \underset{A}{\overset{}{\int}}P(\xi = x)\mu(dx) \; \forall A \in \mathcal{B}(\mathbb{R})$$
Последнее равенство означает, что у дискретной случайной величины $\xi$ есть плотность вероятности $f(x) = P(\xi = x), \; x = x_1, x_2, \dots$ по считающей мере. При $x \not = x_1, x_2, \dots$ значения этой плотности не важны, их можно положить равными нулю.

\begin{definition}\label{cha:1/def:9}
	\blue{Математическим ожиданием} случайной величины $\xi$ называется число $E \xi = \underset{\Omega}{\overset{}{\int}}\xi(w) P(dw)$ в предположении, что $\underset{\Omega}{\overset{}{\int}}|\xi(w)|P(dw) < \infty$. Если $\underset{\Omega}{\overset{}{\int}}|\xi(w)|P(dw) = \infty$, то будем говорить, что $E \xi$ не существует.
\end{definition}

Если $f(x)$ - плотность вероятности случайной величины $\xi$ по мере $\mu$, а $\varphi(x)$ - борелевская функция, то:
$$E \varphi(x) = \underset{R}{\overset{}{\int}}\varphi(x) P_{\xi} (dx) = \underset{R}{\overset{}{\int}}\varphi(x) f(x) \mu(dx)$$
В частности, если $\xi$ - абсолютно непрерывная случайная величина в терминологии 2-го курса (т.е. $\mu$ - мера Лебега), то в случае $\underset{R}{\overset{}{\int}}|\varphi(x)|f(x) dx < \infty$ пишем $E \varphi(x) = \underset{R}{\overset{}{\int}}\varphi(x) f(x) dx$.

Если $\xi$ дискретна со значениями $x_1, x_2, \dots$ и соответствующими вероятностями $p_1, p_2, \dots$, то $E \varphi (\xi) = \underset{i \ge 1}{\overset{}{\sum}} \varphi (x_i) p_i$ (если ряд сходится абсолютно).

\section{Случайные вектора}\label{cha:1/sec:2}

Обозначим $\mathcal{B}(\mathbb{R}^k)$ борелевскую сигма-алгебру подмножеств $\mathbb{R}^n$.

\begin{definition}\label{cha:1/def:10}
	Вектор $\xi = (\xi_1, \dots, \xi_k)^T$ называется \blue{$k$-мерным случайным вектором}, если $\xi$ - измеримое отображение $\xi: (\Omega, \mathcal{F}) \to (\mathbb{R}^k, \mathcal{B}(\mathbb{R}^k)$
\end{definition}

Известно: $\xi$ - случайный вектор тогда и только тогда, когда каждая компонента $\xi_i$ - одномерная случайная величина.

\begin{definition}\label{cha:1/def:11}
	Функция распределения случайного вектора $\xi$: $F(x_1, \dots, x_k) = P(\xi_1 \le x_1, \dots, \xi_k \le x_k), \; x_i \in \mathbb{R}$, а распределение $P_{\xi}(A) = P(w: \xi(w) \in A), \; A \in \mathcal{B}(\mathbb{R}^k)$.
\end{definition}

\begin{definition}\label{cha:1/def:12}
	Плотность вероятности вектора $\xi$ по мере $\mu$ ($\mu$ распределена на элементам $\mathcal{B}(\mathbb{R}^k)$) - борелевская функция $f(x) \ge 0, \; x = (x_1, \dots, x_n)$, такая что:
	$$P_{\xi}(A) = \underset{A}{\overset{}{\int}}f(x) \mu (dx) \; \forall A \in \mathcal{B}(\mathbb{R}^k)$$
\end{definition}

\begin{definition}\label{cha:1/def:13}
	Случайные величины $\{\xi_1, \dots, \xi_k\}$ независимы, если
	$$P(\xi_1 \in A_1, \dots, \xi_k \in A_k) = \underset{i=1}{\overset{k}{\Pi}} P(\xi_i \in A_i) \; \forall A_i \in \mathcal{B}(\mathbb{R})$$
\end{definition}

\begin{propose}[\textit{необходимые и достаточные условия независимости}]\label{cha:1/propose:1}
	$$F(x_1, \dots, x_k) = F_{\xi_1} (x_1) F_{\xi_2}(x_2) \dots F_{\xi_k}(x_k) \; \forall (x_1, \dots, x_k)$$
	$$f(x_1, \dots, x_k) = f_{\xi_1}(x_1)\dots f_{\xi_k}(x_k)$$
\end{propose}

\section{Сходимости случайных векторов}\label{cha:1/sec:3}

Пусть случайные векторы $\xi, \xi_1, \xi_2, \dots$ размера $k$ со значениями в $(\mathbb{R}^k, \mathcal{B}(\mathbb{R}^k))$ определены на некотором вероятностном пространстве $(\Omega, \mathcal{F}, P)$. Пусть $|\cdot|$ означает евклидову норму вектора, т.е. $|\xi| = \sqrt{\xi_1^2 + \dots + \xi_k^2}$.

\begin{definition}\label{cha:1/def:14}
	Говорят, что последовательность $\{\xi_n\}$ \blue{сходится слабо} к $\xi$ ($\xi_n \xrightarrow[n\to \infty]{W} \xi$), если для любой непрерывной и ограниченной функции $g: \mathbb{R}^k \to \mathbb{R}^1$ имеет место сходимость:
	$$\underset{\mathbb{R}^k}{\overset{}{\int}}g(x) P_n(dx) \xrightarrow[n \to \infty]{}\underset{\mathbb{R}^k}{\overset{}{\int}}g(x) P(dx)\eqno(1)$$
	($P_n$ и $P$ - распределения $\xi_n$ и $\xi$ соотвественно)
\end{definition}

\begin{definition}\label{cha:1/def:15}
	Обозначим $F_n(x)$ и $F(x), \; x = (x_1, \dots, x_n)$, функции распределения векторов $\xi_n$ и $\xi$. Тогда сходимость $(1)$ эквивалентна \blue{сходимости в основном}:
	$$F_n(x) \Rightarrow F(x) \; \Leftrightarrow \; F_n(x) \to F(x) \;\; \forall x \in \mathbb{C}(\mathcal{F})\eqno(2)$$
\end{definition}

\begin{definition}\label{cha:1/def:16}
	Пусть $\varphi_n(t)$ и $\varphi(t), \; t \in \mathbb{R}^k$, будут характеристические функции $\xi_n$ и $\xi$, т.е. $\varphi(t) := E e^{i t^T \xi}$. Тогда сходимость $(2)$ эквивалентна сходимости:
	$$\varphi_n (t) \xrightarrow[n \to \infty]{}\varphi(t) \;\; \forall t \in \mathbb{R}^k\eqno(3)$$
\end{definition}

\begin{definition}\label{cha:1/def:17}
	Если выполнено любое из соотношений $(1)-(3)$, будем писать:
	$$\xi_n \xrightarrow[n \to \infty]{d}\xi\eqno(4)$$
	и говорить, что $\{\xi_n\}$ сходится к $\xi$ \blue{по распределению}.
\end{definition}

\begin{remark}\label{cha:1/remark:1}
	Сходимость $(4)$ не следует из сходимости $\xi_{in} \xrightarrow[]{d}\xi_i, \; i = 1, \dots, k$, компонент векторов $\xi_n$ и $\xi$.
\end{remark}

\begin{definition}\label{cha:1/def:18}
	Говорят, что последовательность $\{\xi_n\}$ сходится \blue{по вероятности} к вектору $\xi$ ($\xi_n \xrightarrow[n \to \infty]{P}\xi$), если:
	$$\forall \varepsilon > 0 \;\; P(|\xi_n - \xi| > \varepsilon) \xrightarrow[n \to \infty]{}0 \eqno(5)$$
\end{definition}

\begin{remark}\label{cha:1/remark:2}
	Понятно, что сходимость $(5)$ эквивалентна сходимости компонент $\xi_{in} \xrightarrow[]{P} \xi_i$ для всех $i=1,2,\dots, k$.
\end{remark}

\begin{remark}\label{cha:1/remark:3}
	Сходимость по вероятности $(5)$ влечет сходимость по распределению $(4)$. Обратное верно только в частных случаях.
\end{remark}

\begin{definition}\label{cha:1/def:19}
	Говорят, что последовательность $\{\xi_n\}$ \blue{сходится п.н.} (почти наверно или с вероятностью единица) и пишут $\xi_n \xrightarrow[n\to \infty]{\text{п.н.}}\xi$, если:
	$$P(w:\xi_n(w) \to \xi(w)) = 1\eqno(6)$$
\end{definition}

\begin{remark}\label{cha:1/remark:4}
	Сходимость п.н. $(6)$ влечет сходимость по вероятности $(5)$. Значит верна следующая цепочка:
	$$\xi_n \xrightarrow[n\to \infty]{\text{п.н.}}\xi \; \Rightarrow \; \xi_n \xrightarrow[n \to \infty]{P}\xi \; \Rightarrow \; \xi_n \xrightarrow[n \to \infty]{d}\xi$$
\end{remark}

\begin{theorem}[\red{непрерывности}]\label{cha:1/the:1}
	Пусть векторы $\{\xi_n\}, \xi$ определены на $(\Omega, \mathcal{F}, P), \;\; \xi_n, \xi \in \mathbb{R}^k$. Пусть $A \in \mathcal{B}(\mathbb{R}^k)$ и $P(\xi \in A) = 1$ (т.е. $A$ - носитель $\xi$). Пусть борелевская $H: \mathbb{R}^k \to \mathbb{R}^1$ непрерывна на множестве $A$. Тогда:
	\begin{itemize}
		\item[1)] если $\xi_n \xrightarrow[n \to \infty]{d}\xi$, то $H(\xi_n)\xrightarrow[n \to \infty]{d}H(\xi)$
		\item[2)] если $\xi_n \xrightarrow[n \to \infty]{P}\xi$, то $H(\xi_n)\xrightarrow[n \to \infty]{P}H(\xi)$
		\item[3)] если $\xi_n \xrightarrow[n \to \infty]{\text{п.н.}}\xi$, то $H(\xi_n)\xrightarrow[n \to \infty]{\text{п.н.}}H(\xi)$
	\end{itemize}
\end{theorem}
\begin{Proof}
	Докажем пункт 3. Два других пункта будут доказаны на практических занятиях.

	Итак, в силу непрерывности функции $H(x)$ на $A$:
	$$\begin{gathered}
		(w:\xi_n(w) \to \xi(w)) \cap (w: \xi(w)\in A) \subseteq (w: H(\xi_n(w))) \to H(\xi(w)) \; \Rightarrow \\
		\Rightarrow \; 1 = P(\xi_n(w) \to \xi(w)) = P(\xi_n(w) \to \xi(w), \xi(w) \in A) \le P(H(\xi_n(w)) \to H(\xi(w))) 
	\end{gathered}$$
\end{Proof}

\section{ЗБЧ и ЦПТ}\label{cha:1/sec:4}

Пусть на $(\Omega, \mathcal{F}, P)$ задана бесконечная последовательность случайных величин $\xi_1, \xi_2, \dots$

\begin{definition}\label{cha:1/def:20}
	Если $\{\xi_i\}$ независимы и одинаково распределены с конечным средним, $E|\xi_1| < \infty$, то
	$$n^{-1} \underset{i=1}{\overset{n}{\sum}}\xi_i \xrightarrow[n\to \infty]{\text{п.н.}}E \xi_1\eqno(7)$$
	Соотношение $(7)$ - \blue{усиленный закон больших чисел Колмогорова}.
\end{definition}

\begin{definition}\label{cha:1/def:21}
	Если $\{\xi_i\}$ некоррелированные случайные величины, может быть, разнораспределенные, но с одинаковым средним $m = E \xi_i$ и $D \xi_i \le C < \infty$, то
	$$n^{-1}\underset{i=1}{\overset{n}{\sum}}\xi_i \xrightarrow[n \to \infty]{P}m = E \xi_i\eqno(8)$$
	Соотношение $(8)$ - \blue{слабый закон больших чисел}.
\end{definition}

\begin{definition}\label{cha:1/def:22}
	Если $\{\xi_i\}$ - н.о.р.с.в., $E \xi_1 = m, \; 0 < D\xi_1 = \sigma^2 < \infty$, то
	$$\frac{1}{\sqrt{n}\sigma} \underset{i=1}{\overset{n}{\sum}}(\xi_i - m) \xrightarrow[n \to \infty]{d}\xi \sim N(0,1)\eqno(9)$$
	Соотношение $(9)$ - \blue{центральная предельная теорема}, точнее ее вариант, т.е.:
	$$n^{\frac{1}{2}} (\overline{\xi} - m)\xrightarrow[n \to \infty]{d}N(0, \sigma^2), \text{ где } \overline{\xi} := n^{-1} \underset{i=1}{\overset{n}{\sum}}\xi_i$$
\end{definition}



