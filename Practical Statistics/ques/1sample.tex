\chapter{Проверка гипотез о параметрах одной выборки}\label{cha:1sample}

\section{Гипотеза о мат. ожидании нормального распределения}\label{cha:1sample/sec:mo}

\subsection{Z-test (1 sample) (известная дисперсия)}\label{cha:1sample/sec:mo/subsec:ztest}

\subsubsection*{Теория}\label{cha:1sample/sec:mo/subsec:ztest/subsubsec:theory}

Применяется в предположении, что выборка нормальна: \\$X = (X_1, \dots, X_n), \; X_i \sim N(a, \sigma^2)$\\
\red{Дисперсия $\sigma^2$ известна, строим гипотезу про $a$}: $H_0: a = a_0$, альтернатива любая.
$$Z = \frac{\overline{X} - a_0}{\frac{\sigma}{\sqrt{n}}} \Hosim N(0,1)$$
Критическое множество зависит от альтернативы.

\subsection{t-test (1 sample) (неизвестная дисперсия)}\label{cha:1sample/sec:mo/subsec:ttest}

\subsubsection*{Теория}\label{cha:1sample/sec:mo/subsec:ttest/subsubsec:theory}

Применяется в предположении, что выборка нормальна: \\$X = (X_1, \dots, X_n), \; X_i \sim N(a, \sigma^2)$\\
\red{Дисперсия $\sigma^2$ неизвестна, строим гипотезу про $a$}: $H_0: a = a_0$, альтернатива любая.
$$t = \frac{\overline{X} - a_0}{\frac{S}{\sqrt{n}}} \Hosim t(n-1), \;\; S = \sqrt{S^2} = \sqrt{\frac{1}{n-1} \underset{i=1}{\overset{n}{\sum}}(X_i - \overline{X})^2}$$
Критическое множество зависит от альтернативы.

\subsubsection*{Python}\label{cha:1sample/sec:mo/subsec:ttest/subsubsec:python}

\paragraph*{Ручная реализация}

\begin{lstlisting}[language=Python]
	alpha = .05
	n = len(data)
	data_mean = np.mean(iq)
	data_s = np.std(data, ddof=1)
	t_stat = (data_mean - 110) * np.sqrt(n) / data_s
	iq_crit = t.ppf(1-alpha, n-1)
	t_p_value = 1 - t.cdf(t_stat, n-1)
	print ("{0:.7f}".format(t_p_value))
\end{lstlisting}

\paragraph*{Встроенная реализация}

Параметр \blue{popmean = $a_0$}. Тест для двусторонней альтернативы, чобы получить односторонний pvalue, делим полученный pvalue на 2.

\begin{lstlisting}[language=Python]
	ans = ttest_1samp(iq, popmean=110)
	pvalue = ans[1]
\end{lstlisting}

\subsubsection*{Пример}\label{cha:1sample/sec:ttest/subsec:python}

\begin{problem}
	 Оцениваем средний уровень IQ профессоров университета города N. Можно ли утверждать на уровне значимости 10$\%$, что средний уровень IQ профессоров выше 110 баллов? Данные в файле IQ.txt (решить задачу в предположении нормальности данных).
\end{problem}
\begin{solution}
	\begin{itemize}
		\item[1 способ] 
			\begin{lstlisting}[language=Python]
				import numpy as np
				import pandas as pd
				from scipy.stats import t
				iq_data = pd.read_csv("IQ.txt")
				iq = iq_data['iq']
				alpha = .1
				n = len(iq)
				iq_mean = np.mean(iq)
				iq_s = np.std(iq, ddof=1)
				t_stat = (iq_mean - 110) * np.sqrt(n) / iq_s
				iq_crit = t.ppf(1-alpha, n-1)
				t_p_value = 1 - t.cdf(t_stat, n-1)
				print("{0:.7f}".format(t_p_value))
			\end{lstlisting}
		\item[2 способ] 
			\begin{lstlisting}[language=Python]
				import numpy as np
				import pandas as pd
				from scipy.stats import ttest_1samp
				iq_data = pd.read_csv("IQ.txt")
				iq = iq_data['iq']
				ans = ttest_1samp(iq, popmean=110)
				pvalue = ans[1]/2
				print("{0:.7f}".format(pvalue))
			\end{lstlisting}
	\end{itemize}
\end{solution}

\begin{problem}
	 В городе Ивановск проведено выборочное исследование доходов жителей. По выборке из 500 человек получено среднее 23800 руб. и среднее квадратическое отклонение 400 руб. Можно ли утверждать на уровне значимости 5$\%$, что средний доход жителей составляет менее 25000 руб? Решить задачу в предположении нормальности данных.
\end{problem}
\begin{solution}
	\begin{lstlisting}[language=Python]
		import numpy as np
		import pandas as pd
		from scipy.stats import t
		alpha = .05
		n = 500
		mean_ivan = 23800
		s_ivan = 400 * np.sqrt(n) / np.sqrt(n-1)
		t_statistics = (mean_ivan - 25000) * np.sqrt(n) / s_ivan
		crit_ivan = t.ppf(alpha, n-1)
		ivan_p_value = t.cdf(t_statistics, n-1)
		print("{0:.7f}".format(ivan_p_value))
	\end{lstlisting}
\end{solution}

\section{Гипотеза о дисперсии нормального распределения}\label{cha:1sample/sec:var}

\subsection{Известное м.о.}\label{cha:1sample/sec:var/subsec:mo}

\subsubsection*{Теория}\label{cha:1sample/sec:var/subsec:mo/subsubsec:theory}

Применяется в предположении, что выборка нормальна: \\$X = (X_1, \dots, X_n), \; X_i \sim N(a, \sigma^2)$\\
\red{М.о. $a$ известно, строим гипотезу про $\sigma^2$}: $H_0: \sigma = \sigma_0$, альтернатива любая.
$$T = \frac{\underset{i=1}{\overset{n}{\sum}}(X_i - a)^2}{\sigma_0^2} \Hosim \chi^2 (n)$$
Критическое множество зависит от альтернативы.

\subsection{Неизвестное м.о.}\label{cha:1sample/sec:var/subsec:nemo}

\subsubsection*{Теория}\label{cha:1sample/sec:var/subsec:nemo/subsubsec:theory}

Применяется в предположении, что выборка нормальна: \\$X = (X_1, \dots, X_n), \; X_i \sim N(a, \sigma^2)$\\
\red{М.о. $a$ неизвестно, строим гипотезу про $\sigma^2$}: $H_0: \sigma = \sigma_0$, альтернатива любая.
$$T = \frac{(n-1)S^2}{\sigma_0^2} \Hosim \chi^2 (n-1), \;\; S^2 = \frac{1}{n-1} \underset{i=1}{\overset{n}{\sum}}(X_i - \overline{X})^2$$
Критическое множество зависит от альтернативы.

\subsubsection*{Пример}\label{cha:1sample/sec:var/subsec:nemo/subsubsec:python}

\begin{problem}
	 Партия изделий принимается, если дисперсия размеров не превышает 0.2. Исправленная выборочная дисперсия для 30 изделий оказалась равной 0.3. Можно ли принять партию на уровне значимости 5$\%$? Решить задачу в предположении нормальности данных.
\end{problem}
\begin{solution}
	\begin{lstlisting}[language=Python]
		import numpy as np
		import pandas as pd
		from scipy.stats import chi2
		alpha = .05
		n = 30
		s2_izd = 0.3
		T = (n-1) * s2_izd / (0.2)
		crit_izd = chi2.ppf(1 - alpha, n-1)
		p_value_izd = 1 - chi2.cdf(T, n-1)
		print("{0:.7f}".format(p_value_izd))
	\end{lstlisting}
\end{solution}

\section{Гипотеза о параметрах гамма-распределения}\label{cha:1sample/sec:gamma}

\subsection{Критерий Вальда}\label{cha:1sample/sec:gamma/subsec:vald}

\subsubsection*{Теория}\label{cha:1sample/sec:gamma/subsec:vald/subsubsec:theory}

Применяется в предположении, что имеем выборку с гамма-распределение: \\ $X = (X_1, \dots, X_n), \;\; X_i \sim Pois (\lambda)$\\
\red{Cтроим гипотезу про $E X_i = \lambda$}: $H_0: \lambda = \lambda_0$, альтернатива любая. По ЦПТ получаем:
$$T = \frac{\underset{i=1}{\overset{n}{\sum}}X_i - n \lambda_0}{\sqrt{n \lambda_0}} \Hosim N(0,1)$$
Критическое множество зависит от альтернативы.\\

Существует асимптотический критерий. \red{Cтроим гипотезу про $E X_i = \mu$}: $H_0: \mu = \mu_0$, альтернатива любая. По лемме Слуцкого получаем:
$$T = \frac{\underset{i=1}{\overset{n}{\sum}}X_i - n \mu_0}{\sqrt{n S^2}} \Hosim N(0,1), \;\; S^2 = \frac{1}{n-1} \underset{i=1}{\overset{n}{\sum}}(X_i - \overline{X})^2$$
Критическое множество зависит от альтернативы.\\

Если $|T(X)| > Z_{1-\alpha/2}$, то гипотеза $H_0$ отвергается, иначе $H_0$ принимается.

\subsubsection*{Пример}\label{cha:1sample/sec:gamma/subsec:vald/subsubsec:python}

\begin{problem}
	 Рассмотрим гамма-распределение с плотностью $f(x) = \frac{\theta^{-\alpha}}{\Gamma(\alpha)} x^{\alpha-1} e^ {\frac{-x}{\theta}}, \ x \ge 0.$ Гипотеза $H_0$: $\alpha \theta = E[X_1] = 1$. Альтернатива $H_1$: $E[X_1] \neq 1$.
\end{problem}
\begin{solution}
	Асимтотическое решение:
	\begin{lstlisting}[language=Python]
		import numpy as np
		import pandas as pd
		from scipy.stats import gamma, norm
		alpha = .05
		
		def calc_wald_statistics(X, assumed_mean): 
	    	X = np.array(X)
	    	n = len(X)
	    	return (X.sum() - n * assumed_mean) / np.sqrt(n * X.var(ddof=1))

		norm_threshold = norm.ppf(1.0 - 0.5 * alpha)
		h0_cws = calc_wald_statistics (gamma.rvs(a = 1, scale = 1, size=2000), 1)
		p_value = 2*np.min([ norm.cdf(h0_cws), 1 - norm.cdf(h0_cws) ])
		print("{0:.7f}".format(p_value))
	\end{lstlisting}
	P.S. a - это параметр альфа, scale - параметр тета\\

	Точное решение:
	\begin{lstlisting}[language=Python]
		import numpy as np
		import pandas as pd
		from scipy.stats import gamma, norm
		alpha = .05
		samples_count = 1000 
		iters_count = 10000
		h0_a = 2.0
		h0_scale = 0.5
		h0_samples = gamma.rvs(a = h0_a, scale = h0_scale, size=(iters_count, samples_count))
		
		def calc_wald_statistics_multirow(X, samples_count, assumed_mean): 
		    X = np.array(X)[:, : samples_count]
		    n = X.shape[1] (n=samples_count
		    return (X.sum(axis=1) - n * assumed_mean) /np.sqrt(n * X.var(ddof=1, axis=1)) 

		h0_stat_values = calc_wald_statistics_multirow(h0_samples, 1000, h0_a * h0_scale)
		ans = np.sum(np.abs(h0_stat_values) > norm_threshold) / float(iters_count)
		print("{0:.7f}".format(ans))
	\end{lstlisting}
\end{solution}

\section{Гипотеза о параметрах распределения Коши}\label{cha:1sample/sec:cauchy}

\subsection{Метод выборочных квантилей}\label{cha:1sample/sec:cauchy/subsec:quant}

\subsubsection*{Теория}\label{cha:1sample/sec:cauchy/subsec:quant/subsubsec:theory}

не нашел у себя

\subsubsection*{Пример}\label{cha:1sample/sec:cauchy/subsec:quant/subsubsec:python}

\begin{problem}
	 Рассмотрим распределение Коши с плотностью $f(x) = \frac{1}{\pi (1 + (x - x_0)^2)}.$ Гипотеза $H_0$: $x_0 = 0$. Альтернатива $H_1$: $x_0 \neq 0$.
\end{problem}
\begin{solution}
Мы не можем здесь рассматривать статистику Вальда. Но знаем, что при $H_0$ статистика

$$ T(X) = \frac{\sqrt{n} \widehat{z}_{0.5}}{\frac{\pi}{2}} $$ стремится к $N(0, 1)$ с ростом количества элементов выборки. Тест снова устроен следующим образом:

Если $|T(X)| > z_{1-\alpha/2}$, то гипотеза $H_0$ отвергается, иначе $H_0$ принимается. Проведем тест с уровнем значимости $\alpha = 0.05$.


	\begin{lstlisting}[language=Python]
		import numpy as np
		import pandas as pd
		from scipy.stats import cauchy

		def calc_statistics(X):
		    X = np.array(X)
		    n=len(X)
		    return 2 * np.sqrt(n) * np.percentile(X, 50, interpolation='lower') / (np.pi)

		alpha = 0.05
		norm_threshold = norm.ppf(1.0 - 0.5 * alpha)
		h0_sample = cauchy.rvs(size = 1000)
		h0_cs = calc_statistics (h0_sample)
		h1_sample = cauchy.rvs(size = 1000, loc=2)
		h1_cs = calc_statistics (h1_sample) 
	\end{lstlisting}
\end{solution}

\section{Непараметрические критерии о математическом ожидании}\label{cha:1sample/sec:neparam}

\subsection{Одновыборочный критерий знаков}\label{cha:1sample/sec:neparam/subsec:znak}

\subsubsection*{Теория}\label{cha:1sample/sec:neparam/subsec:znak/subsubsec:theory}

Применяется в предположении, что имеем выборку $Z = (Z_1, \dots, Z_n)$, удовлетворяющую следующим условия:
\begin{itemize}
	\item[1)] все $Z_i$ независимы
	\item[2)] все $Z_i$ получены из непрерывной совокупности с медианой $\theta$:
	$$P(Z_i < \theta) = P(Z_i > \theta) = \frac{1}{2}, \; i = \ton n$$ 
\end{itemize}
\red{Cтроим гипотезу про медиану $\theta$}: $H_0: \theta = \theta_0$, альтернатива любая.
Модифицируем $Z_i$: $\tilde{Z_i} = Z_i - \theta_0$. Если $\tilde{Z_i} = 0$, то обрасываем этот элемент и уменьшаем $n$. Рассматриваем $\psi_i = \begin{cases}
	1, \; \tilde{Z_i} > 0 \\
	0, \; \tilde{Z_i} < 0 
\end{cases}$. $B = \underset{i=1}{\overset{n}{\sum}}\psi_i \Hosim Bin (n, \frac{1}{2})$ - статистика (число успехов в $n$ испытаниях). В качестве квантилей для левосторонней и правосторонней альтернатив равны $X_{\alpha} - 1$ и $X_{1-\alpha}+1$ соответственно. \\

Существует асимптотический критерий. Гипотезы и альтернативы аналогичные.
$$B^{*} = \frac{B - \frac{n}{2}}{\sqrt{\frac{n}{4}}} \Hosim N(0,1)$$

\subsubsection*{Пример}\label{cha:1sample/sec:neparam/subsec:znak/subsubsec:python}

\begin{problem}
	 В городе N проведены выборочные обследования доходов жителей. Проверить на уровне значимости 3$\%$ утверждение о том, что средняя зарплата жителей в городе N менее 40000 руб. Данные в файле City.txt.
\end{problem}
\begin{solution}
	\begin{lstlisting}[language=Python]
		import numpy as np
		import pandas as pd
		from scipy.stats import binom_test
		from scipy.stats import binom
		city = pd.read_csv("City.txt") 
		city = city['City']
		z = city - 40000
		z = z[z!=0]
		b = sum(z > 0)
		n = len(z)
		sign_res=binom_test(b, n,  p=0.5)
		ans = sign_res/2.0
		ans = binom.cdf(b, n, 0.5)
	\end{lstlisting}
	Асимтотический критерий:
	\begin{lstlisting}[language=Python]
		import numpy as np
		import pandas as pd
		from scipy.stats import binom_test
		from scipy.stats import binom
		from scipy.stats import norm
		city = pd.read_csv("City.txt") 
		city = city['City']
		z = city - 40000
		z = z[z!=0]
		b = sum(z > 0)
		n = len(z)
		b_star = (b - n*0.5) / np.sqrt(n*0.25)
		ans = norm.cdf(b_star, loc=0, scale=1)
	\end{lstlisting}
\end{solution}

\subsection{Одновыборочный знако-ранговый критерий Вилкоксона}\label{cha:1sample/sec:neparam/subsec:znak}

\subsubsection*{Теория}\label{cha:1sample/sec:neparam/subsec:znak/subsubsec:theory}

Wilcoxon signed-rank test\\

Применяется в предположении, что имеем выборку $Z = (Z_1, \dots, Z_n)$, удовлетворяющую следующим условия:
\begin{itemize}
	\item[1)] все $Z_i$ независимы
	\item[2)] все $Z_i$ получены из непрерывной и симметричной относительно $\theta$ совокупности 
\end{itemize}
\red{Cтроим гипотезу про медиану $\theta$}: $H_0: \theta = \theta_0$, альтернатива любая.
Модифицируем $Z_i$: $\tilde{Z_i} = Z_i - \theta_0$. Если $\tilde{Z_i} = 0$, то обрасываем этот элемент и уменьшаем $n$. Сортируем $|\tilde{Z_1}|, |\tilde{Z_2}|, \dots, |\tilde{Z_n}|$ по возрастанию и присваиваем ранги, равные порядковому номеру элемента в последовательности. Если соседние элементы равны, то присваиваем им ранги, равные друг другу и среднему арифметическому их порядковых номеров. Рассматриваем $\psi_i = \begin{cases}
	1, \; \tilde{Z_i} > 0 \\
	0, \; \tilde{Z_i} < 0 
\end{cases}$. $T^{+} = \underset{i=1}{\overset{n}{\sum}}R_i \cdot \psi_i$\\

Существует асимптотический критерий. Гипотезы и альтернативы аналогичные. Применяется для $n > 20$.
$$T^{*} = \frac{T^{+} - \frac{n(n+1)}{4}}{\sqrt{\frac{n(n+1)(2n+1)}{24}}} \Hosim N(0,1)$$

\subsubsection*{Пример}\label{cha:1sample/sec:neparam/subsec:znak/subsubsec:python}

\begin{problem}
	 В городе N проведены выборочные обследования доходов жителей. Проверить на уровне значимости 3$\%$ утверждение о том, что средняя зарплата жителей в городе N менее 40000 руб. Данные в файле City.txt.
\end{problem}
\begin{solution}
	Асимтотический критерий:
	\begin{lstlisting}[language=Python]
		import numpy as np
		import pandas as pd
		from scipy.stats import wilcoxon
		city = pd.read_csv("City.txt") 
		city = city['City']
		z = city - 40000
		z = z[z!=0]
		signed_rank_res = wilcoxon(z)
		ans = signed_rank_res.pvalue/2.0
	\end{lstlisting}
\end{solution}















