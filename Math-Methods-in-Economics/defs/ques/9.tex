\chapter{Устойчивость замкнутой модели Леонтьева.}\label{cha:9}

$$\begin{cases}
	A \pi = \pi \\
	\pi \geq 0
\end{cases} \text{-- замкнутая модель Леонтьева (} \forall \; i: \; \sum\limits_{j = 1}^n a_{ij} = 1 \text{).}$$

Пусть $A \geq 0, \; r_i = \sum\limits_{j = 1}^n a_{ij}, \; s_j = \sum\limits_{i = 1}^n a_{ij}, \; r = \underset{i}{min}r_i, \; R = \underset{i}{max}r_i, \; s = \underset{i}{min}s_i, \; S = \underset{i}{max}s_i$

\begin{clair}
	$A \geq 0 \Rightarrow r \leq \lambda_A \leq R, s \leq \lambda_A \leq S.$
\end{clair}

\begin{conseq}
	$A \geq 0$ -- замкнутая, тогда $\lambda_A = 1.$
\end{conseq}

\begin{definition}
	$A > 0$ -- неразложимая, $\lambda_A = 1$ (иначе $A = \dfrac{1}{\lambda_A}A$). $p_A$ -- левый Фробениусов собственный вектор: $(p_A, p_A) = 1.$
\end{definition}

\begin{definition}
	Норма на $\mathbb{R}^n: \; \| x\|_A = (p_A, |x|), \; |x| = (|x_1|, \ldots, |x_n|).$
\end{definition}

\begin{definition}
	A -- устойчива, если $\forall \; x \; \exists \lim\limits_{k \to \infty}A^k x$
\end{definition}

\begin{remark}
	Выберем правый Фробениусов собственный вектор: $\| x_A\|_A = 1. \; \| Ax\|_A \leq \| x_A\|_A$ и, если $x \geq 0,$ то $\| Ax\|_A = \| x_A\|_A \; (p_A |Ax| \leq p_A A |x| = p_A |x| = \| x \|_A, \; p_A |Ax| = p_A A x = p_A x = p_A |x| = \| x\|_A. )$
\end{remark}

\begin{clair}
	$x \geq 0, $ если $\exists \lim\limits_{k \to \infty}A^k x,$ то $z = \lim\limits_{k \to \infty}A^kx = \mu x_A, \; \mu = \| x\|_A.$
\end{clair}

\begin{definition}[Импримитивная(циклическая) матрица]
	Неразложимая матрица А называется импримитивной(циклической), если $\exists$ разбиение $\{ 1, \ldots, n\} = S_0 \bigsqcup \ldots \bigsqcup S_{m - 1}, \; S_i \cap S_j = \emptyset \; i \neq j, \; \forall i \; S_i \neq \emptyset, $ при этом $a_{ij} > 0 \Rightarrow$:

	$$\begin{cases}
		i \in S_r, \; j \in S_{r - 1}, \; 1 \geq r \geq m - 2 \\
		i \in S_0, \; j \in S_{m - 1}
	\end{cases}$$

	То есть, матрица импримитивна, если одновременной перестановкой столбцов и строк она приводится к виду: 

	\begin{gather*}
		\begin{pmatrix}
		  0 & \ldots & 0 & A_{m-1}\\
		  A_0 & 0 & \ldots & 0\\
		  0 & A_1 & \ldots & 0\\
		  \ldots & \ldots & \ldots & 0\\
		  0 & \ldots & 0 &  A_{m-2}\\
		\end{pmatrix}
	\end{gather*}

	Иначе А -- примитивная.
\end{definition}

\begin{example}
		\begin{gather*}
			\begin{pmatrix}
			  0 & 1\\
			  1 & 0 \\
			\end{pmatrix} \text{--циклическая, неразложимая.}
			\begin{pmatrix}
			  0 & 1\\
			  1 & 1 \\
			\end{pmatrix} \text{--примитивная.}
			\begin{pmatrix}
			  1 & 1\\
			  0 & 1 \\
			\end{pmatrix} \text{--разложимая.}
		\end{gather*}
\end{example}

\begin{theorem}
	Неразложимая матрица $A \geq 0 устойчива, \; \lambda_A = 1 \Longleftrightarrow $ A -- примитивна.
\end{theorem}

\begin{lemma}
	А -- примитивная, тогда у некоторой степени А первая строка положительная: $\exists k: \; \forall \; j a_{1j}^k > 0, \; A^k = (a_{ij}^k)$.
\end{lemma}

\begin{lemma}
	Если матрица $A^k$ устойчива при некотором k, то А -- устойчива.
\end{lemma}

\begin{definition}[Оператор сжатия]
	Оператор Р действует на линейном нормированном пространстве как оператор сжатия, если $\exists \gamma: \; 0 < \gamma < 1: \; \forall \; v \in L: \; \| Pv\| \leq \gamma \| v\|, \; \gamma $ -- коэффициент сжатия. 
\end{definition}

\begin{lemma}
	$L_A:= Ann(p_A) = \{ v | (v, p_A) = 0\}.$ Если оператор А действует на $L_A$ как оператор сжатия, то А -- устойчива.
\end{lemma}

\begin{lemma}
	Если неразложимая матрица А с $\lambda_A = 1$ имеет положительную строку, то оператор А действует на пространстве $L_A:= Ann(p_A) = \{ v | (v, p_A) = 0\}$ как оператор сжатия.
\end{lemma}

\begin{conseq}
	Если неразложимая матрица А с $\lambda_A = 1$ имеет положительную строку, то A -- устойчива.
\end{conseq}

\begin{theorem}
	A > 0 -- неразложимая с $\lambda_A = 1$, тогда А -- устойчивая $ \Longleftrightarrow \forall \; \lambda \in Spec(A) \setminus 1$ выполнено $|\lambda| < 1.$
\end{theorem}