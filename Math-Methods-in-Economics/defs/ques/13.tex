\chapter{Модель Гейла сбалансированного роста. Существование состояния равновесия.}\label{cha:13}

$z:= (x,y) \in \mathbb{R}^{2n}$

\begin{definition}[Модель Гейла]
	Модель Гейла -- подмножество $Z \subset \mathbb{R}^{2n}_+:$
	\begin{enumerate}
		\item $Z$ -- выпуклый замкнутый конус
		\item если $(0,y) \in Z \Rightarrow y = 0$
		\item $\forall \; i: \; \exists (x,y) \in Z: \; y_i > 0 \;\; (3'. \exists (x,y)\in Z: \; y \gg 0)$
	\end{enumerate}
\end{definition}

\begin{itemize}
	\item $z = (x,y) \in Z$ -- производственный процесс
	\item x -- вектор затрат, y -- вектор выпуска
\end{itemize}

\begin{clair}
	(A, B) -- модель Неймана, в А нет нулевых строк (все товары производятся), тогда $Z = \{ (Au, Bu) | u \geq 0\}$ -- модель Гейла.
\end{clair}

\begin{definition}[Траектория(план)]
	Модель Гейла с началом $y_0$ -- последовательность $z_t = (x_{t-1}, y_t) \in Z, \; t \in \mathbb{N}, \; x_t \leq y_t.$
\end{definition}

\begin{definition}[Траектория цен]
	Последовательность $p_t \in \mathbb{R}_+^n, \; t \in \mathbb{N}_0: \; \forall \; (x,y) \in Z: \; p_{t-1}x \geq p_t y. \; \pi_t(\xi) := p_t y_t, \; \xi = \{ z_t\}, \; \pi_t(\xi)$ монотонно убывает.
\end{definition}

\begin{definition}[Состояние равновесия]
	Тройка $(\alpha, \vec{z}, \vec{p}), \; \alpha > 0, \; \vec{z} \in Z, \; \vec{p} > 0,$ -- состояние равновесия, если:
	\begin{enumerate}
		\item $\alpha \vec{x} \leq \vec{y}$
		\item $\forall \; (x, y) \in Z: \; \alpha p x \geq py$.
	\end{enumerate}

	Если (p, y) > 0, то положение равновесия невырожденное, $\alpha$ -- темп роста.
\end{definition}

\begin{theorem}
	У $\forall$ модели Гейла $\exists$ состояние равновесия.
\end{theorem}

\begin{remark}
	Состояние равновесия $(\alpha, \vec{z}, p)$ может быть вырожденным.
\end{remark}

\begin{clair}
	В модели Гейла может быть не более n темпов роста.
\end{clair}

\begin{definition}[Число Фробениуса модели Гейла]
	Обозначим $\alpha'(p) = \inf \{ \alpha | \alpha p x  \geq p y \; \forall \; (x, y) \in Z\}.$ Тогда $\alpha_F = \underset{p > 0}{\inf}\alpha'(p)$ -- число Фробениуса модели Гейла.

	Отметим, что $\exists \vec{p} > 0: \; \alpha'(\vec{p}) = \alpha_F.$
\end{definition}

\begin{clair}
	\begin{enumerate}
		\item $\alpha_F \leq \alpha_N$.
		\item $\forall \; \alpha \in [\alpha_F, \alpha_N] \; \exists $ состояние равновесия $(\alpha, z, p): \; \alpha'(p) \leq \alpha \leq \alpha(z)$.
		\item Если $(\alpha, z, p)$ невырожденно, то $\alpha = \alpha(z) = \alpha'(p).$
	\end{enumerate}
\end{clair}

\begin{remark}
	Существуют модели Гейла без (ненулевых) темпов роста.
\end{remark}