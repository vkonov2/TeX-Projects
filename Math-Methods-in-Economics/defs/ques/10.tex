\chapter{Нелинейная модель Леонтьева.}\label{cha:10}

\begin{itemize}
	\item n товаров.
	\item $f_{ij}(x_j)$ -- количество i-ого товара, необходимого для производства товара j в количестве $x_j$.
	\item Условие замкнутости (бесприбыльности): $x_j = \sum\limits_{i = 1}^nf_{ij}(x_j)$.
	\item  Нелинейная(замкнутая) модель Леонтьева: $x_i = \sum\limits_{j = 1}^nf_{ij}(x_j)$.
\end{itemize}

\begin{sign}
	$D_k^{n - 1} = \{x = (x_1, \ldots, x_n) | x_i \geq 0, \; \sum x_i = k\}$ - (n-1)-мерный симплекс.
\end{sign}

\begin{theorem}[\red{Брауэр}]
	У любого непрерывного отображения $F: \; \delta_k^{n-1} \rightarrow \delta_k^{n-1} \; \exists$ неподвижная точка $x \in \delta_k^{n-1}: \; F(x) = x$
\end{theorem}

\begin{theorem}
	В нелинейной замкнутой модели Леонтьева $\exists$ равновесие x такое, что $\sum x_i = k$.
\end{theorem}