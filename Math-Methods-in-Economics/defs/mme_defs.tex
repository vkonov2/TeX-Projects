\documentclass[article,final,14pt]{scrreprt}
\usepackage{cmap}
% \documentclass[14pt,pdf,hyperref={unicode}]{beamer}
\usepackage[utf8x]{inputenc}
\usepackage[russian]{babel}
\usepackage{amsfonts}
\usepackage{enumerate}
\usepackage{amsthm}
\usepackage{amssymb}
\usepackage{vmargin}
\usepackage{amsmath}
\usepackage{graphicx}
\usepackage{listings}
\usepackage{color}
\usepackage{multicol}
\usepackage{pb-diagram}
\usepackage[Bjornstrup]{fncychap}
\usepackage{fancyhdr}
\usepackage[svgnames,dvipsnames,x11names]{xcolor}
\usepackage{esint}
\usepackage{tocloft}
\usepackage{hyperref}
\usepackage{tikz}
\usepackage{epigraph}
\usepackage{setspace}
\usepackage{microtype}

\usetikzlibrary{calc}

\setpapersize{A4}
\setmarginsrb{2cm}{1.5cm}{1cm}{1.5cm}{0pt}{10mm}{0pt}{20mm}

\setlength\epigraphwidth{.5\textwidth}

% \usepackage[left=3.5cm,right=0cm,top=2cm,bottom=2cm]{geometry}

\usepackage{indentfirst}
\setlength{\parindent}{0.6cm}
% \sloppy

\DeclareGraphicsExtensions{.pdf,.png,.jpg}
\graphicspath{{ques/images/}}

\definecolor{linkcolor}{HTML}{610B0B}
\definecolor{urlcolor}{HTML}{6495ED}
\definecolor{lightgrey}{HTML}{9BABE7}
\definecolor{currentfancycolout}{HTML}{000000}

\hypersetup{pdfstartview=FitH,  linkcolor=linkcolor,urlcolor=urlcolor, colorlinks=true, pagecolor=linkcolor}

\setcounter{tocdepth}{2}
\linespread{1}

\fancyhead[RO]{\colorbox{currentfancycolout}{\color{white}{\textbf{\large \thepage}}}}  %% odd-right 
\fancyhead[LE]{\colorbox{currentfancycolout}{\color{white}{\textbf{\large \thepage}}}}  %%% even-left
\fancyhead[LO]{\colorbox{lightgrey}{\textbf{\thesection}}}% odd-left
\fancyhead[RE]{\colorbox{lightgrey}{\textbf{\thesection}}}% even-right 
\fancyhead[CE]{\rightmark}% odd-center, with the name of the Section
\fancyhead[CO]{\textsc{\leftmark}}% Even-center, with the name of the Chapter.
% \fancyfoot[L,R,C]{}

\makeatletter
\renewcommand*\env@matrix[1][*\c@MaxMatrixCols c]{%
  \hskip -\arraycolsep
  \let\@ifnextchar\new@ifnextchar
  \array{#1}}
\makeatother

\usepackage{enumitem}
% \setlist{noitemsep}
% \setlist[1]{\labelindent=\parindent} % < Usually a good idea
% \setlist[itemize]{leftmargin=*}
% \setlist[itemize,1]{label=$\triangleleft$}
% \setlist[enumerate]{labelsep=*, leftmargin=1.5pc}
% \setlist[enumerate,1]{label=\arabic*., ref=\arabic*}
% \setlist[enumerate,2]{label=\emph{\alph*}),
% ref=\theenumi.\emph{\alph*}}
% \setlist[enumerate,3]{label=\roman*), ref=\theenumii.\roman*}
% \setlist[description]{font=\sffamily\bfseries}

\begin{document}

\pagestyle{fancy}

% \fancyhf{} % очистили все колонтитулы
% \lhead{тратата} % левый верхний колонтитул
% \chead{} % центральный верхний
% \rhead{\textbf{\large \thepage}} % правый верхний
% \lfoot{} % левый нижний
% \cfoot{\textbf{\large \thepage}} % центральный нижний
% \rfoot{} % правый нижний

\renewcommand{\headrulewidth}{2pt} % линия под верхним к.
\renewcommand{\footrulewidth}{0pt} % линия над нижним к. 

\renewcommand\qedsymbol{$\blacksquare$}
\renewcommand\contentsname{Содержание}
\renewcommand\cftchapfont{\large\mdseries}

\newtheorem{theorem}{Теорема}[chapter]
\newtheorem{problem}{Задача}
\newtheorem{lemma}{Лемма}[chapter]
\newtheorem{clair}{Утверждение}[chapter]
\newtheorem{definition}{Определение}[chapter]
\newtheorem{propose}{Предложение}[chapter]
\newtheorem{property}{Свойство}[chapter]
\newtheorem{condition}{Условие}[chapter]
\newtheorem{properties}{Свойства}[chapter]
\newtheorem{conseq}{Следствие}[chapter]
\newtheorem{remem}{Напоминание}[chapter]
\newtheorem{example}{Пример}[chapter]
\newtheorem{rulee}{Правило}[chapter]

\newtheorem*{question}{Вопрос}
\newtheorem*{remark}{Замечание}
\newtheorem*{chck}{Проверка}
\newtheorem*{sign}{Обозначение}
\newtheorem*{uprazh}{Упражнение}

\newenvironment{Proof}       
	{\par\noindent{\bf Доказательство.}}
	{\hfill$\blacksquare$}
\newenvironment{solution}       
	{\par\noindent{\bf Решение.}}
	{\hfill$\blacksquare$}

\newcommand{\red}[1]{\textbf{\color{red}#1}}
\newcommand{\blue}[1]{\textbf{\color{blue}#1}}
\newcommand{\green}[1]{\textbf{\color{green}#1}}

\newcommand{\RNumb}[1]{\uppercase\expandafter{\romannumeral #1\relax}}

\def\ton#1{1,2,\dots,#1}
\def\Set#1#2{\left\{#1\colon#2\right\}}
\def\MYdef{\mathrel{\stackrel{\rm def}=}}
\def\QUdef{\mathrel{\stackrel{\rm ?}=}}
\def\Ddef{\mathrel{\stackrel{\rm d}=}}
\def\PNdef{\mathrel{\stackrel{\rm \text{п.н.}}=}}

\begin{titlepage}
  \begin{center}
    \large
 
 	МОСКОВСКИЙ ГОСУДАРСТВЕННЫЙ УНИВЕРСИТЕТ ИМЕНИ М. В. ЛОМОНОСОВА 
    
    \includegraphics[scale=0.6]{mm.jpg} 
     
    Механико-математический факультет
    \vspace{0.25cm} 
      
    экономический поток
    \vspace{0.8cm} 
     
    {\LARGE \textit{Алгебраические методы в экономике}}
    
    \vspace{0.8cm} 
    3 курс, группы 331-332

    \vspace{0.25cm} 
    6 семестр
\end{center}
\vfill
 
\newlength{\ML}
\settowidth{\ML}{«\underline{\hspace{0.7cm}}» \underline{\hspace{1cm}}}
\hfill\begin{minipage}{7cm}
  \begin{flushright}
    Лектор, семинарист $\;\;$\\
    д. ф.-м. н., профессор $\;\;$
    В.А.~Артамонов $\;\;$\\
    «\underline{\hspace{0.7cm}}» \underline{\hspace{2cm}} 2021 г. $\;\;$
  \end{flushright}
\end{minipage}%
\vfill
\bigskip
 
\begin{center}
  Москва, 2021 г.
\end{center}
% \tikz[remember picture,overlay] \node[opacity=0.3,inner sep=0pt] at (8.5,12.5){\includegraphics[scale=1.13]{background}};
\clearpage
\end{titlepage}
\newpage

% \begin{center}
	{\Large \textbf{Техническая информация}}
\end{center}

\vspace{0.5cm}
Данный PDF содержит примерную программу по предмету "Уравнения с частными производными"$\;$ 5-ого и 6-ого семестров.

\vspace{0.5cm}
Собрали и напечатали по мотивам лекций студенты 3-го курса Конов Марк и Гащук Елизавета.

\vspace{0.5cm}
Авторы выражают огромную благодарность лектору, семинаристу, кандидату физико-математических наук, доценту Капустиной Татьяне Олеговне за прочитанный курс по предмету <<Уравнения с частными производными>>.

\vspace{0.5cm}
Добавления и исправления принимаются на почты \href{}{vkonov2@yandex.ru}, \href{}{gashchuk2011@mail.ru}.

\vspace{0.5cm}
\begin{center}
	{\Large \textbf{ПРИЯТНОГО ИЗУЧЕНИЯ}}
\end{center}

\newpage



% \begin{center}
	{\Large \textbf{Программа экзамена по предмету}}
	
	{\Large \textbf{<<Уравнения с частными производными>>}}
\end{center}

\begin{enumerate}
\item Линейные уравнения с частными производными второго порядка. Понятие характеристики. [4], § 2.2. Классификация
уравнений второго порядка. Приведение к каноническому виду в случае двух независимых переменных. [1], глава I, §
1(1). Приведение к каноническому виду уравнений с постоянными коэффициентами в случае трех и более независимых
переменных. [5], приложение.

\item Задача Коши для линейного уравнения второго порядка. Теорема Коши-Ковалевской (без доказательства). [2], § 1.4(7).

\item Корректность постановки задачи. Пример Адамара некорректной задачи. [2], § 1.4(6,8).

\item Задача Коши для уравнения струны. Формула Даламбера. [5], глава I, § 2 Решение неоднородного уравнения, принцип
Дюамеля. [2], § 5.1.6.

\item Полуограниченная струна. Методы четного и нечетного продолжения. Решение задачи для полуограниченной струны с
неоднородным граничным условием. [5], глава I, § 5

\item Задача Коши для волнового уравнения в R2 и R3 . Энергетическое неравенство. Единственность решения задачи
Коши. [4], § 5.1.1.

\item Формула Кирхгофа решения задачи Коши для волнового уравнения в $ \mathbb{R}^3 $ . [4], § 5.1.2.

\item Формула Пуассона решения задачи Коши для волнового уравнения в $ \mathbb{R}^2 $ . Метод спуска. [4], § 5.1.3.

\item Ограниченная струна. Метод Фурье. [5], глава II, § 6, 7.2, 8.2.

\item Принцип максимума для уравнения теплопроводности в ограниченной области. Единственность решения первой
краевой задачи. [1], глава III, § 1(5,6).

\item Задача Коши для уравнения теплопроводности. Принцип максимума в неограниченной области. Единственность
решения задачи Коши в классе ограниченных функций. [1], глава III, § 1(7).

\item Формула Пуассона решения задачи Коши для уравнения теплопроводности. [1], глава III, § 3(1).

\item Метод Фурье для уравнений Лапласа и Пуассона в круге и кольце. [5], глава II, § 9.2.

\item Обобщенные функции. Действия над обобщенными функциями. [2], § 2.1, 2.2. Фундаментальное решение линейного
дифференциального оператора с постоянными коэффициентами. [2], § 3.1(1,2).

\item Формулы Грина. [4], § 3.1.

\item Фундаментальное решение оператора Лапласа в $ \mathbb{R}^2 $ и $ \mathbb{R}^3 $ . [4], § 3.2.

\item Функция Грина оператора Лапласа, ее симметрия. Представление решения задачи Дирихле через функцию Грина. [4], §
3.6. Метод отражений. Метод конформных отображений. [5], глава IV, § 4, 5

\item Свойства гармонических функций: теорема о потоке, теоремы о среднем по сфере и по шару, принцип максимума,
теорема Лиувилля. [4], § 3.5.

\item Краевые задачи для уравнений Лапласа и Пуассона. [4], § 3.4. Единственность решения задачи Дирихле. [4], § 3.5.
Задача Неймана: условие разрешимости [6], § 5.2, теорема о множестве решений [5], глава IV, § 2(8).
\end{enumerate}


 

\tableofcontents

\chapter{Описание выпуклых многогранников}
\label{cha:1}

\epigraph{
	\textit{Как многогранен этот мир и многолик!}

    \textit{А мы в нем только временные странники...}

    \textit{Мы - путники, пришедшие на миг...}

    \textit{Мы - отражающие вечность многогранники!}}
{-- Аллиса Невдомек}


Барицентрическая комбинация точек с неотрицательными коэффициентами называется \textit{выпуклой}.

Совокупность выпуклых комбинаций некоторой системы точек M называется их \textit{выпуклой оболочкой} $conv M$. 

Выпуклая комбинация $m + 1$ точки, находящихся в общем положении, называется \textit{$m$-мерным симплексом}.

Например, одномерный симплекс – это отрезок, двумерный симплекс – это треугольник, трехмерный симплекс – тетраэдр.

\begin{propose}[]\label{cha:1/propose:1}
	Выпуклая комбинация точек не зависит от выбора внешней точки. Выпуклая комбинация точек n-мерного аффинного пространства совпадает с выпуклой комбинацией не более, чем $n + 1$ точек из них.
\end{propose}
\begin{Proof}
	Рассмотрим выпуклую комбинацию точек $A_0, \dots, A_m$, где $m > n$:
	$$O + \underset{j=0}{\overset{m}{\sum}}\lambda_j \overline{OA_j}, \; \lambda_j > 0, \; \underset{j=0}{\overset{m}{\sum}}\lambda_j = 1$$
	Так как векторы $A_0A_1, \dots, A_0A_m$ линейно зависимы, то существует нетривиальная линейная комбинация $\displaystyle c_1 \overline{A_0 A_1} + \dots + c_m \overline{A_0 A_m} = 0$. Полагая $A_0A_j = OA_j − OA_0$, получим равенство:
	$$\alpha_0 \overline{O A_0} + \dots + \alpha_m \overline{O A_m} = 0, \; \alpha_j = \begin{cases}
		c_j, \text{ если } j=\ton m \\
		-c_1 - \dots - c_m, \text{ если } j = 0
	\end{cases}$$
	Тогда $\underset{j=0}{\overset{m}{\sum}}\alpha_j = 0$.  Без ограничения общности можно считать, что некоторое $\alpha_j > 0$. Пусть $\theta = \underset{j: \alpha_j > 0}{\min}\frac{\lambda_j}{\alpha_j}$. В этом случае $\theta > 0$, причем:
	$$O + \underset{j=0}{\overset{m}{\sum}}\lambda_j \overline{O A_j} = O + \underset{j=0}{\overset{m}{\sum}}(\lambda_j - \theta \alpha_j) \overline{OA_j}$$
	Для положительных $\alpha_j$ имеем $\displaystyle \lambda_j - \theta \alpha_j = \alpha_j \left( \frac{\lambda_j}{\alpha_j} - \theta \right)$. Для отрицательных $\alpha_j$ это неравенство тем более имеет место. При этом хотя бы для одного $j$ справедливо равенство $\lambda_j − \theta \alpha_j = 0$. Кроме того:
	$$\underset{j=0}{\overset{m}{\sum}}(\lambda_j - \theta \alpha_j) = \underset{j=0}{\overset{m}{\sum}}\lambda_j - \theta \underset{j=0}{\overset{m}{\sum}}\alpha_j = 1$$
\end{Proof}

\begin{definition}\label{cha:1/def:1}
	Множество точек \blue{выпукло}, если оно вместе с любыми двумя своими точками содержит соединяющиий их отрезок.
\end{definition}

\begin{definition}\label{cha:1/def:2}
	Наименьшее по включению выпуклое множество, содержащее данное множество точек $\{A_0, \dots, A_m\}$ называется \blue{выпуклым замыканием} и обозначается $conv\{A_0, \dots, A_m\}$.
\end{definition}

\begin{theorem}[]\label{cha:1/the:1}
	Выпуклое замыкание $M = conv\{A_0, \dots, A_m\}$ состоит из всех точек следующего вида:
	$$A = O + \underset{j=0}{\overset{m}{\sum}}\alpha_j \overline{O A_j}, \; \underset{j=0}{\overset{m}{\sum}}\alpha_j = 1, \; \alpha_j \ge 0\eqno(1)$$
\end{theorem}
\begin{Proof}
	Пусть $A$ из (1) и
	$$B = O + \underset{j=0}{\overset{m}{\sum}}\beta_j \overline{O A_j} \in M, \; \underset{j=0}{\overset{m}{\sum}}\beta_j = 1, \; \beta_j \ge 0$$
	Пусть $0 \le \alpha, \beta \le 1$ и $\alpha + \beta = 1$. Тогда:
	$$\begin{gathered}
		O + \alpha \overline{OA} + \beta \overline{OB} = O + \alpha \underset{j=0}{\overset{m}{\sum}}\alpha_j \overline{OA_j} + \beta \left( \underset{j=0}{\overset{m}{\sum}}\beta_j \overline{OA_j} \right) = \\
		= O + \underset{j=0}{\overset{m}{\sum}} (\alpha \alpha_j + \beta \beta_j) \overline{OA_j} \in M \\
		\text{ т.к. } \underset{j=0}{\overset{m}{\sum}} (\alpha \alpha_j + \beta \beta_j) = \alpha \underset{j=0}{\overset{m}{\sum}}\alpha_j + \beta \underset{j=0}{\overset{m}{\sum}}\beta_j = \alpha + \beta = 1
	\end{gathered}$$
	Следовательно, $M$ – выпуклое множество. Кроме того, $\{A_0, \dots, A_m\} \in M$.

	Покажем обратное включение. Пусть $N$ — выпуклое множество, содержащее $A_0, \dots, A_m$. Индукцией по $m$ установим, что точка $A$ из (1) в лежит в $N$.

	Если $m = 0$, то $conv A_0 = A_0 \in N$.

	Пусть $m = 1$. Тогда при $\alpha_0, \alpha_1 \ge 0$ и $\alpha_0 + \alpha_1 = 1$, откуда получаем:
	$$A = O + \alpha_0 \overline{OA_0} + \alpha_1 \overline{OA_1} \in \left[ A_0, A_1 \right] \subseteq N\eqno(2)$$
	Пусть для случая $m − 1$ утверждение доказано и $m \ge 2$. Возьмем точку $A$ из (1). Можно считать, например, что $1 > \alpha_m > 0$. Тогда: 
	$$\begin{gathered}
		B = O + \underset{j=0}{\overset{m-1}{\sum}}\frac{\alpha_j}{1-\alpha_m}\overline{OA_j} \in N \text{ по индукции}, \\
		\text{т.к. } \underset{j=0}{\overset{m-1}{\sum}}\frac{\alpha_j}{1-\alpha_m} = \frac{1-\alpha_m}{1-\alpha_m} = 1
	\end{gathered}$$
	При этом как и в (2):
	$$A = O + (1-\alpha_m)\underset{j=1}{\overset{m-1}{\sum}}\frac{\alpha_j}{1-\alpha_m}\overline{OA_j} + \alpha_m \overline{OA_m} \in \left[ B, A_m \right] \subseteq N$$
	Следовательно, $M \subseteq N$.
\end{Proof}

\begin{definition}\label{cha:1/def:3}
	Выпуклое замыкание конечного числа точек называется \blue{выпуклым многогранником}.
\end{definition}

\begin{theorem}[]\label{cha:1/the:2}
	В n-мерном евклидовом пространстве выпуклое множество размерности n обладает внутренними точками.
\end{theorem}
\begin{Proof}
	Выпуклое множество размерности n содержит $n+1$ точку в общем положении и, следовательно, в нем лежит порожденный ими n-мерный симплекс, обладающий внутренними точками.
\end{Proof}

\begin{definition}\label{cha:1/def:4}
	Совокупность всех внутренних точек множества $M \subseteq S$ называется его \textit{внутренностью} и обозначается $int M$, а совокупность граничных точек — его \textit{границей} и обозначается $bd M$.
\end{definition}

\begin{definition}\label{cha:1/def:5}
	Множество M, полученное присоединением к M его граничных точек, называется \blue{замыканием} M. Таким образом, $\overline{M} = M \cup bd M$. Множество M замкнуто, если $M = \overline{M}$.
\end{definition}

\begin{propose}[]\label{cha:1/propose:2}
	Внутренность выпуклого множества является выпуклым множеством.
\end{propose}
\begin{Proof}
	Пусть $A, B \in int M$ и $C \in [A, B]$. Покажем, что $C$ является внутренней точкой множества $M$. Это вытекает из элементарных геометрических соображений. Существует такое $r > 0$, что шары с центрами $A$ и $B$ радиуса $r$ лежат в $M$. 

	\begin{center}
		\includegraphics[width = \textwidth]{1_1}
	\end{center}

	Пусть $C_1$ – любая точка, удаленная от $C$ на расстояние $\le r$. Тогда в упомянутых шарах можно выбрать, соответственно, точки $A_1, B_1$ так, чтобы выполнялись равенства $\overline{AA_1} = \overline{CC_1} = \overline{BB_1}$. По условию $A_1, B_1 \in M$ и в силу выпуклости множества $M$ справедливо включение $[A_1, B_1] \in M$. Но $C_1 \in [A_1, B_1]$ и потому $C_1 \in M$. Таким образом, окрестность точки $C$ радиуса $r$ лежит в $M$, т. е. $C$ – внутренняя точка. Значит, $[A, B] \subseteq int M$ и $int M$ – выпуклое множество.

\end{Proof}

\begin{propose}[]\label{cha:1/clair:2}
	Замыкание $\overline{M}$ выпуклого множества $M$ является выпуклым множеством.
\end{propose}
\begin{Proof}
	Пусть $A, B \in M$ и:
	$$C = O + \lambda \overline{OA} + \mu \overline{OB} \in [A, B], \; \lambda, \mu \ge 0, \; \lambda + \mu = 1$$
	Из очевидных соображений для любого $\varepsilon > 0$ существует такое $\delta > 0$, что как только точки $A_1$ и $B_1$ находятся в $\delta$-окрестности, соответственно, точек $A$ и $B$, то точка $\displaystyle C_1 = O + \lambda \overline{OA_1} + \mu \overline{OB_1} \in [A_1, B_1]$ лежит в $\varepsilon$-окрестности точки $C$. По условию точки $A_1$, $B_1$ можно выбрать лежащими в $M$ и в силу выпуклости M точка $C_1$ лежит в M . Итак, в любой окрестности точки C лежат точки из M , поэтому C либо внутренняя, либо граничная точка M. В обоих случаях $C \in \overline{M}$. Нами доказано, что $[A,B] \subseteq \overline{M}$ и, следовательно, $\overline{M}$ – выпуклое множество.

\end{Proof}

\begin{lemma}\label{cha:1/lemma:1}
	Выпуклое замыкание объединения двух выпуклых множеств M и N совпадает с объединением отрезков, соединяющих пары точек этих множеств:
	$$[M \cup N] = \cup [A, B], \; A \in M, B \in N$$
\end{lemma}
\begin{Proof}
	По теореме \ref{cha:1/the:1} точка $C \in [M \cup N ]$ обладает представлением следующего вида:
	$$\begin{gathered}
		C = O + \underset{j=1}{\overset{p}{\sum}}\lambda_j \overline{OA_j} + \underset{j=1}{\overset{q}{\sum}}\mu_j \overline{O B_j}, \; \lambda_j, \mu_j \ge 0 \\
		\underset{j=1}{\overset{p}{\sum}}\lambda_j + \underset{j=1}{\overset{q}{\sum}}\mu_j = 1, \; A_1, \dots, A_p \in M, \; B_1, \dots, B_q \in N
	\end{gathered}$$
	Положим:
	$$\lambda = \underset{j=1}{\overset{p}{\sum}}\lambda_j \ge 0, \; \mu = \underset{j=1}{\overset{q}{\sum}}\mu_j \ge 0$$
	Тогда $\lambda + \mu = 1$ и:
	$$A = O + \underset{j=1}{\overset{p}{\sum}}\frac{\lambda_j}{\lambda} \overline{OA_j}, \; B = O + \underset{j=1}{\overset{q}{\sum}}\frac{\mu_j}{\mu} \overline{OB_j}$$
	Следовательно, $C \in [A, B]$, $A \in M$, $B \in N$, поскольку $\displaystyle C = O + \lambda \overline{OA} + \mu \overline{OB}$.
\end{Proof}

\begin{definition}\label{cha:1/def:6}
	Точка выпуклого множества называется \blue{угловой}, если она не принадлежит внутренности отрезка, целиком лежащего в этом множестве.
\end{definition}

\begin{conseq}[]\label{cha:1/conseq:1}
	Пусть M выпуклое множество и точка $A \not \in M$. Тогда A является угловой точкой выпуклого замыкания $[M \cup A]$.
\end{conseq}
\begin{Proof}
	По лемме \ref{cha:1/lemma:1} концы отрезка $[P, Q] \subseteq [M \cup A]$ принадлежат, соответственно, отрезкам: $Q \in [A,B]$, $B \in M$, и $P \in [A, C]$, $C \in M$. 

	\begin{center}
		\includegraphics[width = \textwidth]{1_2}
	\end{center}

	Из треугольника $ABC$ видно, что при любых возможных положениях точек $P, Q$ точка A не принадлежит внутренности отрезка $[P,Q]$. Это рассуждение включает в себя и предельный случай $B = C$.
\end{Proof}

\begin{theorem}[]\label{cha:1/the:3}
	Выпуклый многогранник совпадает с выпуклым замыканием своих угловых точек.
\end{theorem}
\begin{Proof}
	Выбросим последовательно те порождающие точки выпуклого многогранника $M$ , которые принадлежат выпуклому замыканию остальных точек. Оставшиеся точки порождают многогранник $M$, причем по следствию \ref{cha:1/conseq:1} каждая из них является угловой.
\end{Proof}

\begin{theorem}[]\label{cha:1/the:4}
	Выпуклый многогранник является замкнутым множеством.
\end{theorem}
\begin{Proof}
	Из предложения \ref{cha:1/propose:1} вытекает, что каждая точка выпуклого многогранника M принадлежит симплексу некоторой размерности, целиком лежащему в M, и порождено некоторым множеством угловых точек, находящихся в общем положении. Таким образом, M совпадает с объединением конечного числа симплексов. Но каждый симплекс является замкнутым множеством, и поэтому M – замкнутое множество.
\end{Proof}























\chapter{Задача Коши для линейного уравнения второго порядка. Теорема Коши-Ковалевской (без доказательства).}
\label{cha:2}

\begin{definition}[\red{Задача Коши}]\label{cha:2/def:1}
	\hfill \break
	Пусть $S \subset \mathbb{R}^n$ - поверхность, $dim S = n-1$, $S: F(x_1, \dots, x_n) = 0$.
	$$\begin{cases}
		\underset{i,j=1}{\overset{n}{\sum}}a_{i j} u''_{x_i x_j} + \underset{i=1}{\overset{n}{\sum}}b_i u'_{x_i} + c u = f (x_1, \dots, x_n), \;\; a_i, b_i, c, f\text{ - функции от }x_1, \dots, x_n \\
		u|_s = \varphi (\overrightarrow{x}), \; S \subset \mathbb{R}^n \\
		\frac{\partial u}{\partial \overrightarrow{l}}|_S = \psi (\overrightarrow{x}), \text{ где } \overrightarrow{l} \text{ - не касательное направление к поверхности } S
	\end{cases}$$
\end{definition}

\begin{remem}\label{cha:1/remem:1}
	\blue{Аналитическая функция} вещественного переменного - это функция, которая совпадает со своим рядом Тейлора в окрестности любой точки области определения.
\end{remem}

\begin{theorem}[\red{Коши-Ковалевской}]\label{cha:2/the:1}
	Пусть выполняется:
	\begin{itemize}
		\item[$1)$] $a_{i j} (\overline{x}), b_i (\overline{x}), c, f, \psi, \xi$ - аналитические функции в области $\Omega: s \subset \mathbb{R}$\\
		$S: F(\overline{x}) = 0 \; \Rightarrow \; F$ - тоже аналитическая функция в $\Omega$
		\item[$2)$] поверхность $S$ не характеристическая, т.е. $A(\overrightarrow{n}) \not = 0$ ни в одной точке $S$.
	\end{itemize}
	Тогда $\exists ! \; u(\overline{x})$ - решение задачи Коши в некоторой области $Q: S \subset Q \subset \Omega$ и $u(\overline{x})$ - аналитическая функция в $Q$.
\end{theorem}

\begin{remark}\label{cha:2/remark:1}
	Решение задачи Коши непрерывно зависит от начальных условий.
\end{remark}
\begin{Proof}
	Если $\varphi_n \xrightarrow[]{\text{равн-но}} \varphi, \; \psi_n \xrightarrow[]{\text{равн-но}} \psi$, то $u_n \xrightarrow[]{\text{равн-но}} u$ при $t < t_0$.\\
	$\begin{cases}
		\varphi_n(x-t) \xrightarrow[]{\text{равн-но}} \varphi (x-t) \\
		\varphi_n (x+t) \xrightarrow[]{\text{равн-но}} \varphi (x+t)
	\end{cases} \Rightarrow \; |\underset{x-t}{\overset{x+t}{\int}} \psi d\xi - \underset{x-t}{\overset{x+t}{\int}} \psi_n d\xi| \le \underset{x-t}{\overset{x+t}{\int}} |\psi - \psi_n| d\xi \le \\
	\le 2 t \cdot max |\psi - \psi_n| \to 0$, т.к. $|\psi - \psi_n| \xrightarrow[]{\text{равн-но}} 0$.
\end{Proof}


\chapter{Замкнутость конечно порожденного конуса}
\label{cha:3}

\epigraph{
	\textit{Взгляд, постоянно обращенный назад, и исключительное, замкнутое общество — начало выражаться в речах и мыслях, в приемах и одежде; новый цех — цех выходцев — складывался и костенел рядом с другими.}}
{-- Герцен А.И.}

\begin{definition}\label{cha:3/def:1}
	\blue{Конусом} K в $\mathbb{A}^n$ с вершиной в $O \in \mathbb{A}^n$ называется множество точек в $\mathbb{A}^n$, обладающее следующим свойством: если $A \in K, \; \lambda \in \mathbb{R} \lambda \ge 0$, то $O + \lambda \overline{OA} \in K$.
\end{definition}

\begin{propose}\label{cha:3/propose:1}
	Конус K является выпуклым множеством тогда и только тогда, когда вместе с точками $P, Q \in K$ он содержит точку $O + (\overline{OP} + \overline{OQ}) \in K$.
\end{propose}
\begin{Proof}
	Пусть $K$ выпуклое множество и $P, Q \in K$. Тогда $K$ содержит точку $O + \left( \frac{1}{2}\overline{OP} + \frac{1}{2}\overline{OQ} \right)$ и поэтому содержит точку $O + 2\left( \frac{1}{2}\overline{OP} + \frac{1}{2}\overline{OQ} \right) = O + \left( \overline{OP} + \overline{OQ} \right)$. Обратно, если выполнено указанное условие, то по определению конуса $O + \alpha \overline{OP}, \; O + (1 - \alpha) \overline{OQ} \in K$, и поэтому $O + \alpha \overline{OP} + (1-\alpha)\overline{OQ} \in K$, т.е. $[P,Q] \subseteq K$ и конус $K$ выпуклый.
\end{Proof}

\begin{definition}\label{cha:3/def:2}
	Говорят, что конус K с вершиной O \blue{порождается точками}
	$$A_1, \dots, A_m\eqno(3)$$
	если он состоит из всех точек вида $O + i \underset{i=1}{\overset{m}{\sum}}\lambda_i \overline{OA_i}$, где $\lambda_i \ge 0$. Конус K называется \textit{конечнопорожденным}, если он порождается некоторым конечным множеством точек.
\end{definition}

\begin{propose}\label{cha:3/propose:2}
	Конечнопорожденный конус является замкнутым выпуклым множеством.
\end{propose}
\begin{Proof}
	В силу предложения \ref{cha:3/propose:1} конус K является выпуклым. Докажем его замкнутость. Доказательство восходит к доказательству предложения \ref{cha:1/propose:1}. Пусть конус K с вершиной в точке O порождается точками (3). Рассмотрим точку:
	$$O + \lambda_1 \overline{OA_{i_1}} + \dots + \lambda_k \overline{OA_{i_k}} \in K, \; \lambda_j > 0\eqno(4)$$
	Предположим, что векторы $\overline{OA_{i_1}}, \dots, \overline{OA_{i_k}}$ линейно зависимы и $\displaystyle \alpha_1 \overline{OA_{i_1}} + \dots + \alpha_k \overline{OA_{i_k}} = 0$.

	Без ограничения общности можно предполагать, что, например, $\alpha_1 > 0$. Выберем индекс t так, чтобы $\theta = \frac{\lambda_t}{\alpha_t}$ было бы минимальным положительным числом среди всех чисел $\frac{\lambda_t}{\alpha_t}$ , где $\lambda_t$ из (4), $\alpha_t > 0$. Тогда в (5) получаем равенство:
	$$O + \lambda_1 \overline{OA_{i_1}} + \dots + \lambda_k \overline{OA_{i_k}} = O + (\lambda_1 - \theta \alpha_1)\overline{OA_{i_1}} + \dots + (\lambda_k - \theta \alpha_k)\overline{OA_{i_k}}$$
	причем все коэффициенты $\lambda_j−\theta \alpha_j \ge 0$, и один из этих коэффициентов равен нулю. Таким образом, точка (4) лежит в конусе, порожденном точками $A_{i_1}, \dots, A_{i_{t−1}}$, $A_{i_{t+1}}, \dots, A_{i_m}$. Отсюда вытекает, что каждая точка из K лежит в некотором конусе $K_{j_1,\dots,j_s}$, порождаемом точками $A_{j_1}, \dots, A_{j_s}$, причем векторы $\displaystyle e_1 = \overline{OA_{j_1}}, \dots, e_s = \overline{OA_{j_s}}$ независимы. Дополним эти векторы до базиса $e_1, \dots, e_n$ всего линейного пространства и возьмем точку O в качестве начала координат. Тогда в этой системе координат конус $K_{j_1,\dots,j_s}$ задается неравенствами и уравнениями $x_1 \ge 0, \dots ,x_s \ge 0, x_{s+1} = \dots = x_n = 0$. Следовательно, конус K является объединением конечного числа замкнутых конусов вида $K_{j_1,\dots,j_s}$ и потому конус K замкнут.
\end{Proof}


\chapter{Теорема отделимости для замкнутого выпуклого конуса и замкнутого выпуклого компакта вне конуса}
\label{cha:4}

\epigraph{
	\textit{Памятник возвышался в цветах; его пьедестал образовал конус цветов, небывалый ворох, сползающий осыпями жасмина, роз и магнолий.}}
{-- Грин Александр}

\begin{theorem}[]\label{cha:4/the:1}
	Пусть K – замкнутый выпуклый конус с вершиной O и N – компактное выпуклое множество, не пересекающееся с K. Тогда существует такая линейная функция g, что $g(x) \ge 0$ для всех $x \in K$, $g(O) = 0$ и $g(y) < 0$ для всех $y \in N$.
\end{theorem}
\begin{Proof}
	По теореме \ref{cha:2/the:1} существует ближайшая к A точка $B\in K$. Пусть $e_1, \dots, e_n$ – базис, и $g = x_1 − \frac{r}{2}$ – аффинная функция, построенная в теореме \ref{cha:2/the:1} и следствии \ref{cha:2/conseq:1}. Покажем, что $g(O) = 0$. Пусть это не так, т.е. $g(O) > 0$. Поскольку $g(A) = −r < 0$, то:
	$$\cos \angle OBA = \frac{\left( BO, BA \right)}{||OB||\cdot ||BA||} = \frac{g(O) \cdot g(A)}{||OB|| \cdot ||BA||} < 0$$
	Таким образом, $\angle OBA > \frac{\pi}{2}$. Следовательно, перпендикуляр, опущенный из A на прямую OB пересекает ее в точке P, лежащей на луче OB, причем точки P и O лежат на этом луче по разные стороны от B.

	Отсюда $\overline{OP} = \lambda \overline{OB}, \lambda > 1$, и поэтому $P \in K$. Но $||AP||<||AB||$, что противоречит выбору B. Полученное противоречие доказывает теорему.
\end{Proof}

\chapter{Полуограниченная струна. Методы четного и нечетного продолжения. Решение задачи для полуограниченной струны с неоднородным граничным условием.}
\label{cha:5}

\textbf{Однородное уравнение с однородным граничным условием $\RNumb{1}$ рода}

$$\begin{cases}
	u_{tt} = a^2 u_{xx}, \; t > 0, \; x > 0 \\
	\left.
  		\begin{array}{ccc}
    		u|_{t=0} = \varphi(x) \\
    		u_t |_{t=0} = \psi (x)
  		\end{array}
	\right\} \text{ - начальные условия для } x > 0 \\
	u|_{x=0} = 0, \; t > 0
\end{cases}\eqno(*)$$

Нечетно отразим начальные условия:
\begin{multicols}{2}
	$\varphi (x) = 
		\begin{cases}
			\varphi (x), \; x \ge 0 \\
			- \varphi (-x), \; x < 0
		\end{cases}$
	\columnbreak
	$\psi (x) = 
		\begin{cases}
			\psi (x), \; x \ge 0 \\
			- \psi (-x), \; x < 0
		\end{cases}$
\end{multicols}

Т.е. свели задачу к виду:
$$\begin{cases}
	u_{tt} = a^2 u_{xx}, \; t > 0, \; x \in \mathbb{R} \\
	\left.
  		\begin{array}{ccc}
    		u|_{t=0} = \varphi(x) \\
    		u_t |_{t=0} = \psi (x)
  		\end{array}
	\right\}, \; x \in \mathbb{R}
\end{cases}$$

По формуле Даламбера:
$$u(t,x) = \frac{1}{2} \left( \varphi(x+at) + \varphi(x-at) \right) + \frac{1}{2a} \underset{x-at}{\overset{x+at}{\int}} \psi(y) dy$$

Проверим, что $u|_{x=0} = 0$:
$$u|_{x=0} = \frac{1}{2}\left( \varphi(at) + \varphi(-at) \right) + \frac{1}{2a} \underset{-at}{\overset{at}{\int}}\psi(y) dy = 0, \text{ т.к. функции нечетные.}$$

Т.е. если надо решить $*$, то продолжим нечетным образом начальные условия. Тогда функция $\frac{1}{2} \left( \varphi(x+at) + \varphi(x-at) \right) + \frac{1}{2a} \underset{x-at}{\overset{x+at}{\int}} \psi(y) dy$ определена при $x \in \mathbb{R}$, $t>0$ и $u|_{x=0} = 0$. Кроме того, эта функция при $x>0$ удовлетворяет условиям $u|_{t=0} = \varphi(x), \; u_t |_{t=0} = \psi (x)$. Т.е. рассматривая полученную функцию при $x \ge 0, \; t \ge 0$, получим функцию, удовлетворяющую $*$.

$$u(t,x) = \begin{cases}
	\frac{1}{2} \left( \varphi(x+at) + \varphi(x-at) \right) + \frac{1}{2a} \underset{x-at}{\overset{x+at}{\int}} \psi(y) dy, \; x > at \\
	\frac{1}{2} \left( \varphi(x+at) - \varphi(at-x) \right) + \frac{1}{2a} \underset{at-x}{\overset{x+at}{\int}} \psi(y) dy, \; x < at
\end{cases}$$

\textbf{Однородное уравнение с однородным граничным условием $\RNumb{2}$ рода}

$$\begin{cases}
	u_{tt} = a^2 u_{xx}, \; t > 0, \; x > 0 \\
	\left.
  		\begin{array}{ccc}
    		u|_{t=0} = \varphi(x) \\
    		u_t |_{t=0} = \psi (x)
  		\end{array}
	\right\} \text{ - начальные условия для } x > 0 \\
	u_x |_{x=0} = 0, \; t > 0
\end{cases}$$

В этом случае отразим начальные условия четным образом:

\begin{multicols}{2}
	$\varphi (x) = 
		\begin{cases}
			\varphi (x), \; x \ge 0 \\
			\varphi (-x), \; x < 0
		\end{cases}$
	\columnbreak
	$\psi (x) = 
		\begin{cases}
			\psi (x), \; x \ge 0 \\
			\psi (-x), \; x < 0
		\end{cases}$
\end{multicols}

$$\begin{cases}
	u_{tt} = a^2 u_{xx}, \; t > 0, \; x \in \mathbb{R} \\
	\left.
  		\begin{array}{ccc}
    		u|_{t=0} = \varphi(x) \\
    		u_t |_{t=0} = \psi (x)
  		\end{array}
	\right\}, \; x \in \mathbb{R}
\end{cases}$$

По формуле Даламбера:
$$u(t,x) = \frac{1}{2} \left( \varphi(x+at) + \varphi(x-at) \right) + \frac{1}{2a} \underset{x-at}{\overset{x+at}{\int}} \psi(y) dy$$

$$u_x |_{x=0} = \frac{1}{2} \left( \varphi'(at) + \varphi'(-at) \right) + \frac{1}{2} \left( \psi (at) - \psi (-at) \right) = 0$$

\newpage
Аналогично предыдущему получаем:
$$u(t,x) = \begin{cases}
	\frac{1}{2} \left( \varphi(x+at) + \varphi(x-at) \right) + \frac{1}{2a} \underset{x-at}{\overset{x+at}{\int}} \psi(y) dy, \; x > at \\
	\frac{1}{2} \left( \varphi(x+at) - \varphi(at-x) \right) + \frac{1}{2a} \left( \underset{0}{\overset{x+at}{\int}}\psi(y)dy + \underset{0}{\overset{at-x}{\int}}\psi(y)dy \right), \; x < at
\end{cases}$$

\textbf{Однородное уравнение с неоднородным граничным условием $\RNumb{3}$ рода}

$$\begin{cases}
	u_{tt} = a^2 u_{xx}, \; t >0 , \; x >0 \\
	\left.
  		\begin{array}{ccc}
    		u|_{t=0} = \varphi(x) \\
    		u_t |_{t=0} = \psi (x)
  		\end{array}
	\right\}, \; x > 0 \\
	(\alpha u_x + \beta u)|_{x=0} = \mu (t), \; t > 0
\end{cases}$$

В этом случае решение ищется в виде:
$$u(t,x) = \begin{cases}
	f(x+at) + g(x-at), \; x >at \text{ (по формуле Даламбера)} \\
	f(x+at)+h(x-at), x < at
\end{cases}$$

Т.е. по формуле Даламбера найдем решение при $x>at$ и также найдем $f$, остается найти $h$. Воспользуемся начальным граничным условием $(\alpha u_x + \beta u)|_{x=0} = \mu(t)$. Выразим $h(y) = \dots + C$. Чтобы найти $C$, воспользуемся тем, что при $x = at \; g(0) = h(0)$.\\

\textbf{Неоднородное уравнение с граничным условием $\RNumb{3}$ рода}

$$\begin{cases}
	u_{tt} = a^2 u_{xx} + f(t,x), \; t >0 , \; x >0 \\
	\left.
  		\begin{array}{ccc}
    		u|_{t=0} = \varphi(x) \\
    		u_t |_{t=0} = \psi (x)
  		\end{array}
	\right\}, \; x > 0 \\
	(\alpha u_x + \beta u)|_{x=0} = \mu (t), \; t > 0
\end{cases}\eqno(**)$$

Рассмотрим вспомогательную функцию $w(t,x): \; w_{tt} = a^2 w_{xx} + f(t,x)$ ($w$ подбирается так, чтобы это было верно).\\

Пусть теперь $u = w + v$, тогда подставим это в $**$ и получим:

$$\begin{cases}
	v_{tt} = a^2 v_{xx}\\
	v|_{t=0} = \varphi_1 (x) \\
	v_t |_{t=0} = \psi_1 (x) \\
	(\alpha v_x + \beta v)|_{x=0} = \mu_1 (t)
\end{cases}$$

Алгоритм решения данной системы описан ранее.


\chapter{Задача Коши для волнового уравнения в $\mathbb{R}^2$ и $\mathbb{R}^3$ . Энергетическое неравенство. Единственность решения задачи Коши.}
\label{cha:6}

\textbf{Напоминание}

\begin{enumerate}
	\item \blue{формула Гаусса-Остроградского}
		$$\underset{\Omega}{\overset{}{\iint}} \frac{\partial w}{\partial x_i} d \overline{x} = \underset{\partial \Omega}{\overset{}{\oint}} w \cdot n_i \cdot d \sigma$$
	\item \blue{формула интегрирования по частям}\\
		$$\begin{gathered}
			w = f \cdot g, \; \underset{\Omega}{\overset{}{\iint}} (\frac{\partial f}{\partial x_i} g + \frac{\partial g}{\partial x_i} f) d \overline{x} = \underset{\partial \Omega}{\overset{}{\oint}}  f \cdot g \cdot n_i d \sigma \; \Rightarrow \\
			\Rightarrow \; \underset{\Omega}{\overset{}{\iint}} \frac{\partial f}{\partial x_i} g d \overline{x} = \underset{\partial \Omega}{\overset{}{\oint}} f \cdot g \cdot n_i d\sigma - \underset{\Omega}{\overset{}{\iint}} \frac{\partial g}{\partial x_i} \cdot f d \overline{x}
		\end{gathered}$$
	\item \blue{теорема о потоке}
		$$\begin{gathered}
			w = \frac{\partial f}{\partial x_i}, \; \underset{\Omega}{\overset{}{\iint}} \frac{\partial^2 f}{\partial x_i^2} \cdot d \overline{x} = \underset{\partial \Omega}{\overset{}{\int}} \frac{\partial f}{\partial x_i} \cdot n_i d\sigma \text{ - суммируем по } i: \\
			\underset{\Omega}{\overset{}{\iint}} \triangle f d \overline{x} = \underset{\partial \Omega}{\overset{}{\oint}} (\overline{\nabla} f, n) d\sigma = \underset{\partial \Omega}{\overset{}{\oint}} \frac{\partial f}{\partial \overline{n}} d\sigma
		\end{gathered}$$
	\item \blue{производная интеграла по параметру}
		$$F(t) = \underset{a(t)}{\overset{b(t)}{\int}}h(t,x) dx, \; F'(t) = \underset{a(t)}{\overset{b(t)}{\int}}\frac{\partial h}{\partial t} (t,x) dx + h(t, b(t)) \cdot b'(t) - h(t, a(t)) \cdot a'(t)$$
\end{enumerate}

\textbf{Волновое уравнение}
$$u_{tt} = a^2 \triangle_{\overline{x}} u = a^2 (u_{x_1 x_1} + \dots + u_{x_n x_n})$$
Рассмотрим задачу Коши для волнового уравнения:
$$\begin{cases}
	u_{tt} = a^2 \triangle_{\overline{x}} u, \; t > 0, \; x \in \mathbb{R}^n \\
	\left.
  		\begin{array}{ccc}
    		u|_{t=0} = \varphi(x) \\
    		u_t |_{t=0} = \psi (x)
  		\end{array}
	\right\}, \; \overline{x} \in \mathbb{R}^n
\end{cases}$$

Рассмотрим случай \red{n = 2}: 
$$\begin{gathered}
	u_{tt} - a^2 (u_{x_1 x_1} + u_{x_2 x_2}) = 0 \\
	A(\lambda, \mu_1, \mu_2) = \lambda^2 - a^2 (\mu_1^2 + \mu_2^2) \text{ - характеристический многочлен} \\
	\overline{n} = (n_t, n_1, n_2), \; A(\overline{n}) = (n_t)^2 - a^2 (n_1^2 + n_2^2) = 0
\end{gathered}$$
Т.о. характеристической поверхностью является конус:
\begin{figure}[h]
	\begin{multicols}{2}
		\includegraphics[width=70mm]{cha6im1}
		\columnbreak
		\includegraphics[width=110mm]{cha6im2}
	\end{multicols}
\end{figure} 

Получаем уравнение:
$$a^2 (t_0 -t)^2 = x_1^2 + x_2^2, \; a^2 (t_0 - t)^2 - (x_1^2 + x_2^2) = 0 = F(t, x_1, x_2)$$
Нормаль к конусу:
$$\overline{n} = (F_t, F_{x_1}, F_{x_2}) = (2a^2(t- t_0), -2x_1, -2x_2) = 2(a^2(t-t_0), -x_1, -x_2)$$
Получаем уравнение характеристик:
$$a^4 (t-t_0)^2 - a^2 (x_1^2 + x_2^2) = 0$$
\begin{definition}\label{lec:6/def:1}
	\blue{Интегралом энергии} называется величина
	$$E(t) := \frac{1}{2} \underset{K(t)}{\overset{}{\varoiint}}\left[ (u_t)^2 + a^2 \left( (u_{x_1})^2 + (u_{x_2})^2 \right)^2 \right] d \overline{x} = \frac{1}{2} \underset{|\overline{x}| \le a (t_0 - t)}{\overset{}{\varoiint}}\left[ (u_t)^2 + a^2 \left| \overline{\nabla_x} u \right|^2 \right] d \overline{x}$$
\end{definition}

\begin{theorem}[\red{Энергетическое неравенство}]\label{lec:6/the:1}
	Пусть $u(t,x)$ удовлетворяет волновому уравнению в конусе $\Omega$. Тогда: 
	$$\forall t_1, t_2 \in (0, t_0), t_1 < t_2: \; E(t_1) \ge E(t_2)$$
\end{theorem}
\begin{Proof}
	Хотим показать, что $E(t)$ убывает по $t$. Идея доказательства состоит в том, чтобы рассмотреть $E'(t)$ и показать, что она $\le 0$.\\
	Запишем $E(t)$ в следующем виде:
	$$E(t) = \frac{1}{2} \underset{0}{\overset{a(t_0-t)}{\int}}\underset{|\overline{x}| = \tau}{\overset{}{\oint}}\left[ (u_t)^2 + a^2 \left( (u_{x_1})^2 + (u_{x_2})^2 \right)^2 \right] d \sigma d \tau$$
	Найдем производную:
	$$\begin{gathered}
		E'(t) = \frac{2}{2}\underset{K(t)}{\overset{}{\varoiint}} \left[ u_t u_{tt} + a^2 (u_{x_1} u_{x_1 t} + u_{x_2} u_{x_2 t}) \right] d \overline{x} - \frac{a}{2} \underset{|\overline{x}| = a(t_0 - t)}{\overset{}{\varoiint}}\left[ (u_t)^2 + a^2 \left| \overline{\nabla_x} u \right|^2 \right] d \sigma = \\
		= \underset{K(t)}{\overset{}{\varoiint}} u_t \triangle u d \overline{x} + a^2 \Bigg[ \underset{|\overline{x}| = a(t_0 - t)}{\overset{}{\varoiint}} (u_{x_1} u_{x_1} n_1 + u_{x_2} u_{x_2} n_2) d \sigma - \\
		- \underset{K(t)}{\overset{}{\varoiint}} (u_{x_1 x_1} u_t + u_{x_2 x_2} u_t) d \overline{x} \Bigg] - \frac{a}{2} \underset{|\overline{x}| = a(t_0 - t)}{\overset{}{\varoiint}}\left[ (u_t)^2 + a^2 \left| \overline{\nabla_x} u \right|^2 \right] d \sigma \\
		(\text{применили формулу интегрирования по частям для } f = u_t \text{ и } g = u_{x_1})
	\end{gathered}$$
	Таким образом получаем:
	$$\begin{gathered}
		E'(t) = -\frac{a}{2} \underset{|\overline{x}| = a(t_0 - t)}{\overset{}{\varoiint}}\Big[ (u_t)^2 + a^2 (u_{x_1})^2 (n_1^2 + n_2^2) + a^2 (u_{x_2})^2 (n_1^2 + n_2^2) + \\
		+ a^2 (u_{xx})^2 (n_1^2 + n_2^2) - 2 a u_t u_{x_1} n_1 - 2 a u_t u_{x_2} n_2\Big] d \sigma = -\frac{a}{2} \underset{|\overline{x}| = a(t_0 - t)}{\overset{}{\varoiint}}\Big[ a^2 (u_{x_1})^2 n_2^2 + \\
		+ a^2 (u_{x_2})^2 n_1^2 - 2 a^2 u_{x_1} u_{x_2} n_1 n_2 + (u_t - a u_{x_1} n_1 - a u_{x_2} n_2)^2\Big] d \sigma = \\
		= -\frac{a}{2} \underset{|\overline{x}| = a(t_0 - t)}{\overset{}{\varoiint}}\Big[(u_t - a u_{x_1} n_1 - a u_{x_2} n_2)^2 + (a u_{x_1} n_2 - a u_{x_2} n_1)^2 \Big] d \sigma \le 0
	\end{gathered}$$
	Т.е. $E'(t) \le 0$ (не возрастает), что и требовалось доказать.
\end{Proof}

\begin{conseq}[]\label{lec:6/the:1}
	Решение задачи Коши единственно.
\end{conseq}
\begin{Proof}
	Имеем задачу Коши:
	$$\begin{cases}
		u_{tt} = a^2 \triangle u, \; t > 0, \; x \in \mathbb{R}^n \\
		\left.
	  		\begin{array}{ccc}
	    		u|_{t=0} = \varphi(x) \\
	    		u_t |_{t=0} = \psi (x)
	  		\end{array}
		\right\}, \; \overline{x} \in \mathbb{R}^n
	\end{cases}$$
	Предположим, что существует два различных решения $u_1$ и $u_2$. Рассмотрим их разность $z = u_1 - u_2$, для нее задача Коши имеет вид:
 	$$\begin{cases}
		z_{tt} = a^2 \triangle z, \; t > 0, \; z \in \mathbb{R}^n \\
		\left.
	  		\begin{array}{ccc}
	    		z|_{t=0} = 0 = z (0, x_1, x_2) \\
	    		z_t |_{t=0} = 0
	  		\end{array}
		\right\}, \; \overline{x} \in \mathbb{R}^n
	\end{cases}$$
	$$\begin{gathered}
		E(t) = \underset{K(t)}{\overset{}{\varoiint}} \left[ (z_t)^2 + a^2 \left( (z_{x_1})^2 + (z_{x_2})^2 \right)^2 \right] d \overline{x} \\
		E(0) = \underset{K(0)}{\overset{}{\varoiint}} \left[ (z_t)^2|_{t=0} + a^2 \left( (z_{x_1})^2 + (z_{x_2})^2 \right)^2|_{t=0} \right] d \overline{x} = 0 \\
		\text{(из начальных условиях задачи Коши)}
	\end{gathered}$$
	Из энергетического неравенства имеем: $\forall t: E(t) \le E(0) = 0$. Тогда получаем, что $E(t) \equiv 0 \; \Rightarrow \; z_t \equiv z_{x_1} \equiv z_{x_2} \equiv 0 \; \Rightarrow \; z \equiv 0$. Т.к. $z|_{t=0} = 0$, то $z \equiv 0 \; \Rightarrow \; u_1 \equiv u_2$.
\end{Proof}


\chapter{Неотр-ные матрицы. Т-ма Фробениуса-Перрона.}\label{cha:7}

Элементы теории неотрицательных матриц. Теорема Фробениуса-Перрона.

Пусть $A = (a_{ij})_{i, j = 1}^n \in Mat_{n \times n}$.\\

$A$ неотрицательная ($A \ge 0$), если $\forall i, j \; a_{ij} \ge 0$.

$A$ положительна ($A > 0$), если $A \ge 0$ и $A \not = 0$.

$A$ строго положительна ($A >> 0$), если $\forall i, j \; a_{ij} > 0$.\\

\textbf{Свойства}:
\begin{itemize}
	\item[$\bullet$]
		$A \ge 0, \; v \ge 0 \; \Rightarrow \; A v \ge 0$
	\item[$\bullet$]
		$A >> 0, \; v >> 0 \; \Rightarrow \; A v >> 0$
	\item[$\bullet$]
		$A > 0, \; v >> 0 \; \Rightarrow \; A v > 0$
\end{itemize}

\begin{definition}\label{cha:7/def:1}
	Матрица $A$ называется \textit{разложимой}, если $\exists S, T \subset \{1, \dots, n\}: \; S, T \not = \emptyset, \; S \bigcap T = \emptyset, \; S \bigcup T = \{1, \dots, n\}$ и $\forall i \in S, \forall j \in T: a_{ij} = 0$.
\end{definition}

\begin{definition}\label{cha:7/def:2}
	\textit{Перестановка рядов $i$ и $j$} = перестановка строк $i$ и $j$ $+$ перестановка столбцов $i$ и $j$.
\end{definition}

\begin{definition}\label{cha:7/def:3}
	Матрица $A$ \textit{разложима}, если она перестановкой рядов переводится к виду: $\begin{pmatrix}
		A_1 & A_2 \\
		0 & A_4
	\end{pmatrix}$ или $\begin{pmatrix}
		A_1 & 0 \\
		A_3 & A_4
	\end{pmatrix}$, где $A_1, A_4$ - квадртаные матрицы.
\end{definition}

\textbf{Пример}:
\begin{itemize}
	\item[$\bullet$]
		$\begin{pmatrix}
		1 & 0 \\
		0 & 1
	\end{pmatrix}$ - разложима
	\item[$\bullet$]
		$\begin{pmatrix}
		1 & 1 \\
		1 & 1
	\end{pmatrix}$, $\begin{pmatrix}
		0 & 1 \\
		1 & 0
	\end{pmatrix}$ - не разложимы
\end{itemize}

\begin{clair}[]\label{cha:7/clair:1}
	$A$ неразложима, тогда в $A$ нет нулевых строк и столбцов.
\end{clair}
\begin{Proof}
	От противного: пусть $i$-ая строка нулевая: $\forall j \; a_{ij} = 0$. Пусть $S = \{i\}, \; T = \{1, \dots, i-1, i+1, \dots, n\} \; \Rightarrow \; A$ разложима - противоречие. Аналогично для столбцов.
\end{Proof}

\begin{clair}[]\label{cha:7/clair:2}
	$A \ge 0 $ - неразложима, $x >> 0$, тогда $A x >> 0$.
\end{clair}
\begin{Proof}
	В $A$ нет нулевых строк $\Rightarrow \; \forall i \; \exists j_0: a_{i j_0} > 0 \; \Rightarrow \; (A x)_i = \underset{j=1}{\overset{n}{\sum}}a_{ij}x_j \ge a_{i j_0} x_{j_0} > 0$.
\end{Proof}

\begin{clair}[]\label{cha:7/clair:3}
	$A \ge 0$ - неразложима $\Rightarrow \; (I+A)^{n-1} >> 0$.
\end{clair}
\begin{Proof}
	Достаточно доказать, что $\forall x > 0: \; (I+A)^{n-1} \cdot x >> 0$.

	Пусть $x > 0, \; y := (I+A)x$. Если $x >> 0$, то $y = x+Ax \ge x \; \Rightarrow \; y >> 0$ и т.д., получаем требуемое.

	Докажем, что если $x \not\gg 0$, то $\Set{i}{y_i = 0} \subsetneq \Set{i}{x_i = 0}$. Имеем $y = x + A x \ge x \; \Rightarrow \; \Set{i}{y_i = 0} \subseteq \Set{i}{x_i = 0}$.

	Пусть $\Set{i}{y_i = 0} = \Set{i}{x_i = 0} = \{1, \dots, k$ (без ограничения общности). Тогда $x = (\underset{k}{0, \dots, 0}, \underset{n-k}{u})^T, \; u >> 0, \; y = (\underset{k}{0, \dots, 0}, \underset{n-k}{v})^T, \; v >> 0 \; \Rightarrow \; \begin{pmatrix}
		0 \\ \vdots \\ 0 \\ v
	\end{pmatrix} = \begin{pmatrix}
		0 \\ \vdots \\ 0 \\ u
	\end{pmatrix} + \begin{pmatrix}
		A_1 & A_2 \\ A_3 & A_4
	\end{pmatrix}\begin{pmatrix}
		0 \\ \vdots \\ 0 \\ u
	\end{pmatrix} \; \Rightarrow \; A_2 u = 0, u >> 0, \; A_2 \ge 0 \; \Rightarrow \; A_2 = 0$, т.е. разложима - противоречие, значит $x > 0$ и у $x$ не более $n-1$ нулевых компонент. Т.о. $y >> 0$ и $(I+A)^{n-1} \cdot x >> 0$.
\end{Proof}

\begin{conseq}[]\label{cha:7/conseq:1}
	$A \ge 0$ - неразложима $\Rightarrow \; \forall i, j \exists k (i, j) \le n: a_{ij}^{(k)} > 0$, где $A^k = (a_{ij}^{(k)})_{i, j = 1}^{n}$. 
\end{conseq}
\begin{Proof}
	Пусть $i \not = j \; \Rightarrow \; 0 << (I+A)^{n-1} = \underset{k=0}{\overset{n-1}{\sum}}C_{n-1}^k \cdot A^k \; \Rightarrow \; \underset{k=1}{\overset{n-1}{\sum}}a_{ij}^{(k)} > 0 \; \Rightarrow \; \exists k: C_{n-1}^k \cdot a_{ij}^{(k)} > 0 \; \Rightarrow \; a_{ij}^{(k)} > 0$.

	Пусть $i = j \; \Rightarrow \; A \cdot (I+A)^{n-1} >> 0$. Т.к. $A$ неразложима, то $(I+A)^{n-1} >> 0$. Имеем: $A \cdot (I+A)^{n-1} = \underset{k=0}{\overset{n-1}{\sum}}C_{n-1}^k \cdot A^{k+1} = \underset{k=1}{\overset{n}{\sum}}C_{n-1}^{k-1} \cdot A^k >> 0 \; \Rightarrow \; \dots \; \Rightarrow \; \exists k: a_{ij}^{(k)} > 0$. 
\end{Proof}

\textbf{Пример}: $A = \begin{pmatrix}
	0 & 1 \\ 1 & 0
\end{pmatrix}, \; A^{2k} = \begin{pmatrix}
	1 & 0 \\ 0 & 1
\end{pmatrix}, A^{2k+1} = \begin{pmatrix}
	0 & 1 \\ 1 & 0
\end{pmatrix}$.

\begin{theorem}[\red{Перрона-Фробениуса}]\label{cha:7/the:1}
	Пусть $A \ge 0$ - неразложимая матрица, тогда:
	\begin{itemize}
		\item[$1$)]
			$\exists \lambda_A > 0: \forall \lambda$ - с.з. $A: |\lambda| \le \lambda_A$
		\item[$2$)]
			$\lambda_A$ - с.з. $A$ кратности 1, называемое \blue{числом Фробениуса} и $\exists$ соответствующий с.в. $x_A >> 0: A x_A = \lambda_A \cdot x_A$, называемый \blue{вектором Фробениуса}.
	\end{itemize}
\end{theorem}
\begin{Proof}
	Пусть $x > 0$, $r(x) := \underset{i = \overline{1 n}}{\min} (A x)_i = \underset{i = \overline{1 n}}{\min} \; x_i$. Если $x_i = 0$, то считаем, что $\frac{(A x)_i}{x_i} = +\infty$.

	$r'(x) = \sup \Set{\rho}{\rho x \le A x}$. Тогда $r(x) = r'(x)$, т.к.:
	$$\begin{gathered}
		\forall i: r(x) \cdot x_i \le (A x)_i \; \Rightarrow \; r(x) \cdot x \le A x \; \Rightarrow \; r(x) \le r'(x) \\
		r'(x) \cdot x \le A x \; \Rightarrow \; \forall i \; r'(x) \cdot x_i \le (A x)_i \; \Rightarrow \; r'(x) \le \frac{(A x)_i}{x_i} \; \forall i \; \Rightarrow \; r'(x) \le r(x)
	\end{gathered}$$
	Т.о. имеем, что $r(x) = r'(x)$. Пусть $r := \underset{x > 0}{\sup} \; r(x)$, $M := \Set{x > 0}{\underset{i=1}{\overset{n}{\sum}}x_i = 1}$ - замкнутое и ограниченное множество (компакт).

	$$\begin{gathered}
		\forall x >0 \; \forall \lambda > 0 \; r (\lambda x) = r(x) \; \Rightarrow \; r = \underset{x \in M}{\sup} \; r(x) \\
		\text{т.к. } \forall x > 0 \exists \lambda > 0: \lambda x \in M
	\end{gathered}$$

	Пусть $x > 0$ и $Z = (I+A)^{n-1} \cdot x >> 0$. $r(x) \cdot x \le A x$, т.е. $(A - r(x) \cdot I) \cdot x \ge 0$. Значит имеем:
	$$\begin{gathered}
		0 \le (I+A)^{n-1} \cdot (A - r(x) \cdot I) \cdot x = (A - r(x) \cdot I) \cdot (I + A)^{n-1} \cdot x = \\
		= (A - r(x) \cdot I) \cdot Z \; \Rightarrow  \; r(x) \cdot Z \le A z \; \Rightarrow \; r(x) \le r(z)
	\end{gathered}$$
	$N:= \Set{(I+A)^{n-1} \cdot x}{x \in M}, \; \forall z \in N z >> 0$. $N$ - компакт, т.к. $M$ - компакт.

	Функция $r(x)$ непрерывна на $N \subset \mathbb{R}_{++}^{n} = \Set{x}{x >>0}$, тогда по теореме Вейерштрасса достигает своего $max$ и $min$ на компакте:
	$$\exists z \in N: r(z) = \underset{x \in N}{\sup} \; r(x) = \underset{x \in M}{\sup} \; r(x) = \underset{x > 0}{\sup} \; r(x) = r$$

	Имеем: $r = r(z), \; r \cdot z \le A z$. Докажем, что $r \cdot z = A z$. 

	От противного: пусть $r \cdot z < A z$, т.е. $(A - r I)\cdot z > 0$, тогда имеем:
	$$\begin{gathered}
		w := (I+A)^{n-1} \cdot Z \; \Rightarrow \; (A - r I)\cdot w = (A -r I) \cdot (I+A)^{n-1} \cdot z = \\
		= \underbrace{(I+A)^{n-1}}_{>>0} \underbrace{(A -r I) \cdot z}_{>0} >> 0 \; \Rightarrow \; r w \le A w \; \Rightarrow \;  r(w) > r \text{ - противоречие (} r \text{ - } \sup \text{)}
	\end{gathered}$$
	Значит имеем: $r\cdot z = A z$, т.е. $Z$ - с.в. $A$  с с.з. $r$.\\

	Теперь докажем, что $\forall$ с.з. $\lambda: |\lambda| \le r$.

	Пусть $A y = \lambda y, \lambda \in \mathbb{C}, \; y \in\mathbb{C}^n, y \not =0, \; |y| = \begin{pmatrix}
		|y_1| \\ \vdots \\ |y_n|
	\end{pmatrix}$.
	$$|\lambda| \cdot |y| = |\lambda y| = |Ay| \le |A| \cdot |y| = A \cdot |y| \; \Rightarrow \; |\lambda| \le r(|y|) \le r$$
	Докажем, что $z$ - единственный с точностью до пропорциональности с.в. с с.з. $r$. Пусть $A y = r y \; \Rightarrow \; A |y| \ge |A y| = r |y| \; \Rightarrow \; r \le r(|y|) \le r \; \Rightarrow \; r(|y|) = r$ и $A |y| = r|y|, \; |y| > 0$. Тогда:
	$$0 << (I+A)^{n-1}\cdot |y| = (1+r)^{n-1} \cdot |y| \; \Rightarrow \; |y| >> 0 \; \Rightarrow \; |y_i| > 0, \; y_i \not = 0$$
	Т.о. в $y$ нет нулевых координат.

	Пусть $y_1, y_2$ - с.в. с с.з. $r$ и они не пропорциональны, тогда $\exists \lambda, \mu \in \mathbb{C}: \; y = \lambda y_1 + \mu y_2 \not = 0, \; \exists i: y_i = 0$ - противоречие. Т.о. все с.в. с с.з. $r$ пропорциональны.
\end{Proof}

\begin{theorem}[]\label{cha:7/the:2}
	Пусть $A \ge 0$, тогда $\exists \lambda_A \ge 0$:
	\begin{itemize}
		\item[$1$)]
			$\forall$ с.з. $\lambda: |\lambda| \le \lambda_A$
		\item[$2$)]
			$\lambda_A$ - с.з. $A$
		\item[$3$)]
			$\exists x_A > 0: A x_A = \lambda_A \cdot x_A$
	\end{itemize}
\end{theorem}
\begin{Proof}
	Рассмотрим $B = \begin{pmatrix}
		1 & 1 \\ 1 & 1
	\end{pmatrix}$ и последовательность $A_n = A + \frac{1}{n} b$. $A \ge 0, B >>0 \; \Rightarrow \; \forall n A_n >> 0$ - неразложимая. $\underset{n \to \infty}{\lim} A_n = A \; \Rightarrow$ последовательность $\{A_n\}$ ограничена, значит, $||A_n||$ - ограничено.

	$\lambda_n = \lambda_{A_n} \le ||A_n|| \; \Rightarrow \; \lambda_n$ - ограниченная последовательность.

	Пусть $x_n$ - вектор Фробениуса, $x_n = x_{A_n}$. Можно считать, что $\forall n: ||x_n|| = 1$. Из того, что $\lambda_n$ и $x_n$ ограничены, то $\exists n_k$ - подпоследовательность: $\exists \underset{k \to \infty}{\lim} \lambda_{n_k} = \lambda \ge 0, \; \exists \underset{k \to \infty}{\lim} x_{n_k} = x \ge 0, ||x||=1 \; \Rightarrow \; x > 0$.

	$\forall n \; A_{n_k} x_{n_k} = \lambda_{n_k} \cdot x_{n_k} \; \Rightarrow \; Ax = \lambda x$ при $k \to \infty$.

	Пусть $\lambda'$ - с.з. $A$ и $y \not = 0$ - соответствующий с.в.: $Ay = \lambda' y \; \Rightarrow \; |\lambda'|\cdot |y| = |A y| \le |A| \cdot |y| = A |y| \le A |y| + \frac{1}{n} B |y| = A_n |y| \; \forall n$.

	Имеем: $|\lambda'| \le r_{A_n} (|y|) \le \lambda_n \; \Rightarrow \; \forall n_k: |\lambda'| \le \lambda_{n_k} \; \Rightarrow \; |\lambda'| \le \lambda$ при $k \to \infty$. Имеем $\lambda_A = \lambda, \; x_a = x$.
\end{Proof}



















\chapter{Приложение теоремы фон Неймана к теории конечных антагонистических
игр}
\label{cha:8}

\epigraph{
	\textit{В этой игре, по ее бесплотности и страшности, действительно было что-то адово, аидово.}}
{-- Цветаева М.И.}

В большинстве конечных игр двух лиц седловая точка отсутствует. Применение оптимальных стратегия гарантирует выигрыш, равный a. Возникает вопрос: нельзя ли гарантировать \textit{средний} выигрыш, больший a, если применять не одну, как говорят, \textit{чистую} стратегию, а чередовать стратегии по некоторому вероятностному закону? Таким комбинированные стратегии в теории игр называются \textit{смешанными} стратегиями. Ясно, что чистая стратегия является частным случаем смешанной, когда вероятность выбора одной стратегии равна 1, а остальных — 0. Сейчас мы увидим, что ответ на поставленный вопрос положительный.

Смешанную стратегию первого игрока $\alpha$ будем обозначать строкой $p =$ \\
$(p_1, \dots, p_n)$, где $p_i$ – вероятность выбора стратегии $\alpha_i$. Тогда $p_i \ge 0$ и $p_1 + \dots + p_n = 1$. Аналогично, смешанную стратегию второго игрока $\beta$ обозначим строкой $q = (q_1, \dots, q_m)$, где $q_i$ – вероятность выбора стратегии $\beta_i$, то есть $q_i \ge 0$ и $q_1 + \dots + q_m = 1$. Строки p, q можно рассматривать как координаты точек соответственно n-мерного и m-мерного пространств. Ограничения на p, q показывают, что допустимые значения p,q пробегают симплексы P, Q соответствующих пространств. Выбор i-ой стратегии игроком $\alpha$ и j-ой стратегии игроком $\beta$ — независимые события. Следовательно, вероятность их наступления равна $p_i q_j$. Поэтому математическое ожидание выигрыша при применении пары смешанных стратегий p, q равна числу 
$$\delta(p, q) = \underset{i=1}{\overset{n}{\sum}}\underset{j=1}{\overset{m}{\sum}}a_{ij} p_i q_j = p A^t q \eqno(15)$$
где $A = (a_{ij})$ – матрица игры размера $n \times m$. Величина $\delta(p, q)$ линейна по каждому из своих аргументов p,q. Важно отметить, что допустимые значения вектора $A^t q$ являются выпуклым многогранником как образ симплекса при аффинном отображении. Дело в том, что при аффинном отображении выпуклая линейная оболочка векторов переходит в выпуклую линейную оболочку их образов.

Теперь определим гарантированный средний выигрыш и проигрыш игроков $\alpha$ и $\beta$ и их оптимальные смешанные стратегии. Если игрок $\alpha$ применяет стратегию p, то игрок $\beta$ выбирает смешанную стратегию, реализующую $\psi(p) = \underset{q \in Q} \delta(p, q)$. Тогда гарантированный средний выигрыш игрока $\alpha$ равен $\underset{p \in P}{\max} \; \underset{q \in Q}{\min} \delta (p, q)$. Из симметричных соображений, если игрок $\beta$ выбирает стратегию q, то игрок $\alpha$ ответит на нее стратегией, реализующей $\varphi(q) = \underset{p \in P} \delta(p, q)$. Средний проигрыш игрока $\beta$ не превосходит $\underset{q \in Q}{\min} \; \underset{p \in P}{\max} \delta (p, q)$.

По теореме фон Неймана оба числа существуют и равны между собой:
$$ \underset{p \in P}{\max} \; \underset{q \in Q}{\min} \delta (p,q) = \underset{q \in Q}{\min} \; \underset{p \in P}{\max} \delta (p,q) \eqno(16)$$
Пусть обе части (15) реализуются на паре смешанных стратегий $(p^*,q^*)$, и $\delta^* = \delta(p^*,q^*)$ – число, равное обеим частям (15). Оптимальные смешанные стратегии $p^*$, $q^*$ обладают необходимым свойством устойчивости: при любом одностороннем отклонении от оптимальной стратегии выигрыш меняется в направлении, невыгодном отклонившейся стороне. Действительно, переписав неравенства 2) из теоремы фон Неймана, получим соответственно:
$$\delta (p^*, q^*) \ge \delta (p, q^*), \; \delta (p^*, q) \ge \delta (p, q^*)$$

Говорят, что игра имеет решение, если существует пара смешанных стратегий, являющаяся седловой точкой и обладающая сформулированным выше условием устойчивости. Только что доказанные утверждения могут быть коротко сформулированы в виде следующей основополагающей в теории игр теоремы.

\begin{theorem}[]\label{cha:8/the:1}
	Каждая конечная игра двух лиц имеет решение в области смешанных стратегий.
\end{theorem}



\chapter{Устойчивость замкнутой модели Леонтьева.}\label{cha:9}

$$\begin{cases}
	A \pi = \pi \\
	\pi \geq 0
\end{cases} \text{-- замкнутая модель Леонтьева (} \forall \; i: \; \sum\limits_{j = 1}^n a_{ij} = 1 \text{).}$$

Пусть $A \geq 0, \; r_i = \sum\limits_{j = 1}^n a_{ij}, \; s_j = \sum\limits_{i = 1}^n a_{ij}, \; r = \underset{i}{min}r_i, \; R = \underset{i}{max}r_i, \; s = \underset{i}{min}s_i, \; S = \underset{i}{max}s_i$

\begin{clair}
	$A \geq 0 \Rightarrow r \leq \lambda_A \leq R, s \leq \lambda_A \leq S.$
\end{clair}

\begin{conseq}
	$A \geq 0$ -- замкнутая, тогда $\lambda_A = 1.$
\end{conseq}

\begin{definition}
	$A > 0$ -- неразложимая, $\lambda_A = 1$ (иначе $A = \dfrac{1}{\lambda_A}A$). $p_A$ -- левый Фробениусов собственный вектор: $(p_A, p_A) = 1.$
\end{definition}

\begin{definition}
	Норма на $\mathbb{R}^n: \; \| x\|_A = (p_A, |x|), \; |x| = (|x_1|, \ldots, |x_n|).$
\end{definition}

\begin{definition}
	A -- устойчива, если $\forall \; x \; \exists \lim\limits_{k \to \infty}A^k x$
\end{definition}

\begin{remark}
	Выберем правый Фробениусов собственный вектор: $\| x_A\|_A = 1. \; \| Ax\|_A \leq \| x_A\|_A$ и, если $x \geq 0,$ то $\| Ax\|_A = \| x_A\|_A \; (p_A |Ax| \leq p_A A |x| = p_A |x| = \| x \|_A, \; p_A |Ax| = p_A A x = p_A x = p_A |x| = \| x\|_A. )$
\end{remark}

\begin{clair}
	$x \geq 0, $ если $\exists \lim\limits_{k \to \infty}A^k x,$ то $z = \lim\limits_{k \to \infty}A^kx = \mu x_A, \; \mu = \| x\|_A.$
\end{clair}

\begin{definition}[Импримитивная(циклическая) матрица]
	Неразложимая матрица А называется импримитивной(циклической), если $\exists$ разбиение $\{ 1, \ldots, n\} = S_0 \bigsqcup \ldots \bigsqcup S_{m - 1}, \; S_i \cap S_j = \emptyset \; i \neq j, \; \forall i \; S_i \neq \emptyset, $ при этом $a_{ij} > 0 \Rightarrow$:

	$$\begin{cases}
		i \in S_r, \; j \in S_{r - 1}, \; 1 \geq r \geq m - 2 \\
		i \in S_0, \; j \in S_{m - 1}
	\end{cases}$$

	То есть, матрица импримитивна, если одновременной перестановкой столбцов и строк она приводится к виду: 

	\begin{gather*}
		\begin{pmatrix}
		  0 & \ldots & 0 & A_{m-1}\\
		  A_0 & 0 & \ldots & 0\\
		  0 & A_1 & \ldots & 0\\
		  \ldots & \ldots & \ldots & 0\\
		  0 & \ldots & 0 &  A_{m-2}\\
		\end{pmatrix}
	\end{gather*}

	Иначе А -- примитивная.
\end{definition}

\begin{example}
		\begin{gather*}
			\begin{pmatrix}
			  0 & 1\\
			  1 & 0 \\
			\end{pmatrix} \text{--циклическая, неразложимая.}
			\begin{pmatrix}
			  0 & 1\\
			  1 & 1 \\
			\end{pmatrix} \text{--примитивная.}
			\begin{pmatrix}
			  1 & 1\\
			  0 & 1 \\
			\end{pmatrix} \text{--разложимая.}
		\end{gather*}
\end{example}

\begin{theorem}
	Неразложимая матрица $A \geq 0 устойчива, \; \lambda_A = 1 \Longleftrightarrow $ A -- примитивна.
\end{theorem}

\begin{lemma}
	А -- примитивная, тогда у некоторой степени А первая строка положительная: $\exists k: \; \forall \; j a_{1j}^k > 0, \; A^k = (a_{ij}^k)$.
\end{lemma}

\begin{lemma}
	Если матрица $A^k$ устойчива при некотором k, то А -- устойчива.
\end{lemma}

\begin{definition}[Оператор сжатия]
	Оператор Р действует на линейном нормированном пространстве как оператор сжатия, если $\exists \gamma: \; 0 < \gamma < 1: \; \forall \; v \in L: \; \| Pv\| \leq \gamma \| v\|, \; \gamma $ -- коэффициент сжатия. 
\end{definition}

\begin{lemma}
	$L_A:= Ann(p_A) = \{ v | (v, p_A) = 0\}.$ Если оператор А действует на $L_A$ как оператор сжатия, то А -- устойчива.
\end{lemma}

\begin{lemma}
	Если неразложимая матрица А с $\lambda_A = 1$ имеет положительную строку, то оператор А действует на пространстве $L_A:= Ann(p_A) = \{ v | (v, p_A) = 0\}$ как оператор сжатия.
\end{lemma}

\begin{conseq}
	Если неразложимая матрица А с $\lambda_A = 1$ имеет положительную строку, то A -- устойчива.
\end{conseq}

\begin{theorem}
	A > 0 -- неразложимая с $\lambda_A = 1$, тогда А -- устойчивая $ \Longleftrightarrow \forall \; \lambda \in Spec(A) \setminus 1$ выполнено $|\lambda| < 1.$
\end{theorem}
\section{Инвариантная мера. Мера с гладкой плотностью. Плотность при замене координат. Теорема Лиувилля об инвариантной мере. Построение инвариантной меры на многообразии уровней первых интегралов – локально (Существование инвариантной меры у ограничения системы на инвариантное многообразие.)}\label{chasec10}



\newpage
\chapter{Задача Коши для уравнения теплопроводности. Принцип максимума в неограниченной области. Единственность решения задачи Коши в классе ограниченных функций.}
\label{cha:11}

Задача Коши для уравнения теплопроводности:
$$\begin{cases}
	u_{t} = a^2 u_{xx}, \; t > 0, \; x \in \mathbb{R}\\
	u|_{t = 0} = \varphi (x) \\
	|u| \leq C, \; t \geq 0, \; x \in \mathbb{R}.
\end{cases}$$

\begin{theorem}[\red{Принцип максимума в неограниченной области}] 
Пусть $ m = \underset{\mathbb{R}}{min(\varphi(x))}, \; M = \underset{\mathbb{R}}{max(\varphi(x))}$. Тогда $m \leq u(t, x) \leq M, \; t > 0, 
\; x \in \mathbb{R}.$
\end{theorem}

\begin{Proof}
Докажем для максимума (для минимума аналогично, но с другими знаками).\\

Будем доказывать, что $ M - u(t_0, x_0) \ge 0. $ Введем вспомогательную функцию $\nu(t,x)$, удовлетворяющую следующим условиям:
$$\begin{gathered}
	\nu(t, x) \ge 0, \; \nu_t = a^2\nu_{xx} \; \Rightarrow \; 
	\nu(t, x) = 2a^2t + x^2 \ge 0 , \;\; 
	\nu_t - a^2\nu_{xx} = 2a^2 - 2a^2 = 0
\end{gathered}$$
Рассмотрим следующую функцию:
$$u_\varepsilon(t, x) = M - u(t, x ) + \varepsilon\dfrac{v(t,x)}{\nu(t_0, x_0)}$$
Для нее задача Коши имеет вид:
$$\begin{cases}
	u_\varepsilon(t, x) = M - u(t, x ) + \varepsilon\dfrac{v(t,x)}{\nu(t_0, x_0)} \\
	u_\varepsilon|_{t = 0} = \underset{\ge 0}{\underbrace{M - \varphi(x)}} + \varepsilon\dfrac{x^2}{\nu(t_0, x_0)} \ge 0 \\
	u_\varepsilon|_{x = \pm R} = M - u(t, \pm R) + \varepsilon\dfrac{2a^2t + R^2}{\nu(t_0, x_0)} \ge \underbrace{M - u(t, \pm R)}_{\ge C_1, \text{ т.к. } |u| \leq C} + \varepsilon\dfrac{R^2}{\nu(t_0, x_0)} \geq 0 \;(R \rightarrow \infty)
\end{cases}$$

\begin{center}
\includegraphics[scale=0.4]{cha11im1}
\end{center}

По принципу максимума на стакане $ u_\varepsilon \geq 0 $ в прямоугольнике. Значит $u_\varepsilon(t_0, x_0) \ge 0$, тогда $M - u(t_0, x_0) + \varepsilon \ge 0$, значит $u(t_0, x_0) \le M + \varepsilon$, т.е. $u_(t_0, x_0) \le M$ при $\varepsilon \longrightarrow 0$.

\end{Proof}

\begin{conseq}
Ограниченное решение задачи Коши единственно.
\end{conseq}

\begin{proof}
Допустим, что существуют два разные решения $u_1 \not = u_2$. Рассмотрим их разность $ z = u_1 - u_2 $. Запишем для $z$ задачу Коши:
$$\begin{cases}
	z_{t} = a^2 u_{xx}, \; t > 0, \; x \in \mathbb{R}\\
	z|_{t = 0} = 0 \\
    |z| \leq \tilde{C}, \; t \geq 0, \; x \in \mathbb{R}.
\end{cases}$$

По принципу максимума получаем $ z \equiv 0$, откуда $u_1 = u_2$.
\end{proof}



\section{Теорема Пуанкаре о возвращении.}\label{chasec12}



\newpage
\chapter{Модель Гейла сбалансированного роста. Существование состояния равновесия.}\label{cha:13}

$z:= (x,y) \in \mathbb{R}^{2n}$

\begin{definition}[Модель Гейла]
	Модель Гейла -- подмножество $Z \subset \mathbb{R}^{2n}_+:$
	\begin{enumerate}
		\item $Z$ -- выпуклый замкнутый конус
		\item если $(0,y) \in Z \Rightarrow y = 0$
		\item $\forall \; i: \; \exists (x,y) \in Z: \; y_i > 0 \;\; (3'. \exists (x,y)\in Z: \; y \gg 0)$
	\end{enumerate}
\end{definition}

\begin{itemize}
	\item $z = (x,y) \in Z$ -- производственный процесс
	\item x -- вектор затрат, y -- вектор выпуска
\end{itemize}

\begin{clair}
	(A, B) -- модель Неймана, в А нет нулевых строк (все товары производятся), тогда $Z = \{ (Au, Bu) | u \geq 0\}$ -- модель Гейла.
\end{clair}

\begin{definition}[Траектория(план)]
	Модель Гейла с началом $y_0$ -- последовательность $z_t = (x_{t-1}, y_t) \in Z, \; t \in \mathbb{N}, \; x_t \leq y_t.$
\end{definition}

\begin{definition}[Траектория цен]
	Последовательность $p_t \in \mathbb{R}_+^n, \; t \in \mathbb{N}_0: \; \forall \; (x,y) \in Z: \; p_{t-1}x \geq p_t y. \; \pi_t(\xi) := p_t y_t, \; \xi = \{ z_t\}, \; \pi_t(\xi)$ монотонно убывает.
\end{definition}

\begin{definition}[Состояние равновесия]
	Тройка $(\alpha, \vec{z}, \vec{p}), \; \alpha > 0, \; \vec{z} \in Z, \; \vec{p} > 0,$ -- состояние равновесия, если:
	\begin{enumerate}
		\item $\alpha \vec{x} \leq \vec{y}$
		\item $\forall \; (x, y) \in Z: \; \alpha p x \geq py$.
	\end{enumerate}

	Если (p, y) > 0, то положение равновесия невырожденное, $\alpha$ -- темп роста.
\end{definition}

\begin{theorem}
	У $\forall$ модели Гейла $\exists$ состояние равновесия.
\end{theorem}

\begin{proof}
	$\forall \; z \in Z$ определим технологический темп роста процесса $z = (x, y): \; \alpha(z) = \underset{\alpha}{max}\{\alpha | \alpha x \leq y \} \Rightarrow \alpha(z) \geq 0, \; \alpha(\lambda z) = \alpha(z) \; \forall \; \lambda > 0.$ Обозначим $\lambda_N = \underset{z \in Z \setminus \{ 0\} }{sup} \alpha(z)$ -- число Неймана модели Гейла. Покажем, что $\lambda_N < \infty.$

	Пусть $\lambda_N = \infty \Rightarrow \exists z_n = (x_n, y_n) \in Z: \; \alpha(z_n) > n, \; \| y_n\| = 1 \Rightarrow \| x_n\| \leq \dfrac{\| y_n\|}{\alpha(z_n)} \Rightarrow \| x_n \| \underset{n \to \infty}{\to} 0.$

	С другой стороны, $\{ y_n\}$ -- ограничена, тогда $\exists$ сходящаяся подпоследовательность $y_{n_k} \to y: \; \| y\| = 1.$

	Множество $Z$ замкнуто, тогда $z_{n_k} \underset{k \to \infty}{\to} (0,y) \in Z$ -- противоречие с пунктом 2) определения множеста Гейла, тогда $\alpha_N < \infty.$

	Аналогично доказывается, что $\exists z_n \to \vec{z}\in Z: \; \alpha(z) \to \alpha_N \Rightarrow \alpha(\vec{z}) = \alpha_N.$ Рассмотрим $U = \{ y - \alpha_N x | (x,y) \in Z\}, \; \mathbb{R}^n_{++} = \{ x \gg 0\} \Rightarrow U \cup \mathbb{R}^n_{++} = \emptyset,$ иначе $\exists (x,y)\in Z: \; y - \alpha_N x \gg 0 \Rightarrow \alpha(z) > \alpha_N. \; U, \; \mathbb{R}^n_{++}$ -- выпуклые, тогда по теореме отделимости $\exists p \neq 0: \; \forall \; u \in U, \; \forall \; v \in \mathbb{R}^n_{++}: \; pu \leq pv.$ Так как $0 \in U \Rightarrow pv \geq 0 \Rightarrow p > 0.$

	$\exists v_n \to 0 \Rightarrow pu \leq 0 \; \forall u \Rightarrow \alpha_N px \geq py \; \forall (x, y) \in Z \Rightarrow (\alpha_N, \vec{z}, p) $ -- состояние равновесия.
\end{proof} 

\begin{remark}
	Состояние равновесия $(\alpha, \vec{z}, p)$ может быть вырожденным.
\end{remark}

\begin{clair}
	В модели Гейла может быть не более n темпов роста.
\end{clair}
	
\begin{proof}
	$\forall \; \alpha > 0$ обозначим $Z(\alpha) = \{ z = (x, y) \in Z | \alpha(z) \geq \alpha\}, \; l(z) = \{ i | y_i > 0\}, \; \vec{z}(\alpha)$ -- вектор из Z, имеющий наибольшее количество ненулевых компонент, а $n(\alpha) = \#l(\vec{z}(\alpha)).$

	Тогда $n(\alpha)$ -- корректно определено, так как $Z(\alpha)$ -- выпуклый конус. Пусть $\alpha_1 < \alpha_2 \Rightarrow Z(\alpha_1) \subset Z(\alpha_2), \; n(\alpha_1) \geq n(\alpha_2).$ Пусть $\alpha_1, \alpha_2$ -- темпы роста и $(\alpha_i, \vec{z}_i, p_i)$ -- состояния равновесия i = 1,2. Можно считать, что $z_i$ имеет наибольшее число ненулевых компонент. Пусть $n(\alpha_1) = n(\alpha_2) \Rightarrow l(z_1) = l(z_2), \text{ и } \exists \gamma > 0: \; y_1 \leq \gamma y_2 \Rightarrow \gamma p_1 y_2 \geq p_1 y_1 \Rightarrow p_1 y_2 > 0 \Rightarrow \alpha_1 p_1 x_2 \geq p_1 y_2$ (по определению p) $\Rightarrow \alpha_2 p_2 \leq y_2  \Rightarrow \alpha_2 p_1 x_2 \leq p_1 y_2 \leq \alpha_1 p_1 x_2, $ но $p_1 y_2 > 0  \Rightarrow p_1 x_1 > 0  \Rightarrow \alpha_2 \leq \alpha_1$ -- противоречие.
\end{proof}

\begin{definition}[Число Фробениуса модели Гейла]
	Обозначим $\alpha'(p) = \inf \{ \alpha | \alpha p x  \geq p y \; \forall \; (x, y) \in Z\}.$ Тогда $\alpha_F = \underset{p > 0}{\inf}\alpha'(p)$ -- число Фробениуса модели Гейла.

	Отметим, что $\exists \vec{p} > 0: \; \alpha'(\vec{p}) = \alpha_F.$
\end{definition}

\begin{clair}
	\begin{enumerate}
		\item $\alpha_F \leq \alpha_N$.
		\item $\forall \; \alpha \in [\alpha_F, \alpha_N] \; \exists $ состояние равновесия $(\alpha, z, p): \; \alpha'(p) \leq \alpha \leq \alpha(z)$.
		\item Если $(\alpha, z, p)$ невырожденно, то $\alpha = \alpha(z) = \alpha'(p).$
	\end{enumerate}
\end{clair}

\begin{remark}
	Существуют модели Гейла без (ненулевых) темпов роста.
\end{remark}
\section{Инвариантная мера уравнений Эйлера-Пуассона и интегрируемость в квадратурах. Понятие о трех классических случаях интегрируемости Эйлера, Лагранжа и Ковалевской.}\label{chasec14}



\newpage
\chapter{Формулы Грина.}
\label{cha:15}

\begin{theorem}[\blue{Формула Гаусса - Остроградского}]
	Для функции $w$ имеем следующую формулу:
	\begin{multicols}{2}

			\includegraphics[scale = 0.3]{cha15im1}
		\columnbreak

			\hfill \break
			$$\underset{\Omega}{\int\int} \dfrac{\partial \omega}{\partial x_i}d\bar{x} = \oint\limits_{\partial \Omega} \omega n_i d\sigma$$
	\end{multicols}
\end{theorem}

\begin{theorem}[\blue{Многомерная формула интегрирования по частям}]
	Для функции $ \omega = uv $ имеем следующую формулу:
	$$\begin{gathered}
		\underset{\Omega}{\int\int} u \dfrac{\partial v}{\partial x_i}d\bar{x} = \oint\limits_{\partial \Omega} u v n_i d\sigma - \underset{\Omega}{\int\int} v \dfrac{\partial u}{\partial x_i}d\bar{x}.
	\end{gathered}$$
\end{theorem}

\begin{theorem}[\blue{I формула Грина}]
	Для функции $ \omega = u \dfrac{\partial v}{\partial x_i} $ имеем следующую формулу:
	$$\begin{gathered}
		\underset{\Omega}{\int\int} u \dfrac{\partial^2 v}{\partial x^2_i}d\bar{x} = \oint\limits_{\partial \Omega} u \dfrac{\partial v}{\partial x_i} n_i d\sigma - \underset{\Omega}{\int\int} \dfrac{\partial u}{\partial x_i} \dfrac{\partial v}{\partial x_i} d\bar{x} \mid \sum\limits_{i = 1}^k \Longrightarrow \\
		\Longrightarrow \underset{\Omega}{\int\int} u \Delta v d \bar{x} = \oint\limits_{\partial \Omega} u \dfrac{\partial v}{\partial \bar{n}}d\sigma - \underset{\Omega}{\int\int} (\bar{\nabla}u, \bar{\nabla}v)
		d\bar{x}
	\end{gathered}$$
\end{theorem}\newpage

\begin{theorem}[\blue{II формула Грина}]
$$
\begin{gathered}
\underset{\Omega}{\int\int} v \Delta u d\bar{x} = \oint\limits_{\partial \Omega} v \dfrac{\partial u}{\partial \bar{n}} d\sigma - \underset{\Omega}{\int\int} (\bar{\nabla}v, \bar{\nabla}u)d\bar{x} \\
- \\
\underset{\Omega}{\int\int} u \Delta v d \bar{x} = \oint\limits_{\partial \Omega} u \dfrac{\partial v}{\partial \bar{n}}d\sigma - \underset{\Omega}{\int\int} (\bar{\nabla}u, \bar{\nabla}v)d\bar{x} \\
= \\
\underset{\Omega}{\int\int} (u \Delta v - v \Delta u)d\bar{x} = \oint\limits_{\partial \Omega} (u \dfrac{\partial v}{\partial \bar{n}} - v \dfrac{\partial u}{\partial \bar{n}})d\sigma.
\end{gathered}
$$
\end{theorem}

\begin{theorem}[\blue{Теорема о потоке}]
	Если $ \omega = \dfrac{\partial u}{\partial x_i}$, то справедлива формула:
	$$\begin{gathered}
		\underset{\Omega}{\int\int} \Delta u d\bar{x} = \oint\limits_{\partial \Omega} \dfrac{\partial u}{\partial \bar{n}}d\sigma.
	\end{gathered}$$
\end{theorem}
\chapter{Невырожденные состояния равновесия в модели Неймана.}\label{cha:16}

Пусть (A, B) -- модель Неймана, в А нет нулевых столбцов, в В -- нулевых строк.

\begin{sign}
	$\lambda_N := \inf \{ \lambda | \exists u > 0: \; (A - \lambda B)u \leq 0\}$ -- число Неймана модели (A, B).

	$\lambda_F := \sup \{ \lambda | \exists p > 0: \; p(A - \lambda B) \leq 0\}$ -- число Фробениуса модели (A, B).
\end{sign}

\begin{theorem}
	$A \geq 0, \; B \geq 0, $ в А нет нулевых столбцов, в В -- нулевых строк $\Rightarrow $ в соответствующей модели Неймана $\exists$ невырожденное положение равновесия.
\end{theorem}

\begin{definition}[Продуктивная модель]
	(A, B) -- продуктивна, если $\forall \; c \geq 0 \; \exists x \geq 0: \; (B - A)x \geq c.$
\end{definition}

\begin{theorem}
	(A, B) -- продуктивна $\Longleftrightarrow \lambda_F < 1.$ 
\end{theorem}





\chapter{Функция Грина оператора Лапласа, ее симметрия. Представление решения задачи Дирихле через функцию Грина. Метод отражений. Метод конформных отображений.}
\label{cha:17}

\section*{Функция Грина оператора Лапласа, ее симметрия.}

Рассмотрим фундаментальное решение оператора Лапласса:
$$ \triangle \varepsilon (x) = \delta (x) , \; \varepsilon (x) = 
\begin{cases}
	\displaystyle \frac{1}{2 \pi} \ln |\vec{x}|, \; \vec{x} \in \mathbb{R}^2\\
	\displaystyle - \frac{1}{4 \pi |\vec{x}|}, \; \vec{x} \in \mathbb{R}^3
\end{cases}$$

Рассмотрим задачу Дирихле: 
$\begin{cases}
	\triangle u = f (x) , x \in \Omega\\
	u|_{x \in \partial \Omega} = h(x)
\end{cases}$

\begin{definition}
	\red{Функция Грина} $ G(x, y) $ задачи Дирихле для области $ \Omega $ имеет вид:
	$$\begin{cases}
		\Delta_x G(x, y) = \delta(x - y), \; x \in \Omega \\
		G(x, y)|_{x \in \Omega} = 0
	\end{cases} \forall \; y \in \Omega$$
\end{definition}

\begin{definition}[\blue{эквивалентное определение функции Грина}]
	\red{Функция Грина} представляется в виде $ G(x,y) = \varepsilon(x - y) + g(x, y) $, где:
	$$
	\begin{cases}
		\Delta_x g(x, y) = \delta(x - y), \; x \in \Omega \\
		g(x, y)|_{x \in \Omega} = -\varepsilon(x - y)
	\end{cases} \; \forall \; y \in \Omega
	$$
\end{definition}\newpage

\begin{theorem}[\red{Симметрия функции Грина}]
	$G(x, y) = G(y, x)$
\end{theorem}
\begin{Proof}
	\begin{multicols}{2}
		\includegraphics[scale=0.25]{cha17im1}
		\columnbreak
		$$\begin{gathered}
			\\
			\text{Рассмотрим две функции:}\\
			G_1 = G(x, y_1), \;  G_2 = G(x, y_2) \\ \\
			\text{Рассмотрим область:}\\
			\Omega_1 = \Omega \setminus (\left\{ {|x - y_1| < \alpha} \right\} \cup \left\{ {|x - y_2| < \alpha} \right\})
		\end{gathered}$$
	\end{multicols}

	Применим 2-ую формула Грина для $G_1$, $G_2$ в области $\Omega_1$:
	$$\begin{gathered}
		0 = \iint \limits_{\Omega_1} (G_1 \underbrace{\triangle_x G_2}_{ = 0} - G_2 \underbrace{\triangle_x G_1}_{ = 0}) d \vec{x} = 0 = \oint \limits_{\partial \Omega_1} \left(G_1 \frac{\partial G_2}{\partial n_x} - G_2 \frac{\partial G_1}{\partial n_x}\right) d \sigma = \\
		= \underbrace{\oint \limits_{\partial \Omega}\left(G_1 \frac{\partial G_2}{\partial n_x} - G_2 \frac{\partial G_1}{\partial n_x}\right) d \sigma}_{I_1} + \underbrace{\oint \limits_{S_1}\left(G_1 \frac{\partial G_2}{\partial n_x} - G_2 \frac{\partial G_1}{\partial n_x}\right) d \sigma}_{I_2} + \underbrace{\oint \limits_{S_2}\left(G_1 \frac{\partial G_2}{\partial n_x} - G_2 \frac{\partial G_1}{\partial n_x}\right) d \sigma }_{I_3}\\
		\left( S_1 \text{ и } S_2  \text{ - вырезанные окружности}\right)
	\end{gathered}$$

	$I_1 = 0$, т.к. $G_1$ и $G_2$ равны 0 на $\Omega$.

	Рассмотрим $I_2$:
	$$\begin{gathered} 
		\oint \limits_{S_1} \left(G(x, y_1) \frac{\partial}{\partial n_x} G(x, y_2) - G(x, y_2) \frac{\partial}{\partial n_x} G(x, y_1) d \sigma_x \right) = \\
		=\oint \limits_{S_1} (\varepsilon (x - y_1) + \underbrace{g(x, y_1)}_{\underset{\alpha \rightarrow 0}{\longrightarrow 0}}) \frac{\partial}{\partial n_x} G(x, y_2) - G(x, y_2) \frac{\partial}{\partial n_y} (\varepsilon (x - y_1) + \underbrace{g(x, y_1)}_{\underset{\alpha \rightarrow 0}{\longrightarrow 0}}) d \sigma_x \\
	\end{gathered}$$
	Сделаем замену: $|x - y_1| = \rho, \frac{\partial}{\partial n_x} = - \frac{\partial}{\partial \rho}$. Тогда при $\alpha \to 0$ интеграл стремится к:
	$$\begin{gathered} \int \limits_{0}^{2 \pi} ( \dfrac{1}{2 \pi} \ln \rho ( - \frac{\partial G_2}{\partial \rho} ) \rho) |_{\rho = \alpha} d \theta - \int \limits_{0}^{2 \pi} ( G_2 ( - \frac{1}{2 \pi \rho} ) \rho) |_{\rho = \alpha} d \theta  = \\
	= \underbrace{\alpha \ln \alpha \frac{1}{2 \pi} \int \limits_{0}^{2 \pi}(- \frac{\partial G_2}{\partial \rho}) |_{\rho = \alpha} d \theta}_{\underset{\alpha \rightarrow 0}{\longrightarrow 0}} + \frac{1}{2 \pi} \int \limits_{0}^{2 \pi} G_2 |_{\rho = \alpha} d \theta
	 \end{gathered}$$ 

	$$ = \Big|\text{по теореме о среднем}\Big| =  G(x^*, y_2) |_{x^* \in S_1} \underbrace{\frac{1}{2 \pi} \int \limits_{0}^{2 \pi} d \theta}_{=1} \xrightarrow[\alpha \to 0]{} G(y_1, y_2) $$

	Для $\oint \limits_{S_2}$ все аналогично, только меняем местами $ G_1 $ и $ G_2 $. Тогда $\oint \limits_{S_2} \underset{\alpha \rightarrow 0}{\longrightarrow} G(y_2, y_1)$.
	
	В итоге получаем, что $G(y_1, y_2) - G(y_2, y_1) = 0$. Т.к. $ y_1, y_2 $ -- произвольные точки из $\Omega$, то $G(y_1, y_2) = G(y_2, y_1) \; \forall \; y_1, y_2 \in \Omega.$

\end{Proof}

\section*{Представление решения задачи Дирихле через функцию Грина.}

Имеем задачу Дирихле: 
$\begin{cases}
	\triangle u = f (x) , x \in \Omega\\
	u|_{x \in \partial \Omega} = h(x)
\end{cases}$\\

Воспользуемся 2-ой формулой Грина для $ u(y), G(x, y): $
$$\begin{gathered} 
	\iint\limits_{\Omega}\left(u(y) \underbrace{\triangle_y G(x,y)}_{=\delta (y-x)} - G(x,y)\underbrace{\triangle_y u(y)}_{=f(y)}\right)dy = \oint\limits_{\partial \Omega}\left(\underbrace{u(y)}_{=h(y)}\dfrac{\partial G(x,y)}{\partial n_y} - \underbrace{G(x,y)}_{=0}\dfrac{\partial u(y)}{\partial n_y}\right)d\sigma \\
	\iint\limits_{\Omega}\underbrace{u(y)\delta(y - x)}_{=u(x)\delta(y-x)}dy =  u(x) \underbrace{\iint\limits_{\Omega}\delta(y - x)dy}_{=1} = u(x)
\end{gathered}$$
Тогда получаем формулу:
$$u(x) = \displaystyle \iint\limits_{\Omega} G(x,y)f(y)dy + \oint\limits_{\partial \Omega}h(y)\dfrac{\partial G(x,y)}{\partial n_y}d\sigma_y$$

\section*{Метод отражений.}

\begin{center}
	\includegraphics[scale=0.4]{cha17im2}
\end{center}
	
Воспользуемся тем, что одной из физических интерпретаций функции Грина задачи Дирихле в области $ \Omega_0 $ является потенциал поля, создаваемого в точке $ x \in \Omega_0 $ точечным зарядом величины $ q = \dfrac{1}{4\pi} $, расположенным в точке $ y \in \Omega_0 $, если граница $ \partial \Omega_0 $ области $ \Omega_0 $ является заземленной идеально проводящей поверхностью.

Предположим, что вне области $ \Omega_0 $ можно расположить фиктивные электрические заряды таким образом, чтобы суммарный потенциал поля, создаваемого зарядом $ q = \dfrac{1}{4\pi} $, расположенным в точке $ y \in \Omega_0 $, и этими фиктивными зарядами, на границе $ \partial \Omega_0 $ обращался в нуль. Такие фиктивные заряды называются электростатическими изображениями заряда, помещенного в точку $ y \in \Omega_0 $. Потенциал поля, порожденного зарядами, находящимися вне области, представляет собой гармоническую внутри области $ \Omega_0 $ функцию, то есть искомое гармоническое слагаемое в функции Грина.

Тогда в условиях вышеописанной задачи потенциал есть $ u(x) = \varepsilon(x - y) + g(x,y) = G(x, y), \; g(x, y)$ -- гармоническая. $ G_0(x, y) $ -- функция Грина для половинки, тогда
$$\begin{cases}
	\Delta_x G_0(x, y) = \delta(x - y), \; x \in \Omega_0 \\
	G_0(x, y)|_{x \in \Omega_0} = 0
\end{cases} \; \forall y \in \Omega_0$$
\begin{theorem}[]\label{lec:17/the:1}
	$ G_0(x, y) = G(x, y) - G(x, \bar{y}) $.
\end{theorem}
\begin{Proof}
	\begin{enumerate}
		\item 
			\begin{itemize}
				\item[$\bullet$]
					$G_0(x, y)|_{x \in \Omega} = 0, \text{ т.к. } G(x, y)|_{x \in \Omega} = 0$
				\item[$\bullet$]
					$G_0(x, y)|_{\text{на } l} = 0, \text{ т.к. суммарный потенциал поля обращается }\text{в нуль на } l$
			\end{itemize}
			Значит $ G_0(x, y)|_{x \in \Omega_0} = 0 $.
		\item 
			$ \triangle_x G_0(x, y) = \delta(x-y) $, т.к.:
			$$\triangle_x G_0(x, y) = \triangle_x G(x, y) - \triangle_x G(x, \bar{y}) = \delta(x-y) - \underbrace{\delta(x-\bar{y})}_{= 0 \text{ в } \Omega_0 \; \left(\bar{y} \not\in \Omega_0\right)}$$
	\end{enumerate}
	Таким образом, $ G_0(x, y) = G(x, y) - G(x, \bar{y}). $
\end{Proof}

\section*{Метод конформных отображений.}

\begin{center}
		\includegraphics[scale=0.4]{cha17im3}
		
		$ \varphi(x): \; \mathbb{R}^2 \longrightarrow \mathbb{C} $
		
		$ \varphi(x) = \xi(x_1, x_2) + i \eta(x_1, x_2) $
\end{center}

\begin{definition}
	$ \varphi(x) $ -- \red{конформное отображение}, если $ \varphi(x) $ есть аналитическая функция ($ \mathbb{C} \text{ - дифференцируема, } \varphi'(x) \neq 0$).
\end{definition}

\begin{definition}
	$ \varphi(x) $ - \red{аналитическая функция} $\Leftrightarrow$ выполнены условия \textit{Коши - Римана:} 
	$$\begin{cases}
		\dfrac{\partial \xi}{\partial x_1} = \dfrac{\partial \eta}{\partial x_2} \\
		-\dfrac{\partial \xi}{\partial x_2} = \dfrac{\partial \eta}{\partial x_1}
	\end{cases}$$
\end{definition}

Функция $ \varphi_y(x) $ -- аналитическая, $ \frac{\partial \varphi_y(x)}{\partial x} \neq 0. $ Пусть $ \varphi_y(x) $ такая, что удовлетворяет условиям:
$$\begin{gathered} 
	x \to \varphi_y(x), \; y \to 0 \\
	\Omega \to |\varphi_y| < 1, \; \partial\Omega \to |\varphi_y| = 1
\end{gathered}$$
Получаем, что $G(x, y) \to G(\varphi_y, 0)$.
$$\begin{gathered} 
	G(\varphi_y, 0) = \varepsilon(\varphi_y - 0) = \varepsilon(\varphi_y) = \dfrac{1}{2\pi}\log|\varphi_y| \\
	\triangle \varepsilon(\varphi_y) = \xi_{\varphi_y\varphi_y} + \eta_{\varphi_y\varphi_y} = \delta(\varphi_y)
\end{gathered}$$
Значит:
$\begin{cases}
	\displaystyle \Delta_{\varphi_y} G(\varphi_y, 0) = \delta(\varphi_y - 0), |\varphi_y| < 1 \\
	\displaystyle G(\varphi_y, 0)|_{|\varphi_y| = 1} = 0
\end{cases}$

\begin{clair}
	$ G(x, y) = \dfrac{1}{2\pi} \ln|\varphi_y(x)|$.
\end{clair}
\begin{Proof}
	\begin{enumerate}
		\item 
			$ \displaystyle \partial\Omega \to |\varphi_y| = 1 \; \Rightarrow \; G(x, y)|_{x \in \partial \Omega} = \dfrac{1}{2\pi} \ln|\varphi_y(x)| \Big|_{|\varphi_y| = 1} = \dfrac{1}{2\pi} \ln1 = 0$.
		\item 
			Рассмотрим случай $x \neq y$:

			$\ln z = \ln|z| + iArg(z) $ - аналитическая при $ z \neq 0$, тогда $\varphi_y(x) $ -- аналитическая при $ \varphi_y(x) \not = 0$.
			$$G(x, y) = \dfrac{1}{2\pi}\ln|\varphi_y(x)| = \dfrac{1}{2\pi} \Re e(\ln(\varphi_y(x))) \text{ - гармоническая при } \\
			x \not = y.$$
			Тогда $\triangle_x G(x,y) = 0$ при $x \not = y$.
		\item 
			Рассмотрим случай $x \to y$.
			$$\begin{gathered} 
			\varphi_y(x) = \underbrace{\varphi_y(y)}_{= 0} + \underbrace{\dfrac{d \varphi_y(x)}{dx}\Big|_{x = y}}_{\neq 0}\cdot(x - y) + \underbrace{\alpha(x,t)}_{\underset{x \rightarrow y}{\longrightarrow} 0}(x - y)
			\end{gathered}$$
			Тогда 
			$$\begin{gathered} 
			G(x, y) = \dfrac{1}{2\pi}\ln|\varphi_y(x)| = \dfrac{1}{2\pi}\ln\left|\dfrac{d \varphi_y(x)}{dx}\Big|_{x = y}(x - y) + \alpha(x,t)(x - y)\right| = \\
			= \dfrac{1}{2\pi} \ln \left(|x - y|\left|\left.\dfrac{d \varphi_y(x)}{dx}\right|_{x = y} + \alpha(x,t)\right|\right) = \\
			= \underbrace{\dfrac{1}{2\pi} \ln|x - y|}_{=\epsilon (x-y)} + \dfrac{1}{2\pi} \ln\Big|\underbrace{\dfrac{d \varphi_y(x)}{dx}|_{x = y}}_{\not = 0} + \underbrace{\alpha(x,t)}_{\to 0}\Big| = \\
			= \varepsilon(x - y) + g(x, y), \; |g(x, y)| < C \; x \longrightarrow y.
			\end{gathered}$$
			Отсюда следует, что 
			$$\begin{gathered} 
			\triangle_x G(x, y) = \triangle_x (\varepsilon(x - y) + g(x, y)) = \delta(x - y) + \triangle_x g(x, y) = \\
			= \underbrace{\delta(x - y)}_{= 0, \text{ при } x \neq y} + \triangle_x g(x, y)
			\end{gathered}$$
			$$\begin{cases}
				\triangle_x g(x, y) = 0, \; x \neq y \\
				|g(x,y)| < C, \; x \to y
			\end{cases} \Rightarrow \;  \triangle_x g(x, y) = 0$$
			Получаем, что 
			$\displaystyle \triangle_x G(x, y) = \delta(x - y)$.
	\end{enumerate}
\end{Proof}













\chapter{Теорема Раднера.}\label{cha:18}

Пусть $Z \subset \mathbb{R^{2n}_+}$ -- модель Гейбла и $u: \mathbb{R}^n \to \mathbb{R}$ -- функция полезности, $z = (x, y) \in \mathbb{R}^n \times \mathbb{R}^n.$

$z_1, \ldots, z_T \in Z$ называется траекторией, если $\forall \; t \; x_{t + 1} \leq y_t.$

\begin{problem}
	$$\begin{cases}
		u(y_T) \to \max\\
		z_t \in Z\\
		x_{t + 1} \leq y_t\\
		y_0 \leq x_1
	\end{cases}$$ Решение этой задачи -- оптимальная траектория. 
\end{problem}

Рассмотрим условия:

\begin{enumerate}
	% \item $ \vec{z} = (\vec{x}, \alpha \vec{x}) \in Z, \; \vec{x} > 0.$
	\item $ z = (x, \alpha x) \in Z, \; x > 0.$
	% \item $\exists p > 0: \; \forall \; z = (x, y) \in Z, \; z \neq \lambda \vec{z}: \; \lambda p x > py.$
	\item $\exists p > 0: \; \forall \; z = (x, y) \in Z, \; z \neq \lambda z: \; \lambda p x > py.$
	\item $\forall \; x \gg 0 \; \exists L > 0: \; (x, Lx) \in Z.$
	\item u -- непрерывна и неортицательна.
	\item u -- однородная степени 1: $u(\lambda y) = \lambda u(y) \forall \; \lambda > 0.$
	\item u(x) > 0.
	\item $\exists k > 0: \; \forall \; y \geq 0: \; u(y) \leq kpy.$
\end{enumerate}

\begin{sign}
	$S(\varepsilon, T, \{ z_t \}) = \#\{ t = 1, \ldots, T | s(y_t, x) \geq \varepsilon \}.$
\end{sign}

\begin{lemma}
	$C_{\varepsilon} := \{ (x, y) \in z | s(x, y) \geq \varepsilon \} \Rightarrow \exists \delta > 0: \; \forall (x,y) \in C_{\varepsilon}: \; (\alpha - \delta)px \geq py.$
\end{lemma}

\begin{proof}
	Пусть $\exists z_k = (x_k, y_k) \in C_{\varepsilon}: \; (\alpha - \dfrac{1}{k})px_k < p y_k.$ Можно считать, что $\exists \lim\limits_{k \to \infty} z_k = (x, y) \neq 0. Z$ -- замкнутое множество $\Rightarrow (x, y) \in Z.$

	$$\begin{gathered}
		(\alpha - \dfrac{1}{k})px_k < p y_k \Rightarrow  \alpha p x \leq py \Rightarrow (x, y) = \lambda z \Rightarrow y = \lambda \alpha x \text{ и } s(x, y) = 0 \\
		y_k \to y \Rightarrow \exists k_0: \; s(y_k, x) < \varepsilon \text{ -- противоречие с } z_{k_0} = (x_{k_0}, y_{k_0}) \in C_{\varepsilon}.
	\end{gathered}$$
\end{proof}

\begin{theorem}[Раднера]
	Если модель Гейбла $Z$ удовлетворяет условиям 1)-7), то х -- слабая магистраль, то есть $\forall \; \varepsilon > 0 \; \exists Q(\varepsilon) = Q: \; \forall$ оптимальных траекторий $\{ x_t \}: \; S(\varepsilon, T, \{ z_t \}) \leq Q.$
\end{theorem}

\begin{proof}
	Рассмотрим оптимальный проект $\{ z_t\}_{t = 1}^T$. Рассмотрим $\{ z'_t\}: \; z'_1 = (x_1, Lx) \in Z, \; z'_t = (\underset{= \alpha^{t - 2}Lz \in Z, \; z = (x, \alpha x)}{\alpha^{t - 2}Lx}, \alpha^{t - 1}Lx), \; t > 1.$

	$z'_t$ -- траектория такая, что $x'_{t+1} = y'_t, \; x'_1 < x_1 \Rightarrow u(y'_T) \leq u(y_T).$

	$$\begin{gathered}
		u(y'_T) = \alpha^{T - 2}Lu(x) > 0. \text{ Если } z_t \in C_{\varepsilon}, \text{ то } py_t \leq (\alpha - \delta)px_t \leq (\alpha - \delta)py_{t - 1}. \;\\
		z_t \in Z \setminus C_{\varepsilon} \Rightarrow py_z \leq \alpha p x_t \leq \alpha py_{t - 1}.\\
		\text{Пусть } S = S(\varepsilon, T, \{ z_t \}) = \#\{ t | z_t \in C_{\varepsilon} \} \Rightarrow py_T \leq (\alpha - \delta)^s \alpha^{T - s}px_1 \Rightarrow u(y_T) \leq k p y_T \leq \\ 
		\leq k(\alpha - \delta)^s \alpha^{T - s} p x_1\\
		1 - \dfrac{u(y_T)}{u(y'_T)} = \dfrac{k(\alpha - \delta)^s \alpha^{T - s} p x_1}{\alpha^{T - 2}Lu(x)} = d(\dfrac{\alpha - \delta}{\alpha})^s, \text{ где } d = \dfrac{k\alpha^2px_1}{Lu(x)}\\
		d(1 - \dfrac{\delta}{2})^s \geq 1 \Rightarrow s \leq -\dfrac{\log d}{\log (1 - \dfrac{\delta}{2})} = Q(\varepsilon).
	\end{gathered}$$

	$Q(\varepsilon)$ зависит от модели Гейбла $Z, u, \varepsilon, x_1$ и не зависит от $T, \{ z_t\}.$
\end{proof}

% \cleardoublepage
\phantomsection
\addcontentsline{toc}{chapter}{Список используемой литературы}
\begin{thebibliography}{}
	\bibitem{0}
		Курс лекций И.М.Никонова, механико-математический факультет МГУ им. М.В.Ломоносова, 2021 г.
	\bibitem{1}
		Курс семинаров И.М.Никонова, механико-математический факультет МГУ им. М.В.Ломоносова, 2021 г.
\end{thebibliography}

\end{document}
