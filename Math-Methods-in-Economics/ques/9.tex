\chapter{Устойчивость замкнутой модели Леонтьева.}\label{cha:9}

$$\begin{cases}
	A \pi = \pi \\
	\pi \geq 0
\end{cases} \text{-- замкнутая модель Леонтьева (} \forall \; i: \; \sum\limits_{j = 1}^n a_{ij} = 1 \text{).}$$

Пусть $A \geq 0, \; r_i = \sum\limits_{j = 1}^n a_{ij}, \; s_j = \sum\limits_{i = 1}^n a_{ij}, \; r = \underset{i}{min}r_i, \; R = \underset{i}{max}r_i, \; s = \underset{i}{min}s_i, \; S = \underset{i}{max}s_i$

\begin{clair}
	$A \geq 0 \Rightarrow r \leq \lambda_A \leq R, s \leq \lambda_A \leq S.$
\end{clair}

\begin{proof}
	$Ax_A = \lambda_A x_A \Rightarrow \forall \; i: \; \sum\limits_{j = 1}^n a_{ij} (x_A)_j = \lambda_A (x_A)_i.$

	$$\begin{gathered}
		\sum\limits_{i = 1}^n \sum\limits_{j = 1}^n a_{ij} (x_A)_j = \lambda_A \sum\limits_{i = 1}^n (x_A)_i \Longleftrightarrow \sum\limits_{j = 1}^n (x_A)_j \underset{= s_j}{\sum\limits_{i = 1}^n a_{ij}} = \underset{\geq s\sum\limits_{j = 1}^n (x_A)_j \text{ и } \leq S\sum\limits_{j = 1}^n (x_A)_j}{\sum\limits_{j = 1}^n s_j (x_A)_j} = \\
		= \lambda_A \sum\limits_{i = 1}^n (x_A)_i.
	\end{gathered}$$

	$\Rightarrow s \leq \lambda_A \leq S. \text{ Аналогично с  } r \leq \lambda_A \leq R, \text{ достаточно транспонировать.}$
\end{proof}

\begin{conseq}
	$A \geq 0$ -- замкнутая, тогда $\lambda_A = 1.$
\end{conseq}

\begin{definition}
	$A > 0$ -- неразложимая, $\lambda_A = 1$ (иначе $A = \dfrac{1}{\lambda_A}A$). $p_A$ -- левый Фробениусов собственный вектор: $(p_A, p_A) = 1.$
\end{definition}

\begin{definition}
	Норма на $\mathbb{R}^n: \; \| x\|_A = (p_A, |x|), \; |x| = (|x_1|, \ldots, |x_n|).$
\end{definition}

\begin{definition}
	A -- устойчива, если $\forall \; x \; \exists \lim\limits_{k \to \infty}A^k x$
\end{definition}

\begin{remark}
	Выберем правый Фробениусов собственный вектор: $\| x_A\|_A = 1. \; \| Ax\|_A \leq \| x_A\|_A$ и, если $x \geq 0,$ то $\| Ax\|_A = \| x_A\|_A \; (p_A |Ax| \leq p_A A |x| = p_A |x| = \| x \|_A, \; p_A |Ax| = p_A A x = p_A x = p_A |x| = \| x\|_A. )$
\end{remark}

\begin{clair}
	$x \geq 0, $ если $\exists \lim\limits_{k \to \infty}A^k x,$ то $z = \lim\limits_{k \to \infty}A^kx = \mu x_A, \; \mu = \| x\|_A.$
\end{clair}

\begin{proof}
	$Az = \lim\limits_{k \to \infty}A^{k + 1}x = z, \; A$ -- неразложимая, $\lambda_A = 1 \Rightarrow z = \mu x_A. \; x \geq 0 \Rightarrow \forall \; k: \; A^kx \geq 0, \; \| A^kx\|_A = \| x\|_A \Rightarrow \| z\|_A = \| x\|_A. \| z\|_A = \| \mu x_A\|_A = \mu \| x_A\|_A = \mu = \| x\|_A.$
\end{proof}

\begin{definition}[Импримитивная(циклическая) матрица]
	Неразложимая матрица А называется импримитивной(циклической), если $\exists$ разбиение $\{ 1, \ldots, n\} = S_0 \bigsqcup \ldots \bigsqcup S_{m - 1}, \; S_i \cap S_j = \emptyset \; i \neq j, \; \forall i \; S_i \neq \emptyset, $ при этом $a_{ij} > 0 \Rightarrow$:

	$$\begin{cases}
		i \in S_r, \; j \in S_{r - 1}, \; 1 \geq r \geq m - 2 \\
		i \in S_0, \; j \in S_{m - 1}
	\end{cases}$$

	То есть, матрица импримитивна, если одновременной перестановкой столбцов и строк она приводится к виду: 

	\begin{gather*}
		\begin{pmatrix}
		  0 & \ldots & 0 & A_{m-1}\\
		  A_0 & 0 & \ldots & 0\\
		  0 & A_1 & \ldots & 0\\
		  \ldots & \ldots & \ldots & 0\\
		  0 & \ldots & 0 &  A_{m-2}\\
		\end{pmatrix}
	\end{gather*}

	Иначе А -- примитивная.
\end{definition}

\begin{example}
		\begin{gather*}
			\begin{pmatrix}
			  0 & 1\\
			  1 & 0 \\
			\end{pmatrix} \text{--циклическая, неразложимая.}
			\begin{pmatrix}
			  0 & 1\\
			  1 & 1 \\
			\end{pmatrix} \text{--примитивная.}
			\begin{pmatrix}
			  1 & 1\\
			  0 & 1 \\
			\end{pmatrix} \text{--разложимая.}
		\end{gather*}
\end{example}

\begin{theorem}
	Неразложимая матрица $A \geq 0 устойчива, \; \lambda_A = 1 \Longleftrightarrow $ A -- примитивна.
\end{theorem}

\begin{lemma}
	А -- примитивная, тогда у некоторой степени А первая строка положительная: $\exists k: \; \forall \; j a_{1j}^k > 0, \; A^k = (a_{ij}^k)$.
\end{lemma}

\begin{proof}
	$R := \{ r | a_{11}^r > 0\}, \; m = \text{ НОД}(R)$

	\begin{enumerate}
		\item m > 1 $\Rightarrow A$ -- циклическая, докажем это:

		$S_i := \{ j | \exists k \equiv -i (mod \; m): \; a_{1j}^k > 0\}, \; i = \bar{0, m - 1}. S_i \neq \emptyset, $ иначе у $A^{m - i}$ первая строка нулевая. $S_r \cap S_t = \emptyset, \; r \neq t.$ Пусть $j \in S_r \cap S_i \Rightarrow \exists k, l: \; a_{1j}^k > 0, \; a_{1j}^l > 0, \; k \equiv -r, \; l \equiv -t (mod \; m).$

		Так как А -- неразложимая, то $\exists q: \; a{1j}^q > 0 \Rightarrow a_{11}^{k + q} \geq a_{1j}^k a_{1j}^q > 0, \; a_{11}^{l + q} \geq a_{1j}^l a_{1j}^q > 0 \Rightarrow m | (k + q), \; m | (l + q) \Rightarrow m | (k - l) \Rightarrow r \equiv t (mod \; m) \Rightarrow$ такие множества задают искомое разбиение множества $\{ 1, \ldots, n\}$.

		\item действительо, $l = sd + q, \; s \geq d, \; 0 \leq q \leq d \Rightarrow l = s (\sum |\beta_i|l_i) + q (\sum \beta_i l_i) = \sum (s|\beta_i| + q|\beta_i|)l_i = \sum\alpha_i l_i, \; \forall \; i: \; \alpha_i \geq 0 \Rightarrow a_{11}^l \geq \prod (a_{11}^{l_i})^{\alpha_i} > 0.$ Рассмотрим $t_j: \; a_{1j}^{t_j} > 0, \; t := \max t_j \Rightarrow $ в $A^{Q + t}$ первая строка положительна: $a_{1j}^{Q + t} \geq a_{1j}^{Q + t - t_j}a_{1j}^{t_j} > 0.$
	\end{enumerate}
\end{proof}

\begin{lemma}
	Если матрица $A^k$ устойчива при некотором k, то А -- устойчива.
\end{lemma}

\begin{proof}
	Пусть $x \in \mathbb{R}^n, \; \lim\limits_{s \to \infty}(A^k)^sx = \mu x_A.$ Докажем, что $\lim\limits_{m \to \infty}A^m = \mu x_A.$

	$\forall \; \varepsilon > 0 \; \exists S \in \mathbb{N}: \; \forall \; s > S: \; \| A^{sk}x - \mu x_A \| \leq \varepsilon \Rightarrow \forall \; m \geq Sk: \; m = sk : m = sk + r, \; 0 \leq r < k \Rightarrow \| A^{m}x - \mu x_A \| = \| A^{r}(A^{sk}(x - \mu x_A)) \| \leq \| A^{sk}(x - \mu x_A) \| < \varepsilon.$
\end{proof}

\begin{definition}[Оператор сжатия]
	Оператор Р действует на линейном нормированном пространстве как оператор сжатия, если $\exists \gamma: \; 0 < \gamma < 1: \; \forall \; v \in L: \; \| Pv\| \leq \gamma \| v\|, \; \gamma $ -- коэффициент сжатия. 
\end{definition}

\begin{lemma}
	$L_A:= Ann(p_A) = \{ v | (v, p_A) = 0\}.$ Если оператор А действует на $L_A$ как оператор сжатия, то А -- устойчива.
\end{lemma}

\begin{proof}
	Заметим, что $L_A$ инвариантно относительно оператора А. Пусть $x = \mu x_A + z, \; z \in L_A \Rightarrow \| A^{k}x - \mu x_A \| = \| A^{k}(x - \mu x_A) \| = \| A^{k}z \| \leq \gamma^k \| z \| \underset{k \to \infty}{\to}0 \Rightarrow \lim\limits_{k \to \infty}A^kx = \mu x_A.$
\end{proof}

\begin{lemma}
	Если неразложимая матрица А с $\lambda_A = 1$ имеет положительную строку, то оператор А действует на пространстве $L_A:= Ann(p_A) = \{ v | (v, p_A) = 0\}$ как оператор сжатия.
\end{lemma}

\begin{proof}
	Без ограничения общности первая строка А -- положительная. Обозначим строки А как $a_i \Rightarrow a_1 > 0.$ Пусть $\delta > 0, \; \delta p_A \leq a_1.$ Пусть $p_A^1$ -- первая координата $p_A$. Покажем, что $\gamma = 1 - \delta p_A^1$ -- коэффициент сжатия A на $L_A$. 

	Действительно, пусть $z \in L_A.$ Можно считать, что $(a_1, z) \geq 0$, иначе z = -z. Тогда $\| Az \| = \sum\limits_{i = 1}^n p_A^i | (a_i, z) | = p_A^1 (a_1, z) + \sum\limits_{i = 2}^n p_A^i | (a_i, z) | \leq p_A^1 (a_1, z) + \sum\limits_{i = 2}^n p_A^i (a_i, |z|) = p_A^1 (a_1, z - |z|) + \sum\limits_{i = 1}^n p_A^i (a_i, |z|) = p_A^1 (a_1, z - |z|) + (p_A, A|z|) \leq p_A^1 (\delta p_A,z - |z|) + (p_A, A|z|), \text{ так как } z - |z| \leq 0.$ Но $z \in L_A \Rightarrow \| Az\| = (1 - \delta p_A) |z| = \gamma \| z \|.$
\end{proof}

\begin{conseq}
	Если неразложимая матрица А с $\lambda_A = 1$ имеет положительную строку, то A -- устойчива.
\end{conseq}

\begin{theorem}
	A > 0 -- неразложимая с $\lambda_A = 1$, тогда А -- устойчивая $ \Longleftrightarrow \forall \; \lambda \in Spec(A) \setminus 1$ выполнено $|\lambda| < 1.$
\end{theorem}

\begin{proof}
	Достаточно рассмотреть одну жорданову клетку $J_{\lambda}$ ЖНФ для А (над $\mathbb{C}^n$):

	\begin{gather*}
		\begin{pmatrix}
		  \lambda & 1 & 0 & 0\\
		  0 & \lambda & 1 & 0\\
		  0 & \ldots & \ldots & \ldots\\
		  0 & \ldots & \ldots & 1\\
		  0 & \ldots & 0 &  \lambda \\
		\end{pmatrix}
	\end{gather*}

	\begin{itemize}
		\item Если $|\lambda| < 1 \Rightarrow \lim \limits_{k \to \infty}J_{\lambda}^k = 0$
		\item Если $|\lambda| \geq 1, \; \lambda \neq 1 \Rightarrow \nexists \lim \limits_{k \to \infty}J_{\lambda}^k$
	\end{itemize}

	С собственным значением 1 одна жорданова клетка размера $1 \times 1.$ Клетка одна, так как есть только один с точностью до пропорциональности левый фробениусов вектор $p_A: \; (p_A, p_A) = 1.$ Действительно, $0 = p_A(A - E)e_2 = (p_A, \mu X_A).$
\end{proof}