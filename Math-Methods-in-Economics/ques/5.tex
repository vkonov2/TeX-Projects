\chapter{Функции коллективного выбора. Теорема Эрроу.}\label{cha:5}

Пусть $S = \{y_1, \dots, y_n\}$ - множество избирателей, $M = \{x_1, \dots, x_m\}$ - множество кандидатов. Т.о. имеем $n$ систем индивидуального предпочтения, каждая из которых устанавливает линейный порядок на множестве всех кандидатов.

Пусть $p$ - произвольное правило голосования. Применяя правило $p$, построим множество $M_1 = \{a_1 = \dots = a_s\}$, состоящее из победителей. 

Применим правило $p$ ко множеству проигравших $M \setminus M_1$. Опять получаем множество выигравших $M_2 = \{a_{s+1} = \dots = a_{s+l}\}$.

Выкинем из $M$ объединение $M_1 \bigcup M_2$ и проделаем ту же операцию, и т.д. В результате мы упорядочим множество $M$ согласно решению коллектива: 
$$M_1 \succ M_2 \succ \dots$$

Таким образом, правило голосования позволяет построить систему коллективного предпочтения. И из правила $p$ строим \red{функцию коллективного предпочтения} $\succeq = f(\overset{1}{\succ}, \dots, \overset{n}{\succ})$, где $\overset{k}{\succ}$ - система индивидуального предпочтения.

\textbf{Требования} к функции коллективного предпочтения:
\begin{itemize}
	\item[$1$.]
		\blue{полнота}

		Для любых кандидатов $a$ и $b$ коллективный порядок устанавливает, что либо $a \prec b$, либо $b \prec a$, либо $a = b$.
	\item[$2$.]
		\blue{транзитивность}

		Для любых трех кандидатов $a, b, c$ таких, что $a \preceq b$ и $b \preceq c$, выполняется $a \preceq c$, причем равенство имеет место, если и только $a = b = c$.
	\item[$3$.]
		\blue{единогласие}

		Если все избиратели считают, что $a$ лучше $b$, значит и в коллективном предпочтении $a$ должен быть лучше $b$:
		$$\forall k \; a \overset{k}{\prec} b \; \Rightarrow \; a \prec b$$
	\item[$4$.]
		\blue{независимость}

		Положение любых двух кандидатов в коллективном предпочтении зависит только от их взаимного расположения в индивидуальных предпочтениях и не зависит от расположения других кандидатов. 

		Т.о. если для профиля вида:
		\begin{tabular}{ | c | c | }
			\hline
				группы избирателей & кандидаты \\ \hline
				$A$ & $\cdot \cdot \cdot$ $a$ $\cdot \cdot \cdot$ $b$ $\cdot \cdot \cdot$ \\
				$S \setminus A$ & $\cdot \cdot \cdot$ $b$ $\cdot \cdot \cdot$ $a$ $\cdot \cdot \cdot$ \\
			\hline
		\end{tabular}
		имеем $a \prec b$, то и для всех профилей такого вида $a \prec b$.
\end{itemize}

\begin{remark}\label{cha:5/remark:1}
	Множество функций коллективного выбора, удовлетворяющих аксимомам 1-4, непусто. Примером таких функций могут служить \red{функции диктатора}, а именно функции вида $f(\overset{1}{\succ}, \dots, \overset{n}{\succ}) = \overset{k}{\succ}$ для некоторого $k$. 
\end{remark}

\begin{theorem}[\red{Эрроу}]\label{cha:5/the:1}
	Пусть $f$ - функция коллективного предпочтения, удовлетворяющая аксиомам 1-4, и предположим, что имеется не менее 3 кандидатов. Тогда $f$ - функция диктатора.

	(диктатура описывается аксиомами, а демократия - отрицание диктатуры)
\end{theorem}
\begin{Proof}
	Введем некоторые понятия.

	Произвольное подмножество $A$ множества избирателей называется \blue{коалицией}.

	\begin{definition}\label{cha:5/def:1}
		Коалиция $A$ назывется \blue{f-решающей для кандидата $a$ против кандидата $b$}, тогда и только тогда, когда из того, что все члены коалиции $A$ ставят $a$ выше $b$, а все члены, не входящие в $A$, ставят $b$ выше $a$, вытекает, что в коллективном предпочтении $a \prec b$:
		$$(\forall y_k \in A, \; a \overset{k}{\prec} b) \bigcup (\forall y_k \not \in A, \; b \overset{k}{\prec}a) = a \prec b$$
		Обозначение: $A = f(a, b)$.
	\end{definition}

	Коалиция $A$, такая, что для любых двух кандидатов $a$ и $b$ коалиция $A$ является $f$-решающей для $a$ против $b$ назывется просто \blue{$f$-решающей}.

	\begin{lemma}[]\label{cha:5/lemma:1}
		$\exists$ пара кандидатов $(a, b)$, для которой найдется коалиция $D$, состоящая из одного избирателя $d$, такая, что $D = f(a, b)$.
	\end{lemma}
	\begin{Proof}
		Обозначим через $K$ множество всех коалиций, для каждой из которых $\exists$ пара кандидатов $(a, b)$, таких, что эта коалиция является $f$-решающей для $a$ против $b$. 

		Отметим, что множество $K$ не пусто, т.к., в силу аксиомы единогласия, множество всех избирателей $S$ образует $f$-решающую коалицию $\forall$ пары кандидатов $(a, b)$.

		Рассмотрим в $K$ коалицию $D$, состоящую из наименьшего числа избрателей. Покажем, что $D$ состоит ровно из одного элемента.

		Предположим противное, т.е. $D = \{d\} \bigcup E$, где $E$ - некоторое непустое множество кандидатов. Если $S$ состоит из $\ge 2$ кандидатов, рассмотрим профиль:

		\begin{center}
			\begin{tabular}{ | c | c | c | c |}
				\hline
					группа избирателей & $\{d\}$ & $E$ & $S \setminus D$ \\ \hline
					кандидаты & $a$ & $c$ & $b$ \\
					кандидаты & $b$ & $a$ & $c$ \\
					кандидаты & $c$ & $b$ & $a$ \\
				\hline
			\end{tabular}
		\end{center}

		Если же $S$ состоит из двух кандидатов, рассмотрим профиль:

		\begin{center}
			\begin{tabular}{ | c | c | c |}
				\hline
					группа избирателей & $\{d\}$ & $E$ \\ \hline
					кандидаты & $a$ & $c$ \\
					кандидаты & $b$ & $a$ \\
					кандидаты & $c$ & $b$ \\
				\hline
			\end{tabular}
		\end{center}

		Т.к. $D = f(a,b)$, то $a \prec b$. Предположим, что $c \prec b$. Тогда $E = f(c, b)$, что противоречит минимальности коалиции $D$. Значит, $b \preceq c$, и, по аксиоме транзитивности, $a \prec c$. Но тогда $\{d\} = f(a,c)$, что опять же противоречит минимальности коалиции $D$.

		Т.о., мы получили противоречие к предположению, что $D$ состоит из $\ge 1$ элемента, а значит, $D$ содержит ровно $1$ элемент.
	\end{Proof}

	\begin{lemma}[]\label{cha:5/lemma:2}
		Коалиция $D$ из леммы $5.1$ является $f$-решающей.
	\end{lemma}
	\begin{Proof}
		Пусть $c$ - произвольный кандидат. Рассмотрим профиль:

		\begin{center}
			\begin{tabular}{ | c | c | }
				\hline
					группы избирателей & кандидаты \\ \hline
					$\{d\}$ & $\dots a \prec \dots \prec b \prec \dots \prec c \dots$ \\
					$S \setminus \{d\}$ & $\dots b \prec \dots \prec c \prec \dots \prec a \dots$ \\
				\hline
			\end{tabular}
		\end{center}

		Т.к. $\{d\} = f(a,b)$, то $a \prec b$. В силу аксиомы единогласия, $b \prec c$, значит, по транзитивности, $a \prec c$. Значит, $\{d\} = f(a, c)$.

		Пусть $e$ - еще один кандидат. Рассмотрим профиль:

		\begin{center}
			\begin{tabular}{ | c | c | }
				\hline
					группы избирателей & кандидаты \\ \hline
					$\{d\}$ & $\dots e \prec \dots \prec a \prec \dots \prec c \dots$ \\
					$S \setminus \{d\}$ & $\dots c \prec \dots \prec e \prec \dots \prec a \dots$ \\
				\hline
			\end{tabular}
		\end{center}

		По аксиоме единогласия, $e \prec a$. Т.к. $a \prec c$, то по транзитивности для данного профиля имеем $e \prec c$. Но тогда $\{d\} = f(e,c)$.

		Т.о. мы показали, что для любых двух кандидатов $e$ и $c$ коалиция $\{d\}$ является $f$-решающей для $e$ против $d$. Значит, коалиция $\{d\}$ есть $f$-решающая коалиция.
	\end{Proof}

	\begin{lemma}[]\label{cha:5/lemma:3}
		Избиратель $d$ является диктатором.
	\end{lemma}
	\begin{Proof}
		Мы показали, что $d$ может навязывать свое мнение по поводу любых двух кандидатов $a$ и $b$ при условии, что мнение остальных избирателей противоположно - в этом пока проявляется зависимость от мнения других.

		Надо показать, что как бы не голосовали остальные избиратели, коллективное мнение совпадает с мнением $d$. \\

		Рассмотрим такие профили голосования, в которых у избирателя $d$ порядок вида $\dots a \overset{d}{\prec} \dots \overset{d}{\prec} c \overset{d}{\prec} \dots \overset{d}{\prec} b \dots$, а все остальные избиратели ставят $c$ выше, чем $a$ и $b$.

		Т.к. $\{d\}$ является $f$-решающей коалицией, то для таких профилей $a \prec c$. В силу аксиомы единогласия $c \prec b$, тогда по транзитивности $a \prec b$. 

		Исключая из соотношений $c$ и используя аксиому независимости, получаем, что если $d$ ставит $a$ выше $b$, то и в коллективном порядке $a \prec b$. В силу произвольности $a$ и $b$, получаем, что $d$ - диктатор.
	\end{Proof}

	Доказательство теоремы закончено.
\end{Proof}

















