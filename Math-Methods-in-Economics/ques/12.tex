\chapter{Модель расширяющейся экономики фон Неймана. Сбалансированный рост в модели фон Неймана.}\label{cha:12}

\begin{itemize}
	\item n товаров, m производственных процессов.
	\item Базисные процессы $(a^j, b^j)$, где $a^j$ -- вектор затрат, $b^j$ -- вектор выпуска.
	\item Смешанный процесс: $x = (x^1, \ldots, x^m)$ -- вектор интенсивностей, $x \rightarrow (Ax, Bx)$.
	\item A -- матрица затрат, B -- матрица выпуска (неотрицательная матрица).
	\item Время t дискретно.
\end{itemize}

При этом:

\begin{enumerate}
	\item Условие замкнутости: $Ax_t \leq Bx_{t-1}.$
	\begin{definition}
		Последовательность $\{ x_i\}_{i = 1}^t$, удовлетворяющая условию замкнутости, называется планом, $x_0$ -- вектор запросов.
	\end{definition}
	\item Условие нулевого дохода: $p_{t-1}A \geq p_tB$ (базисный процесс не приносит прибыли), где $\{ p_t\}$ -- траектория цен.
	\item Сохранение денежной массы: $p_t A x_t = p_{t + 1} B x_t.$
	\item Отсутствие прибыли: $p_t B x_{t-1} = p_t A x_t.$
\end{enumerate}

\begin{definition}[Стационарная траектория]
	$x_t = \partial^t x_0, \; p_t = \mu^{-t}p_0.$
\end{definition}

\begin{definition}[Стационарный план]
	$\partial A x_0 \leq B x_0.$
\end{definition}

\begin{definition}[Стационарная траектория цен]
	$\mu p_0 A \geq p_0 B.$
\end{definition}

\begin{definition}[Состояние динамического равновесия]
	$(\partial, \mu, p, x): \; \partial A x \leq Bx, \; \mu p A \geq p B, \; \mu p A x = pBx, \; \partial pAx = pBx, \; \partial, \mu, p, x > 0$
\end{definition}

\begin{remark}
	$pAx \neq 0 \Rightarrow \partial = m.$
\end{remark}

\begin{definition}[Невырожденное положение равновесия]
	$(\alpha, x, p): \alpha, x, p > 0, \; \alpha A x \leq Bx, \; \alpha p A \geq pB, \; pAx > 0.$
\end{definition}
