\chapter{Уравнение Слуцкого. Его следствия.}\label{cha:3}

Изучим поведение потребителя, стесненного бюджетными ограничениями.

Пусть потребительноское множество $X = \mathbb{R}_{+}^n$, функция полезности $u(x)$ гладкая и удолетворяет ограничениями:

$$\begin{gathered}
	\frac{\partial u}{\partial x_i} > 0, \; \underset{x_i \to 0}{\lim} \frac{\partial u}{\partial x_i} = \infty, \; \underset{x_i \to \infty}{\lim} \frac{\partial u}{\partial x_i} = 0, \; i = \ton n \\
	\text{Гауссиан } U(x) = \left( \frac{\partial^2 u}{\partial x_i \partial x_j} (x) \right) \text{ отрицательно определен } \forall x \in X
\end{gathered}$$

Пусть $p$ - система цен, $K$ - капитал потребителя.

Из свойств функции $u(x)$ вытекает, что функция спроса $\Phi(p, K)$ является однозначной, и при заданных $p$ и $K$ единственное значение $x^{*}(p,K)$ функции $\Phi(p,K)$ определяется следующей задачей математического программирования:
$$\begin{gathered}
	u(x) \to max \\
	< p,x > = K, \; x \ge 0
\end{gathered}$$
Кроме этого, имеем $x^{*} (p, K) >> 0$, тогда для определения точки $x^{*}(p,K)$ воспользуемся теоремой Лагранжа. Выпишем функцию Лагранжа:
$$L(x, \lambda) = u(x) - \lambda \left( <p,x> - K \right)$$
Тогда сущетсвует такое $\lambda^{*}$, что:
$$<p, x^{*}> - K = 0\eqno(1)$$
$$\frac{\partial u}{\partial x_i}(x^{*}) - \lambda^{*} p_i = 0, \; i = \ton n\eqno(2)$$

Заметим, что уравнение $(2)$ - это условие того, что бюджетная плоскость касается поверхности уровня функции полезности (градиент функции полезности сонаправлен с нормалью $p$ к бюджетной плоскости).

Рассмотрим влияение изменения цены ровно одного продукта (например, $p_n$) на поведение потребителя. Для этого продифференцируем полученные уравнения по $p_n$:
$$<p, \frac{\partial x^{*}}{\partial p_n}> = -x_n^{*}\eqno(3)$$
$$U \frac{\partial x^{*}}{\partial p_n} - \frac{\partial {\lambda}^{*}}{\partial p_n}p = (0, \dots, 0, \lambda^{*}) \MYdef \Lambda^{*}\eqno(4)$$

Воспользовавшись невырожденностью матрицы $U$ (она невырождена в силу ее отрицательной определенности), выразим из уравнения $(4)$ $\frac{\partial x^{*}}{\partial p_n}$ и подставим полученное значение в $(3)$, откуда найдем $\frac{\partial {\lambda}^{*}}{\partial p_n}$:
$$\frac{\partial {\lambda}^{*}}{\partial p_n} = -\frac{x_n^{*} + p U^{-1}\Lambda^{*}}{p U^{-1} p}$$

Обозначим $\mu = - (p U^{-1} p)^{-1}$ и $[M]^{(i)}$ - $i$-ый столбец матрицы $M$. Заметим, что $U^{-1}\Lambda^{*} = \lambda^{*} [U^{-1}]^{(n)}$. Тогда полученную формулу можно переписать так:
$$\frac{\partial {\lambda}^{*}}{\partial p_n} = \mu x_n^{*} + \mu \lambda^{*} p[U^{-1}]^{(n)}$$

Пусть $z = [U^{-1}]^{(n)}$, тогда:
$$\frac{\partial {\lambda}^{*}}{\partial p_n} = \mu x_n^{*} + \mu \lambda^{*} <z, p>$$

Отсюда получаем непосредственнуб формулу для $\frac{\partial x^{*}}{\partial p_n}$:
$$\frac{\partial x^{*}}{\partial p_n} = \mu x_n^{*} U^{-1} p + \lambda^{*} \left( \mu <p, z> U^{-1} p + z \right)\eqno(5)$$

Поймем экономический смысл правой части.

Чтобы выяснить смысл первого слагаемого, рассмотрим, что происходит с решением $x^{*}(p, K)$, если цены $p$ остаются неизменные, а капитал $K$ меняется. Продифференцируем $(1)$ и $(2)$ по $K$:
$$<p, \frac{\partial x^{*}}{\partial K}> = 1\eqno(5)$$
$$U \frac{\partial x^{*}}{\partial K} - \frac{\partial {\lambda}^{*}}{\partial K}p = 0\eqno(6)$$

Выражая из $(7)$ величину $\frac{\partial x^{*}}{\partial K}$ и подставляя полученное значение в $(6)$, найдем $\frac{\partial {\lambda}^{*}}{\partial K}$:
$$\frac{\partial {\lambda}^{*}}{\partial K} = \frac{1}{p U^{-1}p} = -\mu$$
Значит имеем:
$$\frac{\partial x^{*}}{\partial K} = -\mu U^{-1} p$$
Значит получаем:
$$\frac{\partial x^{*}}{\partial p_n} = -\frac{\partial x^{*}}{\partial K} x_n^{*} + \lambda^{*} \left( \mu <p, z> U^{-1} p + z \right)$$

Выясним теперь смысл второго слагаемого уравнения $(5)$. Для этого рассмотрим влияние компенсированного изменения цены $p$, т.е. такого изменения, при котором одновременно меняется капитал $K$ так, чтобы максимальное значение функции полезности на соответствующей бюджетной плоскости оставалось неизменным. Т.о. мы предполагаем, что капитал $K$ зависит от $p$, т.е. является функцией $K(p)$, и имеет место следующееся условие: $u(x^{*}(p, K(p))) = const$.

Рассмотрим теперь функцию $x^{*}(p,K)$ как функцию от $p$, подставив вместо $K$ соответствующую функцию $K(p)$.

\begin{definition}\label{cha:3/def:1}
	Полную производную функции $x^{*}$ по $p_j$, т.е. величину $\frac{\partial x^{*}}{\partial p_i} = \frac{\partial x^{*}}{\partial K} \cdot \frac{\partial K}{\partial p_i}$, называется \red{компенсированной производной} по $p_i$ и обозначается $\left(\frac{\partial x^{*}}{\partial p_i}\right)_{comp}$.
\end{definition}

Вычислим компенсированную производную при $i = n$. Для этого продифференцируем уравнение $(1)$ по $p_n$. Имеем:
$$x_n^{*} + \left< p, \left(\frac{\partial x^{*}}{\partial p_n}\right)_{comp} \right> - \frac{\partial K}{\partial p_n} = 0\eqno(9)$$

Т.к. при каждом $p$ функция $u(x^{*})$ остается неизменной, получаем, что вектор $\left(\frac{\partial x^{*}}{\partial p_n}\right)_{comp}$ касается поверхности $u = const$, поэтому этот вектор перпендикулярен градиенту функции $u$, т.е. вектору $\frac{\partial u}{\partial x}$. 

С другой стороны, по определению $x^{*}$, бюджетная плоскость касается в точке $x^{*}$ поверхности $u = const$, поэтому нормаль $p$ к бюджетной плоскости сонаправлена с нормальню $\frac{\partial u}{\partial x}$ к поверхности $u = const$ (из $(2)$). Значит, второе слагаемое в $(9)$ равно $0$. Отсюда получаем, что:
$$x_n^{*} = \frac{\partial K}{\partial p_n}\eqno(10)$$

Продифференцируем еще раз $(2)$ по $p_n$. Имеем:
$$U\left(\frac{\partial x^{*}}{\partial p_n}\right)_{comp} = \frac{\partial \lambda^{*}}{\partial p_n}p + \Lambda^{*}\eqno(11)$$

Воспользуемся тем, что $\left< p, \left(\frac{\partial x^{*}}{\partial p_n}\right)_{comp} \right> = 0$, выразим $\left(\frac{\partial x^{*}}{\partial p_n}\right)_{comp}$ из $(11)$ и умножим полученное выражение скалярно на $p$, найдем выражение для $\frac{\partial \lambda^{*}}{\partial p_n}$:
$$\frac{\partial \lambda^{*}}{\partial p_n} = \mu p U^{-1}\Lambda^{*} = \mu \lambda^{*} <z, p>$$

Подставим полученное выражение в $(11)$, получаем:
$$\begin{gathered}
	\left(\frac{\partial x^{*}}{\partial p_n}\right)_{comp} = \mu \lambda^{*} <z, p> U^{-1} p + U^{-1} \Lambda^{*} = \\
	= \mu \lambda^{*} <z, p> U^{-1} p + \lambda^{*} z = \lambda^{*} \left( \mu <z, p> U^{-1} p + z \right)
\end{gathered}$$

Сравнивая полученное выражение и второе слагаемое в $(5)$, получаем, что это слагаемое равно компенсированной производной функции $x^{*}$ по $p_n$. Т.о. доказали теорему.

\begin{theorem}[]\label{cha:3/the:1}
	Имеет место соотношение:
	$$\frac{\partial x^{*}}{\partial p_n} = \left(\frac{\partial x^{*}}{\partial p_n}\right)_{comp} - \left(\frac{\partial x^{*}}{\partial K}\right) x_n^{*}\eqno(12)$$
	Данное уравнение называется \red{уравнением Слуцкого}.
\end{theorem}

\textbf{Замечение}\\

Выражение для $\left(\frac{\partial x^{*}}{\partial p_n}\right)_{comp}$, полученное при выражении уравнения Слуцкого, можно пеперписать так:
$$\left(\frac{\partial x^{*}}{\partial p_n}\right)_{comp} = \lambda^{*} \left[ \mu U^{-1} p' p U^{-1} + U^{-1} \right]^{(n)}$$
где $p'$ обозначает вектор $p$, рассмтариваемый как столбец, в отличие от вектора $p$, рассматриваемого как вектор-строка.

\begin{definition}\label{cha:3/def:2}
	Матрица $H = \mu U^{-1} p' p U^{-1} + U^{-1}$ называется \red{матрицей Слуцкого}.
\end{definition}

\textbf{Свойства матрицы Слуцкого}:
\begin{itemize}
	\item[$1)$]
		\textit{матрица $H$ симметрична}
		\begin{Proof}
			$U$ - симметричная, тогда $U^{-1}$ тоже симметричная. Кроме того, матрица $p' p$ - это $n \times n$ матрица, у которой $(i, j)$-элемент равен $p_i p_j$, поэтому она тоже симметрична. Получаем:

			$$(\mu U^{-1} p' p U^{-1})^T = \mu (U^{-1})^T (p' p)^T (U^{-1})^T = \mu U^{-1} p' p U^{-1}$$

			Значит, матрица $\mu U^{-1} p' p U^{-1}$ симметрична. Т.к. сумма симметричных матриц является симметричной матрицей, то утвреждение доказано.
		\end{Proof}
	\item[$2)$]
		\textit{Имеет место соотношение: $p H = H p' = 0$}
		\begin{Proof}
			Докажем, что $p H = 0$ (второе свойство вытекает из симметричности матрицы $H$). Имеем:

			$$pH = \mu p U^{-1} p' p U^{-1} + p U^{-1} = -\frac{1}{p U^{-1}p'} (p U^{-1} p') p U^{-1} + p U^{-1} = 0$$
		\end{Proof}
	\item[$3)$]
		\textit{Матрица $H$ является полуотрицательно определенной, т.е. $\forall v \in \mathbb{R}^n \; v H v' \le 0$. Более того, $v H v' = 0 \; \Leftrightarrow$ векторы $v$ и $p$ коллинеарны.}
		\begin{Proof}
			Рассмотрим скалярное произведение с матрицей $-U^{-1}$ (она симметричная и положительно определенная), и пусть $w \in \mathbb{R}^n$ - вектор, являющийся ортогональной проекцией $v$ относительно введенного скалярного произведения на подпространство, ортогональное к $p$. Т.е. $v = \alpha p + w, \; \alpha \in \mathbb{R}$ и $w U^{-1} p' = 0$. Имеем:

			$$\begin{gathered}
				vHv' = (\alpha p + w)H(\alpha p' + w') = \alpha^2 pHp' + \alpha pHw' + \alpha wHp' + wHw' = \\
				= wHw' = w (\mu U^{-1} p' p U^{-1} + U^{-1}) w' = \\
				= \mu (w U^{-1}p')pU^{-1}w' + w U^{-1}w' = w U^{-1} w' \le 0
			\end{gathered}$$
			Равенство достигается тогда и только тогда, когда $w = 0$.
		\end{Proof}
\end{itemize}

\begin{conseq}[]\label{cha:3/conseq:1}
	Возрастание цены товара при соответствущей коменсации дохода приводит к снижению спроса на него:
	$$\left(\frac{\partial x_n^{*}}{\partial p_n}\right)_{comp} < 0$$
\end{conseq}
\begin{Proof}
	Т.к. $\left(\frac{\partial x^{*}}{\partial p_n}\right)_{comp} = \lambda^{*} [H]^{(n)}$, то $\left(\frac{\partial x_n^{*}}{\partial p_n}\right)_{comp} = \lambda^{*} h_{nn}$, где $h_{nn}$ - самый нижний диагональный элемент матрицы $H$.

	Пусть $e_i$ - базисный орт, тогда $h_{nn} = e_n H e_n$. Т.к. $p >> 0$, то $p$ и $e_n$ не коллинеарны, тогда по свойству $(3)$ матрицы Слуцкого имеем, что $h_{nn} < 0$.
\end{Proof}

\begin{definition}\label{cha:3/def:3}
	Назовем $n$-ый товар \red{ценным}, если $\frac{\partial x_n^{*}}{\partial K} > 0$, т.е. при увеличении дохода потребителя спрос на этот товар также увеличивается.

	Товар, не являющийся ценным, называется \red{малоценным}.
\end{definition}

\begin{conseq}[]\label{cha:3/conseq:2}
	Множество ценных товаров не пусто.
\end{conseq}
\begin{Proof}
	Это следует из уравнения $(6)$ и неотрицательности вектора $p$.
\end{Proof}

\begin{conseq}[]\label{cha:3/conseq:3}
	Спрос на ценные товары при повышении цены на него обязательно падает.
\end{conseq}
\begin{Proof}
	Это следует из того, что правая часть уравнения Слуцкого для $x_n^{*}$ отрицательна.
\end{Proof}

\begin{definition}\label{cha:3/def:4}
	Два товара $i$ и $j$ называются \red{взаимозаменяемыми}, если $\left(\frac{\partial x_j^{*}}{\partial p_i}\right)_{comp} > 0$, т.е. если при возрастании цены на $i$-ый товар при компенсирующем изменении дохода (с одновременным падением спроса на товар $i$) спрос на товар $j$ возрастает. 

	Если $\left(\frac{\partial x_j^{*}}{\partial p_i}\right)_{comp} < 0$, то товары $i$ и $j$ называют \red{взаимодополнительными}.
\end{definition}

\textbf{Пример}

Масло и маргарин являются взаимозаменяемыми продуктами, а бензин а автомобилями - взаимодополнительными.

\begin{conseq}[]\label{cha:3/conseq:4}
	Для каждого товара $i$ существует хотя бы один товар $j$, образующий с $i$ взаимозаменяемую пару.
\end{conseq}
\begin{Proof}
	Пусть без ограничения общности $i = n$. По следствию 3.1 имеем $\left(\frac{\partial x_n^{*}}{\partial p_n}\right)_{comp} < 0$. С другой стороны, было доказано, что $\left< \left(\frac{\partial x^{*}}{\partial p_n}\right)_{comp}, p \right> = 0$, а значит, т.к. $p >> 0$, получаем, что существует такое $j$, что $\left(\frac{\partial x_j^{*}}{\partial p_n}\right)_{comp} > 0$.
\end{Proof}
















