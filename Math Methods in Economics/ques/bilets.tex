\begin{center}
	{\Large \textbf{Программа экзамена по предмету}}
	
	{\Large \textbf{<<Математические методы в экономике>>}}
\end{center}

\begin{enumerate}
\item Элементы теории потребления. Пространство товаров. Множество потребления. Отношения. Отношение предпочтения и функция полезности.

\item Функция спроса. Основные задачи классической теории потребления.

\item Уравнение Слуцкого. Его следствия.

\item Правила голосования.

\item Функции коллективного выбора. Теорема Эрроу о диктаторе.

\item Замкнутая модель Леоньтева. Описание модели.

\item Элементы теории неотрицательных матриц. Теорема Фробениуса-Перрона.

\item Открытая линейная модель Леоньтева. Описание модели. Критерий существования решения в открытой модели.

\item Устойчивость замкнутой модели Леонтьева.

\item Нелинейная модель Леонтьева.

\item Обобщенная модель Леоньтева. Теорема о замещении.

\item Модель расширяющейся экономики фон Неймана. Сбалансированный рост в модели фон Неймана.

\item Модель Гейла сбалансированного роста. Существование состояния равновесия.

\item Альтернатива для систем линейных неравенств.

\item Теорема о системах линейных неравенств.

\item Невырожденные состояния равновесия в модели Неймана.

\item Теорема Моришимы о магистралях.

\item Теорема Раднера.
\end{enumerate}


 