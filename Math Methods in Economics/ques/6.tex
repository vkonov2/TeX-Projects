\chapter{Замкнутая модель Леоньтева. Описание модели.}\label{cha:6}

Пусть есть n стран, $\forall \; i: \; 1 \leq i \leq n, \; \pi_i$ -- национальный доход i-ой страны. $\pi = (\pi_1, \ldots, \pi_n)^T$ -- вектор доходов, $A = (a_{ij})$ -- матрица международного обмена.
\begin{definition}
	$(A\pi)_i = \sum\limits_{j = 1}^na_{ij}\pi_j$.
\end{definition}

\begin{definition}
	$\sum\limits_{j = 1}^na_{ij} = 1 \; \forall \; i$ -- \red{условие замкнутости матрицы А.}
\end{definition}

\begin{question}
	\textit{Возможен ли безубыточный обмен: $A\pi \geq \pi$?}
\end{question}

\begin{clair}
	A -- замкнутая, $A\pi \geq \pi \Rightarrow A\pi = \pi.$
\end{clair}

\begin{Proof}
	Пусть $A\pi > \pi. \; \sum\limits_{i = 1}^n\sum\limits_{j = 1}^n a_{ij}\pi_j > \sum\limits_{i = 1}^n\pi_i,$ но $\sum\limits_{i = 1}^n\sum\limits_{j = 1}^n a_{ij}\pi_j = \sum\limits_{j = 1}^n \pi_j \sum\limits_{i = 1}^n a_{ij} = \sum\limits_{j = 1}^n \pi_j$ -- противоречие.
\end{Proof}

\begin{definition}
	А -- замкнутая, $A\pi = \pi, \; \pi \geq 0$ -- \red{замкнутая линейная модель Леонтьева.}
\end{definition}
