\chapter{Неотр-ные матрицы. Т-ма Фробениуса-Перрона.}\label{cha:7}

Элементы теории неотрицательных матриц. Теорема Фробениуса-Перрона.

Пусть $A = (a_{ij})_{i, j = 1}^n \in Mat_{n \times n}$.\\

$A$ неотрицательная ($A \ge 0$), если $\forall i, j \; a_{ij} \ge 0$.

$A$ положительна ($A > 0$), если $A \ge 0$ и $A \not = 0$.

$A$ строго положительна ($A >> 0$), если $\forall i, j \; a_{ij} > 0$.\\

\textbf{Свойства}:
\begin{itemize}
	\item[$\bullet$]
		$A \ge 0, \; v \ge 0 \; \Rightarrow \; A v \ge 0$
	\item[$\bullet$]
		$A >> 0, \; v >> 0 \; \Rightarrow \; A v >> 0$
	\item[$\bullet$]
		$A > 0, \; v >> 0 \; \Rightarrow \; A v > 0$
\end{itemize}

\begin{definition}\label{cha:7/def:1}
	Матрица $A$ называется \textit{разложимой}, если $\exists S, T \subset \{1, \dots, n\}: \; S, T \not = \emptyset, \; S \bigcap T = \emptyset, \; S \bigcup T = \{1, \dots, n\}$ и $\forall i \in S, \forall j \in T: a_{ij} = 0$.
\end{definition}

\begin{definition}\label{cha:7/def:2}
	\textit{Перестановка рядов $i$ и $j$} = перестановка строк $i$ и $j$ $+$ перестановка столбцов $i$ и $j$.
\end{definition}

\begin{definition}\label{cha:7/def:3}
	Матрица $A$ \textit{разложима}, если она перестановкой рядов переводится к виду: $\begin{pmatrix}
		A_1 & A_2 \\
		0 & A_4
	\end{pmatrix}$ или $\begin{pmatrix}
		A_1 & 0 \\
		A_3 & A_4
	\end{pmatrix}$, где $A_1, A_4$ - квадртаные матрицы.
\end{definition}

\textbf{Пример}:
\begin{itemize}
	\item[$\bullet$]
		$\begin{pmatrix}
		1 & 0 \\
		0 & 1
	\end{pmatrix}$ - разложима
	\item[$\bullet$]
		$\begin{pmatrix}
		1 & 1 \\
		1 & 1
	\end{pmatrix}$, $\begin{pmatrix}
		0 & 1 \\
		1 & 0
	\end{pmatrix}$ - не разложимы
\end{itemize}

\begin{clair}[]\label{cha:7/clair:1}
	$A$ неразложима, тогда в $A$ нет нулевых строк и столбцов.
\end{clair}
\begin{Proof}
	От противного: пусть $i$-ая строка нулевая: $\forall j \; a_{ij} = 0$. Пусть $S = \{i\}, \; T = \{1, \dots, i-1, i+1, \dots, n\} \; \Rightarrow \; A$ разложима - противоречие. Аналогично для столбцов.
\end{Proof}

\begin{clair}[]\label{cha:7/clair:2}
	$A \ge 0 $ - неразложима, $x >> 0$, тогда $A x >> 0$.
\end{clair}
\begin{Proof}
	В $A$ нет нулевых строк $\Rightarrow \; \forall i \; \exists j_0: a_{i j_0} > 0 \; \Rightarrow \; (A x)_i = \underset{j=1}{\overset{n}{\sum}}a_{ij}x_j \ge a_{i j_0} x_{j_0} > 0$.
\end{Proof}

\begin{clair}[]\label{cha:7/clair:3}
	$A \ge 0$ - неразложима $\Rightarrow \; (I+A)^{n-1} >> 0$.
\end{clair}
\begin{Proof}
	Достаточно доказать, что $\forall x > 0: \; (I+A)^{n-1} \cdot x >> 0$.

	Пусть $x > 0, \; y := (I+A)x$. Если $x >> 0$, то $y = x+Ax \ge x \; \Rightarrow \; y >> 0$ и т.д., получаем требуемое.

	Докажем, что если $x \not\gg 0$, то $\Set{i}{y_i = 0} \subsetneq \Set{i}{x_i = 0}$. Имеем $y = x + A x \ge x \; \Rightarrow \; \Set{i}{y_i = 0} \subseteq \Set{i}{x_i = 0}$.

	Пусть $\Set{i}{y_i = 0} = \Set{i}{x_i = 0} = \{1, \dots, k$ (без ограничения общности). Тогда $x = (\underset{k}{0, \dots, 0}, \underset{n-k}{u})^T, \; u >> 0, \; y = (\underset{k}{0, \dots, 0}, \underset{n-k}{v})^T, \; v >> 0 \; \Rightarrow \; \begin{pmatrix}
		0 \\ \vdots \\ 0 \\ v
	\end{pmatrix} = \begin{pmatrix}
		0 \\ \vdots \\ 0 \\ u
	\end{pmatrix} + \begin{pmatrix}
		A_1 & A_2 \\ A_3 & A_4
	\end{pmatrix}\begin{pmatrix}
		0 \\ \vdots \\ 0 \\ u
	\end{pmatrix} \; \Rightarrow \; A_2 u = 0, u >> 0, \; A_2 \ge 0 \; \Rightarrow \; A_2 = 0$, т.е. разложима - противоречие, значит $x > 0$ и у $x$ не более $n-1$ нулевых компонент. Т.о. $y >> 0$ и $(I+A)^{n-1} \cdot x >> 0$.
\end{Proof}

\begin{conseq}[]\label{cha:7/conseq:1}
	$A \ge 0$ - неразложима $\Rightarrow \; \forall i, j \exists k (i, j) \le n: a_{ij}^{(k)} > 0$, где $A^k = (a_{ij}^{(k)})_{i, j = 1}^{n}$. 
\end{conseq}
\begin{Proof}
	Пусть $i \not = j \; \Rightarrow \; 0 << (I+A)^{n-1} = \underset{k=0}{\overset{n-1}{\sum}}C_{n-1}^k \cdot A^k \; \Rightarrow \; \underset{k=1}{\overset{n-1}{\sum}}a_{ij}^{(k)} > 0 \; \Rightarrow \; \exists k: C_{n-1}^k \cdot a_{ij}^{(k)} > 0 \; \Rightarrow \; a_{ij}^{(k)} > 0$.

	Пусть $i = j \; \Rightarrow \; A \cdot (I+A)^{n-1} >> 0$. Т.к. $A$ неразложима, то $(I+A)^{n-1} >> 0$. Имеем: $A \cdot (I+A)^{n-1} = \underset{k=0}{\overset{n-1}{\sum}}C_{n-1}^k \cdot A^{k+1} = \underset{k=1}{\overset{n}{\sum}}C_{n-1}^{k-1} \cdot A^k >> 0 \; \Rightarrow \; \dots \; \Rightarrow \; \exists k: a_{ij}^{(k)} > 0$. 
\end{Proof}

\textbf{Пример}: $A = \begin{pmatrix}
	0 & 1 \\ 1 & 0
\end{pmatrix}, \; A^{2k} = \begin{pmatrix}
	1 & 0 \\ 0 & 1
\end{pmatrix}, A^{2k+1} = \begin{pmatrix}
	0 & 1 \\ 1 & 0
\end{pmatrix}$.

\begin{theorem}[\red{Перрона-Фробениуса}]\label{cha:7/the:1}
	Пусть $A \ge 0$ - неразложимая матрица, тогда:
	\begin{itemize}
		\item[$1$)]
			$\exists \lambda_A > 0: \forall \lambda$ - с.з. $A: |\lambda| \le \lambda_A$
		\item[$2$)]
			$\lambda_A$ - с.з. $A$ кратности 1, называемое \blue{числом Фробениуса} и $\exists$ соответствующий с.в. $x_A >> 0: A x_A = \lambda_A \cdot x_A$, называемый \blue{вектором Фробениуса}.
	\end{itemize}
\end{theorem}
\begin{Proof}
	Пусть $x > 0$, $r(x) := \underset{i = \overline{1 n}}{\min} (A x)_i = \underset{i = \overline{1 n}}{\min} \; x_i$. Если $x_i = 0$, то считаем, что $\frac{(A x)_i}{x_i} = +\infty$.

	$r'(x) = \sup \Set{\rho}{\rho x \le A x}$. Тогда $r(x) = r'(x)$, т.к.:
	$$\begin{gathered}
		\forall i: r(x) \cdot x_i \le (A x)_i \; \Rightarrow \; r(x) \cdot x \le A x \; \Rightarrow \; r(x) \le r'(x) \\
		r'(x) \cdot x \le A x \; \Rightarrow \; \forall i \; r'(x) \cdot x_i \le (A x)_i \; \Rightarrow \; r'(x) \le \frac{(A x)_i}{x_i} \; \forall i \; \Rightarrow \; r'(x) \le r(x)
	\end{gathered}$$
	Т.о. имеем, что $r(x) = r'(x)$. Пусть $r := \underset{x > 0}{\sup} \; r(x)$, $M := \Set{x > 0}{\underset{i=1}{\overset{n}{\sum}}x_i = 1}$ - замкнутое и ограниченное множество (компакт).

	$$\begin{gathered}
		\forall x >0 \; \forall \lambda > 0 \; r (\lambda x) = r(x) \; \Rightarrow \; r = \underset{x \in M}{\sup} \; r(x) \\
		\text{т.к. } \forall x > 0 \exists \lambda > 0: \lambda x \in M
	\end{gathered}$$

	Пусть $x > 0$ и $Z = (I+A)^{n-1} \cdot x >> 0$. $r(x) \cdot x \le A x$, т.е. $(A - r(x) \cdot I) \cdot x \ge 0$. Значит имеем:
	$$\begin{gathered}
		0 \le (I+A)^{n-1} \cdot (A - r(x) \cdot I) \cdot x = (A - r(x) \cdot I) \cdot (I + A)^{n-1} \cdot x = \\
		= (A - r(x) \cdot I) \cdot Z \; \Rightarrow  \; r(x) \cdot Z \le A z \; \Rightarrow \; r(x) \le r(z)
	\end{gathered}$$
	$N:= \Set{(I+A)^{n-1} \cdot x}{x \in M}, \; \forall z \in N z >> 0$. $N$ - компакт, т.к. $M$ - компакт.

	Функция $r(x)$ непрерывна на $N \subset \mathbb{R}_{++}^{n} = \Set{x}{x >>0}$, тогда по теореме Вейерштрасса достигает своего $max$ и $min$ на компакте:
	$$\exists z \in N: r(z) = \underset{x \in N}{\sup} \; r(x) = \underset{x \in M}{\sup} \; r(x) = \underset{x > 0}{\sup} \; r(x) = r$$

	Имеем: $r = r(z), \; r \cdot z \le A z$. Докажем, что $r \cdot z = A z$. 

	От противного: пусть $r \cdot z < A z$, т.е. $(A - r I)\cdot z > 0$, тогда имеем:
	$$\begin{gathered}
		w := (I+A)^{n-1} \cdot Z \; \Rightarrow \; (A - r I)\cdot w = (A -r I) \cdot (I+A)^{n-1} \cdot z = \\
		= \underbrace{(I+A)^{n-1}}_{>>0} \underbrace{(A -r I) \cdot z}_{>0} >> 0 \; \Rightarrow \; r w \le A w \; \Rightarrow \;  r(w) > r \text{ - противоречие (} r \text{ - } \sup \text{)}
	\end{gathered}$$
	Значит имеем: $r\cdot z = A z$, т.е. $Z$ - с.в. $A$  с с.з. $r$.\\

	Теперь докажем, что $\forall$ с.з. $\lambda: |\lambda| \le r$.

	Пусть $A y = \lambda y, \lambda \in \mathbb{C}, \; y \in\mathbb{C}^n, y \not =0, \; |y| = \begin{pmatrix}
		|y_1| \\ \vdots \\ |y_n|
	\end{pmatrix}$.
	$$|\lambda| \cdot |y| = |\lambda y| = |Ay| \le |A| \cdot |y| = A \cdot |y| \; \Rightarrow \; |\lambda| \le r(|y|) \le r$$
	Докажем, что $z$ - единственный с точностью до пропорциональности с.в. с с.з. $r$. Пусть $A y = r y \; \Rightarrow \; A |y| \ge |A y| = r |y| \; \Rightarrow \; r \le r(|y|) \le r \; \Rightarrow \; r(|y|) = r$ и $A |y| = r|y|, \; |y| > 0$. Тогда:
	$$0 << (I+A)^{n-1}\cdot |y| = (1+r)^{n-1} \cdot |y| \; \Rightarrow \; |y| >> 0 \; \Rightarrow \; |y_i| > 0, \; y_i \not = 0$$
	Т.о. в $y$ нет нулевых координат.

	Пусть $y_1, y_2$ - с.в. с с.з. $r$ и они не пропорциональны, тогда $\exists \lambda, \mu \in \mathbb{C}: \; y = \lambda y_1 + \mu y_2 \not = 0, \; \exists i: y_i = 0$ - противоречие. Т.о. все с.в. с с.з. $r$ пропорциональны.
\end{Proof}

\begin{theorem}[]\label{cha:7/the:2}
	Пусть $A \ge 0$, тогда $\exists \lambda_A \ge 0$:
	\begin{itemize}
		\item[$1$)]
			$\forall$ с.з. $\lambda: |\lambda| \le \lambda_A$
		\item[$2$)]
			$\lambda_A$ - с.з. $A$
		\item[$3$)]
			$\exists x_A > 0: A x_A = \lambda_A \cdot x_A$
	\end{itemize}
\end{theorem}
\begin{Proof}
	Рассмотрим $B = \begin{pmatrix}
		1 & 1 \\ 1 & 1
	\end{pmatrix}$ и последовательность $A_n = A + \frac{1}{n} b$. $A \ge 0, B >>0 \; \Rightarrow \; \forall n A_n >> 0$ - неразложимая. $\underset{n \to \infty}{\lim} A_n = A \; \Rightarrow$ последовательность $\{A_n\}$ ограничена, значит, $||A_n||$ - ограничено.

	$\lambda_n = \lambda_{A_n} \le ||A_n|| \; \Rightarrow \; \lambda_n$ - ограниченная последовательность.

	Пусть $x_n$ - вектор Фробениуса, $x_n = x_{A_n}$. Можно считать, что $\forall n: ||x_n|| = 1$. Из того, что $\lambda_n$ и $x_n$ ограничены, то $\exists n_k$ - подпоследовательность: $\exists \underset{k \to \infty}{\lim} \lambda_{n_k} = \lambda \ge 0, \; \exists \underset{k \to \infty}{\lim} x_{n_k} = x \ge 0, ||x||=1 \; \Rightarrow \; x > 0$.

	$\forall n \; A_{n_k} x_{n_k} = \lambda_{n_k} \cdot x_{n_k} \; \Rightarrow \; Ax = \lambda x$ при $k \to \infty$.

	Пусть $\lambda'$ - с.з. $A$ и $y \not = 0$ - соответствующий с.в.: $Ay = \lambda' y \; \Rightarrow \; |\lambda'|\cdot |y| = |A y| \le |A| \cdot |y| = A |y| \le A |y| + \frac{1}{n} B |y| = A_n |y| \; \forall n$.

	Имеем: $|\lambda'| \le r_{A_n} (|y|) \le \lambda_n \; \Rightarrow \; \forall n_k: |\lambda'| \le \lambda_{n_k} \; \Rightarrow \; |\lambda'| \le \lambda$ при $k \to \infty$. Имеем $\lambda_A = \lambda, \; x_a = x$.
\end{Proof}


















