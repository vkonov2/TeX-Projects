\chapter{Неотр-ные матрицы. Т-ма Фробениуса-Перрона.}\label{cha:7}

Элементы теории неотрицательных матриц. Теорема Фробениуса-Перрона.

Пусть $A = (a_{ij})_{i, j = 1}^n \in Mat_{n \times n}$.\\

$A$ неотрицательная ($A \ge 0$), если $\forall i, j \; a_{ij} \ge 0$.

$A$ положительна ($A > 0$), если $A \ge 0$ и $A \not = 0$.

$A$ строго положительна ($A >> 0$), если $\forall i, j \; a_{ij} > 0$.\\

\textbf{Свойства}:
\begin{itemize}
	\item[$\bullet$]
		$A \ge 0, \; v \ge 0 \; \Rightarrow \; A v \ge 0$
	\item[$\bullet$]
		$A >> 0, \; v >> 0 \; \Rightarrow \; A v >> 0$
	\item[$\bullet$]
		$A > 0, \; v >> 0 \; \Rightarrow \; A v > 0$
\end{itemize}

\begin{definition}\label{cha:7/def:1}
	Матрица $A$ называется \textit{разложимой}, если $\exists S, T \subset \{1, \dots, n\}: \; S, T \not = \emptyset, \; S \bigcap T = \emptyset, \; S \bigcup T = \{1, \dots, n\}$ и $\forall i \in S, \forall j \in T: a_{ij} = 0$.
\end{definition}

\begin{definition}\label{cha:7/def:2}
	\textit{Перестановка рядов $i$ и $j$} = перестановка строк $i$ и $j$ $+$ перестановка столбцов $i$ и $j$.
\end{definition}

\begin{definition}\label{cha:7/def:3}
	Матрица $A$ \textit{разложима}, если она перестановкой рядов переводится к виду: $\begin{pmatrix}
		A_1 & A_2 \\
		0 & A_4
	\end{pmatrix}$ или $\begin{pmatrix}
		A_1 & 0 \\
		A_3 & A_4
	\end{pmatrix}$, где $A_1, A_4$ - квадртаные матрицы.
\end{definition}

\textbf{Пример}:
\begin{itemize}
	\item[$\bullet$]
		$\begin{pmatrix}
		1 & 0 \\
		0 & 1
	\end{pmatrix}$ - разложима
	\item[$\bullet$]
		$\begin{pmatrix}
		1 & 1 \\
		1 & 1
	\end{pmatrix}$, $\begin{pmatrix}
		0 & 1 \\
		1 & 0
	\end{pmatrix}$ - не разложимы
\end{itemize}

\begin{clair}[]\label{cha:7/clair:1}
	$A$ неразложима, тогда в $A$ нет нулевых строк и столбцов.
\end{clair}

\begin{clair}[]\label{cha:7/clair:2}
	$A \ge 0 $ - неразложима, $x >> 0$, тогда $A x >> 0$.
\end{clair}

\begin{clair}[]\label{cha:7/clair:3}
	$A \ge 0$ - неразложима $\Rightarrow \; (I+A)^{n-1} >> 0$.
\end{clair}

\begin{conseq}[]\label{cha:7/conseq:1}
	$A \ge 0$ - неразложима $\Rightarrow \; \forall i, j \exists k (i, j) \le n: a_{ij}^{(k)} > 0$, где $A^k = (a_{ij}^{(k)})_{i, j = 1}^{n}$. 
\end{conseq}

\textbf{Пример}: $A = \begin{pmatrix}
	0 & 1 \\ 1 & 0
\end{pmatrix}, \; A^{2k} = \begin{pmatrix}
	1 & 0 \\ 0 & 1
\end{pmatrix}, A^{2k+1} = \begin{pmatrix}
	0 & 1 \\ 1 & 0
\end{pmatrix}$.

\begin{theorem}[\red{Перрона-Фробениуса}]\label{cha:7/the:1}
	Пусть $A \ge 0$ - неразложимая матрица, тогда:
	\begin{itemize}
		\item[$1$)]
			$\exists \lambda_A > 0: \forall \lambda$ - с.з. $A: |\lambda| \le \lambda_A$
		\item[$2$)]
			$\lambda_A$ - с.з. $A$ кратности 1, называемое \blue{числом Фробениуса} и $\exists$ соответствующий с.в. $x_A >> 0: A x_A = \lambda_A \cdot x_A$, называемый \blue{вектором Фробениуса}.
	\end{itemize}
\end{theorem}

\begin{theorem}[]\label{cha:7/the:2}
	Пусть $A \ge 0$, тогда $\exists \lambda_A \ge 0$:
	\begin{itemize}
		\item[$1$)]
			$\forall$ с.з. $\lambda: |\lambda| \le \lambda_A$
		\item[$2$)]
			$\lambda_A$ - с.з. $A$
		\item[$3$)]
			$\exists x_A > 0: A x_A = \lambda_A \cdot x_A$
	\end{itemize}
\end{theorem}

















