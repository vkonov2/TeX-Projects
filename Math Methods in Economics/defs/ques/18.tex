\chapter{Теорема Раднера.}\label{cha:18}

Пусть $Z \subset \mathbb{R^{2n}_+}$ -- модель Гейбла и $u: \mathbb{R}^n \to \mathbb{R}$ -- функция полезности, $z = (x, y) \in \mathbb{R}^n \times \mathbb{R}^n.$

$z_1, \ldots, z_T \in Z$ называется траекторией, если $\forall \; t \; x_{t + 1} \leq y_t.$

\begin{problem}
	$$\begin{cases}
		u(y_T) \to \max\\
		z_t \in Z\\
		x_{t + 1} \leq y_t\\
		y_0 \leq x_1
	\end{cases}$$ Решение этой задачи -- оптимальная траектория. 
\end{problem}

Рассмотрим условия:

\begin{enumerate}
	% \item $ \vec{z} = (\vec{x}, \alpha \vec{x}) \in Z, \; \vec{x} > 0.$
	\item $ z = (x, \alpha x) \in Z, \; x > 0.$
	% \item $\exists p > 0: \; \forall \; z = (x, y) \in Z, \; z \neq \lambda \vec{z}: \; \lambda p x > py.$
	\item $\exists p > 0: \; \forall \; z = (x, y) \in Z, \; z \neq \lambda z: \; \lambda p x > py.$
	\item $\forall \; x \gg 0 \; \exists L > 0: \; (x, Lx) \in Z.$
	\item u -- непрерывна и неортицательна.
	\item u -- однородная степени 1: $u(\lambda y) = \lambda u(y) \forall \; \lambda > 0.$
	\item u(x) > 0.
	\item $\exists k > 0: \; \forall \; y \geq 0: \; u(y) \leq kpy.$
\end{enumerate}

\begin{sign}
	$S(\varepsilon, T, \{ z_t \}) = \#\{ t = 1, \ldots, T | s(y_t, x) \geq \varepsilon \}.$
\end{sign}

\begin{lemma}
	$C_{\varepsilon} := \{ (x, y) \in z | s(x, y) \geq \varepsilon \} \Rightarrow \exists \delta > 0: \; \forall (x,y) \in C_{\varepsilon}: \; (\alpha - \delta)px \geq py.$
\end{lemma}

\begin{theorem}[Раднера]
	Если модель Гейбла $Z$ удовлетворяет условиям 1)-7), то х -- слабая магистраль, то есть $\forall \; \varepsilon > 0 \; \exists Q(\varepsilon) = Q: \; \forall$ оптимальных траекторий $\{ x_t \}: \; S(\varepsilon, T, \{ z_t \}) \leq Q.$
\end{theorem}