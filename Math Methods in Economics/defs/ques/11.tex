\chapter{Обобщенная модель Леоньтева. Теорема о замещении.}\label{cha:11}

Имеем n товаров, $m \geq n$ отраслей.

Отрасль производит 1 товар, i-ому товару в соответствие поставим $M_i$ -- набор отраслей, производящих товар i. $\hat{I} = (e_{ij}), e_{ij} = 1, \text{ если } j \in M_i, \text{ иначе } e_{ij} = 0, \; \hat{A} = (a_{ij})$ -- матрица прямых затрат. $a_{ij}$ -- количество i-ого товара, необходимого для производства единицы товара j-ой отраслью, $x = (x_1, \ldots, x_n)^T$ -- вектор интенсивностей, $\hat{I}x$ -- вектор выпуска, $\hat{A}x$ -- вектор затрат.

\begin{definition}[Обобщенная модель Леонтьева]
	$\hat{I}x - \hat{A}x = c, \; x \geq 0$ -- обобщенная модель Леонтьева.
\end{definition}

\begin{definition}
	$l = (l_1, \ldots, l_n)$ -- вектор трудовых затрат, lx -- величина трудовых затрат.
\end{definition}

\begin{problem}
	$$\begin{cases}
		lx \to min\\
		\hat{I}x - \hat{A}x \geq c\\
		x \geq 0
	\end{cases}$$ -- задача минимизации трудовых затрат.
\end{problem}

\begin{definition}[Подмодель]
	$\sigma \subset \{1, \ldots, m \}$ -- подмодель, если $\sigma = \{j_1, \ldots, j_n \},\\ 
	j_k \in M_k, \; A_{\sigma} = ((a_{\sigma})_{ik})_{i,k = 1}^n, \; (a_{\sigma})_{ik} = a_{ij_k}.$
\end{definition}

\begin{theorem}[Самуэльсона о замещении]
	Пусть обобщенная модель Леонтьева с матрицей $\hat{A}$ продуктивна, тогда существует подмодуль $\sigma$ такой, что среди решений задачи минимизации трудовых затрат есть решение $\hat{x}: \; \hat{x}_j = 0 \; \forall \; j \notin \sigma, \; \hat{x}_j > 0 \; j \in \sigma.$ При этом $\sigma$ зависит от l и не зависит от c.
\end{theorem}