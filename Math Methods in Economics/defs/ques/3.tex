\chapter{Уравнение Слуцкого. Его следствия.}\label{cha:3}

Изучим поведение потребителя, стесненного бюджетными ограничениями.

Пусть потребительноское множество $X = \mathbb{R}_{+}^n$, функция полезности $u(x)$.

Пусть $p$ - система цен, $K$ - капитал потребителя.

Из свойств функции $u(x)$ вытекает, что функция спроса $\Phi(p, K)$ является однозначной, и при заданных $p$ и $K$ единственное значение $x^{*}(p,K)$ функции $\Phi(p,K)$ определяется следующей задачей математического программирования:
$$\begin{gathered}
	u(x) \to max \\
	< p,x > = K, \; x \ge 0
\end{gathered}$$
Кроме этого, имеем $x^{*} (p, K) >> 0$, тогда для определения точки $x^{*}(p,K)$ воспользуемся теоремой Лагранжа. Выпишем функцию Лагранжа:
$$L(x, \lambda) = u(x) - \lambda \left( <p,x> - K \right)$$
Тогда сущетсвует такое $\lambda^{*}$, что:
$$<p, x^{*}> - K = 0\eqno(1)$$
$$\frac{\partial u}{\partial x_i}(x^{*}) - \lambda^{*} p_i = 0, \; i = \ton n\eqno(2)$$

Заметим, что уравнение $(2)$ - это условие того, что бюджетная плоскость касается поверхности уровня функции полезности (градиент функции полезности сонаправлен с нормалью $p$ к бюджетной плоскости).

Рассмотрим теперь функцию $x^{*}(p,K)$ как функцию от $p$, подставив вместо $K$ соответствующую функцию $K(p)$.

\begin{definition}\label{cha:3/def:1}
	Полную производную функции $x^{*}$ по $p_j$, т.е. величину $\frac{\partial x^{*}}{\partial p_i} = \frac{\partial x^{*}}{\partial K} \cdot \frac{\partial K}{\partial p_i}$, называется \red{компенсированной производной} по $p_i$ и обозначается $\left(\frac{\partial x^{*}}{\partial p_i}\right)_{comp}$.
\end{definition}

\begin{theorem}[]\label{cha:3/the:1}
	Имеет место соотношение:
	$$\frac{\partial x^{*}}{\partial p_n} = \left(\frac{\partial x^{*}}{\partial p_n}\right)_{comp} - \left(\frac{\partial x^{*}}{\partial K}\right) x_n^{*}\eqno(12)$$
	Данное уравнение называется \red{уравнением Слуцкого}.
\end{theorem}

\textbf{Замечение}\\

Выражение для $\left(\frac{\partial x^{*}}{\partial p_n}\right)_{comp}$, полученное при выражении уравнения Слуцкого, можно пеперписать так:
$$\left(\frac{\partial x^{*}}{\partial p_n}\right)_{comp} = \lambda^{*} \left[ \mu U^{-1} p' p U^{-1} + U^{-1} \right]^{(n)}$$
где $p'$ обозначает вектор $p$, рассмтариваемый как столбец, в отличие от вектора $p$, рассматриваемого как вектор-строка.

\begin{definition}\label{cha:3/def:2}
	Матрица $H = \mu U^{-1} p' p U^{-1} + U^{-1}$ называется \red{матрицей Слуцкого}.
\end{definition}

\textbf{Свойства матрицы Слуцкого}:
\begin{itemize}
	\item[$1)$]
		\textit{матрица $H$ симметрична}
	\item[$2)$]
		\textit{Имеет место соотношение: $p H = H p' = 0$}
	\item[$3)$]
		\textit{Матрица $H$ является полуотрицательно определенной, т.е. $\forall v \in \mathbb{R}^n \; v H v' \le 0$. Более того, $v H v' = 0 \; \Leftrightarrow$ векторы $v$ и $p$ коллинеарны.}
\end{itemize}

\begin{conseq}[]\label{cha:3/conseq:1}
	Возрастание цены товара при соответствущей коменсации дохода приводит к снижению спроса на него:
	$$\left(\frac{\partial x_n^{*}}{\partial p_n}\right)_{comp} < 0$$
\end{conseq}

\begin{definition}\label{cha:3/def:3}
	Назовем $n$-ый товар \red{ценным}, если $\frac{\partial x_n^{*}}{\partial K} > 0$, т.е. при увеличении дохода потребителя спрос на этот товар также увеличивается.

	Товар, не являющийся ценным, называется \red{малоценным}.
\end{definition}

\begin{conseq}[]\label{cha:3/conseq:2}
	Множество ценных товаров не пусто.
\end{conseq}

\begin{conseq}[]\label{cha:3/conseq:3}
	Спрос на ценные товары при повышении цены на него обязательно падает.
\end{conseq}

\begin{definition}\label{cha:3/def:4}
	Два товара $i$ и $j$ называются \red{взаимозаменяемыми}, если $\left(\frac{\partial x_j^{*}}{\partial p_i}\right)_{comp} > 0$, т.е. если при возрастании цены на $i$-ый товар при компенсирующем изменении дохода (с одновременным падением спроса на товар $i$) спрос на товар $j$ возрастает. 

	Если $\left(\frac{\partial x_j^{*}}{\partial p_i}\right)_{comp} < 0$, то товары $i$ и $j$ называют \red{взаимодополнительными}.
\end{definition}

\textbf{Пример}

Масло и маргарин являются взаимозаменяемыми продуктами, а бензин а автомобилями - взаимодополнительными.

\begin{conseq}[]\label{cha:3/conseq:4}
	Для каждого товара $i$ существует хотя бы один товар $j$, образующий с $i$ взаимозаменяемую пару.
\end{conseq}
















