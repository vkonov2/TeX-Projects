\chapter{Открытая линейная модель Леоньтева. Описание модели. Критерий существования решения в открытой модели.}\label{cha:8}

Пусть у нас есть n отраслей, все выпускают разные товары.

\begin{definition}
	\begin{gather*}
		\begin{pmatrix}
		  \bar{a}_{11} & \ldots & \bar{a}_{1n} & c_1\\
		  \ldots & \ldots & \ldots & \ldots\\
		  \bar{a}_{n1} & \ldots & \bar{a}_{nn} & c_n\\
		  v_1 & \ldots & v_n &  \\
		\end{pmatrix}
	\end{gather*} -- балансовая матрица, где $\bar{a}_{ij}$ -- количество i-ого товара, потребляемого j-ой отраслью, $v_i$ -- валовый выпуск i-ой отрасли, $c_i$ -- количество i-ого товара, потребляемого непроизводственным сектором.

	$v_i = \sum\limits_{j = 1}^n \bar{a}_{ij} + c_i$ -- балансовые соотношения, $a_{ij} \dfrac{\bar{a}_{ij}}{v_j}$ -- количество i-ого товара, необходимого для производства единицы товара j-ой отраслью. $a_{ij}$ -- коэффициенты пряых затрат, $A = (a_{ij})$ -- матрица прямых затрат/
\end{definition}

\begin{definition}
	$v_i = \sum\limits_{j = 1}^n \bar{a}_{ij} + c_i, \; i = \bar{1,n} \Leftrightarrow v = Av + c, $ где A -- матрица прямых затрат, $c = (c_1, \ldots, c_n)^T$(вектор спроса), $v = (v_1, \ldots, v_n)^T$(вектор интенсивности), Av -- вектор прямых затрат. $v \geq 0, \; v = Av + c$ -- \red{открытая модель Леонтьева}.
\end{definition}

\begin{definition}
	Модель продуктивна, если $\forall \; c \geq 0 \; \exists \; v \geq 0: \; v = Av + c.$ То есть, пусть 

	\begin{equation*}
		\begin{cases}
			x - Ax = c \\
			x \geq 0
		\end{cases} 
	\end{equation*}--
	открытая модель Леонтьева. Модель продуктивна, если $\forall \; c \geq 0 \; \exists $ решение системы.
\end{definition}

\begin{theorem}
	Модель Леонтьева продуктивна $\Leftrightarrow \lambda_A < 1.$
\end{theorem}

\begin{clair}
	x > 0: x - Ax $\gg 0 \Rightarrow$ модель продуктивна.
\end{clair}

\begin{clair}
	А -- неразложима и $\exists x > 0: \; x - Ax = c > 0 \Rightarrow $ модель продуктивна.
\end{clair}

\begin{clair}
	$\exists p > 0: \; p(I - A) \gg 0 \Rightarrow $ модель продуктивна.
\end{clair}