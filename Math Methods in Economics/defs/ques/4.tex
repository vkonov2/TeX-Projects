\chapter{Правила голосования.}\label{cha:4}

Пусть $M = \{ x_1, \ldots, x_m \}$ -- множество кандидатов, $S = \{ y_1, \ldots, y_m \}$ -- множество избирателей и решение принимается голосованием. У каждого избирателя $y_k$ есть схема предпочтений: $x_{i_1} \stackrel{k}{\succ} \ldots \stackrel{k}{\succ} x_{i_m}$.

Требуется построить функцию, определяющую коллективный порядок на множестве кандидатов, то есть правило, которое для любых заданных порядков $\stackrel{1}{\succ}, \ldots \stackrel{k}{\succ}$ определяет \red{коллективный порядок} $\succeq$:

\begin{gather*}
	\succeq = f(\stackrel{1}{\succ}, \ldots \stackrel{k}{\succ}).
\end{gather*}

\begin{definition}
	$a$ -- победитель, если $\forall \; b \in M: \; a \succeq b, \; a = p (\stackrel{1}{\succ}, \ldots \stackrel{n}{\succ})$ -- правило голосования.
\end{definition}

\begin{definition}
	$\{ \stackrel{1}{\succ}, \ldots \stackrel{n}{\succ} \}$ -- профиль голосования, то есть множество всех индивидуальных предпочтений.
\end{definition}

\begin{example}
	Пусть имеется 10 избирателей и три кандидата: a, b, c. Тогда профиль голосования удобно представить таблицей:\vspace{0.5cm}

	\begin{tabular}{ | l | l | l | l | l | }
		\hline
			кол-во избирателей & 2 & 3 & 5 \\ \hline
			кандидаты & a & b & c \\
			кандидаты & b & a & b \\
			кандидаты & c & c & a \\
		\hline
	\end{tabular}
\end{example}

\begin{question}
	\textit{Как определить победителя?}
\end{question}

\begin{enumerate}
	\item \underline{Метод относительного большинства.}

	Каждый избиратель отдает голос ровно за одого кандидата, победит тот, кто наберет наибольшее число голосов.

	\item \underline{Метод абсолютного большинства.}

	Каждый избиратель голосует ровно за одного, побеждает тот, кто набрал > 50% голосов. Если такого нет или таких больше одного, проводят второй тур между кандидатами с наибольшим числом голосов.

	\item \underline{Метод Борда.}

	$y_k: \; \underset{m-1 \text{ балл}}{x_{i_1}} \stackrel{k}{\succ} \ldots \stackrel{k}{\succ} \underset{0 \text{ балл}}{x_{i_m}}$. Тогда каждому кандидату припишем балл по такой схеме: $x_{i_l} \rightarrow \alpha_{k, i_l} = m - l, $ тогда:

	\begin{gather*}
		\alpha_j = \sum \limits_{k = 1}^n \alpha_{k, i_l} \text{ -- сумма баллов для } x_j.
	\end{gather*}

	Тогда победитель - это тот, кто имеет наибольшее $\alpha_j.$

	\item \underline{Обобщенный метод Борда.}

	$y_k: \; x_{i_1} \stackrel{k}{\succ} \ldots \stackrel{k}{\succ} x_{i_m}, \; x_{i_l} \rightarrow \alpha_{k, i_l} = s_{m - l}: \; 0 = s_0 \leq s_1 \leq \ldots \leq s_{m - 1} > 0,$ дальше аналогично методу Барда из пункта 3.

	\begin{remark}
		\begin{itemize}
			\item $s_{m - l} = m - l$ -- обычный метод Борда.
			\item $s_0 = \ldots = s_{m - 2} = 0, \; s_{m - 1} = 1$ -- метод относительного большинства.
		\end{itemize}
	\end{remark}

	\item \underline{Метод Кондорсе.}

	$a, \; b \in M $ -- кандидаты. $K_{a, b}$ -- количество избирателей, считающих a лучше b: 

	\begin{gather*}
		r_a := \{ \#(b \in M\setminus\{ a \}) \mid K_{a, b} \geq K_{b, a} \},
	\end{gather*}

	где $\#$ -- количество. Тогда победитель -- кандидат с наибольшим $r_a$.
\end{enumerate}

\begin{example}

	\begin{tabular}{ | l | l | l | l | }
		\hline
			5 & 3 & 5 & 4 \\ \hline
			a & a & b & c \\
			d & d & c & d \\
			c & b & d & b \\
			b & c & a & a \\
		\hline
	\end{tabular}\vspace{0.5cm}
	\begin{enumerate}
		\item $a \rightarrow 8, \; b \rightarrow 5, \; c \rightarrow 4, \; d \rightarrow 0 \Rightarrow $ победитель -- а.
		\item $8 < \dfrac{17}{2} \Rightarrow $ проводим второй тур между $a, \; b:$

		\begin{table}[h!]
		\caption{Второй тур}
		\begin{center}
		\begin{tabular}{ | l | l | l | l | }
			\hline
				5 & 3 & 5 & 4 \\ \hline
				a & a & b & b \\
				b & b & a & a \\
			\hline
		\end{tabular}
		\end{center}
		\end{table}\vspace{0.5cm}

		Тогда победитель -- b.\vspace{0.5cm}

		\item 

		\begin{tabular}{ | l | l | l | l | l | }
		\hline
			5 & 3 & 5 & 4 & баллы\\ \hline
			a & a & b & c & 3 \\
			d & d & c & d & 2 \\
			c & b & d & b & 1 \\
			b & c & a & a & 0 \\
		\hline
		\end{tabular}\vspace{0.5cm}

		$\alpha_a = 5 \cdot 3 + 3 \cdot 3 + 5 \cdot 0 + 4 \cdot 0 = 24$

		$\alpha_b = 5 \cdot 0 + 3 \cdot 1 + 5 \cdot 3 + 4 \cdot 1 = 22$

		$\alpha_c = 5 \cdot 1 + 3 \cdot 0 + 5 \cdot 2 + 4 \cdot 3 = 27$

		$\alpha_d = 5 \cdot 2 + 3 \cdot 2 + 5 \cdot 1 + 4 \cdot 2 = 29$\vspace{0.5cm}

		Тогда победитель -- d.

		\item -

		\item $a \preceq b: \; 8:9$

		$a \preceq c: \; 8:9$

		$b \preceq c: \; 8:9$

		$d \preceq c: \; 8:9$

		$a \preceq d: \; 8:9$

		$b \preceq d: \; 5:12$\vspace{0.5cm}

		Поэтому $r_a = 0, \; r_b = 1, \; r_c = 3, \; r_d = 2 \Rightarrow $ победитель -- c.

	\end{enumerate}	

	\begin{clair}
		Методы 1), 2), 3), 5) различны.
	\end{clair}
\end{example}\newpage

\begin{example}

	\begin{tabular}{ | l | l | l | l | }
		\hline
			3 & 6 & 4 & 4\\ \hline
			c & a & b & b\\
			a & b & a & c\\
			b & c & c & a\\
		\hline
	\end{tabular}\vspace{0.5cm}

	5). $b \preceq a: \; 8:9$

	$c \preceq a: \; 7:10$\vspace{0.5cm}

	Победитель -- a.\vspace{0.5cm}

	4). $\alpha_a = 6s_2 + 7s_1$

	$\alpha_b = 8s_2 + 6s_1$

	$\alpha_a = 3s_2 + 4s_1$\vspace{0.5cm}

	Пусть побеждает a $ \Rightarrow 6s_2 + 7s_1 \geq 8s_2 + 6s_1 \Rightarrow s_1 \geq 2s_2$ и $ 6s_2 + 7s_1 \geq 3s_2 + 4s_1 \Rightarrow s_2 \geq s_1.$ Тогда

	\begin{gather*}
		s_1 \geq 2s_2 \geq 2s_1 \Rightarrow s_1 = s_2 \text{ -- противоречие.}
	\end{gather*}

	\begin{clair}
		Методы 2), 5) не являются частными случаями обобщенного метода Борда.
	\end{clair}

\end{example}