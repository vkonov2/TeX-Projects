\begin{center}
	{\Large \textbf{Программа экзамена по предмету}}
	
	{\Large \textbf{<<Уравнения с частными производными>>}}
\end{center}

\begin{enumerate}
\item Линейные уравнения с частными производными второго порядка. Понятие характеристики. [4], § 2.2. Классификация
уравнений второго порядка. Приведение к каноническому виду в случае двух независимых переменных. [1], глава I, §
1(1). Приведение к каноническому виду уравнений с постоянными коэффициентами в случае трех и более независимых
переменных. [5], приложение.

\item Задача Коши для линейного уравнения второго порядка. Теорема Коши-Ковалевской (без доказательства). [2], § 1.4(7).

\item Корректность постановки задачи. Пример Адамара некорректной задачи. [2], § 1.4(6,8).

\item Задача Коши для уравнения струны. Формула Даламбера. [5], глава I, § 2 Решение неоднородного уравнения, принцип
Дюамеля. [2], § 5.1.6.

\item Полуограниченная струна. Методы четного и нечетного продолжения. Решение задачи для полуограниченной струны с
неоднородным граничным условием. [5], глава I, § 5

\item Задача Коши для волнового уравнения в R2 и R3 . Энергетическое неравенство. Единственность решения задачи
Коши. [4], § 5.1.1.

\item Формула Кирхгофа решения задачи Коши для волнового уравнения в $ \mathbb{R}^3 $ . [4], § 5.1.2.

\item Формула Пуассона решения задачи Коши для волнового уравнения в $ \mathbb{R}^2 $ . Метод спуска. [4], § 5.1.3.

\item Ограниченная струна. Метод Фурье. [5], глава II, § 6, 7.2, 8.2.

\item Принцип максимума для уравнения теплопроводности в ограниченной области. Единственность решения первой
краевой задачи. [1], глава III, § 1(5,6).

\item Задача Коши для уравнения теплопроводности. Принцип максимума в неограниченной области. Единственность
решения задачи Коши в классе ограниченных функций. [1], глава III, § 1(7).

\item Формула Пуассона решения задачи Коши для уравнения теплопроводности. [1], глава III, § 3(1).

\item Метод Фурье для уравнений Лапласа и Пуассона в круге и кольце. [5], глава II, § 9.2.

\item Обобщенные функции. Действия над обобщенными функциями. [2], § 2.1, 2.2. Фундаментальное решение линейного
дифференциального оператора с постоянными коэффициентами. [2], § 3.1(1,2).

\item Формулы Грина. [4], § 3.1.

\item Фундаментальное решение оператора Лапласа в $ \mathbb{R}^2 $ и $ \mathbb{R}^3 $ . [4], § 3.2.

\item Функция Грина оператора Лапласа, ее симметрия. Представление решения задачи Дирихле через функцию Грина. [4], §
3.6. Метод отражений. Метод конформных отображений. [5], глава IV, § 4, 5

\item Свойства гармонических функций: теорема о потоке, теоремы о среднем по сфере и по шару, принцип максимума,
теорема Лиувилля. [4], § 3.5.

\item Краевые задачи для уравнений Лапласа и Пуассона. [4], § 3.4. Единственность решения задачи Дирихле. [4], § 3.5.
Задача Неймана: условие разрешимости [6], § 5.2, теорема о множестве решений [5], глава IV, § 2(8).
\end{enumerate}


 