\chapter{Линейные уравнения с частными производными второго порядка. Понятие характеристики. Классификация уравнений второго порядка. Приведение к каноническому виду в случае двух независимых переменных. Приведение к каноническому виду уравнений с постоянными коэффициентами в случае трех и более переменных.}
\label{cha:1}

Пусть $u(x,y)$ - неизвестная функция.

\begin{definition}\label{cha:1/def:1}
	Общий вид линейного уравнения с частными производными $\RNumb{2}$ порядка:
	$$a(x,y) u_{xx} + 2 b(x,y) u_{xy} + c(x,y) u_{yy} + d(x,y) u_x + f(x,y) u_y + h(x,y) u = m (x,y)$$
	где $a,b,c,d,f,m,h$ - функции от $(x,y)$.
\end{definition}

\begin{definition}\label{cha:1/def:2}
	Сделаем замену в уравнении: $\frac{\partial}{\partial x} \to \lambda, \; \frac{\partial}{\partial y} \to \mu$. Тогда \blue{характеристическим многочленом} (\blue{главным символом}) называется $A (\lambda, \mu) = a \lambda^2 + 2 b \lambda \mu + c \mu^2$.
\end{definition}

\begin{definition}\label{cha:1/def:3}
	Линия $y = y(x)$ называется \red{характеристикой}, если на нормали к ней в каждой точке главный символ равен нулю:
	$$A(\overrightarrow{n}) = A (dy, -dx) = a (dy)^2 - 2 dy dx + c (dx)^2 = 0$$
\end{definition}

$\textbf{Классификация уравнений второго порядка.}$

\begin{itemize}
	\item[$\bullet$] если $a \not \equiv 0$, то делим на $(dx)^2$
	\item[$\bullet$] если $a \equiv 0$ и $c \not \equiv 0$, то делим на $(dy)^2$
\end{itemize}

$$a (y')^2 - 2 b y' + c = 0, \; \frac{D}{4} = \delta (x,y) = b^2 - a c$$

\begin{itemize}
	\item[$\bullet$] если $\delta (x,y) > 0$, то уравнение гиперболического типа
	\item[$\bullet$] если $\delta (x,y) < 0$, то уравнение эллиптического типа
	\item[$\bullet$] если $\delta (x,y) = 0$, то уравнение параболического типа
\end{itemize}

\vspace{0.5cm}
$\textbf{Случай двух переменных}: \; u(x,y)$

\begin{enumerate}
	\item \red{$\delta (x,y) > 0$}, \textit{гиперболический тип}\\

	Квадратное уравнение имеет два корня: $y' = f_1 (x,y)$ и $y' = f_2 (x,y)$. \\
	Решаем эти ДУ и записываем решения в виде $\begin{cases}
		\psi_1 (x,y) = C_1 \\ \psi_2 (x,y) = C_2
	\end{cases}.$\\
	\begin{itemize}
		\item[$\bullet$] замена: $\begin{cases}
			\xi = \psi_1 (x,y) \\ \eta = \psi_2 (x,y)
		\end{cases}$\\
		Тогда $\overline{a} = \overline{c} \equiv 0, \; \overline{u_{\xi \eta}} = F(\xi, \eta, \overline{u}, \overline{u_{\xi}}, \overline{u_{\eta}})$ - \blue{$\RNumb{1}$-ый канонический вид гиперболического типа}
		\item[$\bullet$] замена: $\begin{cases}
			\xi = \psi_1 + \psi_2 \\ \eta = \psi_1 - \psi_2
		\end{cases}$\\
		Тогда $\overline{a} = -\overline{c}, \; \overline{b} \equiv 0, \; \overline{u_{\xi \xi}} - \overline{u_{\eta \eta}} = G(\xi, \eta, \overline{u}, \overline{u_{\xi}}, \overline{u_{\eta}})$ - \blue{$\RNumb{2}$-ой канонический вид гиперболического типа}
	\end{itemize}
	\item \red{$\delta (x,y) < 0$}, \textit{эллиптический тип}\\

	$D < 0$, действительных корней нет, тогда находим комплексные: $y' = f (x,y), \; y' = f^{*} (x,y)$, $f, f^{*}$ - комплексно сопряженные.\\
	Записываем решения ДУ в виде первых интегралов: $\begin{cases}
		\varphi (x,y) = C_1 \\ \varphi^{*} (x,y) = C_2
	\end{cases}$\\
	Замена: $\begin{cases}
		\xi = \mathbb{R}e \varphi (x,y) = \frac{1}{2} (\varphi + \varphi^{*}) \\
		\eta = \mathbb{I}m \varphi (x,y) = \frac{1}{2 i} (\varphi - \varphi^{*})
	\end{cases}$\\
	Тогда $\overline{a} = \overline{c}, \; \overline{b} \equiv 0, \; \overline{u_{\xi \xi}} + \overline{u_{\eta \eta}} = H(\xi, \eta, \overline{u}, \overline{u_{\xi}}, \overline{u_{\eta}})$ - \blue{канонический вид эллиптического типа}
	\item \red{$\delta (x,y) = 0$}, \textit{параболический тип}\\

	$D = 0$, имеем ровно один корень $y' = f(x,y)$.\\
	Решаем ДУ и решение записываем в виде $\varphi(x,y) = C$.\\
	Замена: $\begin{cases}
		\xi = \varphi (x,y) \\ \eta = \psi (x,y)
	\end{cases}$, где $\psi$ - любая функция, независимая с $\varphi$, т.е. якобиан $J \not = 0$.\\
	При невырожденной замене знак $D$ сохраняется: $\overline{\delta} = \delta J^2, \; \overline{\delta} = \overline{b}^2 - \overline{a} \overline{c} \equiv 0$.\\
	Тогда $\overline{a} = \overline{b} \equiv 0, \; \overline{u_{\eta \eta}} = G (\xi, \eta, \overline{u}, \overline{u_{\xi}}, \overline{u_{\eta}})$ - \blue{канонический вид параболического типа}.
\end{enumerate}

Рассмотрим уравнение:

$$a_{11} u_{xx} + 2 a_{12} u_{xy} + a_{22} u_yy + b_1 u_x + b_2 u_y + c u = f\eqno(*)$$
где $a_{11}, a_{12}, a_{22}, b_1, b_2, c, f$ - зависят от $(x,y)$.

Сделаем невырожденную замену координат:

$$\begin{cases}
	\xi = \varphi (x,y)\\
	\eta = \psi (x,y)
\end{cases}, \; \; \begin{vmatrix}
	\varphi_x & \varphi_y \\
	\psi_x & \psi_y
\end{vmatrix} \not = 0$$

Тогда по теореме о дифференцировании сложной функции:

$$\begin{gathered}
	\text{\red{$u_x$}} = u_{\xi} \xi_x + u_{\eta} \eta_x \\ \text{\red{$u_y$}} = u_{\xi} \xi_y + u_{\eta} \eta_y
\end{gathered}$$

$$\begin{gathered}
	\text{\red{$u_{xx}$}} = (u_{\xi})_x \xi_x + u_{\xi} \xi_{xx} + (u_{\eta})_x \eta_x + u_{\eta} \eta_{xx} = 
	\\ 
	= u_{\xi \xi} \xi_x^2 + u_{\xi \eta} \eta_x \xi_x + u_{\xi} \xi_{xx} + u_{\eta \xi} \xi_x \eta_x + u_{\eta \eta} \eta_x^2 + u_{\eta} \eta_{xx} = 
	\\ 
	= u_{\xi \xi} \xi_x^2 + 2 u_{\xi \eta} \xi_x \eta_x + u_{\eta \eta} \eta_x^2 + u_{\xi} \xi_{xx} + u_{\eta} \eta_{xx}
\end{gathered}$$

$$\begin{gathered}
	\text{\red{$u_{xy}$}} = (u_{\xi})_y \xi_x + (u_{\eta})_y \eta_x + u_{\xi} \xi_{x y} + u_{\eta} \eta_{x y} = 
	\\
	= u_{\xi \xi} \xi_y \xi_x + u_{\xi \eta} \xi_y \eta_x + u_{\eta \eta} \eta_y \eta_x + u_{\xi} \xi_{xy} + u_{\eta} \eta_{xy} =
	\\
	= u_{\xi \xi} \xi_y \xi_x + u_{\xi \eta} (\eta_y \xi_x + \eta_x \xi_y) + u_{\eta \eta} \eta_y \eta_x + u_{\xi} \xi_{xy} + u_{\eta} \eta_{xy}
\end{gathered}$$

$$\begin{gathered}
	\text{\red{$u_{yy}$}} = (u_{\xi})_y \xi_y + u_{\xi} \xi_{yy} + (u_{\eta})_y \eta_y + u_{\eta} \eta_{yy} = 
	\\ 
	= u_{\xi \xi} \xi_y^2 + u_{\xi \eta} \eta_y \xi_y + u_{\xi} \xi_{yy} + u_{\eta \xi} \xi_y \eta_y + u_{\eta \eta} \eta_y^2 + u_{\eta} \eta_{yy} = 
	\\ 
	= u_{\xi \xi} \xi_y^2 + 2 u_{\xi \eta} \xi_y \eta_y + u_{\eta \eta} \eta_y^2 + u_{\xi} \xi_{yy} + u_{\eta} \eta_{yy}
\end{gathered}$$

Подставим в $(*)$ и получим:

$$\begin{gathered}
	\overline{a_{11}} u_{\xi \xi} + 2 \overline{a_{12}} u_{\xi \eta} + \overline{a_{22}} u_{\eta \eta} + \beta_1 u_{\xi} + \beta_2 u_{\eta} + \gamma u = \delta
	\\
	\overline{a_{11}} = a_{11} \xi_x^2 + 2 a_{12} \xi_x \xi_y + a_{22} \xi_y^2
	\\
	\overline{a_{12}} = a_{11} \xi_x \eta_x + a_{12} (\xi_x \eta_y + \eta_x \eta_y) + a_{22} \xi_y \eta_y
	\\
	\overline{a_{22}} = a_{11} \eta_x^2 + 2 a_{12} \eta_x \eta_y + a_{22} \eta_y^2
\end{gathered}$$

Заметим, что:
$$\overline{a_{12}}^2 - \overline{a_{11}}\cdot \overline{a_{22}} = (a_{12}^2 - a_{11} a_{22})(\xi_x \eta_y - \eta_x \xi_y)^2 = (a_{12}^2 - a_{11} a_{22}) \cdot J^2, \;\; J \not = 0$$

Отсюда следует \red{инвариантность} данного типо уравнения при преобразовании переменных.

\vspace{0.5cm}
\textbf{Уравнения $\RNumb{2}$-порядка с $\ge 3$ независимыми переменными}: $u (x_1, \dots, x_n)$

\begin{definition}\label{cha:1/def:4}
	Общий вид линейного уравнения:
	$$\underset{i,j=1}{\overset{n}{\sum}}a_{i j} u''_{x_i x_j} + \underset{i=1}{\overset{n}{\sum}}b_i u'_{x_i} + c u = f (x_1, \dots, x_n), \;\; a_i, b_i, c = const$$
\end{definition}

\begin{definition}\label{cha:1/def:5}
	Сделаем замену $\frac{\partial}{\partial x_i} \to \lambda_i$. Тогда \blue{характеристическим многочленом} (главным символом) называется 
	$$\begin{gathered}
		A(\lambda_1, \dots, \lambda_n) = \underset{i,j=1}{\overset{n}{\sum}}a_{i j} \lambda_i \lambda_j = \\
		= \pm (\beta_{1 1} \lambda_1 + \dots + \beta_{1 n} \lambda_n)^2 \pm \dots \pm (\beta_{n 1} \lambda_1 + \dots + \beta_{n n} \lambda_n)^2 = \\ = \pm \mu_1^2 \pm \dots \pm \mu_n^2
	\end{gathered}$$
	В матричном виде: $\overrightarrow{\mu} = B \overrightarrow{\lambda}$.
\end{definition}

\textit{Замена}: $\overrightarrow{\xi} = (B^{-1})^T \overrightarrow{x}$\\

\textit{Канонический вид}:
$\;\; \pm \overline{u_{\xi_1 \xi_1}} \pm \dots \pm \overline{u_{\xi_n \xi_n}} = F (\xi_1, \dots, \xi_n, \overline{u}, \overline{u_{\xi_1}}, \dots, \overline{u_{\xi_n}})$

\begin{itemize}
	\item[$\bullet$] если все $+$, то уравнение эллиптического типа
	\item[$\bullet$] если есть $+$ и $-$, то уравнение гиперболического типа
	\item[$\bullet$] если есть $= 0$, то уравнение параболического типа
\end{itemize}

