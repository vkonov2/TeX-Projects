\UseRawInputEncoding

\documentclass[final,fontsize=14pt, DIV=calc]{scrreprt}
% \usepackage{cmap}
\usepackage{scrhack}
\usepackage[dvipsnames]{xcolor}
\usepackage[T1, T2A]{fontenc}
\usepackage[cp1251, utf8]{inputenc}
\usepackage[english, russian]{babel}
\usepackage{amsfonts}
\usepackage{amsthm}
\usepackage{amssymb}
\usepackage{vmargin}
\usepackage{amsmath}
\usepackage{graphicx}
\usepackage{listings}
\usepackage{color}
\usepackage{multicol}
\usepackage{pdfpages}
\usepackage{pdflscape}
\usepackage{float}
\usepackage[breaklinks]{hyperref}
\usepackage{tikz}
\usepackage{microtype}
\usepackage[Lenny]{fncychap}

\usepackage{silence}
\WarningFilter{scrreprt}{Usage of package `fancyhdr'}

\usepackage{fancyhdr}

\usepackage{chngcntr}
% \setpapersize{A4}
\usepackage[a4paper, total={6in, 8in}]{geometry}
\setmarginsrb{2cm}{0.5cm}{1.5cm}{1.5cm}{30pt}{3mm}{0pt}{13mm}
\usepackage{indentfirst}
\usepackage{subcaption}
% \usepackage{parskip}
\usepackage{pdfsync}

\synctex=1

\sloppy
\DeclareGraphicsExtensions{.png,.jpg,.jpeg,.heic,.svg,.pdf}
\graphicspath{{figs/}}

% \definecolor{linkcolor}{HTML}{800000}
% \definecolor{urlcolor}{HTML}{6495ED}

% % \definecolor{linkcolor}{HTML}{610B0B}
% % \definecolor{urlcolor}{HTML}{6495ED}
% \definecolor{lightgrey}{HTML}{9BABE7}
% \definecolor{currentfancycolout}{HTML}{000000}

% \hypersetup{pdfstartview=FitH,  linkcolor=linkcolor,urlcolor=urlcolor, colorlinks=true, pagecolor=linkcolor}

\hypersetup{
  colorlinks,
  citecolor=Orchid,
  linkcolor=Maroon,
  urlcolor=Cerulean}

% \setcounter{tocdepth}{5}
% \linespread{1}

\renewcommand{\thesection}{\arabic{section}.}

\fancyhead[RO]{\colorbox{black}{\color{white}{\textbf{\large \thepage}}}}  %% odd-right 
\fancyhead[L]{\colorbox{black}{\color{white}{\textbf{\large \thepage}}}}  %%% even-left
\fancyhead[LO]{\colorbox{lightgray}{$\rightsquigarrow$}}% odd-left
\fancyhead[R]{\colorbox{lightgray}{\textbf{\thepage}}}% even-right 
\fancyhead[C]{\rightmark}% odd-center, with the name of the Section
\fancyhead[CO]{\textsc{Аналитическая механика, 8 семестр, билет №\thesection}}% Even-center, with the name of the Chapter.
\fancyfoot[L,R,C]{}

\makeatletter
\renewcommand*\env@matrix[1][*\c@MaxMatrixCols c]{%
  \hskip -\arraycolsep
  \let\@ifnextchar\new@ifnextchar
  \array{#1}}
\makeatother

% \numberwithin{section}{part}


% \setcounter{tocdepth}{6}
% \setcounter{secnumdepth}{6}
\linespread{1}

\begin{document}

% \fancyhf{} % очистили все колонтитулы
% \lhead{тратата} % левый верхний колонтитул
% \chead{} % центральный верхний
% \rhead{\textbf{\large \thepage}} % правый верхний
% \lfoot{} % левый нижний
% \cfoot{\textbf{\large \thepage}} % центральный нижний
% \rfoot{} % правый нижний

% \renewcommand{\headrulewidth}{2pt} % линия под верхним к.
% \renewcommand{\footrulewidth}{0pt} % линия над нижним к. 

\ChTitleVar{\bfseries\Large\rmfamily}
\ChNameVar{\bfseries\Large\sffamily}

\renewcommand\qedsymbol{$\blacksquare$}
\renewcommand\contentsname{Содержание}

\newtheorem{theorem}{Теорема}[chapter]
\newtheorem{problem}{Задача}[chapter]
\newtheorem{lemma}{Лемма}[chapter]
\newtheorem{clair}{Утверждение}[chapter]
\theoremstyle{definition}
\newtheorem*{definition}{{\color{Purple} Определение}}
\newtheorem{propose}{Предложение}[chapter]
\newtheorem{property}{Свойство}[chapter]
\newtheorem{condition}{Условие}[chapter]
\newtheorem{properties}{Свойства}[chapter]
\newtheorem*{conseq}{{\color{Purple} Следствие}}
% \newtheorem{conseq}{Следствие}[chapter]
\newtheorem{remem}{Напоминание}[chapter]
\newtheorem{example}{Пример}[chapter]
\newtheorem{rulee}{Правило}[chapter]

\newtheorem*{remark}{Замечание}
\newtheorem*{chck}{Проверка}
\newtheorem{uprazh}{Упражнение}[section]

\newenvironment{Proof}       
	{\par\noindent{\bf Доказательство.}}
	{\hfill$\blacksquare$}
\newenvironment{solution}       
	{\par\noindent{\bf Решение.}}
	{\hfill$\blacksquare$}

\newcommand{\red}[1]{\textbf{\color{red}#1}}
\newcommand{\blue}[1]{\textbf{\color{blue}#1}}
\newcommand{\green}[1]{\textbf{\color{green}#1}}

\newcommand{\RNumb}[1]{\uppercase\expandafter{\romannumeral #1\relax}}

\def\ton#1{1,2,\dots,#1}
\def\Set#1#2{\left\{#1\colon#2\right\}}
\def\MYdef{\mathrel{\stackrel{\rm def}=}}
\def\QUdef{\mathrel{\stackrel{\rm ?}=}}
\def\Ddef{\mathrel{\stackrel{\rm d}=}}
\def\PNdef{\mathrel{\stackrel{\rm \text{п.н.}}=}}

\begin{titlepage}
  \begin{center}
    \large
 
 	МОСКОВСКИЙ ГОСУДАРСТВЕННЫЙ УНИВЕРСИТЕТ ИМЕНИ М. В. ЛОМОНОСОВА 
    
    \includegraphics[scale=0.6]{mm.jpg} 
     
    Механико-математический факультет
    \vspace{0.25cm} 
      
    экономический поток
    \vspace{0.8cm} 
     
    {\LARGE \textit{Алгебраические методы в экономике}}
    
    \vspace{0.8cm} 
    3 курс, группы 331-332

    \vspace{0.25cm} 
    6 семестр
\end{center}
\vfill
 
\newlength{\ML}
\settowidth{\ML}{«\underline{\hspace{0.7cm}}» \underline{\hspace{1cm}}}
\hfill\begin{minipage}{7cm}
  \begin{flushright}
    Лектор, семинарист $\;\;$\\
    д. ф.-м. н., профессор $\;\;$
    В.А.~Артамонов $\;\;$\\
    «\underline{\hspace{0.7cm}}» \underline{\hspace{2cm}} 2021 г. $\;\;$
  \end{flushright}
\end{minipage}%
\vfill
\bigskip
 
\begin{center}
  Москва, 2021 г.
\end{center}
% \tikz[remember picture,overlay] \node[opacity=0.3,inner sep=0pt] at (8.5,12.5){\includegraphics[scale=1.13]{background}};
\clearpage
\end{titlepage}
\newpage
\pagestyle{plain}
\begin{center}
	{\Large \textbf{Техническая информация}}
\end{center}

\vspace{0.5cm}
Данный PDF содержит примерную программу по предмету "Уравнения с частными производными"$\;$ 5-ого и 6-ого семестров.

\vspace{0.5cm}
Собрали и напечатали по мотивам лекций студенты 3-го курса Конов Марк и Гащук Елизавета.

\vspace{0.5cm}
Авторы выражают огромную благодарность лектору, семинаристу, кандидату физико-математических наук, доценту Капустиной Татьяне Олеговне за прочитанный курс по предмету <<Уравнения с частными производными>>.

\vspace{0.5cm}
Добавления и исправления принимаются на почты \href{}{vkonov2@yandex.ru}, \href{}{gashchuk2011@mail.ru}.

\vspace{0.5cm}
\begin{center}
	{\Large \textbf{ПРИЯТНОГО ИЗУЧЕНИЯ}}
\end{center}

\newpage



\tableofcontents
\newpage

\pagestyle{fancy}

\chapter*{Уравнения Лагранжа 2-го рода}
\label{cha1}
\addcontentsline{toc}{chapter}{Уравнения Лагранжа 2-го рода}
\section{Принцип Даламбера-Лагранжа.}
\label{cha1sec1}

% \includepdf[pages={1-2}, offset=80 -72, fitpaper=true]{lit/KugushevCropped.pdf}

\begin{figure}[h!]
	\noindent
	\centering
	\includegraphics[width=\textwidth]{lit/KugushevCropped-part/KugushevCropped-part 1.pdf}
\end{figure}

\begin{figure}[h!]
	\noindent
	\centering
	\includegraphics[width=\textwidth]{lit/KugushevCropped-part/KugushevCropped-part 2.pdf}
\end{figure}

% \includepdf[pages={1}, offset=80 -72]{lit/KugushevCropped-part/KugushevCropped-part 2.pdf}

\newpage

\section{Уравнения Лагранжа второго рода. Разрешимость уравнений Лагранжа относительно старших производных. Обобщенные силы. Случай потенциальных сил, лагранжиан. Первые интегралы уравнений Лагранжа обобщенный интеграл энергии (интеграл Якоби), циклические координаты и циклические интегралы.}\label{chasec2}

\begin{figure}[h!]
	\noindent
	\centering
	\includegraphics[width=\textwidth]{lit/KugushevCropped-part/KugushevCropped-part 3.pdf}
	\includegraphics[width=\textwidth]{lit/KugushevCropped-part/KugushevCropped-part 4-part/KugushevCropped-part 4-part 1.pdf}
\end{figure}

\begin{figure}[h!]
	\noindent
	\centering
	\includegraphics[width=\textwidth]{lit/KugushevCropped-part/KugushevCropped-part 4-part/KugushevCropped-part 4-part 2.pdf}
	\includegraphics[width=\textwidth]{lit/KugushevCropped-part/KugushevCropped-part 5-part/KugushevCropped-part 5-part 1.pdf}
\end{figure}

\begin{figure}[h!]
	\noindent
	\centering
	\includegraphics[width=\textwidth]{lit/KugushevCropped-part/KugushevCropped-part 5-part/KugushevCropped-part 5-part 2.pdf}
\end{figure}

\begin{figure}[h!]
	\noindent
	\centering
	\includegraphics[width=\textwidth]{lit/KugushevCropped-part/KugushevCropped-part 6.pdf}
\end{figure}


% \includepdf[pages={1}, offset=80 -72]{lit/KugushevCropped-part/KugushevCropped-part 4.pdf}
% \includepdf[pages={1}, offset=80 -72]{lit/KugushevCropped-part/KugushevCropped-part 5.pdf}

\newpage

\section{Понижение порядка по Раусу.}\label{chasec3}



\newpage

\chapter*{Вариационные принципы.Симметрии}
\label{cha2}
\addcontentsline{toc}{chapter}{Вариационные принципы.Симметрии}

\section{Поле симметрий. Теорема Нётер о первых интегралах.}\label{chasec4}



\newpage
\section{Вариационные принципы. Функционал действия и его вариация. Принцип Гамильтона. Метрика Якоби. Вариация по Гамильтону и по Мопертюи-Якоби. Принцип Мопертюи-Якоби.}\label{chasec5}



\newpage

\chapter*{Устойчивость положений равновесия. Малые колебания}
\label{cha3}
\addcontentsline{toc}{chapter}{Устойчивость положений равновесия. Малые колебания}

\section{Положения равновесия натуральных лагранжевых систем. Устойчивость положения равновесия лагранжевой системы по Ляпунову. Теорема Лагранжа-Дирихле.}
\label{chasec6}



\newpage
\section{Линеаризация уравнений Лагранжа около положения равновесия. Нормальные координаты. Уравнение малых колебаний.}\label{chasec7}



\newpage
\section{Диссипативные и гироскопические силы. Диссипативность сил Релея. Влияние диссипативных и гироскопических сил на устойчивость положения равновесия (обобщение теоремы Лагранжа-Дирихле при наложении диссипативных и гироскопических сил).}\label{chasec8}



\newpage
\section{Теорема Ляпунова о неустойчивости по первому приближению (формулировка). Степень неустойчивости. Теорема о невозможности гироскопической стабилизации. Четность характеристического полинома линеаризованных уравнений в потенциальном случае. Парность корней характеристического уравнения.}\label{chasec9}



\newpage

\chapter*{Инвариантная мера}
\label{cha4}
\addcontentsline{toc}{chapter}{Инвариантная мера}

\section{Инвариантная мера. Мера с гладкой плотностью. Плотность при замене координат. Теорема Лиувилля об инвариантной мере. Построение инвариантной меры на многообразии уровней первых интегралов – локально (Существование инвариантной меры у ограничения системы на инвариантное многообразие.)}\label{chasec10}



\newpage
\chapter{Задача Коши для уравнения теплопроводности. Принцип максимума в неограниченной области. Единственность решения задачи Коши в классе ограниченных функций.}
\label{cha:11}

Задача Коши для уравнения теплопроводности:
$$\begin{cases}
	u_{t} = a^2 u_{xx}, \; t > 0, \; x \in \mathbb{R}\\
	u|_{t = 0} = \varphi (x) \\
	|u| \leq C, \; t \geq 0, \; x \in \mathbb{R}.
\end{cases}$$

\begin{theorem}[\red{Принцип максимума в неограниченной области}] 
Пусть $ m = \underset{\mathbb{R}}{min(\varphi(x))}, \; M = \underset{\mathbb{R}}{max(\varphi(x))}$. Тогда $m \leq u(t, x) \leq M, \; t > 0, 
\; x \in \mathbb{R}.$
\end{theorem}

\begin{Proof}
Докажем для максимума (для минимума аналогично, но с другими знаками).\\

Будем доказывать, что $ M - u(t_0, x_0) \ge 0. $ Введем вспомогательную функцию $\nu(t,x)$, удовлетворяющую следующим условиям:
$$\begin{gathered}
	\nu(t, x) \ge 0, \; \nu_t = a^2\nu_{xx} \; \Rightarrow \; 
	\nu(t, x) = 2a^2t + x^2 \ge 0 , \;\; 
	\nu_t - a^2\nu_{xx} = 2a^2 - 2a^2 = 0
\end{gathered}$$
Рассмотрим следующую функцию:
$$u_\varepsilon(t, x) = M - u(t, x ) + \varepsilon\dfrac{v(t,x)}{\nu(t_0, x_0)}$$
Для нее задача Коши имеет вид:
$$\begin{cases}
	u_\varepsilon(t, x) = M - u(t, x ) + \varepsilon\dfrac{v(t,x)}{\nu(t_0, x_0)} \\
	u_\varepsilon|_{t = 0} = \underset{\ge 0}{\underbrace{M - \varphi(x)}} + \varepsilon\dfrac{x^2}{\nu(t_0, x_0)} \ge 0 \\
	u_\varepsilon|_{x = \pm R} = M - u(t, \pm R) + \varepsilon\dfrac{2a^2t + R^2}{\nu(t_0, x_0)} \ge \underbrace{M - u(t, \pm R)}_{\ge C_1, \text{ т.к. } |u| \leq C} + \varepsilon\dfrac{R^2}{\nu(t_0, x_0)} \geq 0 \;(R \rightarrow \infty)
\end{cases}$$

\begin{center}
\includegraphics[scale=0.4]{cha11im1}
\end{center}

По принципу максимума на стакане $ u_\varepsilon \geq 0 $ в прямоугольнике. Значит $u_\varepsilon(t_0, x_0) \ge 0$, тогда $M - u(t_0, x_0) + \varepsilon \ge 0$, значит $u(t_0, x_0) \le M + \varepsilon$, т.е. $u_(t_0, x_0) \le M$ при $\varepsilon \longrightarrow 0$.

\end{Proof}

\begin{conseq}
Ограниченное решение задачи Коши единственно.
\end{conseq}

\begin{proof}
Допустим, что существуют два разные решения $u_1 \not = u_2$. Рассмотрим их разность $ z = u_1 - u_2 $. Запишем для $z$ задачу Коши:
$$\begin{cases}
	z_{t} = a^2 u_{xx}, \; t > 0, \; x \in \mathbb{R}\\
	z|_{t = 0} = 0 \\
    |z| \leq \tilde{C}, \; t \geq 0, \; x \in \mathbb{R}.
\end{cases}$$

По принципу максимума получаем $ z \equiv 0$, откуда $u_1 = u_2$.
\end{proof}



\section{Теорема Пуанкаре о возвращении.}\label{chasec12}



\newpage

\section*{Динамика тяжелого твердого тела с неподвижной точкой}
\label{cha5}
\addcontentsline{toc}{chapter}{Динамика тяжелого твердого тела с неподвижной точкой}

\chapter{Модель Гейла сбалансированного роста. Существование состояния равновесия.}\label{cha:13}

$z:= (x,y) \in \mathbb{R}^{2n}$

\begin{definition}[Модель Гейла]
	Модель Гейла -- подмножество $Z \subset \mathbb{R}^{2n}_+:$
	\begin{enumerate}
		\item $Z$ -- выпуклый замкнутый конус
		\item если $(0,y) \in Z \Rightarrow y = 0$
		\item $\forall \; i: \; \exists (x,y) \in Z: \; y_i > 0 \;\; (3'. \exists (x,y)\in Z: \; y \gg 0)$
	\end{enumerate}
\end{definition}

\begin{itemize}
	\item $z = (x,y) \in Z$ -- производственный процесс
	\item x -- вектор затрат, y -- вектор выпуска
\end{itemize}

\begin{clair}
	(A, B) -- модель Неймана, в А нет нулевых строк (все товары производятся), тогда $Z = \{ (Au, Bu) | u \geq 0\}$ -- модель Гейла.
\end{clair}

\begin{definition}[Траектория(план)]
	Модель Гейла с началом $y_0$ -- последовательность $z_t = (x_{t-1}, y_t) \in Z, \; t \in \mathbb{N}, \; x_t \leq y_t.$
\end{definition}

\begin{definition}[Траектория цен]
	Последовательность $p_t \in \mathbb{R}_+^n, \; t \in \mathbb{N}_0: \; \forall \; (x,y) \in Z: \; p_{t-1}x \geq p_t y. \; \pi_t(\xi) := p_t y_t, \; \xi = \{ z_t\}, \; \pi_t(\xi)$ монотонно убывает.
\end{definition}

\begin{definition}[Состояние равновесия]
	Тройка $(\alpha, \vec{z}, \vec{p}), \; \alpha > 0, \; \vec{z} \in Z, \; \vec{p} > 0,$ -- состояние равновесия, если:
	\begin{enumerate}
		\item $\alpha \vec{x} \leq \vec{y}$
		\item $\forall \; (x, y) \in Z: \; \alpha p x \geq py$.
	\end{enumerate}

	Если (p, y) > 0, то положение равновесия невырожденное, $\alpha$ -- темп роста.
\end{definition}

\begin{theorem}
	У $\forall$ модели Гейла $\exists$ состояние равновесия.
\end{theorem}

\begin{proof}
	$\forall \; z \in Z$ определим технологический темп роста процесса $z = (x, y): \; \alpha(z) = \underset{\alpha}{max}\{\alpha | \alpha x \leq y \} \Rightarrow \alpha(z) \geq 0, \; \alpha(\lambda z) = \alpha(z) \; \forall \; \lambda > 0.$ Обозначим $\lambda_N = \underset{z \in Z \setminus \{ 0\} }{sup} \alpha(z)$ -- число Неймана модели Гейла. Покажем, что $\lambda_N < \infty.$

	Пусть $\lambda_N = \infty \Rightarrow \exists z_n = (x_n, y_n) \in Z: \; \alpha(z_n) > n, \; \| y_n\| = 1 \Rightarrow \| x_n\| \leq \dfrac{\| y_n\|}{\alpha(z_n)} \Rightarrow \| x_n \| \underset{n \to \infty}{\to} 0.$

	С другой стороны, $\{ y_n\}$ -- ограничена, тогда $\exists$ сходящаяся подпоследовательность $y_{n_k} \to y: \; \| y\| = 1.$

	Множество $Z$ замкнуто, тогда $z_{n_k} \underset{k \to \infty}{\to} (0,y) \in Z$ -- противоречие с пунктом 2) определения множеста Гейла, тогда $\alpha_N < \infty.$

	Аналогично доказывается, что $\exists z_n \to \vec{z}\in Z: \; \alpha(z) \to \alpha_N \Rightarrow \alpha(\vec{z}) = \alpha_N.$ Рассмотрим $U = \{ y - \alpha_N x | (x,y) \in Z\}, \; \mathbb{R}^n_{++} = \{ x \gg 0\} \Rightarrow U \cup \mathbb{R}^n_{++} = \emptyset,$ иначе $\exists (x,y)\in Z: \; y - \alpha_N x \gg 0 \Rightarrow \alpha(z) > \alpha_N. \; U, \; \mathbb{R}^n_{++}$ -- выпуклые, тогда по теореме отделимости $\exists p \neq 0: \; \forall \; u \in U, \; \forall \; v \in \mathbb{R}^n_{++}: \; pu \leq pv.$ Так как $0 \in U \Rightarrow pv \geq 0 \Rightarrow p > 0.$

	$\exists v_n \to 0 \Rightarrow pu \leq 0 \; \forall u \Rightarrow \alpha_N px \geq py \; \forall (x, y) \in Z \Rightarrow (\alpha_N, \vec{z}, p) $ -- состояние равновесия.
\end{proof} 

\begin{remark}
	Состояние равновесия $(\alpha, \vec{z}, p)$ может быть вырожденным.
\end{remark}

\begin{clair}
	В модели Гейла может быть не более n темпов роста.
\end{clair}
	
\begin{proof}
	$\forall \; \alpha > 0$ обозначим $Z(\alpha) = \{ z = (x, y) \in Z | \alpha(z) \geq \alpha\}, \; l(z) = \{ i | y_i > 0\}, \; \vec{z}(\alpha)$ -- вектор из Z, имеющий наибольшее количество ненулевых компонент, а $n(\alpha) = \#l(\vec{z}(\alpha)).$

	Тогда $n(\alpha)$ -- корректно определено, так как $Z(\alpha)$ -- выпуклый конус. Пусть $\alpha_1 < \alpha_2 \Rightarrow Z(\alpha_1) \subset Z(\alpha_2), \; n(\alpha_1) \geq n(\alpha_2).$ Пусть $\alpha_1, \alpha_2$ -- темпы роста и $(\alpha_i, \vec{z}_i, p_i)$ -- состояния равновесия i = 1,2. Можно считать, что $z_i$ имеет наибольшее число ненулевых компонент. Пусть $n(\alpha_1) = n(\alpha_2) \Rightarrow l(z_1) = l(z_2), \text{ и } \exists \gamma > 0: \; y_1 \leq \gamma y_2 \Rightarrow \gamma p_1 y_2 \geq p_1 y_1 \Rightarrow p_1 y_2 > 0 \Rightarrow \alpha_1 p_1 x_2 \geq p_1 y_2$ (по определению p) $\Rightarrow \alpha_2 p_2 \leq y_2  \Rightarrow \alpha_2 p_1 x_2 \leq p_1 y_2 \leq \alpha_1 p_1 x_2, $ но $p_1 y_2 > 0  \Rightarrow p_1 x_1 > 0  \Rightarrow \alpha_2 \leq \alpha_1$ -- противоречие.
\end{proof}

\begin{definition}[Число Фробениуса модели Гейла]
	Обозначим $\alpha'(p) = \inf \{ \alpha | \alpha p x  \geq p y \; \forall \; (x, y) \in Z\}.$ Тогда $\alpha_F = \underset{p > 0}{\inf}\alpha'(p)$ -- число Фробениуса модели Гейла.

	Отметим, что $\exists \vec{p} > 0: \; \alpha'(\vec{p}) = \alpha_F.$
\end{definition}

\begin{clair}
	\begin{enumerate}
		\item $\alpha_F \leq \alpha_N$.
		\item $\forall \; \alpha \in [\alpha_F, \alpha_N] \; \exists $ состояние равновесия $(\alpha, z, p): \; \alpha'(p) \leq \alpha \leq \alpha(z)$.
		\item Если $(\alpha, z, p)$ невырожденно, то $\alpha = \alpha(z) = \alpha'(p).$
	\end{enumerate}
\end{clair}

\begin{remark}
	Существуют модели Гейла без (ненулевых) темпов роста.
\end{remark}
\section{Инвариантная мера уравнений Эйлера-Пуассона и интегрируемость в квадратурах. Понятие о трех классических случаях интегрируемости Эйлера, Лагранжа и Ковалевской.}\label{chasec14}



\newpage
\chapter{Формулы Грина.}
\label{cha:15}

\begin{theorem}[\blue{Формула Гаусса - Остроградского}]
	Для функции $w$ имеем следующую формулу:
	\begin{multicols}{2}

			\includegraphics[scale = 0.3]{cha15im1}
		\columnbreak

			\hfill \break
			$$\underset{\Omega}{\int\int} \dfrac{\partial \omega}{\partial x_i}d\bar{x} = \oint\limits_{\partial \Omega} \omega n_i d\sigma$$
	\end{multicols}
\end{theorem}

\begin{theorem}[\blue{Многомерная формула интегрирования по частям}]
	Для функции $ \omega = uv $ имеем следующую формулу:
	$$\begin{gathered}
		\underset{\Omega}{\int\int} u \dfrac{\partial v}{\partial x_i}d\bar{x} = \oint\limits_{\partial \Omega} u v n_i d\sigma - \underset{\Omega}{\int\int} v \dfrac{\partial u}{\partial x_i}d\bar{x}.
	\end{gathered}$$
\end{theorem}

\begin{theorem}[\blue{I формула Грина}]
	Для функции $ \omega = u \dfrac{\partial v}{\partial x_i} $ имеем следующую формулу:
	$$\begin{gathered}
		\underset{\Omega}{\int\int} u \dfrac{\partial^2 v}{\partial x^2_i}d\bar{x} = \oint\limits_{\partial \Omega} u \dfrac{\partial v}{\partial x_i} n_i d\sigma - \underset{\Omega}{\int\int} \dfrac{\partial u}{\partial x_i} \dfrac{\partial v}{\partial x_i} d\bar{x} \mid \sum\limits_{i = 1}^k \Longrightarrow \\
		\Longrightarrow \underset{\Omega}{\int\int} u \Delta v d \bar{x} = \oint\limits_{\partial \Omega} u \dfrac{\partial v}{\partial \bar{n}}d\sigma - \underset{\Omega}{\int\int} (\bar{\nabla}u, \bar{\nabla}v)
		d\bar{x}
	\end{gathered}$$
\end{theorem}\newpage

\begin{theorem}[\blue{II формула Грина}]
$$
\begin{gathered}
\underset{\Omega}{\int\int} v \Delta u d\bar{x} = \oint\limits_{\partial \Omega} v \dfrac{\partial u}{\partial \bar{n}} d\sigma - \underset{\Omega}{\int\int} (\bar{\nabla}v, \bar{\nabla}u)d\bar{x} \\
- \\
\underset{\Omega}{\int\int} u \Delta v d \bar{x} = \oint\limits_{\partial \Omega} u \dfrac{\partial v}{\partial \bar{n}}d\sigma - \underset{\Omega}{\int\int} (\bar{\nabla}u, \bar{\nabla}v)d\bar{x} \\
= \\
\underset{\Omega}{\int\int} (u \Delta v - v \Delta u)d\bar{x} = \oint\limits_{\partial \Omega} (u \dfrac{\partial v}{\partial \bar{n}} - v \dfrac{\partial u}{\partial \bar{n}})d\sigma.
\end{gathered}
$$
\end{theorem}

\begin{theorem}[\blue{Теорема о потоке}]
	Если $ \omega = \dfrac{\partial u}{\partial x_i}$, то справедлива формула:
	$$\begin{gathered}
		\underset{\Omega}{\int\int} \Delta u d\bar{x} = \oint\limits_{\partial \Omega} \dfrac{\partial u}{\partial \bar{n}}d\sigma.
	\end{gathered}$$
\end{theorem}
\chapter{Невырожденные состояния равновесия в модели Неймана.}\label{cha:16}

Пусть (A, B) -- модель Неймана, в А нет нулевых столбцов, в В -- нулевых строк.

\begin{sign}
	$\lambda_N := \inf \{ \lambda | \exists u > 0: \; (A - \lambda B)u \leq 0\}$ -- число Неймана модели (A, B).

	$\lambda_F := \sup \{ \lambda | \exists p > 0: \; p(A - \lambda B) \leq 0\}$ -- число Фробениуса модели (A, B).
\end{sign}

\begin{theorem}
	$A \geq 0, \; B \geq 0, $ в А нет нулевых столбцов, в В -- нулевых строк $\Rightarrow $ в соответствующей модели Неймана $\exists$ невырожденное положение равновесия.
\end{theorem}

\begin{definition}[Продуктивная модель]
	(A, B) -- продуктивна, если $\forall \; c \geq 0 \; \exists x \geq 0: \; (B - A)x \geq c.$
\end{definition}

\begin{theorem}
	(A, B) -- продуктивна $\Longleftrightarrow \lambda_F < 1.$ 
\end{theorem}






\chapter*{Гамильтонова механика \RNumb{1}}
\label{cha6}
\addcontentsline{toc}{chapter}{Гамильтонова механика I}

\chapter{Функция Грина оператора Лапласа, ее симметрия. Представление решения задачи Дирихле через функцию Грина. Метод отражений. Метод конформных отображений.}
\label{cha:17}

\section*{Функция Грина оператора Лапласа, ее симметрия.}

Рассмотрим фундаментальное решение оператора Лапласса:
$$ \triangle \varepsilon (x) = \delta (x) , \; \varepsilon (x) = 
\begin{cases}
	\displaystyle \frac{1}{2 \pi} \ln |\vec{x}|, \; \vec{x} \in \mathbb{R}^2\\
	\displaystyle - \frac{1}{4 \pi |\vec{x}|}, \; \vec{x} \in \mathbb{R}^3
\end{cases}$$

Рассмотрим задачу Дирихле: 
$\begin{cases}
	\triangle u = f (x) , x \in \Omega\\
	u|_{x \in \partial \Omega} = h(x)
\end{cases}$

\begin{definition}
	\red{Функция Грина} $ G(x, y) $ задачи Дирихле для области $ \Omega $ имеет вид:
	$$\begin{cases}
		\Delta_x G(x, y) = \delta(x - y), \; x \in \Omega \\
		G(x, y)|_{x \in \Omega} = 0
	\end{cases} \forall \; y \in \Omega$$
\end{definition}

\begin{definition}[\blue{эквивалентное определение функции Грина}]
	\red{Функция Грина} представляется в виде $ G(x,y) = \varepsilon(x - y) + g(x, y) $, где:
	$$
	\begin{cases}
		\Delta_x g(x, y) = \delta(x - y), \; x \in \Omega \\
		g(x, y)|_{x \in \Omega} = -\varepsilon(x - y)
	\end{cases} \; \forall \; y \in \Omega
	$$
\end{definition}\newpage

\begin{theorem}[\red{Симметрия функции Грина}]
	$G(x, y) = G(y, x)$
\end{theorem}
\begin{Proof}
	\begin{multicols}{2}
		\includegraphics[scale=0.25]{cha17im1}
		\columnbreak
		$$\begin{gathered}
			\\
			\text{Рассмотрим две функции:}\\
			G_1 = G(x, y_1), \;  G_2 = G(x, y_2) \\ \\
			\text{Рассмотрим область:}\\
			\Omega_1 = \Omega \setminus (\left\{ {|x - y_1| < \alpha} \right\} \cup \left\{ {|x - y_2| < \alpha} \right\})
		\end{gathered}$$
	\end{multicols}

	Применим 2-ую формула Грина для $G_1$, $G_2$ в области $\Omega_1$:
	$$\begin{gathered}
		0 = \iint \limits_{\Omega_1} (G_1 \underbrace{\triangle_x G_2}_{ = 0} - G_2 \underbrace{\triangle_x G_1}_{ = 0}) d \vec{x} = 0 = \oint \limits_{\partial \Omega_1} \left(G_1 \frac{\partial G_2}{\partial n_x} - G_2 \frac{\partial G_1}{\partial n_x}\right) d \sigma = \\
		= \underbrace{\oint \limits_{\partial \Omega}\left(G_1 \frac{\partial G_2}{\partial n_x} - G_2 \frac{\partial G_1}{\partial n_x}\right) d \sigma}_{I_1} + \underbrace{\oint \limits_{S_1}\left(G_1 \frac{\partial G_2}{\partial n_x} - G_2 \frac{\partial G_1}{\partial n_x}\right) d \sigma}_{I_2} + \underbrace{\oint \limits_{S_2}\left(G_1 \frac{\partial G_2}{\partial n_x} - G_2 \frac{\partial G_1}{\partial n_x}\right) d \sigma }_{I_3}\\
		\left( S_1 \text{ и } S_2  \text{ - вырезанные окружности}\right)
	\end{gathered}$$

	$I_1 = 0$, т.к. $G_1$ и $G_2$ равны 0 на $\Omega$.

	Рассмотрим $I_2$:
	$$\begin{gathered} 
		\oint \limits_{S_1} \left(G(x, y_1) \frac{\partial}{\partial n_x} G(x, y_2) - G(x, y_2) \frac{\partial}{\partial n_x} G(x, y_1) d \sigma_x \right) = \\
		=\oint \limits_{S_1} (\varepsilon (x - y_1) + \underbrace{g(x, y_1)}_{\underset{\alpha \rightarrow 0}{\longrightarrow 0}}) \frac{\partial}{\partial n_x} G(x, y_2) - G(x, y_2) \frac{\partial}{\partial n_y} (\varepsilon (x - y_1) + \underbrace{g(x, y_1)}_{\underset{\alpha \rightarrow 0}{\longrightarrow 0}}) d \sigma_x \\
	\end{gathered}$$
	Сделаем замену: $|x - y_1| = \rho, \frac{\partial}{\partial n_x} = - \frac{\partial}{\partial \rho}$. Тогда при $\alpha \to 0$ интеграл стремится к:
	$$\begin{gathered} \int \limits_{0}^{2 \pi} ( \dfrac{1}{2 \pi} \ln \rho ( - \frac{\partial G_2}{\partial \rho} ) \rho) |_{\rho = \alpha} d \theta - \int \limits_{0}^{2 \pi} ( G_2 ( - \frac{1}{2 \pi \rho} ) \rho) |_{\rho = \alpha} d \theta  = \\
	= \underbrace{\alpha \ln \alpha \frac{1}{2 \pi} \int \limits_{0}^{2 \pi}(- \frac{\partial G_2}{\partial \rho}) |_{\rho = \alpha} d \theta}_{\underset{\alpha \rightarrow 0}{\longrightarrow 0}} + \frac{1}{2 \pi} \int \limits_{0}^{2 \pi} G_2 |_{\rho = \alpha} d \theta
	 \end{gathered}$$ 

	$$ = \Big|\text{по теореме о среднем}\Big| =  G(x^*, y_2) |_{x^* \in S_1} \underbrace{\frac{1}{2 \pi} \int \limits_{0}^{2 \pi} d \theta}_{=1} \xrightarrow[\alpha \to 0]{} G(y_1, y_2) $$

	Для $\oint \limits_{S_2}$ все аналогично, только меняем местами $ G_1 $ и $ G_2 $. Тогда $\oint \limits_{S_2} \underset{\alpha \rightarrow 0}{\longrightarrow} G(y_2, y_1)$.
	
	В итоге получаем, что $G(y_1, y_2) - G(y_2, y_1) = 0$. Т.к. $ y_1, y_2 $ -- произвольные точки из $\Omega$, то $G(y_1, y_2) = G(y_2, y_1) \; \forall \; y_1, y_2 \in \Omega.$

\end{Proof}

\section*{Представление решения задачи Дирихле через функцию Грина.}

Имеем задачу Дирихле: 
$\begin{cases}
	\triangle u = f (x) , x \in \Omega\\
	u|_{x \in \partial \Omega} = h(x)
\end{cases}$\\

Воспользуемся 2-ой формулой Грина для $ u(y), G(x, y): $
$$\begin{gathered} 
	\iint\limits_{\Omega}\left(u(y) \underbrace{\triangle_y G(x,y)}_{=\delta (y-x)} - G(x,y)\underbrace{\triangle_y u(y)}_{=f(y)}\right)dy = \oint\limits_{\partial \Omega}\left(\underbrace{u(y)}_{=h(y)}\dfrac{\partial G(x,y)}{\partial n_y} - \underbrace{G(x,y)}_{=0}\dfrac{\partial u(y)}{\partial n_y}\right)d\sigma \\
	\iint\limits_{\Omega}\underbrace{u(y)\delta(y - x)}_{=u(x)\delta(y-x)}dy =  u(x) \underbrace{\iint\limits_{\Omega}\delta(y - x)dy}_{=1} = u(x)
\end{gathered}$$
Тогда получаем формулу:
$$u(x) = \displaystyle \iint\limits_{\Omega} G(x,y)f(y)dy + \oint\limits_{\partial \Omega}h(y)\dfrac{\partial G(x,y)}{\partial n_y}d\sigma_y$$

\section*{Метод отражений.}

\begin{center}
	\includegraphics[scale=0.4]{cha17im2}
\end{center}
	
Воспользуемся тем, что одной из физических интерпретаций функции Грина задачи Дирихле в области $ \Omega_0 $ является потенциал поля, создаваемого в точке $ x \in \Omega_0 $ точечным зарядом величины $ q = \dfrac{1}{4\pi} $, расположенным в точке $ y \in \Omega_0 $, если граница $ \partial \Omega_0 $ области $ \Omega_0 $ является заземленной идеально проводящей поверхностью.

Предположим, что вне области $ \Omega_0 $ можно расположить фиктивные электрические заряды таким образом, чтобы суммарный потенциал поля, создаваемого зарядом $ q = \dfrac{1}{4\pi} $, расположенным в точке $ y \in \Omega_0 $, и этими фиктивными зарядами, на границе $ \partial \Omega_0 $ обращался в нуль. Такие фиктивные заряды называются электростатическими изображениями заряда, помещенного в точку $ y \in \Omega_0 $. Потенциал поля, порожденного зарядами, находящимися вне области, представляет собой гармоническую внутри области $ \Omega_0 $ функцию, то есть искомое гармоническое слагаемое в функции Грина.

Тогда в условиях вышеописанной задачи потенциал есть $ u(x) = \varepsilon(x - y) + g(x,y) = G(x, y), \; g(x, y)$ -- гармоническая. $ G_0(x, y) $ -- функция Грина для половинки, тогда
$$\begin{cases}
	\Delta_x G_0(x, y) = \delta(x - y), \; x \in \Omega_0 \\
	G_0(x, y)|_{x \in \Omega_0} = 0
\end{cases} \; \forall y \in \Omega_0$$
\begin{theorem}[]\label{lec:17/the:1}
	$ G_0(x, y) = G(x, y) - G(x, \bar{y}) $.
\end{theorem}
\begin{Proof}
	\begin{enumerate}
		\item 
			\begin{itemize}
				\item[$\bullet$]
					$G_0(x, y)|_{x \in \Omega} = 0, \text{ т.к. } G(x, y)|_{x \in \Omega} = 0$
				\item[$\bullet$]
					$G_0(x, y)|_{\text{на } l} = 0, \text{ т.к. суммарный потенциал поля обращается }\text{в нуль на } l$
			\end{itemize}
			Значит $ G_0(x, y)|_{x \in \Omega_0} = 0 $.
		\item 
			$ \triangle_x G_0(x, y) = \delta(x-y) $, т.к.:
			$$\triangle_x G_0(x, y) = \triangle_x G(x, y) - \triangle_x G(x, \bar{y}) = \delta(x-y) - \underbrace{\delta(x-\bar{y})}_{= 0 \text{ в } \Omega_0 \; \left(\bar{y} \not\in \Omega_0\right)}$$
	\end{enumerate}
	Таким образом, $ G_0(x, y) = G(x, y) - G(x, \bar{y}). $
\end{Proof}

\section*{Метод конформных отображений.}

\begin{center}
		\includegraphics[scale=0.4]{cha17im3}
		
		$ \varphi(x): \; \mathbb{R}^2 \longrightarrow \mathbb{C} $
		
		$ \varphi(x) = \xi(x_1, x_2) + i \eta(x_1, x_2) $
\end{center}

\begin{definition}
	$ \varphi(x) $ -- \red{конформное отображение}, если $ \varphi(x) $ есть аналитическая функция ($ \mathbb{C} \text{ - дифференцируема, } \varphi'(x) \neq 0$).
\end{definition}

\begin{definition}
	$ \varphi(x) $ - \red{аналитическая функция} $\Leftrightarrow$ выполнены условия \textit{Коши - Римана:} 
	$$\begin{cases}
		\dfrac{\partial \xi}{\partial x_1} = \dfrac{\partial \eta}{\partial x_2} \\
		-\dfrac{\partial \xi}{\partial x_2} = \dfrac{\partial \eta}{\partial x_1}
	\end{cases}$$
\end{definition}

Функция $ \varphi_y(x) $ -- аналитическая, $ \frac{\partial \varphi_y(x)}{\partial x} \neq 0. $ Пусть $ \varphi_y(x) $ такая, что удовлетворяет условиям:
$$\begin{gathered} 
	x \to \varphi_y(x), \; y \to 0 \\
	\Omega \to |\varphi_y| < 1, \; \partial\Omega \to |\varphi_y| = 1
\end{gathered}$$
Получаем, что $G(x, y) \to G(\varphi_y, 0)$.
$$\begin{gathered} 
	G(\varphi_y, 0) = \varepsilon(\varphi_y - 0) = \varepsilon(\varphi_y) = \dfrac{1}{2\pi}\log|\varphi_y| \\
	\triangle \varepsilon(\varphi_y) = \xi_{\varphi_y\varphi_y} + \eta_{\varphi_y\varphi_y} = \delta(\varphi_y)
\end{gathered}$$
Значит:
$\begin{cases}
	\displaystyle \Delta_{\varphi_y} G(\varphi_y, 0) = \delta(\varphi_y - 0), |\varphi_y| < 1 \\
	\displaystyle G(\varphi_y, 0)|_{|\varphi_y| = 1} = 0
\end{cases}$

\begin{clair}
	$ G(x, y) = \dfrac{1}{2\pi} \ln|\varphi_y(x)|$.
\end{clair}
\begin{Proof}
	\begin{enumerate}
		\item 
			$ \displaystyle \partial\Omega \to |\varphi_y| = 1 \; \Rightarrow \; G(x, y)|_{x \in \partial \Omega} = \dfrac{1}{2\pi} \ln|\varphi_y(x)| \Big|_{|\varphi_y| = 1} = \dfrac{1}{2\pi} \ln1 = 0$.
		\item 
			Рассмотрим случай $x \neq y$:

			$\ln z = \ln|z| + iArg(z) $ - аналитическая при $ z \neq 0$, тогда $\varphi_y(x) $ -- аналитическая при $ \varphi_y(x) \not = 0$.
			$$G(x, y) = \dfrac{1}{2\pi}\ln|\varphi_y(x)| = \dfrac{1}{2\pi} \Re e(\ln(\varphi_y(x))) \text{ - гармоническая при } \\
			x \not = y.$$
			Тогда $\triangle_x G(x,y) = 0$ при $x \not = y$.
		\item 
			Рассмотрим случай $x \to y$.
			$$\begin{gathered} 
			\varphi_y(x) = \underbrace{\varphi_y(y)}_{= 0} + \underbrace{\dfrac{d \varphi_y(x)}{dx}\Big|_{x = y}}_{\neq 0}\cdot(x - y) + \underbrace{\alpha(x,t)}_{\underset{x \rightarrow y}{\longrightarrow} 0}(x - y)
			\end{gathered}$$
			Тогда 
			$$\begin{gathered} 
			G(x, y) = \dfrac{1}{2\pi}\ln|\varphi_y(x)| = \dfrac{1}{2\pi}\ln\left|\dfrac{d \varphi_y(x)}{dx}\Big|_{x = y}(x - y) + \alpha(x,t)(x - y)\right| = \\
			= \dfrac{1}{2\pi} \ln \left(|x - y|\left|\left.\dfrac{d \varphi_y(x)}{dx}\right|_{x = y} + \alpha(x,t)\right|\right) = \\
			= \underbrace{\dfrac{1}{2\pi} \ln|x - y|}_{=\epsilon (x-y)} + \dfrac{1}{2\pi} \ln\Big|\underbrace{\dfrac{d \varphi_y(x)}{dx}|_{x = y}}_{\not = 0} + \underbrace{\alpha(x,t)}_{\to 0}\Big| = \\
			= \varepsilon(x - y) + g(x, y), \; |g(x, y)| < C \; x \longrightarrow y.
			\end{gathered}$$
			Отсюда следует, что 
			$$\begin{gathered} 
			\triangle_x G(x, y) = \triangle_x (\varepsilon(x - y) + g(x, y)) = \delta(x - y) + \triangle_x g(x, y) = \\
			= \underbrace{\delta(x - y)}_{= 0, \text{ при } x \neq y} + \triangle_x g(x, y)
			\end{gathered}$$
			$$\begin{cases}
				\triangle_x g(x, y) = 0, \; x \neq y \\
				|g(x,y)| < C, \; x \to y
			\end{cases} \Rightarrow \;  \triangle_x g(x, y) = 0$$
			Получаем, что 
			$\displaystyle \triangle_x G(x, y) = \delta(x - y)$.
	\end{enumerate}
\end{Proof}













\chapter{Теорема Раднера.}\label{cha:18}

Пусть $Z \subset \mathbb{R^{2n}_+}$ -- модель Гейбла и $u: \mathbb{R}^n \to \mathbb{R}$ -- функция полезности, $z = (x, y) \in \mathbb{R}^n \times \mathbb{R}^n.$

$z_1, \ldots, z_T \in Z$ называется траекторией, если $\forall \; t \; x_{t + 1} \leq y_t.$

\begin{problem}
	$$\begin{cases}
		u(y_T) \to \max\\
		z_t \in Z\\
		x_{t + 1} \leq y_t\\
		y_0 \leq x_1
	\end{cases}$$ Решение этой задачи -- оптимальная траектория. 
\end{problem}

Рассмотрим условия:

\begin{enumerate}
	% \item $ \vec{z} = (\vec{x}, \alpha \vec{x}) \in Z, \; \vec{x} > 0.$
	\item $ z = (x, \alpha x) \in Z, \; x > 0.$
	% \item $\exists p > 0: \; \forall \; z = (x, y) \in Z, \; z \neq \lambda \vec{z}: \; \lambda p x > py.$
	\item $\exists p > 0: \; \forall \; z = (x, y) \in Z, \; z \neq \lambda z: \; \lambda p x > py.$
	\item $\forall \; x \gg 0 \; \exists L > 0: \; (x, Lx) \in Z.$
	\item u -- непрерывна и неортицательна.
	\item u -- однородная степени 1: $u(\lambda y) = \lambda u(y) \forall \; \lambda > 0.$
	\item u(x) > 0.
	\item $\exists k > 0: \; \forall \; y \geq 0: \; u(y) \leq kpy.$
\end{enumerate}

\begin{sign}
	$S(\varepsilon, T, \{ z_t \}) = \#\{ t = 1, \ldots, T | s(y_t, x) \geq \varepsilon \}.$
\end{sign}

\begin{lemma}
	$C_{\varepsilon} := \{ (x, y) \in z | s(x, y) \geq \varepsilon \} \Rightarrow \exists \delta > 0: \; \forall (x,y) \in C_{\varepsilon}: \; (\alpha - \delta)px \geq py.$
\end{lemma}

\begin{proof}
	Пусть $\exists z_k = (x_k, y_k) \in C_{\varepsilon}: \; (\alpha - \dfrac{1}{k})px_k < p y_k.$ Можно считать, что $\exists \lim\limits_{k \to \infty} z_k = (x, y) \neq 0. Z$ -- замкнутое множество $\Rightarrow (x, y) \in Z.$

	$$\begin{gathered}
		(\alpha - \dfrac{1}{k})px_k < p y_k \Rightarrow  \alpha p x \leq py \Rightarrow (x, y) = \lambda z \Rightarrow y = \lambda \alpha x \text{ и } s(x, y) = 0 \\
		y_k \to y \Rightarrow \exists k_0: \; s(y_k, x) < \varepsilon \text{ -- противоречие с } z_{k_0} = (x_{k_0}, y_{k_0}) \in C_{\varepsilon}.
	\end{gathered}$$
\end{proof}

\begin{theorem}[Раднера]
	Если модель Гейбла $Z$ удовлетворяет условиям 1)-7), то х -- слабая магистраль, то есть $\forall \; \varepsilon > 0 \; \exists Q(\varepsilon) = Q: \; \forall$ оптимальных траекторий $\{ x_t \}: \; S(\varepsilon, T, \{ z_t \}) \leq Q.$
\end{theorem}

\begin{proof}
	Рассмотрим оптимальный проект $\{ z_t\}_{t = 1}^T$. Рассмотрим $\{ z'_t\}: \; z'_1 = (x_1, Lx) \in Z, \; z'_t = (\underset{= \alpha^{t - 2}Lz \in Z, \; z = (x, \alpha x)}{\alpha^{t - 2}Lx}, \alpha^{t - 1}Lx), \; t > 1.$

	$z'_t$ -- траектория такая, что $x'_{t+1} = y'_t, \; x'_1 < x_1 \Rightarrow u(y'_T) \leq u(y_T).$

	$$\begin{gathered}
		u(y'_T) = \alpha^{T - 2}Lu(x) > 0. \text{ Если } z_t \in C_{\varepsilon}, \text{ то } py_t \leq (\alpha - \delta)px_t \leq (\alpha - \delta)py_{t - 1}. \;\\
		z_t \in Z \setminus C_{\varepsilon} \Rightarrow py_z \leq \alpha p x_t \leq \alpha py_{t - 1}.\\
		\text{Пусть } S = S(\varepsilon, T, \{ z_t \}) = \#\{ t | z_t \in C_{\varepsilon} \} \Rightarrow py_T \leq (\alpha - \delta)^s \alpha^{T - s}px_1 \Rightarrow u(y_T) \leq k p y_T \leq \\ 
		\leq k(\alpha - \delta)^s \alpha^{T - s} p x_1\\
		1 - \dfrac{u(y_T)}{u(y'_T)} = \dfrac{k(\alpha - \delta)^s \alpha^{T - s} p x_1}{\alpha^{T - 2}Lu(x)} = d(\dfrac{\alpha - \delta}{\alpha})^s, \text{ где } d = \dfrac{k\alpha^2px_1}{Lu(x)}\\
		d(1 - \dfrac{\delta}{2})^s \geq 1 \Rightarrow s \leq -\dfrac{\log d}{\log (1 - \dfrac{\delta}{2})} = Q(\varepsilon).
	\end{gathered}$$

	$Q(\varepsilon)$ зависит от модели Гейбла $Z, u, \varepsilon, x_1$ и не зависит от $T, \{ z_t\}.$
\end{proof}
\chapter{Краевые задачи для уравнений Лапласа и Пуассона. Единственность решения задачи Дирихле. Задача Неймана: условие разрешимости, теорема о множестве решений.}
\label{cha:19}

\begin{definition}
	Имеем несколько видов уравнений:
	\begin{itemize}
		\item[$\bullet$] $ \Delta u = 0 $ -- \blue{уравнение Лапласа}
		\item[$\bullet$] $ \Delta u = f(x) $ -- \blue{уравнение Пуассона}
	\end{itemize}
\end{definition}

\begin{theorem}[\red{Единственность решения задачи Дирихле}]
	$\;$\\
	Имеем задачу Дирихле: 
	$\begin{cases}
		\Delta u(x) = f(x), \; x \in \Omega \\
		u(x) \mid_{\partial \Omega} = h(x)
	\end{cases}$. \\
	Тогда $ \exists ! u(x) $ - решение для $ \forall \text{ непрерывных } f(x), \; h(x)$.
\end{theorem}
\begin{Proof}
	Доказываем \textit{единственность}, т.к. существование уже доказано через функцию Грина.

	Пусть $\exists u_1, u_2 $ - два различных решения. Рассмотрим их разность $ z = u_1 - u_2 $ и задачу Дирихле для нее:
	$$\begin{cases}
		\Delta z(x) = 0, \; x \in \Omega \\
		z(x) \mid_{\partial \Omega} = 0
	\end{cases}$$
	Тогда по принципу максимума max и min достигаются на границе, а значит $ z \equiv 0$, т.е. $u_1 \equiv u_2$.
\end{Proof}

\begin{definition}[\red{Условие разрешимости задачи Неймана}]
	$$\iint\limits_{\Omega} f(x)dx = \oint\limits_{\partial\Omega}h(x)dx$$
\end{definition}

\begin{theorem}[\red{Теорема о множестве решений задачи Неймана}]
	$\;$\\
	Имеем задачу Неймана: 
	$\begin{cases}
		\Delta u(x) = f(x), \; x \in \Omega \\
		\dfrac{\partial u(x)}{\partial \bar{n}} \mid_{\partial \Omega} = h(x)
	\end{cases}$\\
	Если выполнено условия разрешимости и
	$ u_1, u_2 $ - решения задачи Неймана, то $u_1 - u_2 \equiv const$, т.е. решений бесконечно много. Если условие разрешимости не выполнено, то решений нет.
\end{theorem}
\begin{Proof}
	Отстутствие решение в случае невыполнения условия разрешимости очевидно следует из теоремы о потоке.

	Теперь предположим, что 
	$\exists u_1, u_2 $ - два различных решения. Рассмотрим их разность $ z = u_1 - u_2 $ и задачу Неймана для нее:
	$$\begin{cases}
		\Delta z(x) = 0, \; x \in \Omega \\
		\dfrac{\partial z(x)}{\partial \bar{n}} \mid_{\partial \Omega} = 0
	\end{cases}$$
	Условие разрешимости автоматически выполнено. Воспользуемся 1-ой формулой Грина для $ \nu = \omega = z $:
	$$\begin{gathered}
		\iint\limits_{\Omega} \nu \Delta \omega d\bar{x} = \oint\limits_{\partial \omega} \nu \dfrac{\partial \omega}{\partial \bar{n}}d\sigma - \iint\limits_{\Omega}( \bar{\nabla} \nu, \bar{\nabla} \omega)d\bar{x} \\
		\underbrace{\iint\limits_{\Omega} z \Delta z d\bar{x}}_{= 0} = \underbrace{\oint\limits_{\partial z} z \dfrac{\partial z}{\partial \bar{n}}d\sigma}_{= 0} - \iint\limits_{\Omega}|| \bar{\nabla} z||^2d\bar{x} = 0 \; \Rightarrow \; \iint\limits_{\Omega}|| \bar{\nabla} z||^2d\bar{x} = 0
	\end{gathered}$$
	Так как $|| \bar{\nabla} z||^2 \geq 0$ и $|| \bar{\nabla} z|| = (\dfrac{\partial z}{\partial x_1})^2 + \dots + (\dfrac{\partial z}{\partial x_n})^2 \equiv 0$, то: 
	$$\dfrac{\partial z}{\partial x_i} \equiv 0, \; i = 1, \ldots, n \; \Rightarrow \;  z \equiv const \; \Rightarrow \;  u_1 - u_2 \equiv const$$
\end{Proof}












\chapter{Критерий оптимальности допустимого плана транспортной задачи в терминах потенциалов}
\label{cha:20}

\epigraph{
	\textit{Новые замыслы, новые планы, новые разветвления!}}
{-- Салтыков-Щедрин М.Е.}

Пусть в заданных m городах
$$A_1, \dots, A_m\eqno(57)$$
производится некоторый однородный продукт в количествах $a_1, \dots, a_m > 0$.

Этот продукт перевозится в заданные n городов
$$B_1, \dots, B_n\eqno(58)$$
где он полностью потребляется в количествах $b_1, \dots, b_n > 0$. Предполагаются заданными стоимости $c_{ij} \ge 0$ перевозок единицы продукта из $A_i$ в $B_j$.

Назовем \textit{планом перевозок} неотрицательную матрицу $X = (x_{ij})$ размера $m \times n$, в которой $x_{ij} \ge 0$ указывает количество продукта, перевозимого из $A_i$ в $B_j$, $1 \le i \le m$, $1 \le j \le n$. Стоимость перевозок является линейной функцией от X:
$$z(X) = \underset{i,j}{\overset{}{\sum}}c_{ij}x_{ij}, \; x_{ij} \ge 0\eqno(59)$$
В задаче требуется найти такой план перевозок X, чтобы его стоимость была бы минимальной, весь продукт был вывезен из (57) и потребности городов (58) были полностью удовлетворены.

Транспортная задача является задачей линейного программирования. Действительно, весь продукт, производимый в (57), вывозится в (58), где он полностью потребляется, причем все потребности удовлетворены. Таким образом, возникают следующие условия:
$$\begin{gathered}
	\underset{j=1}{\overset{n}{\sum}}x_{ij} = a_i, \; i = \ton m \\
	\underset{i=1}{\overset{m}{\sum}}x_{ij} = b_j, \; j = \ton n \\
	x_{ij} \ge 0
\end{gathered}\eqno(60)$$
Требуется в условиях (60) найти минимум функции (59). Ввиду специфичности условий (60) можно предложить более специальный метод потенциалов решения этой задачи.

Назовем план перевозок X \textit{допустимым}, если выполнены условия (60).

\begin{propose}\label{cha:20/propose:1}
	Полиэдр P, задаваемый условиями (60), непуст тогда и только тогда, когда
	$$\underset{i}{\overset{}{\sum}}a_i = \underset{j}{\overset{}{\sum}}b_j\eqno(61)$$
\end{propose}
\begin{Proof}
	Если $X = (x_{ij})$ – допустимый план, то по (60):
	$$\underset{i}{\overset{}{\sum}}a_i = \underset{i}{\overset{}{\sum}}\underset{j}{\overset{}{\sum}}x_{ij} = \underset{j}{\overset{}{\sum}}\underset{i}{\overset{}{\sum}}x_{ij} = \underset{j}{\overset{}{\sum}}b_j$$
	Обратно, пусть выполнено условие (61). Построим матрицу первоначального плана $X^0 = (x_{ij}^0)$ методом минимального элемента. Выберем клетку $(i,j)$ с минимальным значением $c_{ij}$. В эту клетку ставим число $x_{ij}^0 = \min (a_i, b_j)$. Если $x_{ij}^0 = b_j$, то в остальные клетки столбца j ставим 0, а число $a_i$ заменяем на $a_i - b_j$. Дуальным образом поступаем, если $x_{ij}^0 = a_i$. Если $a_i \le b_j$, то из $A_i$ весь продукт вывезен в $B_j$ и потребность $B_j$ становится равной $b_j - a_i$. Если же $b_j < a_i$, то в $B_j$ весь необходимый продукт завезен, и в $A_i$ осталось $a_i - b_j$ продукта. Таким образом, либо число пунктов $A_t$, либо пунктов $B_s$ уменьшилось.

	Повторяя эту процедуру, получим первоначальный допустимый план $X^0 = (x_{ij}^0)$.
\end{Proof}

\begin{propose}\label{cha:20/propose:2}
	Полиэдр P, задаваемый условиями (60), ограничен.
\end{propose}
\begin{Proof}
	Если $X = (x_{ij})$ – допустимый план, то для любых индексов $i,j$ имеем $0 \le x_{ij} \le \min (a_i, b_j)$.
\end{Proof}

\begin{conseq}[]\label{cha:20/conseq:1}
	Транспортная задача (60) имеет решение тогда и только тогда, когда выполнено равенство (61).
\end{conseq}
\begin{Proof}
	По предложению \ref{cha:20/propose:1} полиэдр P допустимых планов непуст. По предложению \ref{cha:20/propose:2} он ограничен. Следовательно, непрерывная функция $z(X)$ достигает на P минимума.
\end{Proof}

\begin{theorem}[]\label{cha:20/the:1}
	Для того, чтобы допустимый план перевозок X был оптимальным необходимо и достаточно, чтобы существовали числа (потенциалы) \\ $u_1, \dots, u_m, v_1, \dots, v_n$, для которых:
	\begin{itemize}
		\item[1)] $u_i + v_j \le c_{ij}$ при всех $i, j$
		\item[2)] $u_i + v_j = c_{ij}$, если $x_{ij} > 0$
	\end{itemize}
\end{theorem}
\begin{Proof}
	Проверим достаточность. Пусть $X = (x_{ij})$ – допустимый план, и $u_1, \dots, u_m, v_1, \dots, v_n$ из условий теоремы. Предположим, что $Y = (y_{ij})$ – произвольный допустимый план. Из (60) и условий 1), 2) следует, что:
	$$\begin{gathered}
		z(Y) = \underset{i, j}{\overset{}{\sum}}c_{ij} y_{ij} \ge \underset{i, j}{\overset{}{\sum}}(u_i + v_j) y_{ij} = \underset{i}{\overset{}{\sum}}u_i \underset{j}{\overset{}{\sum}}y_{ij} + \underset{j}{\overset{}{\sum}}v_j \underset{i}{\overset{}{\sum}}y_{ij} = \\
		= \underset{i}{\overset{}{\sum}}u_i a_i + \underset{j}{\overset{}{\sum}}v_j b_j = \underset{i}{\overset{}{\sum}}u_i \underset{j}{\overset{}{\sum}}x_{ij} + \underset{j}{\overset{}{\sum}}v_j \underset{i}{\overset{}{\sum}}x_{ij} = \\
		= \underset{i, j}{\overset{}{\sum}}(u_i + v_j) x_{ij} = \underset{i, j}{\overset{}{\sum}}c_{ij} x_{ij} = z(X)
	\end{gathered}$$
	Проверим теперь необходимость. Рассмотрим задачу, двойственную к транспортной задаче. Для этого перепишем ограничения из (60) и целевую функцию в виде:
	$$\begin{gathered}
		-z(X) = \underset{i, j}{\overset{}{\sum}}(-c_{ij}x_{ij}) \to \max \\
		\underset{j=1}{\overset{n}{\sum}}x_{ij} \le a_i, \; \underset{j=1}{\overset{n}{\sum}}(-x_{ij}) \le -a_i, \; i = \ton m \\
		\underset{i=1}{\overset{m}{\sum}}x_{ij} \le b_j, \; \underset{i=1}{\overset{m}{\sum}}(-x_{ij}) \le b_j, \; j = \ton n \\
		x_{ij} \ge 0
	\end{gathered}$$
	Тогда по определению \ref{cha:16/def:1} двойственная задача с переменными
	$$w_1, \dots, w_m, w'_1, \dots, w'_m, s_1, \dots, s_n, s'_1, \dots, s'_n \ge 0$$
	к транспортной задаче имеет вид:
	$$\begin{gathered}
		T' = \underset{i=1}{\overset{m}{\sum}}(w_i - w'_i)a_i + \underset{j=1}{\overset{n}{\sum}}(s_j - s'_j)b_j \to \min \\
		w_i - w'_i + s_j - s'_j \ge -c_{ij}, \; 1 \le i \le m, \; 1 \le j \le n \\
		w_1, \dots, w_m, w'_1, \dots, w'_m, s_1, \dots, s_n, s'_1, \dots, s'_n \ge 0
	\end{gathered}\eqno(62)$$
	Положим $u_i = w'_i - w_i$, $v_j = s'_j - s_j$. Тогда (62) записывается в виде:
	$$\begin{gathered}
		u_i + v_j \le c_{ij} \\
		T = - T' = \underset{i=1}{\overset{m}{\sum}}u_i a_i + \underset{j=1}{\overset{n}{\sum}}v_j b_j \to \max
	\end{gathered}\eqno(63)$$
	Как отмечено в предложении \ref{cha:20/propose:2}, условия (60) задают ограниченный полиэдр P. Таким образом, по следствиию \ref{cha:20/conseq:1} функция $z(X)$ на P достигает минимум в некоторой точке $X^0$. Заметим, что полиэдр Q, задаваемый неравенствами (63) содержит начало координат, поскольку $c_{ij} \ge 0$. Следовательно, Q непусто и по теореме \ref{cha:17/the:1}:
	$$\min \left( z(X) \right) = - \max \left( -z(X) \right) = - \min \left( T' \right) = \max T$$
	Пусть в точке $(u_1, \dots, u_m, v_1, \dots, v_n) \in Q$ достигается максимум. По теореме \ref{cha:18/the:1} о равновесии получаем $u_i + v_j = c_{ij}$, если $x_{ij}^0 > 0$.
\end{Proof}
































\chapter{Построение первоначального плана. Отсутствие в нем циклов}
\label{cha:21}

\epigraph{
	\textit{Я отчасти участвовал в переорганизации общества по новому плану, и только.}}
{-- Достоевский Ф.М.}

\begin{definition}\label{cha:21/def:1}
	Транспортная задача \textit{вырождена}, если существуют такие собственные подмножества индексов $\displaystyle F \subset \{1, \dots, m\}, \; H \subset \{1, \dots, n\}$, что $\underset{i \in G}{\overset{}{\sum}}a_i = \underset{j \in H}{\overset{}{\sum}}b_j$. Другими словами, суммарный запас продукта в пунктах $A_i, i \in G$, совпадает с потреблением в пунктах $B_j, j \in H$.
\end{definition}

Далее мы будет предполагать, что рассматриваемая транспортная задача \textit{невырождена}.

\begin{propose}\label{cha:21/propose:1}
	Пусть задан допустимый план X невырожденной задачи и $i_0 \in \{1, \dots, m\}$. Тогда для любого $j \in \{1, \dots,m\}$ существует такая последовательность клеток $(i_0, j_1), (i_1, j_1), (i_1, j_2), \dots, (i_k, j_k), (i_k, j_{k+1})$, что $j_{k+1} = j$ и во всех клетках этой последовательности в матрице X стоят ненулевые коэффициенты.
	
	Аналогично, для любого $j_0 \in \{1, \dots, n\}$ и любого $i \in \{1, \dots, m\}$. Тогда существует последовательность клеток $(i_1, j_0), (i_1, j_1), (i_2, j_1), \dots, (i_k, j_k), (i_{k+1}, j_k)$, \\ что $i_{k+1} = i$ и во всех клетках этой последовательности в матрице X стоят ненулевые коэффициенты.
\end{propose}
\begin{Proof}
	Рассмотрим первое утверждение. Обозначим через H множество всех таких индексов $l \in \{1, \dots, n\}$, для которых найдется последовательность ненулевых элементов $x_{i_0j_1}, x_{i_1j_1}, \dots, x_{i_kj_k}, x_{i_kj_{k+1}}, \; j_{k+1} = l$. Предположим, что $j \not \in H$. Через G обозначим множество всех таких индексов $t \in \{1, \dots, m\}$, что $x_{tl} \not = 0$ для некоторого $l \in H$. Из определения G вытекает, что если $t \in G$ и $x_{tl} \not = 0$, то $l \in H$. Поэтому $\displaystyle \underset{j \in H}{\overset{}{\sum}}b_j = \underset{i \in G, j \in H}{\overset{}{\sum}}x_{ij} = \underset{i \in G}{\overset{}{\sum}}a_i$. Отсюда $\displaystyle \underset{i \not \in G}{\overset{}{\sum}}a_i = \underset{i \not \in H}{\overset{}{\sum}}b_j > 0$.

	Следовательно, $G \not = \{1, \dots, n\}$, и $H \not = \{1, \dots, m\}$. Это противоречит невырожденности задачи.
	
	Симметрично доказывается второе утверждение.
\end{Proof}

\begin{conseq}[]\label{cha:21/conseq:1}
	Пусть план $X = (x_{ij})$ допустим, и $x_{i_0j_0} = 0$. Тогда существует такая последовательность строк $i_0, i_1, \dots, i_k$ и последовательность столбцов $j_1, \dots, j_k, j_0$ матрицы X, что элементы
	$$x_{i_0j_1}, x_{i_1j_1}, \dots, x_{i_kj_k}, x_{i_kj_0}\eqno(64)$$
	отличны от нуля.
\end{conseq}

\begin{definition}\label{cha:21/def:2}
	Пусть $X = (x_{ij})$ – допустимый план. \textit{Циклом} в X назовем последовательность ненулевых элементов $x_{p_1q_1}, x_{p_1q_2}, \dots, x_{p_sq_s}, x_{p_sq_1}$, причем среди первых и среди вторых индексов в этой последовательности имеются нет совпадений.
\end{definition}

\begin{propose}\label{cha:21/propose:2}
	Если допустимый план не содержит циклов, то в предложении \ref{cha:21/propose:1} по набору $i, j$ искомые последовательности определены однозначно. Кроме того, в (64) последовательность также определена однозначно.
\end{propose}
\begin{Proof}
	 Пусть для заданных $i \in \{1, \dots, m\}$, $j \in \{1, \dots, n\}$ две разные последовательности
	 $$\begin{gathered}
	 	(i_0, j_1), (i_1, j_1), (i_1, j_2), \dots, (i_k, j_k), (i_k, j_{k+1}) \\
	 	(i_0 , j'_1 ), (i'_1 , j'_1 ), (i'_1 , j'_2 ), \dots, (i'_s , j'_s ), (i'_s , j'_{s+1} )
	 \end{gathered}$$
	 где $j_{k+1} = j'_{s+1} = j$ и во всех клетках этих последовательностей в матрице X стоят ненулевые коэффициенты. Погда получаем цикл
	 $$(i_k,j), (i_k,j_k), \dots, (i_1,j_1), (i_0,j_1), (i_0,j'_1), (i'_1,j'_1), \dots, (i'_s,j'_s), (i's_,j)$$
	 что невозможно. Аналогично рассматривается утверждение для (64).
\end{Proof}

Изложим теперь алгоритм решения невырожденной транспортной задачи.

ШАГ 1. \textit{Построение первоначального} плана \textit{методом минимального элемента}. В соответствии с предложением \ref{cha:20/propose:1} строим первоначальный план $X^0 = (x_{ij}^0)$.

\begin{propose}\label{cha:21/propose:3}
	План $X^0$ не содержит циклов. В каждой строке и столбце плана $X^0$ содержится ненулевой элемент. Всего в $X^0$ число ненулевых элементов равно $m + n − 1$.
\end{propose}
\begin{Proof}
	Пусть план $X^0$ содержит цикл $x_{p_1q_1}, x_{p_1q_2}, \dots, x_{p_kq_k}, x_{p_kq_1}$ из определения \ref{cha:21/def:2}. Выберем в этой последовательности тот элемент, который построен первым. Пусть, например, это $x_{p_1q_1}$. Тогда элементы $x_{p_1q_2}, x_{p_sq_1}$, построены позднее, что невозможно, ибо при построении $x_{p_1q_1}$ , либо на месте $(p_1, q_2)$, либо на $(p_s, q_1)$ ставится 0. Аналогично рассматриваются остальные случаи.
\end{Proof}






















\section{Уравнение Гамильтона-Якоби. Его полный интеграл. Разрешимость в квадратурах.}\label{chasec22}



\newpage
\section{Понижение порядка по Уиттекеру. Автономизация системы.}\label{chasec23}



\newpage

\chapter*{Гамильтонова механика \RNumb{2}}
\label{cha7}
\addcontentsline{toc}{chapter}{Гамильтонова механика II}

\section{Симплектическое многообразие. Формулировка теоремы Дарбу о канонических координатах. Гамильтоново векторное поле.}\label{chasec24}



\newpage
\chapter{Сходимость алгоритма решения невырожденной транспортной задачи}
\label{cha:25}

\epigraph{
	\textit{Сходились, расходились, не находя места.}}
{-- Короленко В.Г.}

Приведенный алгоритм конечен. Действительно, на 4-ом шаге получаем новый план $X'$. Если $a_{i_0j_0}$ из (66), то считая $j_{k+1} = j_0$ получаем:
$$\begin{gathered}
	z(X') - z(X) = \underset{i, j}{\overset{}{\sum}}c_{ij} (x'_{ij} - x_{ij}) = \\
	= \underset{s=0}{\overset{k}{\sum}}c_{i_s j_s} (x'_{i_s j_s} - x_{i_s j_s}) + \underset{s=0}{\overset{k}{\sum}}c_{i_s j_{s+1}} (x'_{i_s j_{s+1}} - x_{i_s j_{s+1}}) = \underset{s=0}{\overset{k}{\sum}}c_{i_s j_s} \theta - \underset{s=0}{\overset{k}{\sum}}c_{i_s j_{s+1}} \theta = \\
	= \theta \left[ c_{i_0 j_0} + (u_{i_1} + v_{j_1}) + \dots + (u_{i_{k-1}} + v_{j_{k-1}}) - (u_{i_0} + v_{j_1}) - \dots - (u_{i_k} + v_{j_{k+1}}) \right] = \\
	= \theta [c_{i_0 j_0} - (u_{i_0} + v_{j_0})] = - \theta \alpha_{i_0 j_0} < 0
\end{gathered}$$
При этом для нового плана $X'$ выполнено условие 2) из теоремы \ref{cha:20/the:1}. Таким образом, число шагов не превосходит числа подмножеств клеток с ненулевыми элементами из допустимого плана $X'$. Так как значение $Z(X)$ уменьшается, то у нас не возникает повторений.

Изложение алгоритма завершено.
\section{Теорема Лиувилля о вполне интегрируемых системах.}\label{chasec26}



\newpage
\chapter{Связь нормы матрицы с ее спектральным радиусом}
\label{cha:27}

\epigraph{
	\textit{У меня служба — у нее связи и маленькие средства.}}
{-- Толстой Л.Н.}

\begin{definition}\label{cha:27/def:1}
	\textit{Спектральным радиусом} $\rho(A)$ оператора (матрицы) $A \in \mathcal{L}(V)$ называется максимум модулей собственных значений A.
\end{definition}

\begin{theorem}[]\label{cha:27/the:1}
	Пусть $||\cdot||$ – норма в алгебре линейных операторов $\mathcal{L}(V)$ в конечномерном пространстве V. Если $A \in \mathcal{L}(V)$, то $\rho(A) \le ||A||$.
\end{theorem}
\begin{Proof}
	Пусть $Ax = \lambda x$ для некоторого ненулевого собственного вектора x. Построим матрицу X, столбцами которой будут координаты вектора x. Тогда $AX = \lambda X$, откуда:
	$$||AX|| = |\lambda| ||X|| \le ||A||||X||\eqno(89)$$
	Так как $X \not = 0$, то $||X|| \not = 0$ и поэтому в (89) получаем $|\lambda| \le ||A||$. Отсюда вытекает утверждение, поскольку $\lambda$ – любое собственное значение.
\end{Proof}

\begin{theorem}[]\label{cha:27/the:2}
	Пусть $\rho(A) < 1$. Тогда $A^k \to 0$ при $k \to \infty$.
\end{theorem}
\begin{Proof}
	\textit{Первое доказательство}.

	В силу следствия \ref{cha:26/conseq:1} достаточно доказать сходимость относительно матричной нормы $||\cdot||_E$. Пусть n – размерность пространства, в котором действует оператор A. Без ограничения общности можно предполагать, что $\mathbb{F} = \mathbb{C}$. Случай $n = 1$ очевиден. Пусть для $n − 1$ теорема доказана. В силу теоремы о приведении к жордановой форме существует такой базис, в котором матрица оператора имеет верхнетреугольный вид. Переходя к этому базису, будем считать, что матрица A имеет вид:
	$$A = \begin{pmatrix}[c | c]
		B &  u \\ \hline
		0 &  \lambda
	\end{pmatrix}$$
	Тогда $\rho(B) \le \rho(A) < 1$ и по индукции $B^k \to 0$ при $k \to \infty$.

	\begin{lemma}\label{cha:27/lemma:1}
		$$A^k = \begin{pmatrix}[c | c]
		B^k &  D_k u \\ \hline
		0 &  \lambda^k
	\end{pmatrix}, \; D_k = \underset{j=0}{\overset{k-1}{\sum}}\lambda^j B^{k-1-j}$$
	\end{lemma}
	\begin{Proof}
		Непосредственная проверка, основанная на определении произведения матриц.
	\end{Proof}

	Завершим доказательство теоремы. Так как $B^k \to 0$ при $k \to \infty$, то для любого $1 > \varepsilon > 0$ существует такое натуральное число N, что для всех $k \ge N$ имеем $||B^k|| < \varepsilon$. В частности, последовательность $||B^j||$ ограничена константой C. Кроме того, $|\lambda| < 1$. Таким образом, если $m > 2Nt$, то:
	$$\begin{gathered}
		||D_m|| \le \Big|\Big| \underset{j=0}{\overset{Nt-1}{\sum}}\lambda^j B^{m-1-j}\Big|\Big| + \Big|\Big| \underset{j=Nt}{\overset{m-1}{\sum}}\lambda^j B^{m-1-j}\Big|\Big| \le \\
		\le ||B^{m-1-Nt}||\Big|\Big|\underset{j=0}{\overset{Nt-1}{\sum}}\lambda^j B^{Nt-j}\Big|\Big| + |\lambda^{Nt}|\Big|\Big|\underset{j=Nt}{\overset{m-1}{\sum}}\lambda^{j-Nt}B^{m-1-j}\Big|\Big| \le \\
		\le ||B^N||^t||B^{m-1-2Nt}||\Big|\Big|\underset{j=0}{\overset{Nt-1}{\sum}}\lambda^j B^{Nt-j}\Big|\Big| + |\lambda|^{Nt}\Big|\Big|\underset{j=Nt}{\overset{m-1}{\sum}}\lambda^{j-Nt}B^{m-1-j}\Big|\Big| \le \\
		\le C ||B^N||^t \left( \underset{j=0}{\overset{Nt-1}{\sum}}|\lambda|^j ||B^{Nt-j}|| \right) + |\lambda|^{Nt} \left( \underset{j=Nt}{\overset{m-1}{\sum}}|\lambda^{j-Nt}|||B^{m-1-j}|| \right) \le \\
		\le C^2 ||B^N||^t \left( \underset{j=0}{\overset{Nt-1}{\sum}}|\lambda|^j \right) + C |\lambda|^{Nt} \left( \underset{j=Nt}{\overset{m-1}{\sum}}|\lambda^{j-Nt}| \right) = \\
		= C^2 ||B^N||^t \frac{|\lambda|^{Nt}-1}{|\lambda|-1} + C |\lambda|^{Nt} \frac{|\lambda|^{m-Nt}-1}{|\lambda|-1}
	\end{gathered}$$
	Таким образом, если m (и t) стремятся к $\infty$, то $||D_m|| \to 0$. Отсюда по индукции вытекает утверждение теоремы.
\end{Proof}
\begin{Proof}
	\textit{Второе доказательство}.

	Без ограничения общности можно предполагать, что $\mathbb{F} = \mathbb{C}$. В силу теоремы о приведении к жордановой форме существует такая невырожденная матрица S, что $A = S^{−1}JS$, где J – жорданова форма. При этом $A^k = S^{−1}J^kS$ для всех k. Поэтому достаточно показать, что все элементы матрицы $J^k$ стремятся к нулю при $k \to \infty$. Это утверждение достаточно доказать для одной жордановой клетки. Пусть
	$$J = \begin{pmatrix}
		\lambda & 1 & 0 & \dots & 0 & 0 & 0 \\
		0 & \lambda & 1 & \dots & 0 & 0 & 0 \\
		\vdots & \ddots & \ddots & \ddots & \ddots & \ddots & \vdots \\
		\vdots & \ddots & \ddots & \ddots & \ddots & \ddots & \vdots \\
		0 & 0 & 0 & \dots & \lambda & 1 & 0 \\
		0 & 0 & 0 & \dots & 0 & \lambda & 1 \\
		0 & 0 & 0 & \dots & 0 & 0 & \lambda
	\end{pmatrix}$$
	имеет размер n. Заметим, что $J = \lambda E + B$, где
	$$B = \begin{pmatrix}
		0 & 1 & 0 & \dots & 0 & 0 & 0 \\
		0 & 0 & 1 & \dots & 0 & 0 & 0 \\
		\vdots & \ddots & \ddots & \ddots & \ddots & \ddots & \vdots \\
		\vdots & \ddots & \ddots & \ddots & \ddots & \ddots & \vdots \\
		0 & 0 & 0 & \dots & 0 & 1 & 0 \\
		0 & 0 & 0 & \dots & 0 & 0 & 1 \\
		0 & 0 & 0 & \dots & 0 & 0 & 0
	\end{pmatrix}$$
	Очевидно, что $B^n = 0$. Для любого $k \ge n$ получаем:
	$$J^k = (\lambda E + B)^k = \underset{m=0}{\overset{n}{\sum}}\begin{pmatrix}
		k \\ m
	\end{pmatrix}\lambda^{k-m} B^m$$
	Коэффициент:
	$$\begin{gathered}
		\begin{pmatrix}
			k \\ m
		\end{pmatrix}\lambda^{k-m} = \frac{k(k-1)\dots(k-m+1)}{m!}\lambda^{k-m} = \\
		= \frac{k^m \lambda^k}{m!} \left[ 1 \left( 1-\frac{1}{k} \right) \dots \left( 1 - \frac{m-1}{k} \right) \lambda^{-m} \right] \xrightarrow[k\to \infty]{} 0
	\end{gathered}$$
\end{Proof}













\cleardoublepage
\phantomsection
\addcontentsline{toc}{chapter}{Список используемой литературы}
\begin{thebibliography}{}
	\bibitem{0}
		Курс лекций И.М.Никонова, механико-математический факультет МГУ им. М.В.Ломоносова, 2021 г.
	\bibitem{1}
		Курс семинаров И.М.Никонова, механико-математический факультет МГУ им. М.В.Ломоносова, 2021 г.
\end{thebibliography}

\end{document}
