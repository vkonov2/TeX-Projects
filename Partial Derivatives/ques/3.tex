\chapter{Корректность постановки задачи. Пример Адамара некорректной задачи.}
\label{cha:3}

\begin{definition}\label{cha:3/def:1}
	Задача называется \red{корректной}, если выполнены 3 условия:
	\begin{enumerate}
		\item решение существует
		\item решение единственно
		\item решение непрерывно зависит от данных задачи (начальных условий)
	\end{enumerate}
\end{definition}

\textbf{Пример Адамара некорректной задачи}

$$\begin{cases}
	u_{xx} + u_{yy} = 0 \\
	u|_{y=0} = \varphi (x) \\
	u_{y}|_{y=0} = \psi (x)
\end{cases}$$

Главный символ: $\lambda^2 + \mu^2 \; \Rightarrow$ эллиптический тип.

$(dx)^2 + (dy)^2 = 0$, т.е. $y = 0$ - нехарактеристическое направление, значит сущестует единтвенное решение.

Заметим, что если $\varphi (x) \equiv 0, \psi (x) \equiv 0$, то $u \equiv 0$.\\

Теперь рассмотрим следующую систему:

$$\begin{cases}
	(u_n)_{xx} + (u_n)_{yy} = 0 \\
	u_n|_{y=0} = \frac{1}{n^2} \sin{(nx)} \; (\xrightarrow[n \to \infty]{} 0) \\
	(u_n)_y|_{y=0} = \frac{1}{n} \sin{(nx)} \; (\xrightarrow[n \to \infty]{} 0)
\end{cases}$$

\textbf{Решение}: $u_n = \frac{1}{n^2} \sin{(nx)} e^{n y}$, $u_n \xrightarrow[n \to \infty]{} \infty$, а должно стремиться к 0 (из показанного выше).\\

Т.о. задача Коши для уравнения Лапласа (эллиптического уравнения) поставлена некорректно. 

