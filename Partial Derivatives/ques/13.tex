\chapter{Метод Фурье для уравнений Лапласа и Пуассона в круге и кольце.}
\label{cha:13}

\begin{definition}
	$u \in C^2, \triangle u = 0$, тогда $u$ - гармоническая функция
\end{definition}

\begin{definition}
	$\triangle u = 0$ - уравнение Лапласа
\end{definition}

\begin{definition}
	$\triangle u = f(x)$ - уравнение Пуассона
\end{definition}

1) \blue{Задача Дирихле}: 
	$\begin{cases}
		\triangle u = f(x),  \vec{x} \in \Omega \\
		u|_{\partial \Omega} = h (\vec{x})
	\end{cases}$

\vspace{0.4cm}
2) \blue{Задача Неймана}: 
	$\begin{cases}
		\triangle u = f(x),  \vec{x} \in \Omega \\
		\frac{\partial u}{\partial \vec{n}}|_{\partial \Omega} = g (\vec{x})
	\end{cases}$

	\textit{Условие разрешимости:} 
		$\iint \limits_{\Omega} f(\vec{x}) d \vec{x} = \oint g(\vec{x}) d \sigma $

	(если $\not =$, то решений нет, а если $=$, то решений бесконечно много)

\vspace{0.4cm}
3) \blue{Смешанная задача}: 
	$\begin{cases}
		\triangle u = 0,  \vec{x} \in \Omega \\
		(\alpha \frac{\partial u}{\partial \vec{n}} - \beta u)|_{\partial \Omega} = q (\vec{x})
	\end{cases}$

\vspace{0.5cm}
Решим задачу Дирихле в кольце (окружности радиусов $r_1$ и $r_2$):

$$\begin{cases}
	\triangle u = 0,  r_1^2 < x^2 + y^2 < r_2^2\\
	u|_{x^2 + y^2 = r_1^2} = f_1 (x, y)\\
	u|_{x^2 + y^2 = r_2^2} = f_2 (x, y)
\end{cases}$$
\begin{solution}
	\hfill\break
	1) Перейдем к полярным координатам 
	$$\begin{cases}
		x = r \cos \phi \\
		y = r \sin \phi 
	\end{cases} \begin{cases}
		r = \sqrt{x^2 + y^2} \\
		\phi = arctg \frac{y}{x}
	\end{cases}$$
	В этих координатах задача имеет вид:
	$$\begin{cases}
		\triangle u = u_{rr} + \frac{1}{r^2} u_{\phi \phi} + \frac{1}{r} u_r = 0,  r_1 < r < r_2\\
		u|_{r= r_1} = f_1 (\phi), 0 \leq \phi \leq 2 \pi \\
		u|_{r = r_2} = f_2 (\phi) \\
		u(r, \phi + 2 \pi) = u(r, \phi), \quad u_{\phi} (r, \phi + 2 \pi) = u_{\phi} (r, \phi)
	\end{cases}$$

	2) Решение ищем в виде ряда $u(r, \phi) = \sum \limits_{n = 0}^{\infty} R_n(r) X_n(\phi)$. Подставим в уравнение:
	$$\begin{gathered}
		R''(r)X(\phi) + \dfrac{1}{r}R'(r)X(\phi) + \dfrac{1}{r^2}R(r)X''(\phi) = 0 \\
		\dfrac{r^2 R''(r)}{R(r)} + \dfrac{r R'(r)}{R(r)} = -\dfrac{X''(\phi)}{X(\phi)} = \lambda \; \Rightarrow 
		\begin{cases}
			-\dfrac{X''(\phi)}{X(\phi)} = \lambda \\
			X(\phi + 2 \pi) = X(\phi)
		\end{cases} 
	\end{gathered}$$
	Имеем уравнение: $-X''(\phi) = \lambda X(\phi) \; \Rightarrow \; X''(\phi) + \lambda X(\phi) = 0$
	\begin{itemize}
	\item 
		$ \lambda > 0 \; \Rightarrow \;   X(\phi) = a\cos(\sqrt{\lambda}\phi) + b\sin(\sqrt{\lambda}\phi)$. Тогда базис состоит их $\cos(n\phi)$ и $\sin(n\phi)$, $n \in \mathbb{N}$.
	\item 
		$ \lambda < 0 \; \Rightarrow \;   X(\phi) = c_1 e^{\sqrt{\lambda}\phi} + c_n e^{-\sqrt{\lambda}\phi} $ -- не периодическая, отпадает.
	\item 
		$ \lambda = 0 \; \Rightarrow \; X(\phi) = c_1 \phi + c_2$. Из условия периодичности: $c_1 (\phi + 2\pi) + c_2 = c_1 \phi + c_2 \; \Rightarrow \; c_1 = 0, \; c_2 = 1 \; \Rightarrow \;  X_0(\phi) = 1 $
	\end{itemize}
	Получаем базис: $\{\sin(n \phi), \cos(n \phi), 1\}, \; n \in \mathbb{N}$. Итоговая формула:
	$$u(r, \phi) = R_0(r) + \sum \limits_{m = 1}^{\infty} R_m(r) \cos(m \phi) + \sum \limits_{n = 1}^{\infty} V_n (r) \sin(n \phi)$$
	Пусть $f_i(\phi) = c_i + \sum\limits_{k_i = 1}^\infty \cos(k_i \phi) a_{k_i} + \sum\limits_{s_i = 1}^\infty\sin(s_i \phi) b_{s_i}, \; i = 1, 2, \; c_i, \; a_{k_i}, \; b_{s_i} \in \mathbb{R}$.

	3) Подставляем решение ввиде ряда $u(r, \phi)$ в исходную систему и получаем задачи Коши на $R_n(r)$ и $V_n(r)$:

	$$\begin{cases}
		R_0''(r) + \sum\limits_{m = 1}^\infty R_m''(r)\cos(m \phi) + \sum\limits_{n = 1}^\infty V_n''(r)\sin(n \phi) + \dfrac{1}{r}(R_0'(r) + \sum\limits_{m = 1}^\infty R_m'(r)\cos(m \phi) + \\
		\hspace{0.8cm}+ \sum\limits_{n = 1}^\infty V_n'(r)\sin(n \phi)) - \dfrac{1}{r^2}(\sum\limits_{m = 1}^\infty m^2 R_m(r)\cos(m \phi) + \sum\limits_{n = 1}^\infty n^2 V_n(r) \sin(n \phi)) = 0 \\
		R_0(r_1) + \sum \limits_{n = 1}^{\infty} (R_n(r_1) \cos(n \phi) + V_n (r_1) \sin(n \phi)) = f_1 (\phi) \\
		R_0(r_2) + \sum \limits_{n = 1}^{\infty} (R_n(r_2) \cos(n \phi) + V_n (r_2) \sin(n \phi)) = f_2 (\phi) \\
	\end{cases}$$
	\begin{itemize}
		\item 
			задача на $R_0:$ 
			$\begin{cases}
				R_0''(r) + \dfrac{1}{r} R_0'(r) = 0 \\
				R_0(r_1) = c_1 \\
				R_0(r_2) = c_2
			\end{cases}$
		\item 
			задача на $R_m:$
			$$\begin{gathered}
				\begin{cases}
				r^2 R_{k_{12}}''(r) + r R_{k_{12}}'(r) - k_i^2 R_{k_{12}}(r) = 0, \; k_{12} = k_1 = k_2  \text{ -- номер в ряде,} \\ \text{соответствующий }m \\
				R_{k_{12}}(r_1) = a_{k_1} \\
				R_{k_{12}}(r_2) = a_{k_2} \\
				\end{cases}
			\end{gathered}$$
		\item 
			задача на $ V_n: $
			$$\begin{gathered}
				\begin{cases}
				r^2 V_{k_{12}}''(r) + r V_{k_{12}}'(r) - k_i^2 V_{k_{12}}(r) = 0, \; k_{12} = k_1 = k_2 \text{ -- номер в ряде,} \\ \text{соответствующий } n\\
				V_{k_{12}}(r_1) = b_{k_1} \\
				V_{k_{12}}(r_2) = b_{k_2} \\
				\end{cases}
			\end{gathered}$$
	\end{itemize}

	4) Решаем задачи на $R_0, R_n, V_n$, пишем ответ $u(r, \phi)$, используя найденные $R_0, R_n, V_n$

\end{solution}

\begin{remark}
В случае круга, $ r \in [0, \rho],$ граничные условия:
$\begin{cases}
	u |_{r = \rho} = f(\phi) \\
	| u |_{r = 0} | < \infty
\end{cases}$
\end{remark}






















