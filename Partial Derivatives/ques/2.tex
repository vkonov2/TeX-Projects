\chapter{Задача Коши для линейного уравнения второго порядка. Теорема Коши-Ковалевской (без доказательства).}
\label{cha:2}

\begin{definition}[\red{Задача Коши}]\label{cha:2/def:1}
	\hfill \break
	Пусть $S \subset \mathbb{R}^n$ - поверхность, $dim S = n-1$, $S: F(x_1, \dots, x_n) = 0$.
	$$\begin{cases}
		\underset{i,j=1}{\overset{n}{\sum}}a_{i j} u''_{x_i x_j} + \underset{i=1}{\overset{n}{\sum}}b_i u'_{x_i} + c u = f (x_1, \dots, x_n), \;\; a_i, b_i, c, f\text{ - функции от }x_1, \dots, x_n \\
		u|_s = \varphi (\overrightarrow{x}), \; S \subset \mathbb{R}^n \\
		\frac{\partial u}{\partial \overrightarrow{l}}|_S = \psi (\overrightarrow{x}), \text{ где } \overrightarrow{l} \text{ - не касательное направление к поверхности } S
	\end{cases}$$
\end{definition}

\begin{remem}\label{cha:1/remem:1}
	\blue{Аналитическая функция} вещественного переменного - это функция, которая совпадает со своим рядом Тейлора в окрестности любой точки области определения.
\end{remem}

\begin{theorem}[\red{Коши-Ковалевской}]\label{cha:2/the:1}
	Пусть выполняется:
	\begin{itemize}
		\item[$1)$] $a_{i j} (\overline{x}), b_i (\overline{x}), c, f, \psi, \xi$ - аналитические функции в области $\Omega: s \subset \mathbb{R}$\\
		$S: F(\overline{x}) = 0 \; \Rightarrow \; F$ - тоже аналитическая функция в $\Omega$
		\item[$2)$] поверхность $S$ не характеристическая, т.е. $A(\overrightarrow{n}) \not = 0$ ни в одной точке $S$.
	\end{itemize}
	Тогда $\exists ! \; u(\overline{x})$ - решение задачи Коши в некоторой области $Q: S \subset Q \subset \Omega$ и $u(\overline{x})$ - аналитическая функция в $Q$.
\end{theorem}

\begin{remark}\label{cha:2/remark:1}
	Решение задачи Коши непрерывно зависит от начальных условий.
\end{remark}
\begin{Proof}
	Если $\varphi_n \xrightarrow[]{\text{равн-но}} \varphi, \; \psi_n \xrightarrow[]{\text{равн-но}} \psi$, то $u_n \xrightarrow[]{\text{равн-но}} u$ при $t < t_0$.\\
	$\begin{cases}
		\varphi_n(x-t) \xrightarrow[]{\text{равн-но}} \varphi (x-t) \\
		\varphi_n (x+t) \xrightarrow[]{\text{равн-но}} \varphi (x+t)
	\end{cases} \Rightarrow \; |\underset{x-t}{\overset{x+t}{\int}} \psi d\xi - \underset{x-t}{\overset{x+t}{\int}} \psi_n d\xi| \le \underset{x-t}{\overset{x+t}{\int}} |\psi - \psi_n| d\xi \le \\
	\le 2 t \cdot max |\psi - \psi_n| \to 0$, т.к. $|\psi - \psi_n| \xrightarrow[]{\text{равн-но}} 0$.
\end{Proof}

