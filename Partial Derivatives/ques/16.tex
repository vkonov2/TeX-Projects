\chapter{Фундаментальное решение оператора Лапласа в $R^2$ и $R^3$.}
\label{cha:16}

Рассмотрим оператор Лапласса: $L = \triangle = \frac{\partial^2}{\partial x_1^2} + \dots + \frac{\partial^2}{\partial x_n^2}$.

\begin{theorem}[\red{Фундаментальное решение оператора Лапласса}]\label{lec:16/the:1}
	Фундаментальное решение оператора Лапласса имеет вид:
	$$ \varepsilon (x) = 
	\begin{cases}
		\displaystyle \frac{1}{2 \pi} \ln |\vec{x}|, \; \vec{x} \in \mathbb{R}^2\\
		\displaystyle - \frac{1}{4 \pi |\vec{x}|}, \; \vec{x} \in \mathbb{R}^3 \\
		\displaystyle -\frac{1}{\sigma_n |\vec{x}|^{n-2}}, \; \vec{x} \in \mathbb{R}^n , n \ge 4 
	\end{cases}$$
	где $\sigma_n$ - площадь поверхности единичной сферы в $\mathbb{R}^n$.
\end{theorem}
\begin{Proof}
	Необходимо проверить, что $\triangle \varepsilon (x) = \delta (x) $, т.е. что:
	$$(\triangle \varepsilon (x) , \varphi (x)) = (\delta (x), \varphi (x)) = \varphi (0)$$
	1) Рассмотрим \blue{$\mathbb{R}^2$}, то есть когда $\varepsilon (x) = \displaystyle \frac{1}{2 \pi} \ln |\vec{x}|$.
	$$\begin{gathered}
		(\triangle \varepsilon (x) , \varphi (x)) = (\varepsilon_{x_1 x_1} + \varepsilon_{x_2 x_2} , \varphi (x)) = (\varepsilon (x), \varphi_{x_1 x_1} + \varphi_{x_2 x_2}) = \\
		= (\varepsilon (x) , \triangle \varphi (x)) = \iint \limits_{\mathbb{R}^2} \varepsilon (x) \triangle \varphi (x) d \vec{x} = \lim_{\alpha \to 0} \iint \limits_{\alpha < |\vec{x}| < R} \varepsilon (x) \triangle \varphi (x) d \vec{x} =  \\
		=\Big|\text{2-ая формула Грина}\Big|
		= \lim_{\alpha \to 0} \left(\iint \limits_{\alpha < |\vec{x}| < R} \varphi (x) \triangle \varepsilon (x) d \vec{x} + \oint \limits_{|\vec{x}| = \alpha} (\varepsilon \frac{\partial \varphi}{\partial \vec{n}} - \varphi \frac{\partial \varepsilon}{\partial \vec{n}}) d \sigma\right)
	\end{gathered}$$
	(интеграл $\oint \limits_{|\vec{x}| = R}=0$ , т.к. в силу финитности $\varphi \equiv 0$ на $|x| = R$)

	Введем обозначения:
	$$I_1 =  \iint \limits_{\alpha < |\vec{x}| < R} \varphi (x) \triangle \varepsilon (x) d \vec{x}, \; I_2 = \oint \limits_{|\vec{x}| = \alpha} \varepsilon \frac{\partial \varphi}{\partial \vec{n}} d \sigma, \; I_3 = - \oint \limits_{|\vec{x}| = \alpha} \varphi \frac{\partial \varepsilon}{\partial \vec{n}} d \sigma$$
	Рассмотрим $I_1$. Перейдем в полярные координаты, тогда оператор Лапласса записывается следующим образом:
	$$ \triangle = \frac{\partial^2}{\partial \rho^2} + \frac{1}{\rho} \frac{\partial}{\partial \rho} \left(+ \frac{1}{\rho^2} \frac {\partial^2}{\partial \theta^2}\right) $$
	Так как $\varepsilon = \varepsilon (\rho)$, т.е. не зависит от $\theta$, то $\left(+ \frac{1}{\rho^2} \frac {\partial^2}{\partial \theta^2}\right)$ не нужно.
	$$\triangle \varepsilon = (\frac{\partial^2}{\partial \rho^2} + \frac{1}{\rho} \frac{\partial}{\partial \rho}) (\frac{1}{2 \pi} \ln \rho) = \frac{1}{2 \pi} (\frac {1}{\rho} \frac {1}{\rho} - \frac {1}{\rho^2}) = 0$$
	Таким образом, $I_1 = 0$, т.к. в кольце $(\alpha < \rho < R)$ нет нуля.\\

	Рассмотрим $I_2$:
	$$\begin{gathered}
		I_2 = \int \limits_0^{2 \pi} (\varepsilon \frac{\partial \varphi}{\partial \vec{n}} \rho) |_{\rho = \alpha} d \theta = \Big| \rho \text{ - якобиан полярной замены}\Big| = \\
		= \frac{1}{2 \pi} \int \limits_0^{2 \pi} (\ln \rho \frac{\partial \varphi}{\partial \vec{n}} \rho) |_{\rho = \alpha} d \theta  = \frac{1}{2 \pi} \alpha \ln \alpha \underbrace{\int \limits_0^{2 \pi} \frac{\partial \varphi}{\partial \vec{n}}d \theta}_{= const} \xrightarrow[\alpha \to 0]{} 0
	\end{gathered}$$
	Рассмотрим $I_3$. На внутренней границе $\displaystyle \frac{\partial}{\partial \vec{n}} = -\frac{\partial}{\partial \rho}$. Тогда:
	$$\begin{gathered}
		I_3 = - \int \limits_0^{2 \pi} (\varphi (- \frac{\partial \varepsilon}{\partial \vec{n}}) \rho) |_{\rho = \alpha} d \theta =  \int \limits_0^{2 \pi} (\varphi \frac{1}{2 \pi \rho} \rho) |_{\rho = \alpha} d \theta  = \frac{1}{2 \pi} \int \limits_0^{2 \pi} \varphi (\rho, \theta) |_{\rho = \alpha} d \theta = \\
		= \frac{1}{2 \pi} \int \limits_0^{2 \pi} \varphi (\alpha, \theta) d \theta = \underbrace{\frac{1}{2 \pi} \varphi (\alpha, \theta^{*}) \int \limits_0^{2 \pi} d \theta}_{\text{теорема о среднем}} = \varphi (\alpha, \theta^{*}) \xrightarrow[\alpha \to 0]{} \varphi (0)
	\end{gathered}$$\newpage
	2) Рассмотрим \blue{$\mathbb{R}^3$}, то есть когда $\varepsilon (x) = \displaystyle - \frac{1}{4 \pi |\vec{x}|}, \; \vec{x} \in \mathbb{R}^3$.
	$$\begin{gathered}
		(\triangle \varepsilon (x) , \varphi (x)) = (\varepsilon_{x_1 x_1} + \varepsilon_{x_2 x_2} + \varepsilon_{x_3 x_3}, \varphi (x)) = (\varepsilon (x), \varphi_{x_1 x_1} + \varphi_{x_2 x_2} + \\
		\varphi_{x_3 x_3}) = (\varepsilon (x) , \triangle \varphi (x)) = \iiint \limits_{\mathbb{R}^3} \varepsilon (x) \triangle \varphi (x) d \vec{x} = \lim_{\alpha \to 0} \iiint \limits_{\alpha < |\vec{x}| < R} \varepsilon (x) \triangle 		\varphi (x) d \vec{x} =  \\
		=\Big|\text{2-ая формула Грина}\Big|
		= \lim_{\alpha \to 0} \left(\iiint \limits_{\alpha < |\vec{x}| < R} \varphi (x) \triangle \varepsilon (x) d \vec{x} + \oint \limits_{|\vec{x}| = \alpha} (\varepsilon \frac{\partial \varphi}{\partial \vec{n}} - \varphi \frac{\partial \varepsilon}{\partial \vec{n}}) d \sigma\right) = \\
		= \lim_{\alpha \to 0} \left(\underbrace{\iiint \limits_{\alpha < |\vec{x}| < R} \varphi (x) \triangle \varepsilon (x) d \vec{x}}_{I_1} + \underbrace{\oint \limits_{|\vec{x}| = \alpha} \varepsilon \frac{\partial \varphi}{\partial \vec{n}} d \sigma}_{I_2} + \underbrace{\oint \limits_{|\vec{x}| = \alpha}  -\varphi \frac{\partial \varepsilon}{\partial \vec{n}} d \sigma}_{I_3} \right)
	\end{gathered}$$
	\begin{itemize}
		\item $I_1 = \iiint \limits_{\alpha < |\vec{x}| < R} \varphi (x) \triangle \varepsilon (x) d \vec{x} = 0$, так как 
		$$\begin{gathered}
			\varepsilon = -\dfrac{1}{4\pi \rho}, \; \triangle\varepsilon = \varepsilon_{\rho\rho} + \dfrac{2}{\rho}\varepsilon_{\rho} \\
			\varepsilon_{\rho} = \dfrac{1}{4\pi\rho^2}, \; \varepsilon_{\rho\rho} = \dfrac{-2}{4\pi\rho^3} \Longrightarrow \triangle\varepsilon = 0
		\end{gathered}$$
		\item $I_2 = \oint \limits_{|\vec{x}| = \alpha} \varepsilon \dfrac{\partial \varphi}{\partial \vec{n}}d\sigma $. Сделаем замену:
		$$\begin{gathered}
			\begin{cases}		
				x_1 = \rho \sin\theta\cos\phi \\	
				x_2 = \rho \sin\theta\sin\phi \\	
				x_3 = \rho \cos\theta
			\end{cases},  \theta \in [0, \pi], \; \phi \in [0, 2\pi), 
			\mathbb{J} = 
			\begin{vmatrix}
				x_{1\rho} & x_{1\theta} & x_{1\phi}  \\
				x_{2\rho} & x_{2\theta} & x_{2\phi}  \\
				x_{3\rho} & x_{3\theta} & x_{3\phi}  \\
			\end{vmatrix} =
			\rho^2\sin\theta,
		\end{gathered}$$
		тогда
		$$\begin{gathered}
			I_2 = -\int\limits_{0}^{2\pi} \int\limits_{0}^{\pi} \varepsilon \dfrac{\partial \varphi}{\partial \rho}|_{\rho = \alpha}(\rho^2\sin\theta)
			|_{\rho = \alpha} d\theta d\phi = \\
			= \dfrac{\alpha^2}{4\pi\alpha} \int\limits_{0}^{2\pi} \int\limits_{0}^{\pi}  \dfrac{\partial \varphi}{\partial \rho}|_{\rho = \alpha}\sin\theta
			d\theta d\phi = \dfrac{\alpha}{4\pi} const \underset{\alpha \rightarrow 0}{\longrightarrow} 0
		\end{gathered}$$
		\item $ I_3 = - \oint \limits_{|\vec{x}| = \alpha} \varphi \dfrac{\partial \varepsilon}{\partial \rho} d \sigma $ 
		$$\begin{gathered}
			 I_3 = \dfrac{\alpha^2}{4\pi\alpha^2} \int\limits_{0}^{2\pi} \int\limits_{0}^{\pi} \varphi(\alpha, \theta, \phi) \sin\theta d\theta d \phi = \\
			 = \Big|\text{теорема о среднем}\Big| = 
			 \dfrac{1}{4\pi} \int\limits_{0}^{2\pi} \varphi(\alpha, \theta^{*}, \phi)\underbrace{\int\limits_{0}^{\pi} \sin\theta d\theta }_{= -\cos\theta|_{0}^\pi 
			 = 2} d\phi = \\
			 = \dfrac{1}{2\pi} \int\limits_{0}^{2\pi} \varphi(\alpha, \theta^{*}, \phi) d\phi = \varphi(\alpha, \theta^{*}, \phi^{*}) \underset{\alpha \rightarrow 
			 0}{\longrightarrow} \varphi(0,0,0), \; \theta^{*} \in [0, \pi], \; \phi^{*} \in [0, 2\pi)
		\end{gathered}$$
	\end{itemize}
\end{Proof}







