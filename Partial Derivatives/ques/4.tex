\chapter{Задача Коши для уравнения струны. Формула Даламбера. Решение неоднородного уравнения, принцип Дюамеля.}
\label{cha:4}

\textbf{Уравнение колебаний струны}: $u_{tt} = a^2 u_{xx}$.

\begin{enumerate}
	\item \blue{начальные} условия: $t=0$
	$\begin{cases}
		u|_{t=0} = \varphi(x) \text{ - отклонение}\\
		u_t |_{t=0} = \psi(x) \text{ - скорость}
	\end{cases}$
	\item \blue{граничные} (краевые) условия: $x=0, x=l$
		\begin{itemize}
			\item[$\RNumb{1} \text{ рода}$]: $\begin{cases}
				u|_{x=0} = 0 \\
				u|_{x=0} = \mu(t)
			\end{cases}$
			\item[$\RNumb{2} \text{ рода}$]: $\begin{cases}
				{u_x}|_{x=0} = 0 \\
				{u_x}|_{x=0} = \nu(t)
			\end{cases}$
			\item[$\RNumb{3} \text{ рода}$]: $\begin{cases}
				(\alpha u_x + \beta u)|_{x=0} = \gamma(t) \\
				(\alpha u_x + \beta u)|_{x=l} = \delta(t)
			\end{cases}$
		\end{itemize}
\end{enumerate}

\red{Задача Коши для уравнения струны}:
$$\begin{cases}
	u_{tt} = a^2 u_{xx}, \; t >0, x \in \mathbb{R} \\
	u|_{t=0} = \varphi (x)\\
	u_t |_{t=0} = \psi (x)
\end{cases}$$

$u (t,x)$ - положение точки $x$ струны в момент времени $t$ (отклонение от оси $x$ в одномерном случае).

$$\begin{gathered}
	A(\lambda, \mu) = \lambda^2 - a^2 \mu^2 \\
	A(dx, -dt) = (dx)^2 - a^2 (dt)^2 = 0 \; \Rightarrow \; (x'_t)^2 = a^2 \; \Rightarrow \; x_t = \pm a \; \Rightarrow \\
	\Rightarrow \; x = \pm a t +c \text{ - уравнения характеристик.}
\end{gathered}$$

Делаем замену:
$$\begin{cases}
	\xi = x + a t \\
	\eta = x - a t
\end{cases} \; \Rightarrow \; u_x = u_{\xi} + u_{\eta}, \; u_t = a u_{\xi} - a u_{\eta}$$

$$\begin{gathered}
	u_{xx} = u_{\xi \xi} + 2 u_{\xi \eta} + u_{\eta \eta}, \; u_{tt} = a^2 u_{\xi \xi} - 2 a^2 u_{\xi \eta} + a^2 u_{\eta \eta} \; \Rightarrow \\
	\Rightarrow \; a^2 (u_{\xi \xi} - 2 u_{\xi \eta} + u_{\eta \eta}) = a^2 (u_{\xi \xi} + 2 u_{\xi \eta} + u_{\eta \eta}) = 0 \; \Rightarrow \; u_{\xi \eta} = 0\\
	\frac{\partial u_{\xi}}{\partial \eta} = 0 \; \Rightarrow \; u_{\xi} =  F(\xi) \; \Rightarrow \; u = f(\xi) + c (\eta) \; \Rightarrow \\
	\Rightarrow \; u(x,t) = f(x + a t) + c (x - a t)
\end{gathered}$$

Определим $f$ и $c$ так, чтобы они удовлетворяли начальным условиям:

$$\begin{gathered}
	\varphi (x) = f(x) + c(x) \\
	u_t = a f' (x+a t) - a c' (x-a t) \; \Rightarrow \; \psi (x) = a (f'(x) - c'(x))
\end{gathered}$$

Проинтегрируем:
$$\begin{gathered}\frac{1}{a} \underset{0}{\overset{x}{\int}} \psi (y) dy + K = f(x) - c(x) \; \Rightarrow \\
	 \begin{cases}
	 	f(x) = \frac{1}{2}\left( \frac{1}{a} \underset{0}{\overset{x}{\int}} \psi (y) dy + \varphi(x) \right) + \frac{K}{2}\\
 		c(x) = \frac{1}{2} \left( \varphi(x) - \frac{1}{a} \underset{0}{\overset{x}{\int}} \psi (y) dy \right) - \frac{K}{2}
	 \end{cases} \; \Rightarrow \\
	 u(x,t) = \frac{1}{2} \left( \frac{1}{a} \underset{x - a t}{\overset{x + a t}{\int}}\psi (y) dy + \varphi (x + a t) + \varphi (x - a t) \right) \text{ - \red{формула Даламбера}}
 \end{gathered}$$

 \begin{remark}\label{cha:4/remark:this}
 	Полученное $u(t,x)$ - действительно единственное решение задачи Коши, но в предположении, что $\varphi (x) \in C^2 (\mathbb{R}), \; \psi (x) \in C^1 (\mathbb{R})$.
 \end{remark}

 \textbf{Примеры задачи Коши для разных струн}:

 \begin{enumerate}
 	\item ограниченная струна: $x \in (0, l)$
 	$$\begin{cases}
 		u_{tt} = a^2 u_{xx} + f(t,x), \; x \in (0,l), \; t > 0 \\
 		u|_{t=0} = \varphi (x) \\
 		u_t |_{t=0} = \psi (x) \\
 		u_x |_{x=0} = \nu (t) \\
 		(\alpha u_x + \beta u)|_{x=l} = \gamma (t)
 	\end{cases}$$
 	\item бесконечная струна: $x \in \mathbb{R}$
 	$$\begin{cases}
 		u_{tt} = a^2 u_{xx} + f(t,x), \; x \in \mathbb{R}, \; t > 0 \\
 		u|_{t=0} = \varphi (x) \\
 		u_t |_{t=0} = \psi (x)
 	\end{cases}$$
 	\item полуограниченная струна: $x \in (0, +\infty)$
 	$$\begin{cases}
 		u_{tt} = a^2 u_{xx} + f(t,x), \; x > 0, \; t > 0 \\
 		u|_{t=0} = \varphi (x) \\
 		u_t |_{t=0} = \psi (x) \\
 		u_x |_{x=0} = \nu (t)
 	\end{cases}$$
 \end{enumerate}

 \textbf{Принцип Дюамеля}\\

 Рассмотрим неоднородное уравнение:
$$\begin{cases}
	u_{tt} = a^2 u_{xx} + f(t,x), \; t > 0, \; x \in \mathbb{R} \\
	u|_{t=0} = \varphi (x) \\
	u_t |_{t=0} = \psi (x)
\end{cases}$$

$$u = u_1 + u_2$$

\begin{multicols}{2}
	$\begin{cases}
		(u_1)_{tt} = a^2 (u_1)_{xx}, \; t > 0, \; x \in \mathbb{R} \\
		(u_1)|_{t=0} = \varphi (x) \\
		(u_1)_t |_{t=0} = \psi (x)
	\end{cases}$
	\columnbreak
	$\begin{cases}
		(u_2)_{tt} = a^2 (u_2)_{xx} + f(t,x), \; t > 0, \; x \in \mathbb{R} \\
		(u_2)|_{t=0} = 0 \\
		(u_2)_t |_{t=0} = 0
	\end{cases}$
\end{multicols}

Решение для $u_1$ получаем по формуле Даламбера. Решим систему для $u_2$. Рассмотрим вспомогательную задачу для $v(t, x, \tau)$:
$$\begin{cases}
	v_{tt} = a^2 v_{xx}, \; t > \tau, \; x \in \mathbb{R} \\
	v|_{t = \tau} = 0 \\
	v_t|_{t = \tau} = f(\tau, x)
\end{cases}$$

$$\begin{gathered}
	u_2 (t,x) \QUdef \underset{0}{\overset{t}{\int}}v(t,x,\tau)d\tau \\
	u_2 |_{t=0} = 0 \\
	{u_2}_t = v(t,x,t) + \underset{0}{\overset{t}{\int}}v_t (t,x,\tau)d\tau, \; v(t,x,t) = 0 \; \Rightarrow \; {u_2}_t |_{t=0} = 0 \\
	{u_2}_{xx} = \underset{0}{\overset{t}{\int}}v_{xx}(t,x,\tau)d\tau \\
	{u_2}_{tt} = v_t (t,x,t) + \underset{0}{\overset{t}{\int}}v_{tt}(t,x,\tau)d\tau = f(t,x) + a^2 \underset{0}{\overset{t}{\int}}v_{xx} (t,x,\tau)d\tau = f(t,x) + a^2 u_{xx}
\end{gathered}$$

Введем замену $z = t - \tau$, тогда:
$$\begin{cases}
	v_{zz} = a^2 v_{xx}, \; z >0, \; x \in \mathbb{R} \\
	v|_{z=0} = 0 \\
	v_z |_{z=0} = f(\tau,x)
\end{cases}$$

Отсюда получаем:
$$\begin{gathered}
	v(z,x,\tau) = \frac{1}{2a} \underset{x - a z}{\overset{x + a z}{\int}}f (\tau, y) dy \\
	v(t,x,\tau) = \frac{1}{2a}\underset{x-a(t-\tau)}{\overset{x+a(t-\tau)}{\int}}f(\tau,y)dy \\
	u_2 (t,x) = \frac{1}{2a}\underset{0}{\overset{t}{\int}}\underset{x-a(t-\tau)}{\overset{x+a(t-\tau)}{\int}}f(\tau,y)dy d\tau
\end{gathered}$$

Получаем итоговую формулу:
$$u(t,x) = \frac{1}{2a} \left( \underset{x-at}{\overset{x+at}{\int}}\psi (y) dy + \underset{0}{\overset{t}{\int}} \underset{x-a(t-\tau)}{\overset{x+a(t-\tau)}{\int}}f(\tau,y)dy d\tau \right) + \frac{1}{2}\left( \varphi(x+at) + \varphi (x-at) \right)$$

