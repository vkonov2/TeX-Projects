\chapter{Лекция № 3 - 25.02.2021. Модели страхований жизни} % (fold)
\begin{remem}
	\red{Одноразовая нетто-ставка}
	\[ \pi = EpvY = EZ -\]
	основная часть платежа клиента.
Она не учитывает 
\begin{itemize}
	\item затрат на обслуживание клиента
	\item риск, который несет страховая компания 
\end{itemize}
\end{remem}

Отсюда
\begin{definition}
	\red{Брутто-ставка} - весь платеж.
\end{definition}

Так как для расчета нетто-ставок нужно знать распредление Z, то и высчитывать мы будеи именно ее, а не брутто-ставку.

\begin{remark}
	Считаем, что сумма, выплачиваемая с/к после страхового случая, равна 1.
\end{remark}


Сумма, которую выплачивает с/к фиксированна, в то время как момент выплаты случаен (ибо случано $ T_x$). В то же время считаем, что страховой взнос (страховая премия) платится один раз в момент $ t = 0$(момент заключения контракта).

Прежде, чем мы перейдем к описанию моделей страхования, посчитаем некоторые нетто-ставки.

\begin{example}
	\begin{enumerate}
		\item \red{Нетто-ставка при страховании на всю жизнь(непрервыный случай).}

		Пусть 
		\begin{gather*}
			T = K + S\\
			Z = pv(1) = v^T, 
		\end{gather*}
		где $ v = \frac{1}{1+i} -  $ дисконтирующий множитель.

		Тогда одноразовая нетто-ставка равна
		\[ \overline{A}_x = EZ = \int\limits^{\infty}_{0}v^tf(t)dt = \int\limits^{\infty}_{0}v^t{}_tp_x\mu_xdt.\]
		Это непрерывный случай, причем выплата происходит сразу после наступления страхового случая.

		\item Рассмотрим более простой случай.

		Будем считать, что страховая сумма выплачивается в конце года смерти, т.е. в момент
		\begin{gather*}
			T = K+1\\
			Z = v^{K+1}\\
			P(Z = v^{K+1}| K = k) = P(Z = v^{k+1}) = {}_kp_xq_{x+k} 
		\end{gather*}
		и начнем новую секцию лол
	\end{enumerate}
\end{example}

\section{Модели долгосрочного страхования жизни} % (fold)
\begin{enumerate}
	\item \red{Страхование на всю жизнь(дискретный)} (как в пункте 2) предыдущей секции)

	Нетто-ставка будет такой
	\begin{gather*}
		EZ = A_x = E(v^{k+1}) = \sum\limits_{k = 0}^{\infty}{}_kp_xq_{x+k}\\
		DZ = EZ^2 - (EZ)^2
	\end{gather*}

	\item \red{Страхование на конечный промежуток времени(term insurance).}

	По такому договору страховая сумма выплачивается только в случае, если клиент умрёт в течение n лет после заключения договора (\red{n-year term insurance}).

	В этом случае
	\[
	Z = 
		\begin{cases}
			v^{k+1},\;\;\; k = 0,1, 2, ..., n-1\\
			0 ,\;\;\; k = n, n+1, ...
		\end{cases}
	\] 
	и нетто-ставка
	\begin{gather*}
		EZ = A_{x:n\urcorner}^{1} = \sum\limits_{k=0}^{n-1}v^{k+1}{}_kp_xq_{x+k}\\
		DZ = E(Z^2) - (A_{x:n\urcorner}^{1} )^2
	\end{gather*}

	\item \red{Страхователь прожил n лет после заключения договора (pure endowment).}

	\[
		Z = \begin{cases}
			0,\;\;\; k = 0,1,2,...\\
			v^k, \;\;\; k = n, n+1, ...
		\end{cases}
	\]

	Нетто-ставка
	\begin{gather*}
		EZ = A_{x:n\urcorner}^{\;\;\;\;1} = v^n{}_np_x\\
		DZ = EZ^2 - (A_{x:n\urcorner}^{\;\;\;\;1})^2 = v^{2n}{}_np_x - v^{2n}({}_np_x)^2 = v^{2n}{}_np_x{}_nq_x
	\end{gather*}

	\item \red{Endowments}

	Страховая сумма выплачивается в конце года смерти, если она произошла в первые n-лет, или в конце n-ого года, если смерть не наступила.
	\begin{gather*}
		Z= \begin{cases}
			v^{k+1}, \;\;\; k = 0, 1, 2, ..., n-1\\
			v^n, \;\;\; k = n, n+1,...
		\end{cases}\\
		Z = Z_1 + Z_2\\
		Z_1 = \begin{cases}
			v^{k+1}, \;\;\;k = 0, 1, ..., n-1\\
			0, \;\;\;
		\end{cases}(\text{term insurance})\\
		Z_2 = \begin{cases}
			0, \;\;\; k = 0, 1, ..., n-1\\
			v^n, \;\;\; k = n, n+1, ...\\
		\end{cases}(\text{pure endowments})\\
		EZ = EZ_1 + EZ_2 = A_{x:n\urcorner} = A_{x:n\urcorner}^{1}+ A_{x:n\urcorner}^{\;\;\;\;1}\\
		DZ = DZ_1 + DZ_2 + 2cov(Z_1Z_2) = \\
		=[Z_1Z_2 = 0 \Rightarrow \;\;\; cov(Z_1Z_2) = EZ_1Z_2 - EZ_1EZ_2 = -A_{x:n\urcorner}^{1}A_{x:n\urcorner}^{\;\;\;\;1}]=\\
		=DZ_1+DZ_2 -2 A_{x:n\urcorner}^{1}A_{x:n\urcorner}^{\;\;\;\;1}
	\end{gather*}
	Отсюда видно, что риск от продажи endowments policy меньше, чем от продажи term insurance одному человеку и pure endowments другому.
	
	\item \red{ Отсроченное на m лет страхование на всю жизнь.}

	\[
		Z = \begin{cases}
			0, \;\;\; k = 0, 1, ..., m-1\\
			v^{k+1}, k = m, m+1, ...
		\end{cases}
	\]

	В этом случае нетто-ставка
	\[ {}_{m|}A_x = {}_mp_xv^mA_{x+m} \]
	или
	\[ {}_{m|}A_x = A_x - A_{x:m\urcorner}^{1} = \sum\limits_{k=0}^{\infty}v^{k+1}{}_kp_xq_{x+k} - \sum\limits_{k=0}^{m-1}v^{k+1}{}_kp_xq_{x+k} .\]

	\begin{remark}
		Предположение о выплате страховой суммы в конце года смерти не отражает практику в действительном виде, но имеет то преимущество , что формулы могут быть выражены в цифрах, взятых из таблиц.
	\end{remark}

	\item Вернемся к случаю, когда страховая премия \red{выплачивается в момент смерти.}

	Используем \red{линейную модель} для дробной части времени дожития $ S(x)$:
	\[ {}_uq_x = uq_x, \;\;\; u \in [0,1). \]

	Мы помним: 
	\begin{gather*}
		S(x) \sim R[0,1] \\
		S(x) \text{ и } K(x) - \text{ независимы}.
	\end{gather*}

	Тогда
	\[ Ev^T = \overline{A}_x = \int\limits^{\infty}_{0}v^tf(t)dt.\]

	В нашем случае

	\[ Ev^T = \overline{A}_x = \int\limits^{\infty}_{0}v^tdt \]

	(сумма работает $ K+S $ , а не $ K+1$).

	\begin{gather*}
		T = K + S = (K + 1) - (1 - S).\\
		\overline{A}_x = Ev^T = E[v^{(K+1) - (1-S)}] = E[v^{K+1}v^{-(1-S)}]=\\
		=E(v^{K+1})E(v^{-(1-S)}) = Ev^{K+1}E[(1+i)^{1-S}]\\
		\text{Но } E((1+i)^{1-s}) = \int\limits^{1}_{0}(1+i)^{1-s}ds = \frac{i}{\ln(1+i)}\\
		\Rightarrow\;\;\;\overline{A}_x = E(v^{K+1})E((1+i)^{1-s}) = \frac{i}{ln(1+i)}A_x
 	\end{gather*}
	\begin{notation}
		$\delta = ln(1+i)$
	\end{notation}

	Подобным образом (для непрерывного случая) можно получить форму для endowments:
	\begin{gather*}
		\overline{A}_{x:n\urcorner} = \overline{A}_{x:n\urcorner}^{1} + A_{x:n\urcorner}^{\;\;\;\;1} = \frac{i}{ln(1+i)}A_{x:n\urcorner}^{1}+ A_{x:n\urcorner}^{\;\;\;\;1} = \\
		= \overline{A}_{x:n\urcorner} + (\frac{i}{ln(1+i)} - 1)A_{x:n\urcorner}^{1}
	\end{gather*}

	\item \textbf{Общие типы страхования жизни.}
		\begin{enumerate}
			\item Рассмотрим страхование жизни \red{со страховой суммой изменяющейся от года к году и выплачиваемой в конце года смерти}.

			Пусть $ C_k$-  \red{ страховая сумма, выплачиваемая в конце к-ого года жизни(к+1-ого года после заключения контракта)}

			\begin{gather*}
				Z = C_{K+1}v^{K+1}\\
				\Rightarrow \;\;\; EZ = \sum\limits_{k=0}^{\infty}C_{k+1}v^{k+1}{}_kp_xq_{x+k}
			\end{gather*}


			\begin{gather*}
				 EZ = C_1(A_x - {}_{1|}A_x) + C_2({}_{1|}A_x - {}_{2|}A_x)+...\\
				 \text{или  } EZ = C_1A_x + (C_2 -C_1){}_{1|}A_x + (C_3 -C_2){}_{2|}A_x + ...+(C_{n+1}-C_n){}_{n|}A_x,
			\end{gather*}
			где 
			
				$ C_1A_x - \text{ мат.ожидание современной стоимости } C_1$;

				$C_1A_x - C_1{}_{1|}A_x$ -  мат.ожидание современной стоимости страховой суммы $ C_1$, выплачиваемой в конце первого года страхования (в случае смерти в этом году);

				$C_2{}_{1|}A_x - C_2{}_{2|}A_x$ -  мат.ожидание современной стоимости страховой суммы $ C_2$, выплачиваемой в конце второго года страхования (в случае смерти в этом году) и т.д.
			

			\begin{remark}
				В пункте 7.а) имеем дело с комбинацией отсроченных на разное время страхований с фиксированной для каждого года суммой выплаты.
			\end{remark}

			\item В случае, когда \red{выплата производится только первые n лет}, т.е. когда $ C_{n+1} = C{n+2} = ... =0$ , страхование можно представить в виде комбинации страхований на конечный промежуток времени, начавшихся немедленно:
			\begin{gather*}
				EZ = C_n(A_{x:n\urcorner}^{1} - A_{x:n-1\urcorner}^{1}) + \\
				+C_{n-1}(A_{x:n-1\urcorner}^{1}- A_{x:n-2\urcorner}^{1})+...+c_1 A_{x:1\urcorner}^{1}	
			\end{gather*}
			или
			\begin{gather*}
				EZ = C_n A_{x:n\urcorner}^{1} + (C_{n-1}-C_n)A_{x:n-1\urcorner}^{1} +\\
				+(C_{n-2}-C_{n-1})A_{x:n-2\urcorner}^{1}+...+(C_1-C_2)A_{x:1\urcorner}^{1}, 
			\end{gather*}
			где $ C_n A_{x:n\urcorner}^{1} + (C_{n-1}-C_n)A_{x:n-1\urcorner}^{1}$ - сумма выплачивается только в случае смерти в течение n-ого года.

			\item Если \red{выплата производится немедленно после смерти клиента}, то страховая сумма должна быть функцией времени, т.е. $ C = C(t)\;(t \geq 0).$

			В этом случае $ Z = C(T)v^T,$ и нетто-ставка будет при страховании на всю жизнь (на случай смерти)
			\[ EZ = \overline{A}_x = \int\limits^{\infty}_{0}C(t)v^tf(t)dt. \]

			Фактически же подсчет нетто-ставки может быть сведен к выплачиваниям для дискретной модели.

			\begin{gather*}
				EZ = \sum\limits_{k=0}^{\infty}E(Z|K = k)P(K=k) = \\
				=\sum\limits_{k=0}^{\infty}E(C(K+S)v^{K+S}| K = k)P(K=k)=\\
				=\sum\limits_{k=0}^{\infty}E(C(K+S)(1+i)^{1-S}| K=k)v^{k+1}P(K=k).
			\end{gather*}
			Получаем
			\[
				EZ = \sum\limits_{k=0}^{\infty}C_{k+1}v^{k+1}{}_kp_xq_{x+k},
			\]
			где $ C_{k+1} = E(C(K+S)(1+i)^{1-S}|K=k).$

			Для оценки $ C_{k+1}$ мы нуждаемся в условном распределении $ S$ при условии $ K=k.$ Для этого воспользуемся линейной моделью $ S$, для которой
			\begin{gather*}
				P(S(x) \leq u|K=k) =u\\
				C_{k+1} = \int\limits^{1}_{0}C(k+u)(1+i)^{1-u}du, 
			\end{gather*}
			так как для этой модели \[	P(u < S \leq u +du) = f(u)du = du. \]

			\begin{example}
				Рассмотрим случай показательного возрастания страховой суммы
				\[c(t)=e^{\tau t}.\]

				При этом 
				\begin{gather*}
					C_{k+1} = \int\limits^{1}_{0}e^{\tau (k+u)}(1+i)^{1-u}du=\\
					=e^{\tau k}(1+i)\int\limits^{1}_{0}\frac{e^{\tau u}}{(1+i)^u}du=\\
					=e^{\tau k}(1+i)\frac{(\frac{e^\tau}{1+i})^u}{ln(\frac{e^\tau}{1+i})} |^1_0 = ..=e^{\tau k}\frac{e^\delta - e^\tau}{\delta - \tau}(\; \delta = ln(1+i)),
				\end{gather*}
				а значит \[ EZ = \sum\limits_{k=0}^{\infty}e^{\tau k}\frac{e^\delta - e^\tau}{\delta - \tau}v^{k+1}{}_kp_xq_{x+k} = \overline{A}_x \]
			\end{example}
		\end{enumerate}
\end{enumerate}

\section{Стандартные типы переменного страхования жизни} % (fold)
Рассмотрим \red{стандартно возрастающее}
\begin{itemize}
	\item \red{страхование на всю жизнь}

	Настоящее значение страховой суммы
	\[Z = (K+1)v^{K+1}.\]

	В этом случае нетто-ставку обозначают
	\[(IA)_x\]
	и 
	\[ (IA)_x = \sum\limits_{jk=0}^{\infty}(k+1)v^{k+1}{}_kp_xq_{x+k}.\]

	Соотвественно для

	\item \red{n-year term insurance } мы имеем
	\[
		Z = \begin{cases}
			(k+1)v^{k+1}, \;\;\; k=0, 1, ... , n-1\\
			0, \;\;\; k = n, n+1, ...
		\end{cases}
	\]

	Нетто-ставка обозначается 
	\[ (IA)_{x:n\urcorner}^{1}, \]
	и , используя полученные в предыдущей секции результаты, имеем
	\begin{gather*}
		(IA)_{x:n\urcorner}^{1} = A_x + {}_{1|}A_x + ...+{}_{n-1|}A_x - n{}_{n|}A_x,\\
		\text{ибо } C_1 = 1, C_2 - C_1 = 2-1, ..., C_n - C_{n-1} = n-n + 1 =1\\
		0 = C_{n+1 }= C_{n+2}=...\; \Rightarrow \;\;\; C_{n+1}-C_n = -n.  
	\end{gather*}

	или
	\[ (IA)_{x:n\urcorner}^{1} = n A_{x:n\urcorner}^{1}- A_{x:n-1\urcorner}^{1} - A_{x:n-2\urcorner}^{1} -..-A_{x:1\urcorner}^{1} \]

	\item \red{стандартно убывающее n-year term insurance}( убывающее от n до 0).

	Этот тип страхования удобно использовать при гарантийной оплате заёма, при условии, что долго возвращается частями каждый год, т.е. и страховая сумма уменьшается каждый год.

	Имеем 
	\[
		Z=\begin{cases}
			(n-k)v^{k+1}, \;\;\; k=0,1,...,n-1\\
			0, \;\;\; k = n, n+1, ...
		\end{cases}
	\]
	Тогда нетто-ставку обозначим так (и посчитаем)
	\begin{gather*}
		(DA)_{x:n\urcorner}^{1} = \sum\limits_{k=0}^{n-1}(n-k)v^{k+1}{}_kp_xq_{x+k}\\
		(DA)_{x:n\urcorner}^{1} = A_{x:n\urcorner}^{1} + A_{x:n-1\urcorner}^{1}+ A_{x:n-2\urcorner}^{1}+...+A_{x:1\urcorner}^{1}	
	\end{gather*}

	\item Будем снова считать, что \red{страховая сумма выплачивается сразу после смерти клиента}, т.е. 
	\[ Z = C(T)v^T, \]
	и использовать линейную модель для $ S.$

	Если страховая сумма возрастает ежегодно, то 
	\begin{gather*}
		C(K) = K+1\\
		Z = (K+1)v^{T} = (K+1)v^{(K+1) - (1-S)}=\\
		=(K+1)v^{K+1}(1+i)^{1-S} 
	\end{gather*}
	(если смерть наступит в течение K - ого года , то сумма будет работать не K+1 год, а K+S).

	Для этого случая (используя независимость $ K \text{ и } S$), нетто-ставка получается равной (используем соотношение $ \overline{A}_x = \frac{i}{\delta}A_x $) 
	\[(I\overline{A})_x  = \frac{i}{ln(1+i)}(IA)_x = \frac{i}{\delta}(IA)_x\]

	\item  Теперь рассмотрим случай, когда \red{страховая сумма выплачивается в конце m-ой части года}, в которой наступила смерть.

	\[ T= K + S^{(j)} \;\; \Rightarrow \;\; Z = v^{K+S^{(j)}},\]

	где $ S^{(j)}$ получается из $ S$ округлением до следующего более высокого значения , кратного $ \frac{1}{m}$. Таким образом, всем возможным значениям моментов смерти на полуоткрытом интервале $(0,\frac{1}{m}]$ ставится в соотвествие одно значение $S^{(1)} = \frac{1}{m}$.

	Для $S \in (\frac{1}{m},\frac{2}{m}]$ соответствует $S^{(1)} = \frac{2}{m}$

	...

	Для $S \in (\frac{j-1}{m},\frac{j}{m}]$ соответствует $S^{(j)} = \frac{j}{m}$

	...

	Для $S \in (\frac{m-1}{m},1]$ - $S^{(m)} = 1.$

	Из независимости $K \text{ и } S$ следует независимость между $K \text{ и } S^{(j)}$. К тому же, если $S$ имеет равномерное распределение между 0 и 1, то $S^{(j)}$ имеет дискретное равномерное распредление.

	\begin{gather*}
		s - \text{ момент смерти на } (0,1]\\
		s^{(j)} - \text{ момент выплаты}.
	\end{gather*}

	При этом $i^{(m)}$ - \red{ номинальная годовая процентная ставка}.

	Сумма, равная 1 в момент $t = 0$ к концу первого интервала $(0,\frac{1}{m}]$ будет равна 
	\[ 1 + \frac{i^{(m)}}{m} \]
	и т.д. к концу года
	\[ (1 + \frac{i^{(m)}}{m})^m, \]
	в этот момент она должна быть равна $1+i$, где $i$ - годовая процентная ставка:

	\begin{gather*}
		(1 + \frac{i^{(m)}}{m})^m = 1+i\\
		i^{(m)} = m[(1+i)^{\frac{1}{m}} - 1] = \frac{(1+i)^{\frac{1}{m}}- (1+i)^0}{\frac{1}{m}}\\
		\Rightarrow\;\; \lim\limits_{m\rightarrow\infty}i^{(m)}=\lim\limits_{m\rightarrow\infty}\frac{(1+i)^{\frac{1}{m}}ln(1+i)(-\frac{1}{m^2})}{\frac{1}{m^2}} = ln(1+i) = \delta
		\end{gather*}
	

	т.е. \textbf{$\delta$ можно рассматривать как предел отношения прироста единичного капитала за время $h=\frac{1}{m}$ к $h$}.

	Если говорить об изменениях суммы в течение года, то в прямом направлении - это процентная ставка $i$, в обратном - коэффициент дисконта $d$
	\[\frac{1}{1-d} = 1+i \;\;\;\; d = \frac{i}{1+i}.\]

	Если же говорить об измененниях $m$ раз в течение года, то в \red{прямом направлении} это $i^{(m)}$ - \red{ номинальная годовая процентная ставка} или $\frac{i^{(m)}}{m}$ - процентная ставка за интервал $\frac{1}{m}$.

	При рассмотрении в \red{ обратном направлении } это $d^{(m)}$ - \red{ номинальный годовой дисконт } за интервал $\frac{1}{m}$.

	Можно записывать соотношения между $d$ и $i$ или между $\frac{d^{(m)}}{m}$ и $\frac{i^{(m)}}{m}$, но не между $d^{(m)}$ и $i^{(m)}$, т.к. это только формальные величины.

	По аналогии с формулой $d = \frac{i}{1+i}$ имеем 
	\[  \frac{d^{(m)}}{m} = \frac{\frac{i^{(m)}}{m}}{1+\frac{i^{(m)}}{m}}\;\Rightarrow\;d^{(m)} = \frac{i^{(m)}}{1+\frac{i^{(m)}}{m}}\]

	и видим, что 
	\[ \lim\limits_{m\rightarrow\infty}d^{(m)} = \lim\limits_{m\rightarrow\infty}i^{(m)} = \delta \]

	Вернемся к началу рассматриваемого нами пункта.

	Для расчета нетто-премии снова используем линейную модель для $ S$ и страховую сумму, равную 1.

	\begin{gather*}
		K + S^{(j)} = (K+1) -(1 - S^{(j)})\\
		E[(1+i)^{1-S^{(j)}}] = \frac{1}{m}[(1+i)^{1-\frac{1}{m}}+...+(1+i)^{1- \frac{m}{m}	}]=\\
		=[\frac{1}{m} - \text{ вероятность попадания в интервал длины } \frac{1}{m}]=\\
		=\frac{1+i}{m}\frac{(1+i)^{-\frac{1}{m}}(1-(1+i)^{-\frac{m}{m}})}{1-(1+i)^{-\frac{1}{m}}} = \frac{(1+i)(1+i)^{-\frac{1}{m}}(1- \frac{1}{1+i})}{m \frac{(1+i)^{\frac{1}{m}}-1}{(1+i)^{\frac{1}{m}}}} = \frac{i}{i^{(m)}}
	\end{gather*}

	Тогда 
	\begin{gather*}
		A_x^{(m)} = E(v^{K+1})E[(1+i)^{1-s^{(j)}}] = \frac{i}{i^{(m)}}A_x\\
		E(v^{K+1}) = A_x\\
		A_x^{(m)} = \frac{i}{i^{(m)}}A_x \xrightarrow[m\rightarrow\infty]{}\frac{i}{\delta}A_x=\\
		=\frac{i}{ln(1+i)}A_x = \overline{A}_x\\
		A_x^{(m)} \;\xrightarrow[m\rightarrow\infty]{}\overline{A}_x,
	\end{gather*}

	т.е. $ A_x^{(m)}$ стремится к нетто-ставке, выплачиваемой сразу после смерти клиента.
\end{itemize}

% section стандартные_типы_переменного_страхования_жизни (end)

% section модели_долгосрочного_страхования_жизни (end)
% chapter лекция_3_25_02_2021_модели_страхований_жизни (end)