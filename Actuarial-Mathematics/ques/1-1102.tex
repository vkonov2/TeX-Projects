\chapter{Лекция №1. Страхование жизни}\label{lec:1}


\begin{remark}
	Предполагается спокойное, мирное время. Страхование жизни основано на закономерностях уровня смертности различных групп населения.
\end{remark}

\section{Вероятность выживания} % (fold)
\label{lec:1sec:alive_prob}

Рассмотрим однородную группу людей. Время измеряем в годах.

\begin{definition}
	Введем следующие понятия:
	\begin{itemize}
		\item[$\bullet$] 
			$x$ - возраст клиента на момент заключения договора;
		\item[$\bullet$] 
			$T_x$ - остаточное время жизни, начиная с момента заключения контракта - случайная величина;
		\item[$\bullet$] 
			$F(t) = P(T_x \leq t) = {}_{t}q_x$, где $F(t)$ — функция распределения, обладающая плотностью распределения, а ${}_{t}q_x$ — вероятность того, что с момента $x$ человек проживет время, не большее $t$;
		\item[$\bullet$] 
			$f(t)$ - плотность распределения $T_x$;
		\item[$\bullet$] 
			${}_{t}p_x = P(T_x > t)$ - вероятность того, что с момента $x$ человек проживет время, большее $t$; ${}_{t}p_x = 1 - {}_{t}q_x$;
		\item[$\bullet$] 
			$\omega - x$ — максимальное оставшееся время жизни заключившего договор, где $\omega$ — максимальное время жизни. Его обычно берут равным 100.
	\end{itemize}
\end{definition}

\begin{definition}
	При заключении договора фиксируются:
	\begin{enumerate}
		\item продолжительность заболеваний страхователя;
		\item возраст страхователя;
	\end{enumerate}

	Тогда $\lambda$ - время, прошедшее между медицинским обследованием (началом заболевания) и моментом заключения страхового договора.
\end{definition}
	


\begin{clair}
	\blue{ПРОВЕРИТЬ}
	$$\begin{gathered}
		{}_{\omega - x}p_x = 0 \\
		{}_{0}p_x = 1
	\end{gathered}$$
\end{clair}

\begin{clair}
	\blue{ПРОВЕРИТЬ}
	Пусть 
	$${}_{\frac{t}{t'}}q_x \MYdef P(t < T_x \leq t + t')$$
	Тогда 
	$${}_{\frac{t}{t'}}q_x = {}_{t}p_x - {}_{t+t'}p_x > 0$$
\end{clair}
\begin{Proof}
	$$P(T_x > t) = P(t \leq T_x \leq t+t') + P(T_x > t+t') \; \Rightarrow \; {}_{t}p_x = {}_{\frac{t}{t'}}q_x + {}_{t+t'}p_x. $$
\end{Proof}

\begin{conseq}
	Из доказательства очевидно вытекает, что
	$${}_{\frac{t}{t'}}q_x \;\rightarrow \; 0 \text{ при } t'\;\rightarrow\;0$$
\end{conseq}

\begin{definition}
	Рассмотрим $$P(t \leq T_x \leq t + \Delta t| T_x > t) = \frac{P(t < T_x \leq t + \Delta t)}{P(T_x > t)} $$

	Тогда $$\lim\limits_{\Delta t \rightarrow 0}\frac{P(t \leq T_x \leq t + \Delta t| T_x > t)}{\Delta t} = \lim\limits_{\Delta t \rightarrow 0}\frac{F(t + \Delta t) - F(t)}{\Delta t \cdot \pre{t}p_x} = \frac{f(t)}{1 - F(t)} = \mu_x(t)$$ называется \red{мгновенной смертностью в момент $x + t$} или \red{интенсивностью смерти в возрасте $x+t$}.
\end{definition}

\begin{remark}
	$$\begin{gathered}
		\mu_x (t) = \frac{F'(t)}{1-F(t)} = \frac{-\frac{\partial}{\partial t}(1-F(t))}{1-F(t)}
		= \frac{-\frac{\partial}{\partial t}(\pre{t}p_x)}{\pre{t}p_x} = -\frac{d}{dt}\ln\pre{t}p_x \Rightarrow \pre{t} p_x = e^{-\int\limits_0^t\mu_x(u)du}
	\end{gathered}$$
\end{remark}


\begin{properties}
	Мгновенная смертность обладает следующими свойствами:
	\begin{enumerate}
		\item $\mu_x(t) \geq 0 $ при $t \in [0, \omega - x]$
		\item $\int\limits_0^b\mu_x(t)dt = \infty$, где $b > \omega - x$
	\end{enumerate}
\end{properties}

\section{Модели смертности} % (fold)
\label{lec:1sec:death_mod}

\begin{description}
	\item[\blue{Модель Муавра}] \begin{gather*}
		T_x \text{ равномерно распределено на } [0, \omega - x]\\
		f(t) = \begin{cases}
			\frac{1}{\omega - x} \;\;\; 0 \leq t \leq \omega-x\\
			0 \;\;\;\;\;\;\; t < 0, t > \omega - x
		\end{cases}
		\end{gather*}
	\item[\blue{Модель Гомпертца}]
		$$\mu_x(t) = A\cdot C^{x+t}, (A>0, C>1)$$
		$$\pre{t}p_x \rightarrow 0 \text{ при } t \rightarrow \infty$$
		Эта модель предполагает, что человек может жить бесконечно долго.

	\item[\blue{Модель Мэйкхэма}]
		$$\mu_x(t) =D+ AC^{x+t}, (A>0, C>1, D>0)$$
		Тут берётся во внимание тот факт, что смерть может наступать не только от старости.
		$${}_tp_x = e^{-\int\limits_0^t (D+AC^{x+u})du} = e^{-Dt - \frac{A(C^{x+t} - C^x)}{\ln C}}$$
\end{description}

\section{Оценка числа людей, доживших до момента t}
\label{lec:1sec:eval}

Пусть $L_x$ — \red{число людей возраста $х$ из однородной группы в момент} $t = 0$. Определим индикатор:
$$X_i(t) = \begin{cases}
	1 \;\;\;\; i-\text{ый человек проживет не меньше, чем } t\\
	0 \;\;\;\; i-\text{ый человек проживет меньше, чем } t
	\end{cases}$$

Тогда $$\begin{gathered}
	EX_i(t) = {}_tp_x \\
	\mathbb{D}X_i(t) = {}_tp_x - ({}_tp_x)^2 = {}_tp_x \cdot {}_tq_x
\end{gathered}$$

Положим $$L_{x+t} = X_1(t) +...+ X_{L_x}(t)$$

Тогда $$\begin{gathered}
	EL_{x+t} = L_x \cdot {}_tp_x = l_{x+t}\;\;(\Rightarrow EL_x = L_x = l_x)\\
	\mathbb{D}L_{x+t} = L_x \cdot {}_tp_x \cdot {}_tq_x\;\text{(в силу независимости смертей)}.
\end{gathered}$$

\begin{remark}
	Видно, что $${}_tp_x = \frac{l_{x+t}}{l_x}$$
\end{remark}