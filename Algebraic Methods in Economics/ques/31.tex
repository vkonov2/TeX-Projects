\chapter{Вычисление $\lim_{m \to \infty}[{\rho(A)}^{−1}A]^m$ для положительной матрицы $A$}
\label{cha:31}

\epigraph{
	\textit{Совершенства требует только чистая математика; даже прикладная математика довольствуется приблизительными вычислениями.}}
{-- Чернышевский Н.Г.}

\begin{theorem}[]\label{cha:30/the:1}
	Пусть задана квадратная матрица $A \ge 0$ без нулевых строк, причем существуют такие положительные векторы x, y, что:
	$$A x = \rho(A) x, \; y A = \rho(A) y, \; (x,y) = \underset{j}{\overset{}{\sum}}x_j y_j = 1\eqno(91)$$
	где x, y отождествляются со столбцами из координат векторов. Положим $L = (x_iy_j) = x \cdot y$ Тогда:
	$$Lx=x, \; yL=y, \; L^2 =L, \; AL=LA=\rho(A)L$$
	Кроме того, $(\rho(A)^{−1}A − L)^m = (\rho(A)^{−1}A)^m − L$.
\end{theorem}
\begin{Proof}
	Заметим, что $Lx=(x,y)x=x, \; yL=y(x,y)=y$. Отсюда:
	$$L^2 =x(y,x)y=L, \; AL=(Ax)y=\rho(A)xy=\rho(A)L=LA \eqno(92)$$
	Поэтому $\rho(A)^{−1}AL = L = L(\rho(A)^{−1}A)$ и
	$$\begin{gathered}
		\left[ \rho(A)^{-1}A - L \right]^m = \underset{j=1}{\overset{m}{\sum}}\begin{pmatrix}
			m \\ j
		\end{pmatrix} (-1)^jL^j + \rho(A)^{-m}A^m = \\
		= \left[ \underset{j=1}{\overset{m}{\sum}}\begin{pmatrix}
			m \\ j
		\end{pmatrix} (-1)^j \right] L + \rho(A)^{-m}A^m = -L + \rho(A)^{-m}A^m
	\end{gathered}$$
	Отсюда $\displaystyle \rho(A)^{−m}A^m = L + \left[\rho(A)^{−1}A − L\right]^m$.
\end{Proof}

\begin{theorem}[]\label{cha:31/the:1}
	Пусть $A, x, y, L$ из теоремы \ref{cha:30/the:1}, причем $A > 0$. Тогда: 
	$$\underset{m \to \infty}{\lim}[{\rho(A)}^{−1}A]^m = L$$
\end{theorem}
\begin{Proof}
	В силу теоремы \ref{cha:30/the:1} $\displaystyle L + \left[\rho(A)^{−1}A − L\right]^m = \rho(A)^{−m}A^m$. Поэтому в силу теоремы \ref{cha:27/the:2} достаточно показать, что $\displaystyle \rho \left[ \rho(A)^{−1}A − L\right] < 1$. Пусть $[\rho(A)^{−1}A−L]w = \mu w$, $w \not = 0$. По (92) $\displaystyle \mu L w = L [\rho(A)^{−1}A − L] w = [L − L^2]w = 0$.

	Если $\mu \not = 0$, то $Lw = 0$, и поэтому $Aw = \rho(A)\mu w$, откуда $\rho(A)|\mu| \le \rho(A)$, т.е. $|\mu| \le 1$. Кроме того, если $|\mu| = 1$, то $\mu = 1$ по теореме \ref{cha:30/the:2}, и по теореме \ref{cha:30/the:3} получаем $w = x$. В этом случае $0 = Lx = x^tyx = x \not = 0$ по теореме \ref{cha:30/the:1}. Полученное противоречие показывает, что $|\mu| < 1$.
\end{Proof}


























