\chapter{Решение систем уравнений для потенциалов для допустимого плана в
невырожденной задаче без циклов}
\label{cha:22}

\epigraph{
	\textit{Обнимая собой сполна весь цикл человеческих отношений, они оживляют мысль и деятельность не только отдельных индивидуумов, но и целого общества.}}
{-- Салтыков-Щедрин М.Е.}

Сначала см. билет \ref{cha:21}.\\

ШАГ 2. Построение \textit{первоначальной системы потенциалов}. Для каждой из клеток $(i,j)$, в которых находится ненулевой элемент из $X^0$, рассмотрим уравнение
$$u_i + v_j = c_{ij} \eqno(65)$$
с неизвестными $v_i, u_i$. Зафиксируем одну переменную $u_{i_0}$ , например, $u_1$ и придадим ей произвольное значение, например $u_1 = 0$. В силу (65), предложений \ref{cha:21/propose:1} и \ref{cha:21/propose:2} мы однозначно найдем значения всех остальных переменных $u_i, v_j$. Итак, решая систему (65) находим первоначальную систему потенциалов.

ШАГ 3.

\textit{Проверяем, удовлетворяет ли построенный план и система потенциалов условиям теоремы \ref{cha:20/the:1}}.

Заметим, что условие 2) из теоремы \ref{cha:20/the:1} выполнено по построению. Остается лишь проверить условие 1). Если оно выполнено, то полученный план оптимален. В противном случае переходим к следующему шагу.


% \begin{conseq}[]\label{cha:22/conseq:1}
% 	Если в условии невырожденной транспортной задачи числа $a_1, \dots, a_m, b_1, \dots, b_n$ являются целыми, то задача имеет целочисленное решение.
% \end{conseq}

% Предложенный алгоритм может быть использован при решении ряда близких задач.
% \begin{enumerate}
% 	\item[1)] 
% 		Количество производимого в (57) продукта больше количества продукта, потребляемого в (58). Требуется перевести с минимальными затратами из (57) производимый продукт, чтобы полностью удовлетворить потребности в (58).

% 		Решение задачи сводится к общей транспортной задаче введением пункта $B_{n+1}$ с потреблением $b_{n+1} = \underset{i}{\overset{}{\sum}}a_i - \underset{j}{\overset{}{\sum}} b_j$ , причем $c_{i,n+1} = 0$ для всех i.
% 	\item[2)]
% 		Количество потребляемого в (58) продукта больше количества продукта, производимого в (57). Требуется перевести с минимальными затратами из (57) производимый продукт, чтобы полностью удовлетворить потребности в (58).

% 		Решение задачи сводится к общей транспортной задаче введением пункта $A_{m+1}$ с производством $a_{m+1} = \underset{j}{\overset{}{\sum}} b_j - \underset{i}{\overset{}{\sum}} b_i$, причем $c_{m+1,j} = 0$ для всех j.
% 	\item[3)]
% 		 Если же имеется запрет на перевозки из $A_i$ в $B_j$, то полагаем $c_{ij} = \infty$.
% 	\item[4)]
% 		Если от $A_i$ в $B_j$ имеются фиксированные поставки в количестве $d_{ij}$, то заменяем $a_i$, $b_j$ и $c_{ij}$ на $a_i - d_{ij}$, $b_j - d_{ij}$ и $\infty$. Решая получаемую транспортную задачу и находя оптимальный план $X = (x_{rs})$, мы затем общие затраты $Z(X)$ увеличиваем на $c_{ij}d_{ij}$.
% 	\item[5)]
% 		Если между $A_i$ и $B_j$ имеются минимальные поставки в количестве $d_{ij}$, то заменяем $a_i$, $b_j$ на $a_i - d_{ij}$, $b_j - d_{ij}$. Решая получаемую транспортную задачу и находя оптимальный план $X = (x_{rs})$, мы затем заменяем $x_{ij}$ на $d_{ij}$.
% 	\item[6)]
% 		Если от $A_i$ к $B_j$ поставки не должны превышать объем $d_{ij}$, то мы заменим $B_j$ на два объекта $B'_j$, $B''_j$, где $b'_j = d_{ij}$, $b''_j = b_j - d_{ij}$. Кроме того, полагаем $c'_{tj} = c_{tj}$, $c''_{tj} = \infty$ для всех $t = \ton m$.
% \end{enumerate}










