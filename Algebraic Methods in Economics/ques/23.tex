\chapter{Улучшение плана. Существование и единственность пути улучшения плана}
\label{cha:23}

\epigraph{
	\textit{Царь Алексей Михайлович много заботился об улучшении внутреннего положения России.}}
{-- Добролюбов Н.А.}

ШАГ 4. \textit{Улучшение плана X}.

Пусть относительно системы потенциалов $u_i, v_j$ план X не оптимален, т.е. не выполнено условие 1) теоремы \ref{cha:20/the:1}. Выберем отклонение
$$\alpha_{i_0j_0} = u_{i_0} + v_{j_0} - c_{i_0j_0} > 0\eqno(66)$$
Тогда $x_{i_0j_0} = 0$ и по следствию \ref{cha:21/conseq:1} существует последовательность ненулевых элементов $x_{i_0 j_1}, x_{i_1 j_1}, \dots, x_{i_k j_k}, x_{i_k j_0}$. Расставим во всех выбранных клетках $(i_0, j_0), (i_0, j_1), \dots, (i_k, j_0)$ последовательно чередующиеся метки метки $+, -, +, \dots, -$. Пусть $\theta$ – минимальный среди элементов $x_{i_0j_1}, x_{i_1j_2}, \dots, x_{i_{k−1}j_k}, x_{i_kj_0}$, в клетках, помеченных знаком $-$. В клетках со знаком $+$ число $x_{ij}$ увеличиваем на $\theta$, а со знаком $-$ уменьшаем на $\theta$, т.е.
$$\begin{gathered}
	x'_{i_0 j_0} = x_{i_0 j_0} + \theta = \theta \\
	x'_{i_0 j_1} = x_{i_0 j_1} - \theta \\
	x'_{i_1 j_1} = x_{i_1 j_1} + \theta \\
	\dots \dots \dots \\
	x'_{i_{k-1}j_{k-1}} = x_{i_{k-1}j_{k-1}} + \theta \\
	x'_{i_{k-1}j_k} = x_{i_{k-1}j_k} - \theta
\end{gathered}$$
Для остальных пар индексов положим $x'_{ij} = x_{ij}$. Получаем новый допустимый план $X' = (x'_{ij})$.

ШАГ 5.

\textit{Проверка для плана $X'$ и потенциалов $u'_i$, $v'_j$ выполнение условия 1) из теоремы \ref{cha:20/the:1}.}

Если оно не выполнено, то переходим к шагу 4. Если оно выполнено, то оптимальный план построен.
