\chapter{Двойственная задача линейного программирования}
\label{cha:16}

\epigraph{
	\textit{Мы, провинциалы, — да впрочем одни ли мы? — имеем о «жизни» представление несколько двойственное.}}
{-- Салтыков-Щедрин М.Е.}

Рассмотрим задачу линейного программирования из определения \ref{cha:14/def:1}. Без ограничения общности, заменяя коэффициенты $p_i, a_{ij}$ на противоположные числа, можно считать, что $z = −p_1x_1 − \dots − p_nx_n$. Положим $p = ^t (p_1, \dots, p_n)$, $x = ^t (x_1, \dots, x_n)$. Тогда $z = - ^t p x$, где $^t p x$ - произведение $^t p$ и x как матриц. Имеем:
$$g_i(x) = \underset{j=1}{\overset{n}{\sum}}(-a_{ij})x_j + b_j, \; i = \ton m$$
Пусть ранг матрицы $(−a_{ij})$ размера $m \times n$ равен r, и, например, линейные (однородные) части $g_1, \dots, g_r$ линейно независимы. Совершая замену переменных, можно считать, что $g_1 = x_1, \dots, g_r = x_r$, причем
$$g_i(x) = \underset{j=1}{\overset{r}{\sum}}(-a_{ij})x_j + b_j, \; i = \ton m$$
Обозначим через b – столбец $^t(b_1, \dots, b_m)$. Из предыдущих рассуждений вытекает, что без ограничения общности можно предполагать, что $r = n$ и неравенства имеют вид:
$$x \ge 0, \; A x \le b\eqno(40)$$
где $A = (a_{ij}) \in Mat(m \times n, \mathbb{R})$. При этом необходимо найти $\min z$, т.е. найти:
$$\max (^t p x)\eqno(41)$$

\begin{definition}\label{cha:16/def:1}
	Двойственной задачей к задаче (40), (41) называется задача нахождения $\min ^t b_y$, где $y = ^t(y_1, \dots, y_m)$, при условии
	$$^t A y \ge p, \; y \ge 0 \eqno(42)$$
\end{definition}

\begin{propose}\label{cha:16/propose:1}
	Двойственная задача к двойственной задаче совпадает с исходной задачей.
\end{propose}
\begin{Proof}
	В двойственной задаче имеем $\displaystyle (- ^t A)y \le -p, \; y \ge 0$, причем необходимо найти $\max (- ^t by)$. Поэтому задача, двойственная к двойственной, имеет вид $(-A)x \ge -b$, $x \ge 0$, причем необходимо найти $\min (- ^t px)$, т.е. $\max ( ^t px)$.
	
\end{Proof}