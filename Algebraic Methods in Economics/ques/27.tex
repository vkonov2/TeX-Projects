\chapter{Связь нормы матрицы с ее спектральным радиусом}
\label{cha:27}

\epigraph{
	\textit{У меня служба — у нее связи и маленькие средства.}}
{-- Толстой Л.Н.}

\begin{definition}\label{cha:27/def:1}
	\textit{Спектральным радиусом} $\rho(A)$ оператора (матрицы) $A \in \mathcal{L}(V)$ называется максимум модулей собственных значений A.
\end{definition}

\begin{theorem}[]\label{cha:27/the:1}
	Пусть $||\cdot||$ – норма в алгебре линейных операторов $\mathcal{L}(V)$ в конечномерном пространстве V. Если $A \in \mathcal{L}(V)$, то $\rho(A) \le ||A||$.
\end{theorem}
\begin{Proof}
	Пусть $Ax = \lambda x$ для некоторого ненулевого собственного вектора x. Построим матрицу X, столбцами которой будут координаты вектора x. Тогда $AX = \lambda X$, откуда:
	$$||AX|| = |\lambda| ||X|| \le ||A||||X||\eqno(89)$$
	Так как $X \not = 0$, то $||X|| \not = 0$ и поэтому в (89) получаем $|\lambda| \le ||A||$. Отсюда вытекает утверждение, поскольку $\lambda$ – любое собственное значение.
\end{Proof}

\begin{theorem}[]\label{cha:27/the:2}
	Пусть $\rho(A) < 1$. Тогда $A^k \to 0$ при $k \to \infty$.
\end{theorem}
\begin{Proof}
	\textit{Первое доказательство}.

	В силу следствия \ref{cha:26/conseq:1} достаточно доказать сходимость относительно матричной нормы $||\cdot||_E$. Пусть n – размерность пространства, в котором действует оператор A. Без ограничения общности можно предполагать, что $\mathbb{F} = \mathbb{C}$. Случай $n = 1$ очевиден. Пусть для $n − 1$ теорема доказана. В силу теоремы о приведении к жордановой форме существует такой базис, в котором матрица оператора имеет верхнетреугольный вид. Переходя к этому базису, будем считать, что матрица A имеет вид:
	$$A = \begin{pmatrix}[c | c]
		B &  u \\ \hline
		0 &  \lambda
	\end{pmatrix}$$
	Тогда $\rho(B) \le \rho(A) < 1$ и по индукции $B^k \to 0$ при $k \to \infty$.

	\begin{lemma}\label{cha:27/lemma:1}
		$$A^k = \begin{pmatrix}[c | c]
		B^k &  D_k u \\ \hline
		0 &  \lambda^k
	\end{pmatrix}, \; D_k = \underset{j=0}{\overset{k-1}{\sum}}\lambda^j B^{k-1-j}$$
	\end{lemma}
	\begin{Proof}
		Непосредственная проверка, основанная на определении произведения матриц.
	\end{Proof}

	Завершим доказательство теоремы. Так как $B^k \to 0$ при $k \to \infty$, то для любого $1 > \varepsilon > 0$ существует такое натуральное число N, что для всех $k \ge N$ имеем $||B^k|| < \varepsilon$. В частности, последовательность $||B^j||$ ограничена константой C. Кроме того, $|\lambda| < 1$. Таким образом, если $m > 2Nt$, то:
	$$\begin{gathered}
		||D_m|| \le \Big|\Big| \underset{j=0}{\overset{Nt-1}{\sum}}\lambda^j B^{m-1-j}\Big|\Big| + \Big|\Big| \underset{j=Nt}{\overset{m-1}{\sum}}\lambda^j B^{m-1-j}\Big|\Big| \le \\
		\le ||B^{m-1-Nt}||\Big|\Big|\underset{j=0}{\overset{Nt-1}{\sum}}\lambda^j B^{Nt-j}\Big|\Big| + |\lambda^{Nt}|\Big|\Big|\underset{j=Nt}{\overset{m-1}{\sum}}\lambda^{j-Nt}B^{m-1-j}\Big|\Big| \le \\
		\le ||B^N||^t||B^{m-1-2Nt}||\Big|\Big|\underset{j=0}{\overset{Nt-1}{\sum}}\lambda^j B^{Nt-j}\Big|\Big| + |\lambda|^{Nt}\Big|\Big|\underset{j=Nt}{\overset{m-1}{\sum}}\lambda^{j-Nt}B^{m-1-j}\Big|\Big| \le \\
		\le C ||B^N||^t \left( \underset{j=0}{\overset{Nt-1}{\sum}}|\lambda|^j ||B^{Nt-j}|| \right) + |\lambda|^{Nt} \left( \underset{j=Nt}{\overset{m-1}{\sum}}|\lambda^{j-Nt}|||B^{m-1-j}|| \right) \le \\
		\le C^2 ||B^N||^t \left( \underset{j=0}{\overset{Nt-1}{\sum}}|\lambda|^j \right) + C |\lambda|^{Nt} \left( \underset{j=Nt}{\overset{m-1}{\sum}}|\lambda^{j-Nt}| \right) = \\
		= C^2 ||B^N||^t \frac{|\lambda|^{Nt}-1}{|\lambda|-1} + C |\lambda|^{Nt} \frac{|\lambda|^{m-Nt}-1}{|\lambda|-1}
	\end{gathered}$$
	Таким образом, если m (и t) стремятся к $\infty$, то $||D_m|| \to 0$. Отсюда по индукции вытекает утверждение теоремы.
\end{Proof}
\begin{Proof}
	\textit{Второе доказательство}.

	Без ограничения общности можно предполагать, что $\mathbb{F} = \mathbb{C}$. В силу теоремы о приведении к жордановой форме существует такая невырожденная матрица S, что $A = S^{−1}JS$, где J – жорданова форма. При этом $A^k = S^{−1}J^kS$ для всех k. Поэтому достаточно показать, что все элементы матрицы $J^k$ стремятся к нулю при $k \to \infty$. Это утверждение достаточно доказать для одной жордановой клетки. Пусть
	$$J = \begin{pmatrix}
		\lambda & 1 & 0 & \dots & 0 & 0 & 0 \\
		0 & \lambda & 1 & \dots & 0 & 0 & 0 \\
		\vdots & \ddots & \ddots & \ddots & \ddots & \ddots & \vdots \\
		\vdots & \ddots & \ddots & \ddots & \ddots & \ddots & \vdots \\
		0 & 0 & 0 & \dots & \lambda & 1 & 0 \\
		0 & 0 & 0 & \dots & 0 & \lambda & 1 \\
		0 & 0 & 0 & \dots & 0 & 0 & \lambda
	\end{pmatrix}$$
	имеет размер n. Заметим, что $J = \lambda E + B$, где
	$$B = \begin{pmatrix}
		0 & 1 & 0 & \dots & 0 & 0 & 0 \\
		0 & 0 & 1 & \dots & 0 & 0 & 0 \\
		\vdots & \ddots & \ddots & \ddots & \ddots & \ddots & \vdots \\
		\vdots & \ddots & \ddots & \ddots & \ddots & \ddots & \vdots \\
		0 & 0 & 0 & \dots & 0 & 1 & 0 \\
		0 & 0 & 0 & \dots & 0 & 0 & 1 \\
		0 & 0 & 0 & \dots & 0 & 0 & 0
	\end{pmatrix}$$
	Очевидно, что $B^n = 0$. Для любого $k \ge n$ получаем:
	$$J^k = (\lambda E + B)^k = \underset{m=0}{\overset{n}{\sum}}\begin{pmatrix}
		k \\ m
	\end{pmatrix}\lambda^{k-m} B^m$$
	Коэффициент:
	$$\begin{gathered}
		\begin{pmatrix}
			k \\ m
		\end{pmatrix}\lambda^{k-m} = \frac{k(k-1)\dots(k-m+1)}{m!}\lambda^{k-m} = \\
		= \frac{k^m \lambda^k}{m!} \left[ 1 \left( 1-\frac{1}{k} \right) \dots \left( 1 - \frac{m-1}{k} \right) \lambda^{-m} \right] \xrightarrow[k\to \infty]{} 0
	\end{gathered}$$
\end{Proof}











