\chapter{Доказать, что спектральный радиус является простым корнем характеристического многочлена положительной матрицы}
\label{cha:32}

\epigraph{
	\textit{От ветвей вертикально тянутся растительные нити и, врастая в землю, пускают корни, из которых образуются новые деревья.}}
{-- Гончаров И.А.}

\begin{theorem}[]\label{cha:32/the:1}
	Пусть $A > 0$. Тогда $\rho(A)$ является простым корнем характеристического многочлена матрицы A.
\end{theorem}
\begin{Proof}
	Пусть $\rho = \rho(A)$. Существует такая невырожденная комплексная матрица S, что:
	$$S^{-1}AS = \begin{pmatrix}
		\rho &  &  &  &  &  \\
		 & \ddots &  &  & \bigstar & \\
		 &  & \rho &  &  & \\
		 &  &  & \lambda_t &  & \\
		 & 0 &  &  & \ddots & \\
		 &  &  &  &  & \lambda_k
	\end{pmatrix}, \; |\lambda_j| < \rho$$
	Отсюда:
	$$\rho^{-1}A = S \begin{pmatrix}
		1 &  &  &  &  &  \\
		 & \ddots &  &  & \bigstar & \\
		 &  & 1 &  &  & \\
		 &  &  & \mu_t &  & \\
		 & 0 &  &  & \ddots & \\
		 &  &  &  &  & \mu_k
	\end{pmatrix}S^{-1}, \; \mu_j = \frac{\lambda_j}{\rho}, \; |\mu_j| < 1$$
	Кроме того:
	$$(\rho^{-1}A)^m = S \begin{pmatrix}
		1 &  &  &  &  &  \\
		 & \ddots &  &  & \bigstar & \\
		 &  & 1 &  &  & \\
		 &  &  & \mu_t &  & \\
		 & 0 &  &  & \ddots & \\
		 &  &  &  &  & \mu_k
	\end{pmatrix}^m S^{-1} = $$
	$$= S \begin{pmatrix}
		1 &  &  &  &  &  \\
		 & \ddots &  &  & \bigstar & \\
		 &  & 1 &  &  & \\
		 &  &  & \mu_t^m &  & \\
		 & 0 &  &  & \ddots & \\
		 &  &  &  &  & \mu_k^m
	\end{pmatrix} S^{-1} \to S \begin{pmatrix}
		1 &  &  &  &  &  \\
		 & \ddots &  &  & \bigstar & \\
		 &  & 1 &  &  & \\
		 &  &  & 0 &  & \\
		 & 0 &  &  & \ddots & \\
		 &  &  &  &  & 0
	\end{pmatrix}^m S^{-1}\eqno(94)$$
	Заметим, что ранг $\underset{m}{\lim}(\rho^{−1}A)^m$ равен рангу L, т.е. 1, и не меньше по (94) числу единиц на главной диагонали, т.е. кратности $\rho$. Отсюда вытекает утверждение.
\end{Proof}




















