\chapter{Теорема Фробениуса}
\label{cha:35}

\epigraph{
	\textit{Какая глупая фамилия!.. — рассердившись, произнесла она. — И за чем тебе нужен человек с такой странной фамилией?}}
{-- Гейнце Н.Э.}

\begin{propose}\label{cha:35/propose:1}
	Если $\lambda_1, \dots, \lambda_n$ – собственные значения A, то $1+\lambda_1, \dots, 1+\lambda_n$ – собственные значения $E+A$. В частности, $1+\rho(A) \ge \rho(E+A)$. Если $A\ge0$, то $1+\rho(A) = \rho(E+A)$.
\end{propose}
\begin{Proof}
	Перейдя к жордановой форме A получаем требуемые неравенства в силу теоремы \ref{cha:33/the:1}.
\end{Proof}

\begin{propose}\label{cha:35/propose:2}
	Пусть $A \ge 0$ и $A^k > 0$ для некоторого натурального числа k. Тогда $\rho(A)$ – простой корень характеристического многочлена.
\end{propose}
\begin{Proof}
	Если $\lambda_1, \dots, \lambda_n$ – собственные значения A, то $\lambda_1^k, \dots, \lambda_n^k$ – собственные значения $A^k > 0$. Отсюда $\rho(A^k) = \rho(A)^k$. По теореме \ref{cha:33/the:1}, например, $\lambda_1 = \rho(A)$. Если $\lambda_1 = \lambda_2$, то $\displaystyle \lambda_1^k = \lambda_2^k = \rho(A)^k = \rho(A^k)$, что невозможно по теореме Перрона.
\end{Proof}

\begin{theorem}[\red{Фробениуса}]\label{cha:35/the:1}
	Пусть $A \ge 0$ – неразложимая матрица. Тогда:
	\begin{itemize}
		\item[1)] $\rho(A) > 0$
		\item[2)] $\rho(A)$ – собственное значение A
		\item[3)] существует положительный собственный вектор с собственным значением $\rho(A)$
		\item[4)] $\rho(A)$ – простой корень характеристического многочлена матрицы A
	\end{itemize}
\end{theorem}
\begin{Proof}
	Так как матрица A неразложима, то в A нет нулевых строк и столбцов. Поэтому $\rho(A) > 0$ в силу следствия \ref{cha:29/conseq:4}.

	Утверждение 2) вытекает из теоремы \ref{cha:33/the:1}.

	Рассмотрим утверждение 3). По теореме \ref{cha:33/the:1} существует неотрицательный собственный вектор x для матрицы A с собственным значением $\rho(A)$. Тогда $(E + A)x = (1 + \rho(A))x$ и поэтому $(E+A)^{n−1}x = (1+\rho(A))^{n−1}x$. Но матрица $(E+A)^{n−1}$ положительна и поэтому $(1 + \rho(A))^{n−1}x > 0$, откуда вытекает положительность x.

	Последнее утверждение следует из предложений \ref{cha:35/propose:1}, \ref{cha:35/propose:2}.
\end{Proof}


















