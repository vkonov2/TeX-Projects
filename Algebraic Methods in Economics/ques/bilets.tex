\begin{center}
	{\Large \textbf{Программа экзамена по предмету}}
	
	{\Large \textbf{<<Алгебраические методы в экономике>>}}
\end{center}

\begin{enumerate}
	\item Описание выпуклых многогранников.
	\item Теоремы отделимости для замкнутого выпуклого множества и замкнутого
	выпуклого компакта вне него.
	\item Замкнутость конечно порожденного конуса.
	\item Теорема отделимости для замкнутого выпуклого конуса и замкнутого выпуклого компакта вне конуса.
	\item Теорема Фаркаша и ее следствия.
	\item Теорема фон Неймана.
	\item Решение игры в чистых стратегиях.
	\item Приложение теоремы фон Неймана к теории конечных антагонистических
	игр.
	\item Внутренние точки полиэдра.
	\item Грани полиэдров и экстремумы аффинных функций на полиэдрах.
	\item Грани, их размерность.
	\item Теорема Фань Цзы.
	\item Теорема Вейля. Задание многогранников системой аффинных неравенств.
	\item Симплекс-метод. Выбор главных неизвестных. Связь с вершинами полиэдра. Изменение свободных членов уравнений.
	\item Изменение системы главных неизвестных. Достаточные условия сходимости симплекс-метода.
	\item Двойственная задача линейного программирования.
	\item Совпадение ответов прямой и двойственной задач линейного программирования.
	\item Теорема о равновесии.
	\item Матричные игры как задачи линейного программирования. Решение в
	смешанных стратегиях.
	\item Критерий оптимальности допустимого плана транспортной задачи в терминах потенциалов.
	\item Построение первоначального плана. Отсутствие в нем циклов.
	\item Решение систем уравнений для потенциалов для допустимого плана в
	невырожденной задаче без циклов.
	\item Улучшение плана. Существование и единственность пути улучшения плана
	\item Отсутствие циклов в улучшенном плане.
	\item Сходимость алгоритма решения невырожденной транспортной задачи.
	\item Нормированные векторные пространства и алгебры, примеры. Индуцированные нормы на алгебре матриц.
	\item Связь нормы матрицы с ее спектральным радиусом.
	\item Оценка спектрального радиуса для неотрицательной матрицы с помощью
	элементов матрицы.
	\item Теорема о собственных векторах положительной матрицы, для которых
	собственное значение равно по модулю спектральному радиусу матрицы.
	\item Одномерность собственного подпространства положительной матрицы, со-
	ответствующего спектральному радиусу матрицы.
	\item Вычисление $\lim_m[{\rho(A)}^{−1}A]^m$ для положительной матрицы $A$.
	\item Доказать, что спектральный радиус является простым корнем характеристического многочлена положительной матрицы.
	\item Доказать, что для неотрицательной матрицы существует неотрицательный собственный вектор, собственное значение которого равно спектральному радиусу матрицы.
	\item Доказать, что неотрицательная матрица A размера n неразложима тогда
	и только тогда, когда матрица $(E + A)^{n−1}$ положительна.
	\item Теорема Фробениуса.
	\item Сходимость $[{\rho(A)}^{−1}A]^m$ для неотрицательной неразложимой матрицы.
\end{enumerate}


 