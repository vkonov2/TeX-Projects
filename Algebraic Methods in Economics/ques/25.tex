\chapter{Сходимость алгоритма решения невырожденной транспортной задачи}
\label{cha:25}

\epigraph{
	\textit{Сходились, расходились, не находя места.}}
{-- Короленко В.Г.}

Приведенный алгоритм конечен. Действительно, на 4-ом шаге получаем новый план $X'$. Если $a_{i_0j_0}$ из (66), то считая $j_{k+1} = j_0$ получаем:
$$\begin{gathered}
	z(X') - z(X) = \underset{i, j}{\overset{}{\sum}}c_{ij} (x'_{ij} - x_{ij}) = \\
	= \underset{s=0}{\overset{k}{\sum}}c_{i_s j_s} (x'_{i_s j_s} - x_{i_s j_s}) + \underset{s=0}{\overset{k}{\sum}}c_{i_s j_{s+1}} (x'_{i_s j_{s+1}} - x_{i_s j_{s+1}}) = \underset{s=0}{\overset{k}{\sum}}c_{i_s j_s} \theta - \underset{s=0}{\overset{k}{\sum}}c_{i_s j_{s+1}} \theta = \\
	= \theta \left[ c_{i_0 j_0} + (u_{i_1} + v_{j_1}) + \dots + (u_{i_{k-1}} + v_{j_{k-1}}) - (u_{i_0} + v_{j_1}) - \dots - (u_{i_k} + v_{j_{k+1}}) \right] = \\
	= \theta [c_{i_0 j_0} - (u_{i_0} + v_{j_0})] = - \theta \alpha_{i_0 j_0} < 0
\end{gathered}$$
При этом для нового плана $X'$ выполнено условие 2) из теоремы \ref{cha:20/the:1}. Таким образом, число шагов не превосходит числа подмножеств клеток с ненулевыми элементами из допустимого плана $X'$. Так как значение $Z(X)$ уменьшается, то у нас не возникает повторений.

Изложение алгоритма завершено.