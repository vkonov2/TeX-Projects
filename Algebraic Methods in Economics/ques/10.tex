\chapter{Грани полиэдров и экстремумы аффинных функций на полиэдрах}
\label{cha:10}

\epigraph{
	\textit{В языческом сознании не было непроходимой грани между богами и человеком.}}
{-- Бердяев Н.А.}

\begin{definition}\label{cha:10/def:1}
	\blue{Полиэдром} P называется множество всех точек $x \in \mathbb{A}^n$, удовлетворяющей заданной системе аффинных неравенств (5).
\end{definition}

Иначе говоря, полиэдр – это пересечение конечного числа полупространств. 

\textit{Размерностью} полиэдра P называется размерность наименьшей плоскости, содержащей P . Другими словами, размерность P совпадает с рангом системы векторов $\{ \overline{AB} | A, B \in P \}$. 

Пусть $f_1, \dots, f_r$ – все аффинные функции из (5), обращающиеся в нуль в точке A. Обозначим через П плоскость, задаваемую уравнениями $f_1 = \dots = f_r = 0$. Гранью $\Gamma_A$ точки A в P называется пересечение $\Pi \cap P$.

Отметим, что каждая грань является полиэдром, поскольку она задается неравенствами (5) и неравенствами $−f_1 \ge 0, \dots, −f_r \ge 0$.

\begin{definition}\label{cha:10/def:2}
	\textit{Вершиной} полиэдра называется грань нулевой размерности. Грань размерности 1 называется \textit{ребром}.
\end{definition}

\begin{propose}\label{cha:10/propose:1}
	Точка $A \in P$ является внутренней точкой грани $\Gamma_A$.
\end{propose}
\begin{Proof}
	Предположим, что полиэдр P задается неравенствами (5), а грань $\Gamma_A$, $A \in P$, задается уравнениями $f_1 = \dots = f_r = 0$. В этом случае $f_{r+1}(A) > 0, \dots, f_m(A) > 0$. Пусть $U \subset \mathbb{A}^n$ – множество всех таких точек $B \in \mathbb{A}^n$, что $f_{r+1}(B) > 0, \dots, f_m(B) > 0$. Тогда U – открытое подмножество в $\mathbb{A}^n$. Если $\Pi$ – плоскость, задаваемая уравнениями $f_1 = \dots = f_r = 0$, то $U \cap \Pi \subset P$, т. е. A – внутренняя точка $\Gamma_A$.
\end{Proof}

\begin{theorem}[]\label{cha:10/the:1}
	Пусть аффинная функция f достигает экстремума в некоторой внутренней точке полиэдра P. Тогда $f |_P = const$.
\end{theorem}
\begin{Proof}
	Можно считать, что начало координат O является точкой экстремума f и можно считать, что размерность P совпадает с размерностью всего аффинного пространства. Пусть в некоторой системе координат $\displaystyle f(x) = a_0 + \underset{i=1}{\overset{n}{\sum}}a_i x_i$. Так как O – точка экстремума, то $\displaystyle a_j = \frac{\partial f}{\partial x_j} (O) = 0$ для любого $j = \ton m$. Поэтому $f(x) = a_0$.
\end{Proof}

\begin{propose}\label{cha:10/propose:2}
	Пусть f – аффинная функция, принимающая неотрицательные значения на полиэдре P, причем $f(A) = 0$ для некоторой точки $A \in P$. Тогда $f |_{\Gamma_A} = 0$.
\end{propose}
\begin{Proof}
	Точка A является внутренней точкой $\Gamma_A$ по предложению \ref{cha:10/propose:1}. Остается воспользоваться теоремой \ref{cha:10/the:1}.
\end{Proof}

\begin{conseq}[]\label{cha:10/conseq:1}
	Определение грани полиэдра не зависит от системы неравенств, задающих полиэдр.
\end{conseq}













