\chapter{Теорема о равновесии}
\label{cha:18}

\epigraph{
	\textit{Дело в том, что нужно создать физический противовес внутренней душевной тяжести — и равновесие восстановляется.}}
{-- Мамин-Сибиряк Д.Н.}

Отметим интерпретацию двойственной задачи в терминах сопряженных пространств. Пусть как и выше P – полиэдр, задаваемый неравенствами (40), а Q – полиэдр, задаваемый неравенствами (42). В прямой задаче необходимо найти $\underset{x \in P}{\max} (^tpx)$, а в двойственной — $\underset{y \in Q}{^tby}$. Пусть $A = (a_{ij})$ и $O, e_1, \dots, e_n$ – выбранная система координат в $\mathbb{A}^n$. Предположим, что V – векторное пространство с базой $e_1, \dots, e_n$ и $V^*$ – сопряженное пространство. Обозначим через $l_i \in V^*$, $1 \le i \le m$, линейные функции $l_i (x) = \underset{j=1}{\overset{n}{\sum}}a_{ij}x_j$. Тогда полиэдр P задается неравенствами $l_1 (x) \le b_1, \dots, l_m (x) \le b_m, \; x\ge 0$. Рассмотрим конус K c вершиной – нулевой функцией, порождаемый $l_1, \dots, l_m$. Тогда Q можно отождествить с множеством всех таких линейных функций $f \in V^*$, что $f(e_i) \ge p_i$, $1 \le i \le n$.

Установим связь между точками экстремума прямой и двойственной задач линейного программирования. 

\begin{theorem}[\red{о равновесии}]\label{cha:18/the:1}
	Пусть в точке $x^o \in P$ достигается максимум $^tpx$, а в точке $y^o \in Q$ – минимум функции $^tby$. Тогда $y_i^o (a_{i1}x_1^o + \dots + a_{in}x_n^o − b_i) = 0$, где $b_i$ – i-ая координата b.
\end{theorem}
\begin{Proof}
	Так как $x^o, y^o \ge 0$ и $−^ty^oA \le −^tp$, то:
	$$0 \le ^t y^o (b - A x^o) = ^t y^o b - ^t y^o A x^o \le ^t p x^o - ^t p x^o = 0$$
	Следовательно, $^t y^o (b - A x^o) = 0$. Отсюда вытекает утверждение, поскольку $b \le Ax^o$, $y^o \ge 0$.
\end{Proof}
