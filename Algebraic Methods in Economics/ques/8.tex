\chapter{Приложение теоремы фон Неймана к теории конечных антагонистических
игр}
\label{cha:8}

\epigraph{
	\textit{В этой игре, по ее бесплотности и страшности, действительно было что-то адово, аидово.}}
{-- Цветаева М.И.}

В большинстве конечных игр двух лиц седловая точка отсутствует. Применение оптимальных стратегия гарантирует выигрыш, равный a. Возникает вопрос: нельзя ли гарантировать \textit{средний} выигрыш, больший a, если применять не одну, как говорят, \textit{чистую} стратегию, а чередовать стратегии по некоторому вероятностному закону? Таким комбинированные стратегии в теории игр называются \textit{смешанными} стратегиями. Ясно, что чистая стратегия является частным случаем смешанной, когда вероятность выбора одной стратегии равна 1, а остальных — 0. Сейчас мы увидим, что ответ на поставленный вопрос положительный.

Смешанную стратегию первого игрока $\alpha$ будем обозначать строкой $p =$ \\
$(p_1, \dots, p_n)$, где $p_i$ – вероятность выбора стратегии $\alpha_i$. Тогда $p_i \ge 0$ и $p_1 + \dots + p_n = 1$. Аналогично, смешанную стратегию второго игрока $\beta$ обозначим строкой $q = (q_1, \dots, q_m)$, где $q_i$ – вероятность выбора стратегии $\beta_i$, то есть $q_i \ge 0$ и $q_1 + \dots + q_m = 1$. Строки p, q можно рассматривать как координаты точек соответственно n-мерного и m-мерного пространств. Ограничения на p, q показывают, что допустимые значения p,q пробегают симплексы P, Q соответствующих пространств. Выбор i-ой стратегии игроком $\alpha$ и j-ой стратегии игроком $\beta$ — независимые события. Следовательно, вероятность их наступления равна $p_i q_j$. Поэтому математическое ожидание выигрыша при применении пары смешанных стратегий p, q равна числу 
$$\delta(p, q) = \underset{i=1}{\overset{n}{\sum}}\underset{j=1}{\overset{m}{\sum}}a_{ij} p_i q_j = p A^t q \eqno(15)$$
где $A = (a_{ij})$ – матрица игры размера $n \times m$. Величина $\delta(p, q)$ линейна по каждому из своих аргументов p,q. Важно отметить, что допустимые значения вектора $A^t q$ являются выпуклым многогранником как образ симплекса при аффинном отображении. Дело в том, что при аффинном отображении выпуклая линейная оболочка векторов переходит в выпуклую линейную оболочку их образов.

Теперь определим гарантированный средний выигрыш и проигрыш игроков $\alpha$ и $\beta$ и их оптимальные смешанные стратегии. Если игрок $\alpha$ применяет стратегию p, то игрок $\beta$ выбирает смешанную стратегию, реализующую $\psi(p) = \underset{q \in Q} \delta(p, q)$. Тогда гарантированный средний выигрыш игрока $\alpha$ равен $\underset{p \in P}{\max} \; \underset{q \in Q}{\min} \delta (p, q)$. Из симметричных соображений, если игрок $\beta$ выбирает стратегию q, то игрок $\alpha$ ответит на нее стратегией, реализующей $\varphi(q) = \underset{p \in P} \delta(p, q)$. Средний проигрыш игрока $\beta$ не превосходит $\underset{q \in Q}{\min} \; \underset{p \in P}{\max} \delta (p, q)$.

По теореме фон Неймана оба числа существуют и равны между собой:
$$ \underset{p \in P}{\max} \; \underset{q \in Q}{\min} \delta (p,q) = \underset{q \in Q}{\min} \; \underset{p \in P}{\max} \delta (p,q) \eqno(16)$$
Пусть обе части (15) реализуются на паре смешанных стратегий $(p^*,q^*)$, и $\delta^* = \delta(p^*,q^*)$ – число, равное обеим частям (15). Оптимальные смешанные стратегии $p^*$, $q^*$ обладают необходимым свойством устойчивости: при любом одностороннем отклонении от оптимальной стратегии выигрыш меняется в направлении, невыгодном отклонившейся стороне. Действительно, переписав неравенства 2) из теоремы фон Неймана, получим соответственно:
$$\delta (p^*, q^*) \ge \delta (p, q^*), \; \delta (p^*, q) \ge \delta (p, q^*)$$

Говорят, что игра имеет решение, если существует пара смешанных стратегий, являющаяся седловой точкой и обладающая сформулированным выше условием устойчивости. Только что доказанные утверждения могут быть коротко сформулированы в виде следующей основополагающей в теории игр теоремы.

\begin{theorem}[]\label{cha:8/the:1}
	Каждая конечная игра двух лиц имеет решение в области смешанных стратегий.
\end{theorem}


