\chapter{Построение первоначального плана. Отсутствие в нем циклов}
\label{cha:21}

\epigraph{
	\textit{Я отчасти участвовал в переорганизации общества по новому плану, и только.}}
{-- Достоевский Ф.М.}

\begin{definition}\label{cha:21/def:1}
	Транспортная задача \textit{вырождена}, если существуют такие собственные подмножества индексов $\displaystyle F \subset \{1, \dots, m\}, \; H \subset \{1, \dots, n\}$, что $\underset{i \in G}{\overset{}{\sum}}a_i = \underset{j \in H}{\overset{}{\sum}}b_j$. Другими словами, суммарный запас продукта в пунктах $A_i, i \in G$, совпадает с потреблением в пунктах $B_j, j \in H$.
\end{definition}

Далее мы будет предполагать, что рассматриваемая транспортная задача \textit{невырождена}.

\begin{propose}\label{cha:21/propose:1}
	Пусть задан допустимый план X невырожденной задачи и $i_0 \in \{1, \dots, m\}$. Тогда для любого $j \in \{1, \dots,m\}$ существует такая последовательность клеток $(i_0, j_1), (i_1, j_1), (i_1, j_2), \dots, (i_k, j_k), (i_k, j_{k+1})$, что $j_{k+1} = j$ и во всех клетках этой последовательности в матрице X стоят ненулевые коэффициенты.
	
	Аналогично, для любого $j_0 \in \{1, \dots, n\}$ и любого $i \in \{1, \dots, m\}$. Тогда существует последовательность клеток $(i_1, j_0), (i_1, j_1), (i_2, j_1), \dots, (i_k, j_k), (i_{k+1}, j_k)$, \\ что $i_{k+1} = i$ и во всех клетках этой последовательности в матрице X стоят ненулевые коэффициенты.
\end{propose}
\begin{Proof}
	Рассмотрим первое утверждение. Обозначим через H множество всех таких индексов $l \in \{1, \dots, n\}$, для которых найдется последовательность ненулевых элементов $x_{i_0j_1}, x_{i_1j_1}, \dots, x_{i_kj_k}, x_{i_kj_{k+1}}, \; j_{k+1} = l$. Предположим, что $j \not \in H$. Через G обозначим множество всех таких индексов $t \in \{1, \dots, m\}$, что $x_{tl} \not = 0$ для некоторого $l \in H$. Из определения G вытекает, что если $t \in G$ и $x_{tl} \not = 0$, то $l \in H$. Поэтому $\displaystyle \underset{j \in H}{\overset{}{\sum}}b_j = \underset{i \in G, j \in H}{\overset{}{\sum}}x_{ij} = \underset{i \in G}{\overset{}{\sum}}a_i$. Отсюда $\displaystyle \underset{i \not \in G}{\overset{}{\sum}}a_i = \underset{i \not \in H}{\overset{}{\sum}}b_j > 0$.

	Следовательно, $G \not = \{1, \dots, n\}$, и $H \not = \{1, \dots, m\}$. Это противоречит невырожденности задачи.
	
	Симметрично доказывается второе утверждение.
\end{Proof}

\begin{conseq}[]\label{cha:21/conseq:1}
	Пусть план $X = (x_{ij})$ допустим, и $x_{i_0j_0} = 0$. Тогда существует такая последовательность строк $i_0, i_1, \dots, i_k$ и последовательность столбцов $j_1, \dots, j_k, j_0$ матрицы X, что элементы
	$$x_{i_0j_1}, x_{i_1j_1}, \dots, x_{i_kj_k}, x_{i_kj_0}\eqno(64)$$
	отличны от нуля.
\end{conseq}

\begin{definition}\label{cha:21/def:2}
	Пусть $X = (x_{ij})$ – допустимый план. \textit{Циклом} в X назовем последовательность ненулевых элементов $x_{p_1q_1}, x_{p_1q_2}, \dots, x_{p_sq_s}, x_{p_sq_1}$, причем среди первых и среди вторых индексов в этой последовательности имеются нет совпадений.
\end{definition}

\begin{propose}\label{cha:21/propose:2}
	Если допустимый план не содержит циклов, то в предложении \ref{cha:21/propose:1} по набору $i, j$ искомые последовательности определены однозначно. Кроме того, в (64) последовательность также определена однозначно.
\end{propose}
\begin{Proof}
	 Пусть для заданных $i \in \{1, \dots, m\}$, $j \in \{1, \dots, n\}$ две разные последовательности
	 $$\begin{gathered}
	 	(i_0, j_1), (i_1, j_1), (i_1, j_2), \dots, (i_k, j_k), (i_k, j_{k+1}) \\
	 	(i_0 , j'_1 ), (i'_1 , j'_1 ), (i'_1 , j'_2 ), \dots, (i'_s , j'_s ), (i'_s , j'_{s+1} )
	 \end{gathered}$$
	 где $j_{k+1} = j'_{s+1} = j$ и во всех клетках этих последовательностей в матрице X стоят ненулевые коэффициенты. Погда получаем цикл
	 $$(i_k,j), (i_k,j_k), \dots, (i_1,j_1), (i_0,j_1), (i_0,j'_1), (i'_1,j'_1), \dots, (i'_s,j'_s), (i's_,j)$$
	 что невозможно. Аналогично рассматривается утверждение для (64).
\end{Proof}

Изложим теперь алгоритм решения невырожденной транспортной задачи.

ШАГ 1. \textit{Построение первоначального} плана \textit{методом минимального элемента}. В соответствии с предложением \ref{cha:20/propose:1} строим первоначальный план $X^0 = (x_{ij}^0)$.

\begin{propose}\label{cha:21/propose:3}
	План $X^0$ не содержит циклов. В каждой строке и столбце плана $X^0$ содержится ненулевой элемент. Всего в $X^0$ число ненулевых элементов равно $m + n − 1$.
\end{propose}
\begin{Proof}
	Пусть план $X^0$ содержит цикл $x_{p_1q_1}, x_{p_1q_2}, \dots, x_{p_kq_k}, x_{p_kq_1}$ из определения \ref{cha:21/def:2}. Выберем в этой последовательности тот элемент, который построен первым. Пусть, например, это $x_{p_1q_1}$. Тогда элементы $x_{p_1q_2}, x_{p_sq_1}$, построены позднее, что невозможно, ибо при построении $x_{p_1q_1}$ , либо на месте $(p_1, q_2)$, либо на $(p_s, q_1)$ ставится 0. Аналогично рассматриваются остальные случаи.
\end{Proof}





















