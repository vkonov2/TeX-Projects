\chapter{Теорема Фань Цзы}
\label{cha:12}

\epigraph{
	\textit{Истинная метода одна: это собственно процесс ее органической пластики; форма, система — предопределены в самой сущности ее понятия и развиваются по мере стечения условий и возможностей осуществления их.}}
{-- Герцен А.И.}

\begin{theorem}[\red{Фань Цзы}]\label{cha:12/the:1}
	Пусть полиэдр P размерности s задается в $\mathbb{A}^n$ системой неравенств (5), причем ранг линейных частей $f_1, \dots, f_m$ равен r. Если $A \in P$, то $dim \Gamma_A \ge n−r$ и каждая грань в P размерности $d > n − r$ обладает гранью размерности $d − 1$. В частности, $s \ge n − r$, и в P имеется грань размерности s.
\end{theorem}
\begin{Proof}
	Переходя к новому базису, можно считать, что:
	$$f_1 = x_1, \dots, f_r = x_r, \; f_j= \underset{i \le r}{\overset{}{\sum}}a_{ji} x_i + c_j, \; j > r$$
	Обозначим через U подпространство, задаваемое уравнениями $x_1 = \dots = x_r =0$.

	Пусть $A = (x_1^0, \dots, x_n^0) \in P$ и $u = ( \overbrace{0, \dots, 0}^{r}, u_{r+1}, \dots, u_n) \in U$. Тогда $A + u = (x_1^0, \dots, x_r^0, x_{r+1}^0 + u_{r+1}, \dots, x_n^0 + u_n)$. При этом $x_i^0 \ge 0$ при $i = \ton r$ и:
	$$f_j (A+u) = c_j + \underset{i \le r}{\overset{}{\sum}}a_{ji} x_i^0 \ge 0, \; j > r$$
	Поэтому $A+U \subseteq P$. Если $f_j(A) = 0$ для некоторого $j > r$, то, аналогично, $f_j(A+U) = 0$, откуда $\Gamma_A \supseteq A+U$, и поэтому $dim \Gamma_A \ge dim U = n − r$.
	
	Пусть $\Gamma_B$ – грань точки B в P, и $dim \Gamma_B > n−r$. Тогда не все функции $x_1, \dots, x_r$ тождественно равны нулю на $\Gamma_B$. По теореме \ref{cha:11/the:1} в $\Gamma_B$ имеется грань размерности $dim \Gamma_B − 1$.
\end{Proof}

\begin{theorem}[]\label{cha:12/the:2}
	Пусть полиэдр P обладает вершиной и аффинная функция f на P достигает минимума. Тогда этот минимум достигается и в некоторой вершине.
\end{theorem}
\begin{Proof}
	Пусть $c = \underset{P}{\min} f$ и $f(A) = c$. Тогда A – внутренняя точка грани $\Gamma_A$ по предложению \ref{cha:10/propose:1}. По предложению \ref{cha:10/propose:2} функция f постоянна на $\Gamma_A$. Заметим, что $\Gamma_A$ задается неравенствами, линейные части которых те же, что и в неравенствах, задающих P. Поэтому в силу теоремы Фань Цзы $\Gamma_A$ имеет грань меньшей размерности, если $\Gamma_A$ не вершина. Отсюда вытекает утверждение.
\end{Proof}
