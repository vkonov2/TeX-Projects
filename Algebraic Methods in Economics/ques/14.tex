\chapter{Симплекс-метод. Выбор главных неизвестных. Связь с вершинами полиэдра. Изменение свободных членов уравнений}
\label{cha:14}

\epigraph{
	\textit{Эта задача европейской науки и культуры была неведома Востоку; только с прошлого столетия наиболее чуткие люди стран Азии начали принимать великий научный опыт Европы, ее методы мышления и формы жизнедеятельности.}}
{-- Горький Максим}

Пусть в n-мерном аффинном вещественном пространстве $\mathbb{A}^n$ системой аффинных неравенств:
$$f_1(x) \ge 0, \dots, f_d (x) \ge 0\eqno(31)$$
ранга n задан полиэдр M. Кроме того, задана целевая аффинная функция z. Требуется найти минимальное (максимальное) значение функции z на полиэдре M.

% Приведем инвариантную формулировку \textit{симплекс-метода} и схему его численной реализации.

% \section*{Симплекс-метод в инвариантной форме}\label{cha:14/sec:1}

% С системой аффинных неравенств (31), задающих полиэдр (многогранник) M, свяжем множество P точек и прямых пространства $\mathbb{A}^n$, являющихся пересечениями граничных гиперплоскостей $\Pi_{f_i}$. В частности, вершины полиэдра M образуют подмножество в множестве точек P, а ребра M лежат на прямых из P. Описываемый алгоритм является направленным процессом, позволяющим, отправляясь от произвольной точки из P, путем серии итерационных переходов достигнуть вершины полиэдра M, в которой целевая функция принимает минимальное значение на M. При этом мы избегаем полного перебора. Как правило, количество испытуемых точек из P значительно меньше их общего числа. Алгоритм применим при одном существенном предположении об отсутствии \textit{вырождения}, т.е. предполагается, что через каждую точку из P проходит ровно n граничных гиперплоскостей, и, стало быть, ровно n прямых инциндентны каждой точке в P.

% Процедура применения алгоритма распадается на два пункта: нахождение вершины полиэдра M и отыскание вершины M, в которой целевая функция достигает минимума на M.

% \subsubsection*{Отыскание вершины полиэдра}\label{cha:14/sec:1/subsec:1}

% Опишем один шаг итерационного процесса. Выберем произвольную точку $O \in P$. Без ограничения общности рассуждений предположим, что она является решением системы линейных уравнений:
% $$f_1(x) = 0, \dots, f_n(x) = 0\eqno(32)$$
% т.е. ранг первых n аффинных функций из (31) равен n. Повторяя рассуждения из предложения \ref{cha:10/propose:2} мы можем фиксировать систему координат $(O, e_1, \dots, e_n)$, определяемую левыми частями (32). Это означает, что начало координат O совпадает с решением (32), а базис $e_1, \dots, e_n$, дуален к системе дифференциалов $\mathcal{D}f_1, \dots, \mathcal{D}f_n$. Координатная прямая с номером r, инциндентная в P точке O, имеет направляющим вектором $e_r$ и совпадает с пересечением координатных гиперплоскостей $\Pi_i, i \not = r$. Координата с номером r точки $A \in \mathbb{A}^n$ в выбранном репере совпадает с <<уклонением>> $f_r(a)$ этой точки от r-ой координатной гиперплоскости $\Pi_r$. Полиэдр M лежит в положительном ортанте репера, определенном как конус с вершиной в точке O, порожденный <<концами>> координатных векторов $O + e_i \in \mathbb{A}^n$.

% Если уклонения точки O от всех граничных гиперплоскостей неотрицательны: $f_i(O) \ge 0, \; 1 \le i \le d$, то O – вершина полиэдра M. Поэтому предположим, что существует граничная гиперплоскость $\Pi_r$, отделяющая точку O от полиэдра M, т.е. $\displaystyle f_r(O) < 0, \; n+1 \le r \le d$. Если при этом $\mathcal{D}f_r(e_i) \le 0, \; 1 \le i \le n$, то положительный ортант лежит в <<отрицательном>> открытом полупространстве $\mathbb{A}_{f_r}^{-}$ и полиэдр M пуст (система неравенств (31) несовместна). Мы интересуемся случаем, когда M не пуст и, следовательно, существует такое число s, $1 \le s \le n$, что $\mathcal{D}f_r(e_s) > 0$. Рассмотрим малое смещение начала координат вдоль положительной s-ой координатной полуоси $O_t = O + t e_s \in \mathbb{A}^n, \; t \ge 0$. Эта полусоь пересекается с граничными гиперплоскостями. Например, с r-ой граничной гиперплоскостью она пересекается в точке $O_{t_r}$, где $\displaystyle t_r = -\frac{f_r (O)}{\mathcal{D} f_r (e_s)} > 0$. Однако это пересечение с граничной гиперплоскостью может произойти раньше при меньших значениях параметра t. Положим $\displaystyle t_0 = \underset{i}{\min} \left\{ -\frac{f_r (O)}{\mathcal{D} f_r (e_s)} > 0 \right\}$. Пусть $i_0$ определятеся равенством $\displaystyle t_0 = -\frac{f_{i_0} (O)}{\mathcal{D} f_{i_0} (e_s)} > 0$.

% Ввиду отсутствия вырождения индекс $i_0$ определяется однозначно. Из соображений, связанных с непрерывностью аффинных функций и отсутствием вырождения, заключаем, что при смещении начала координат O в точку $O_{t_0}$ все уклонения от граничных гиперплоскостей, корме i-го и s-го, сохранили знак. Важно отметить, что новое начало координат $O_{t_0}$ лежит в положительном полупространстве s-ой координатной гиперплоскости:
% $$f_s (O_{t_0}) = \mathcal{D} f_s (t_0 e_s) = t_0 \mathcal{D} f_s (e_s) > 0$$
% Кроме того, оно продолжает оставаться на остальных координатных гиперплоскостях. Переходим к новому реперу $(O, e'_1, \dots, e'_n)$, определенному набором граничных гиперплоскостей $\displaystyle \Pi_{f_1}, \dots, \Pi_{f_{s-1}}, \Pi_{f_{i_0}}, \Pi_{f_{s+1}}, \dots, \Pi_{f_n}$. Можно посчитать, что:
% $$e'_j = e_j - \frac{\mathcal{D} f_{i_0} (e_j)}{\mathcal{D} f_{i_0}(e_s)}e_s, \; j \not = s \text{ и } e'_s = \frac{1}{\mathcal{D}f_{i_0} (e_s)}e_s$$
% При переходе к новому реперу все координаты векторов, кроме s-ой, сохраняются, поскольку они совпадают с уклонениями от соответствующих координатных гиперплоскостей.

% Переходом к новому реперу заканчивается шаг итерации. При этом могут представиться два случая.

% \textit{СЛУЧАЙ 1}. $i_0 = r$. Граничная гиперплоскость $\Pi_{f_r}$ с номером r не отделяет новое начало координат $O_{t_0}$ от полиэдра M. Промежуточная цель достигнута. Новое начало координат $O_{t_0}$ отделяетсяот полиэдра M на единицу меньшим числом граничных гиперплоскостей, чем старое начало O. Процедуру применяем к другой граничной гиперплоскости, отделяющей новое начало от полиэдра M (если таковые имеются).

% \textit{СЛУЧАЙ 2}. $i_0 \not = r$. Новое начало координат $O_{t_0}$ по-прежнему отделяется r-ой граничной гиперплоскостью $\Pi_{f_r}$ от полиэдра M. Общее количество граничных гиперплоскостей, отделяющих начало от полиэдра M, не возросло (если $f_{i_0} < 0$, то оно уменьшилось на единицу; если $f_{i_0} > 0$, то оно осталось прежним). Однако заведомо модуль уклонения начала от r-ой граничной гиперплоскости уменьшился: $|f_{r_0} (O_{r_0})| < |f_r (O)|$. Поэтому при многократном применении процедуры к гиперплоскости $\Pi_{f_r}$ мы придем к СЛУЧАЮ 1.

% Таким образом, последовательное применение описанных итерационных шагов приводит нас к реперу, начало которого вообще не отделяется от полиэдра M граничными гиперплоскостями, т.е. оно является вершиной M.

% \subsubsection*{Отыскание минимума целевой функции}\label{cha:14/sec:1/subsec:2}

% Целевая функция z также является аффинной функцией. В предыдущем пункте найден репер $(O, e_1, \dots, e_n)$, начало которого совпадает с одной из вершин полиэдра M. Не сужая общности рассуждений, можно предполагать, что координатными гиперплоскостями являются граничные гиперплоскости $\Pi_{f_i}$ , $i = \ton d$. Тогда $f_i(O) = 0$, $i = \ton n$, и, ввиду отсутствия вырождения, $f_j(O) > 0$, $j = n + 1, \dots, m$. Следовательно, из вершины O выходит ровно n ребер полиэдра M, лежащих на соответствующих координатных полуосях положительного ортанта. Если полуось не содержит вершин полиэдра M, кроме O, то она совпадает с ребром полиэдра M.

% Так как M лежит в положительном ортанте, то из неотрицательности величин $\displaystyle \mathcal{D} z(e_i), \; i = \ton n$ вытекает, что и целевая функция достигает в точке O минимума на M. Поэтому предположим, что среди этих величин есть отрицательные числа, например, $\mathcal{D}z(e_s) < 0$. Если $O_t = O + t e_s$, $t > 0$ — любая точка s-ой полуоси, отличная от O, то:
% $$z (O_t) = \mathcal{D}z (t e_s) + z (O) = t \mathcal{D} z (e_s) + z(O) < Z(O)$$
% В случае, когда s-ая полуось совпадает с ребром полиэдра M, целевая функция z на M не имеет минимума, т.е. $\displaystyle z(O_t) \xrightarrow[t\to \infty]{} -\infty$. 

% Рассмотрим случай, когда s-ая полуось содержит вершину полиэдра M, смежную с O. Для того, чтобы граничная гиперплоскость $\Pi_{f_r}$ пересекала s-ую полуось, необходимо и достаточно, чтобы выполнялось неравенство $\mathcal{D}f_r(e_s) < 0$. Пересечение состоится в точке $O_{t_r}$ , где $\displaystyle t_r = -\frac{f_r (O)}{\mathcal{D} f_r (e_s)} > 0$. Совершим итерационный шаг, описанный в предыдущем пункте, переходя к новому реперу $(O, e_1, \dots, e_n)$, где $\displaystyle t_0 = \min \left\{ -\frac{f_r (O)}{\mathcal{D} f_r (e_s)} \right\}$. 

% В силу выбора значения параметра $t = t_0$ новое начало $O_{t_0}$ по-прежнему оказывается вершиной полиэдра M, причем $z(O_{t_0}) < z(O)$. Если все числа $\mathcal{D}z(e_i)$ неотрицательны, то $O_{t_0}$ реализует минимум значений z на M. Если же среди этих чисел есть отрицательное, то повторяем снова итерационный шаг, либо убеждаемся, что минимума значений z на M не существует и т. д.

% Так как значения целевой функции z в вершинах полиэдра M, получаемые при совершении итерационных шагов, строго убывают, то за конечное число шагов мы либо достигнем вершины, реализующей минимум значения z на M, либо установим, что такового минимума не существует.

% Этим заканчивается инвариантное (т. е. не зависящее от системы координат) описание симплекс-метода.

\section*{Численная реализация симплекс-метода}\label{cha:14/sec:2}

\begin{definition}\label{cha:14/def:1}
	Задача линейного программирования состоит в нахождении максимума аффинной функции $z(x)$ при условии (31).
\end{definition}

Мы будем предполагать, что ранг системы функций $f_1(x), \dots, f_d(x)$ равен n. Тогда $d = m + n$, $m \ge 0$. Совершая замену переменных, можно добиться, чтобы $f_1 = x_1, \dots, f_n = x_n$. Тогда система неравенств имеет вид:
$$\begin{cases}
	x_1 \ge 0, \dots, x_n \ge 0 \\
	y_i = f_{n+i} (x) = \underset{j}{\overset{}{\sum}}a_{ij}x_j + b_j \ge 0, \; i = \ton m
\end{cases}\eqno(33)$$
Необходимо найти $minz,$ где$z = p_1x_1 + \dots + p_nx_n + q$. Изложим алгоритм \textit{симплекс-метода} решения этой задачи.

\subsection*{Симплекc-метод. Первый вариант}\label{cha:14/sec:2/subsec:1}

Запишем систему неравенств (33) в следующей эквивалентной форме. Так как $y_j \ge 0$, то существуют такие неотрицательные числа $x_{1+n}, \dots, x_{m+n}$, что неравенства (33) преобразуются в систему линейных уравнений и неравенств вида:
$$\begin{cases}
	x_1 \ge 0, \dots, x_d \ge 0, \; d = n + m \\
	\underset{j=1}{\overset{n}{\sum}}a_{ij}x_j - x_{j+n} = b_j, \; i = \ton m
\end{cases}$$
Таким образом, меняя обозначения, всегда можно считать, что система ограничений, задающих полиэдр M, имеет вид:
$$\begin{cases}
	x_1 \ge 0, \dots, x_n \ge 0 \\
	\underset{j=1}{\overset{n}{\sum}}a_{ij}x_j = b_j, \; i = \ton m
\end{cases}\eqno(34)$$
Если в каком-то уравнении из (34) коэффициент $b_i < 0$, то умножим это уравнение на $-1$. Таким образом, без ограничения общности можно предполагать, что $b_1, \dots, b_m \ge 0$. Кроме того, запишем целевую функцию z в виде $z + a_{m+1,1}x_1 + \dots + a_{m+1,n}x_n = b_{m+1}$, где $a_{m+1,j} =−p_j, j=\ton n$, и $b_{m+1} = q$.

Изложим алгоритм решения этой задачи, называемый \textit{симплекс-методом}. Он применим в случае отсутствия вырождения. Это означает, что в полиэдре M, задаваемом неравенствами (34), из каждой вершины выходят $n - r$ ребер (одномерных граней), где r – ранг матрицы $(a_{ij})$. Это означает, следовательно, что в каждой системе координат, если система неравенств имеет вид (34), выполнено условие $b_1, \dots, b_m > 0$.

\textit{ПЕРВЫЙ ЭТАП – НАХОЖДЕНИЕ ВЕРШИНЫ ПОЛИЭДРА}.

ШАГ 1. Составим матрицу из коэффициентов.
$$\text{\begin{tabular}{ | l | c | c || r |}
	    \hline
	    $x_1$ & $\dots$ & $x_n$ & \\ \hline \hline
	    $a_{11}$ & $\dots$ & $a_{1n}$ & $b_1$ \\ \hline
	    $\vdots$ & $\vdots$ & $\vdots$ & $\vdots$ \\ \hline
	    $a_{m1}$ & $\dots$ & $a_{mn}$ & $b_m$ \\ \hline
	    $a_{m+1,1}$ & $\dots$ & $a_{m+1,n}$ & $b_{m+1}$ \\ \hline
	\end{tabular}}\eqno(35)$$
ШАГ 2. Если в некотором i-ом уравнении все коэффициентов $a_{ij} \le 0$,
где $i = \ton m$, то полиэдр M, задаваемый ограничениями (34) пуст. Действительно, в i-ом уравнении левая часть неположительна, а правая равна $b_i > 0$.

ШАГ 3.
Пусть в i-ом уравнении коэффициент $a_{ir} > 0$. Зафиксируем столбец с номером r и среди всех положительных коэффициентов этого столбца выберем тот, на котором достигается $\underset{j=\ton m}{\min}\left( \frac{b_j}{a_{jr}} \right)$. Предположим, что этот минимум достигается на $a_{sr}$.

Деля s-ое уравнение на $a_{sr}$ можно считать, что $a_{sr} = 1$. Это означает, что все элементы s-ой строки матрицы (35) делятся на $a_{sr}$. Далее из каждой строки номером $t = 1, \dots, s−1, s+1, \dots, m+1$ вычитаем s-ую строку, умноженную на $a_{tr}$. В новой таблице переменная $x_r$ входит с коэффициентом 1 в s-ое уравнение, в остальные же уравнения и в z она входит с нулевым коэффициентом.

\begin{propose}\label{cha:14/propose:1}
	При указанных преобразованиях, все свободные члены уравнений, т.е. коэффициенты $b_i$, $1 \le i \le m$, остаются неотрицательными. Если $\displaystyle \underset{j=\ton m}{\min}\left( \frac{b_j}{a_{jr}} \right)$ достигается на одном элементе $a_{sr}$ из r-го столбца, то все $b_i$, $1 \le i \le m$, остаются положительными.
\end{propose}
\begin{Proof}
	Свободный член в t-ой строке имеет вид $b_t − a_{t,r}b_s$. Если $t \le m$ и $a_{tr} > 0$, то в силу выбора s имеем $\displaystyle \frac{b_s}{a_{sr}} = b_s \le \frac{b_t}{a_{tr}}$. Поэтому $b_t - a_{t,r}b_s \ge 0$. Если $\displaystyle \underset{j=\ton m}{\min}\left( \frac{b_j}{a_{jr}} \right)$ достигается на одном элементе $a_{sr}$, то при $t \not= s$ получаем $b_t − a_{t,r} b_s > 0$.

	Если же $a_{t,r} \le 0$, то $b_t − a_{t,r} b_s \ge b_t > 0$. 
\end{Proof}

Таким образом, совершая многократно ШАГ 3 мы можем как в методе Гаусса привести таблицу (35) к ступенчатому виду, т.е. каждое уравнение с номером $j = \ton m$ является выражением j-го главного неизвестного через свободные. При этом в последней строке ненулевые коэффициенты стоят только при свободных неизвестных.

Придадим свободным неизвестным нулевые значения. В этом случае значения главных неизвестных равны свободным членам и потому положительны. Таким образом, построенное решение X системы лежит в полиэдре M. По следствию \ref{cha:9/conseq:1} решение X является вершиной полиэдра M, из него выходят $n − r$ ребер.

\begin{propose}\label{cha:14/propose:2}
	Пусть все коэффициенты $a_{m+1,1}, \dots, a_{m+1,n} \ge 0$, причем если неизвестная $x_j$ главная, то $a_{m+1,j} = 0$. Тогда в точке X функция $z(x)$ достигает максимума, равного $b_{m+1}$.
\end{propose}
\begin{Proof}
	Заметим, что $z = z(x) = -a_{m+1,1}x_1 - \dots - a_{m+1,n}x_n + b_{m+1}$, причем ненулевые коэффициенты стоят только при свободных неизвестных. Это означает, что для любой точки $u \in M$ имеем $z(u) \le z(X) = b_{m+1}$. Максимальное значение $b_{m+1}$ достигается при нулевых значениях свободных неизвестных. При этом, если главная неизвестная $x_j$ находится в k-ом уравнении, то ее значение равно $b_k$.
\end{Proof}

Предположим теперь, что $a_{m+1,r} < 0$ для некоторого r. В этом случае переходим ко второму этапу (см. ниже ШАГ 4).

% \textit{ВТОРОЙ ЭТАП – НАХОЖДЕНИЕ max z}

% ШАГ 4.
% Пусть $a_{m+1,r} < 0$. Рассмотрим коэффициенты при r-ой переменной.

% \begin{propose}\label{cha:14/propose:3}
% 	Если коэффициенты $a_{1r}, \dots, a_{mr}$ матрицы (35) отрицательны, то максимума у функции z нет.
% \end{propose}
% \begin{Proof}
% 	Придадим свободной переменной $x_r$ произвольное значение $k > 0$, а всем остальным свободным переменным придадим нулевое значение. Тогда значение главного неизвестного из i-го уравнения равно $b_i − a_{ir}k > 0$. Таким образом, получаем точку $X(k) \in M$. При этом $z(X(k)) = −a_{m+1,r}k$ принимает сколь угодно большие значения. Следовательно, функция z не имеет максимума на M.
% \end{Proof}

% Пусть $a_{m+1,r} < 0$ и $a_{ir} > 0$ для некоторого $i=\ton m$. Переходим к ШАГУ 3. Предположим, что у нас были свободные переменные $x_{i_1}, \dots, x_{i_m}$ и в ШАГЕ 3 мы выбрали элемент $a_{sr}$, причем в s-ое уравнение с коэффициентом 1 входила главная переменная $x_{i_j}$. Совершим ШАГ 3 мы заменим $x_{i_j}$ на другую главную переменную $x_s$. Тем самым получим новую систему главных переменных $x_{i_1}, \dots, x_{i_{j−1}}, x_s, x_{i_{j+1}}, \dots, x_{i_m}$.

% \begin{propose}\label{cha:14/propose:4}
% 	При пременении ШАГА 3 значение свободного члена $b_{m+1}$ увеличивается.
% \end{propose}
% \begin{Proof}
% 	Как и в предложении \ref{cha:14/propose:1} свободный член в z заменяется на $b_{m+1} − a_{m+1,r} b_s$ для некоторого s. Но $a_{m+1,r} < 0$, а $b_s > 0$. Поэтому $b_{m+1} − a_{m+1,r}b_s > b_{m+1}$.
% \end{Proof}

% Итак, мы совершаем различные выборы свободных и главных переменных, т.е. получаем вершины из M. При этом мы никогда не вернемся к выбранной ранее вершине, поскольку на каждом шаге значение целевой функции z увеличивается. Итак, совершив конечное число шагов, мы перейдем к вершине M, в которой функция z достигает максимума, либо выясним, что задача не имеет решения.























