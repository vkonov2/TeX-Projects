\chapter{Матричные игры как задачи линейного программирования. Решение в
смешанных стратегиях}
\label{cha:19}

\epigraph{
	\textit{Бог тут не при чем. Ну, рассказывай, — продолжал он, возвращаясь к своему любимому коньку, — как вас немцы с Бонапартом сражаться по вашей новой науке, стратегией называемой, научили.}}
{-- Толстой Л.Н.}

Пусть $p = (p_1, \dots, p_n)$, $q = (q_1, \dots, q_m)$ искомые оптимальные стратегии, соответственно, игроков $\alpha$ и $\beta$. Без ограничения общности рассуждений можно предполагать, что все элементы матрицы игры A положительны. Если бы это было не так, то можно ко всем элементам матрицы A добавить одно и то же достаточно большое положительно число. При этом оптимальные стратегии не изменяются, а средний выигрыш увеличится на это число. Обозначим через v гарантированный выигрыш (гарантированную верхнюю оценку проигрыша). Тогда:
$$v = \underset{i=1}{\overset{n}{\sum}}\underset{j=1}{\overset{m}{\sum}}a_{ij} p_i q_j = \underset{i=1}{\overset{n}{\sum}}p_i \left( \underset{j=1}{\overset{m}{\sum}}a_{ij}q_j \right)\eqno(17)$$
В скобках правой части (17) стоит математическое ожидание проигрыша второго игрока при применении первым игроком i-ой чистой стратегии $\alpha_i$. Так как q – оптимальная стратегия второго игрока, то $\displaystyle \underset{j=1}{\overset{n}{\sum}}a_{ij}a_j \le v$. Однако все участвующие в записи параметры неотрицательны и $p_1 + \dots + p_n = 1$. Поэтому в (17) имеет место строгое равенство $\displaystyle \underset{j=1}{\overset{m}{\sum}}a_{ij}q_j = v$. Из симметричных соображений получаем равенство $\displaystyle \underset{i=1}{\overset{n}{\sum}}a_{ij}p_i = v$. Таким образом, отыскание оптимальных смешанных стратегий сводится к решению следующей задачи линейного программирования: найти $\min v$, где:
$$\begin{gathered}
	\underset{j=1}{\overset{m}{\sum}}a_{ij}q_j = v, \; \underset{i=1}{\overset{n}{\sum}}a_{ij}p_i = v \\
	\underset{i=1}{\overset{n}{\sum}} p_i = 1, \; \underset{j=1}{\overset{m}{\sum}}q_j = 1 \\
	p_i, q_j \ge 0, \; i = \ton n, \; j = \ton m
\end{gathered}$$

Рассмотрим матричную игру с матрицей $A = (a_{ij}) \in Mat(n \times m, \mathbb{R})$. Применим алгоритм симплекс-метода для нахождения седловой точки смешанных стратегий из теоремы \ref{cha:6/the:1}.

Заметим, что если мы прибавим ко всем элементам матрицы A одно и то же число d, то результат игры увеличится на d, но седловые точки не изменятся. Действительно, значение среднего выигрыша:
$$\begin{gathered}
	\underset{i, j}{\overset{}{\sum}}x_i (a_{ij} + d) y_j = \underset{i,j}{\overset{}{\sum}}x_i a_{ij} y_j + d \underset{i, j}{\overset{}{\sum}} x_i y_j = \\
	= \underset{i, j}{\overset{}{\sum}}x_i a_{ij} y_j + d \left( \underset{i}{\overset{}{\sum}}x_i \right) \left( \underset{j}{\overset{}{\sum}}y_j \right) = \underset{i, j}{\overset{}{\sum}}x_i a_{ij} y_j + d \\
	\text{т.к. } \underset{i}{\overset{}{\sum}}x_i = \underset{j}{\overset{}{\sum}}y_j = 1
\end{gathered}$$
Следовательно, можно считать, что все элементы матрицы A неотрицательны, причем в матрице A нет нулевых столбцов. Тогда результат игры v лежит в пределах $\underset{i}{\max} \; \underset{j}{\min} a_{ij} \le v \le \underset{j}{\min} \; \underset{i}{\max} a_{ij}$ и потому положителен.

Положим $F(x,y) = \underset{i, j}{\overset{}{\sum}}a_{ij}x_i y_j$. Напомним, что если $x^* = (x_1^*, \dots, x_n^*)$, $y^* = (y_1^*, \dots, y_m^*)$ - седловая точка, соответствующая оптимальным стратегиям первого и второго игрока, то для любых стратегий x, y первого и второго игрока имеем:
$$F(x, y^*) \le F(x^*, y^*) \le F(x^*, y)\eqno(49)$$
В частности:
$$v = \underset{x \in N}{\max} \; \underset{y \in M}{\min} F(x, y) = \underset{y \in M}{\min} \; \underset{x \in N}{\max}F(x, y) = F(x^*, y^*)$$
Здесь:
$$\begin{gathered}
	N = \Set{x \in \mathbb{A}^n}{x_1 + \dots + x_n = 1, \; x_i \ge 0} \\
	M = \Set{y \in \mathbb{A}^m}{y_1 + \dots + y_m = 1, \; y_i \ge 0}
\end{gathered}\eqno(50)$$
Рассмотрим следующую задачу линейного программирования:
$$\begin{gathered}
	f = u_1 + \dots + u_m \to \max \\
	A u \le \begin{pmatrix}
		1 \\ \vdots \\ 1
	\end{pmatrix}, \; u = \begin{pmatrix}
		u_1 \\ \vdots \\ u_m
	\end{pmatrix} \ge 0
\end{gathered}\eqno(51)$$
Рассмотрим также двойственную к (51) задачу:
$$\begin{gathered}
	g = t_1 + \dots + t_n \to \max \\
	^tA t \le \begin{pmatrix}
		1 \\ \vdots \\ 1
	\end{pmatrix}, \; t = \begin{pmatrix}
		t_1 \\ \vdots \\ t_n
	\end{pmatrix} \ge 0
\end{gathered}\eqno(52)$$

\begin{propose}\label{cha:19/propose:1}
	Полиэдры P, Q, задаваемые, соответственно, неравенствами из (51) и (52), непусты. Кроме того, P ограничено. В частности, $\underset{u \in P}{\max} f = \underset{t \in Q}{\min} g > 0$.
\end{propose}
\begin{Proof}
	Заметим, что начало координат O лежит в P , но не лежит в Q. Кроме того, если все координаты t достаточно большие, то $t \in Q$. Поэтому P, Q непусты.
	
	По предположению матрица A неотрицательна и в A нет нулевых столбцов. Поэтому для любого $j = \ton m$ найдется такой индекс i, что $a_{ij} > 0$, откуда $\displaystyle a_{ij} u_j \le \underset{s=1}{\overset{m}{\sum}}a_{is} u_s \le 1$ и поэтому $u_j \le \frac{1}{a_{ij}}$. Остается воспользоваться теоремой \ref{cha:17/the:1}. Так как $O \not \in Q$, то $\underset{t \in Q}{\min} g > 0$. 
\end{Proof}

Положим $\underset{u \in P}{\max} f = \underset{t \in Q}{\min} g = \frac{1}{v}$. Тогда $v > 0$. Обозначим через $u^*, t^*$ – оптимальные решения задач (51), (52), соответственно. Положим $x^* = vt^*, y^* = vu^*$.

\begin{theorem}[]\label{cha:19/the:1}
	Точка $(x^*, y^*)$ является седловой для рассматриваемой матричной игры с матрицей $A \ge 0$, не содержащей нулевых столбцов.
\end{theorem}
\begin{Proof}
	Заметим, что:
	$$\underset{i=1}{\overset{m}{\sum}}y_i^* = v \underset{i=1}{\overset{m}{\sum}}u_i^* = v \left( \underset{u \in P}{\max} f \right) = 1$$
	т.е. $y^* \in M$, где M из (50). Аналогично, $x^* \in N$.
	По теореме \ref{cha:18/the:1} получаем $\displaystyle t_i^* \left( \underset{j=1}{\overset{m}{\sum}}a_{ij} u_j^* - 1 \right) = 0$. Домножая на $v^2$, получаем $\displaystyle x_i^* \underset{j=1}{\overset{m}{\sum}}a_{ij} y_j^* = x_i^* v$, откуда:
	$$F(x^*, y^*) = \underset{i, j}{\overset{}{\sum}}x_i^* a_{ij} y_j^* = \underset{i}{\overset{}{\sum}}x^* v = v\eqno(53)$$
	поскольку $x^* \in N$. Кроме того, для любых $x \in N$, $y \in M$ по (51) и (53):
	$$F(x, y^*) = \underset{i, j}{\overset{}{\sum}}x_i a_{ij} u_j^* v \le \underset{i}{\overset{}{\sum}}x_i v = v = F(x^*, y^*)$$
	Аналогично $F (x^*, y^*) \le F (x^*, y)$.
\end{Proof}






















